\documentclass{beamer}
\usepackage[utf8]{inputenc}

\usetheme{Madrid}
\usecolortheme{default}
\usepackage{amsmath,amssymb,amsfonts,amsthm}
\usepackage{txfonts}
\usepackage{tkz-euclide}
\usepackage{listings}
\usepackage{adjustbox}
\usepackage{array}
\usepackage{tabularx}
\usepackage{gvv}
\usepackage{lmodern}
\usepackage{circuitikz}
\usepackage{tikz}
\usepackage{graphicx}

\setbeamertemplate{page number in head/foot}[totalframenumber]

\usepackage{tcolorbox}
\tcbuselibrary{minted,breakable,xparse,skins}



\definecolor{bg}{gray}{0.95}
\DeclareTCBListing{mintedbox}{O{}m!O{}}{%
  breakable=true,
  listing engine=minted,
  listing only,
  minted language=#2,
  minted style=default,
  minted options={%
    linenos,
    gobble=0,
    breaklines=true,
    breakafter=,,
    fontsize=\small,
    numbersep=8pt,
    #1},
  boxsep=0pt,
  left skip=0pt,
  right skip=0pt,
  left=25pt,
  right=0pt,
  top=3pt,
  bottom=3pt,
  arc=5pt,
  leftrule=0pt,
  rightrule=0pt,
  bottomrule=2pt,
  toprule=2pt,
  colback=bg,
  colframe=orange!70,
  enhanced,
  overlay={%
    \begin{tcbclipinterior}
    \fill[orange!20!white] (frame.south west) rectangle ([xshift=20pt]frame.north west);
    \end{tcbclipinterior}},
  #3,
}
\lstset{
    language=C,
    basicstyle=\ttfamily\small,
    keywordstyle=\color{blue},
    stringstyle=\color{orange},
    commentstyle=\color{green!60!black},
    numbers=left,
    numberstyle=\tiny\color{gray},
    breaklines=true,
    showstringspaces=false,
}
%------------------------------------------------------------

\title
{1.6.18}
\date{August 26,2025}
\author 
{AI25BTECH11003 - Bhavesh Gaikwad}



\begin{document}


\frame{\titlepage}
\begin{frame}{Question}
\centering
Prove that points A(2,1), B(0,5) and C(-1,2) are collinear.
\end{frame}


\begin{frame}[fragile]
    \frametitle{Theoretical Solution}
B-A=$\myvec{0-2 \\ 5-1}$ = $\myvec{-2 \\ 4}$
$\qquad C-A=\myvec{-1-2 \\ 2-1}$ =$\myvec{-3 \\ 1}$ \\
M = $\myvec{B-A & C-A}$ = $\myvec{-2 & -3 \\ 4 & 1}$\\\\

Row-reduce to compute the rank:\\

$\myvec{-2 & -3\\ 4 & 1}$ $\xrightarrow{\;R_2\leftarrow R_2+2R_1\;}$ $\myvec{-2 & -3 \\ 0 & -5}$\\

The echelon form has two nonzero rows, hence
rank(M)=2$\neq$1

\begin{align}
    \centering
    \boxed{Therefore, \, The \, points \, A(2,1), \,B(0,5) \, and \, C(-1,2) \, are \, not \, collinear.}
\end{align}
\end{frame}


\begin{frame}[fragile]
    \frametitle{C Code}
    \begin{lstlisting}
#include <stdio.h>
int main(void) {
int Ax = 2, Ay = 1;
int Bx = 0, By = 5;
int Cx = -1, Cy = 2;

// Columns of M: B-A and C-A
int m11 = Bx - Ax; 
int m21 = By - Ay; 
int m12 = Cx - Ax; 
int m22 = Cy - Ay; 

int det = m11 * m22 - m12 * m21;
 \end{lstlisting}
\end{frame}


\begin{frame}[fragile]
    \frametitle{C Code}
    \begin{lstlisting}
  printf("Matrix M = [[%d, %d],[%d, %d]]\n", m11, m12, m21, m22);
printf("det(M) = %d\n", det);

if (det != 0) {
    printf("rank(M) = 2 -> Points are NOT collinear.\n");
} else if (m11 != 0 || m21 != 0 || m12 != 0 || m22 != 0) {
    printf("rank(M) = 1 -> Points are collinear.\n");
} else {
    printf("rank(M) = 0 (degenerate).\n");
}
return 0;
}
    \end{lstlisting}
\end{frame}

\begin{frame}[fragile]
    \frametitle{Python Code}
    \begin{lstlisting}
import numpy as np
import sympy as sp
import matplotlib.pyplot as plt

A = np.array([2, 1])
B = np.array([0, 5])
C = np.array([-1, 2])

# Collinearity matrix [B-A  C-A]
M = sp.Matrix.hstack(sp.Matrix(B - A), sp.Matrix(C - A))
print("Collinearity matrix M =\n", np.array(M, dtype=int))
print("rank(M) =", M.rank())
print("Conclusion:", "Collinear" if M.rank() == 1 else "NOT collinear")
\end{lstlisting}
\end{frame}

\begin{frame}[fragile]
    \frametitle{Python Code}
    \begin{lstlisting}
plt.figure(figsize=(4, 4))
pts = np.vstack([A, B, C])
plt.scatter(pts[:, 0], pts[:, 1], color=['tab:blue', 'tab:orange', 'tab:green'], s=80)
for (x, y), label in zip(pts, ['A(2,1)', 'B(0,5)', 'C(-1,2)']):
    plt.annotate(label, (x + 0.05, y + 0.05))  # <-- This must be indented

plt.plot([A[0], B[0]], [A[1], B[1]], 'k--', alpha=0.6)
plt.plot([A[0], C[0]], [A[1], C[1]], 'k--', alpha=0.6)
plt.axhline(0, color='black', linewidth=0.5)
plt.axvline(0, color='black', linewidth=0.5)
plt.grid(True, linestyle=':')
plt.xlim(-2, 4)
plt.ylim(0, 7)
plt.title('Points A, B, C (rank test)')
plt.tight_layout()
plt.savefig('fig1.png', dpi=150)
    \end{lstlisting}
\end{frame}


\begin{frame}{Graph}
   \centering
    \includegraphics[width=\columnwidth, height=0.8\textheight, keepaspectratio]{figs/fig1.png}
    \label{fig:Beamer/figs/fig1.png}
\end{frame}


\end{document}