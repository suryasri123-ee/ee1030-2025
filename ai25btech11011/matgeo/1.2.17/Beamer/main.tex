\documentclass{beamer}
\usepackage[utf8]{inputenc}

\usetheme{Madrid}
\usecolortheme{default}
\usepackage{amsmath,amssymb,amsfonts,amsthm}
\usepackage{txfonts}
\usepackage{tkz-euclide}
\usepackage{listings}
\usepackage{adjustbox}
\usepackage{array}
\usepackage{tabularx}
\usepackage{gvv}
\usepackage{lmodern}
\usepackage{circuitikz}
\usepackage{tikz}
\usepackage{graphicx}

\setbeamertemplate{page number in head/foot}[totalframenumber]

\usepackage{tcolorbox}
\tcbuselibrary{minted,breakable,xparse,skins}



\definecolor{bg}{gray}{0.95}
\DeclareTCBListing{mintedbox}{O{}m!O{}}{%
  breakable=true,
  listing engine=minted,
  listing only,
  minted language=#2,
  minted style=default,
  minted options={%
    linenos,
    gobble=0,
    breaklines=true,
    breakafter=,,
    fontsize=\small,
    numbersep=8pt,
    #1},
  boxsep=0pt,
  left skip=0pt,
  right skip=0pt,
  left=25pt,
  right=0pt,
  top=3pt,
  bottom=3pt,
  arc=5pt,
  leftrule=0pt,
  rightrule=0pt,
  bottomrule=2pt,
  toprule=2pt,
  colback=bg,
  colframe=orange!70,
  enhanced,
  overlay={%
    \begin{tcbclipinterior}
    \fill[orange!20!white] (frame.south west) rectangle ([xshift=20pt]frame.north west);
    \end{tcbclipinterior}},
  #3,
}
\lstset{
    language=C,
    basicstyle=\ttfamily\small,
    keywordstyle=\color{blue},
    stringstyle=\color{orange},
    commentstyle=\color{green!60!black},
    numbers=left,
    numberstyle=\tiny\color{gray},
    breaklines=true,
    showstringspaces=false,
}
%------------------------------------------------------------
%This block of code defines the information to appear in the
%Title page
\title %optional
{1.2.17}
%\subtitle{A short story}

\author % (optional)
{AI25BTECH11011-VARUN}


\begin{document}


\frame{\titlepage}
\begin{frame}{Question}
Three vertices of a parallelogram ABCD are A(3,-1,2),B(1,-2,4),C(-1,1,2). Find the coordinates of the fourth vertex.
\end{frame}



\begin{frame}{Theoretical Solution}

Let the vertices of parallelogram ABCD be
$\Vec{A}\myvec{3\\-1\\2}$, $\Vec{ B}\myvec{1\\-2\\4}$, $\Vec{C}\myvec{-1\\1\\2}$.
In any parallelogram, the diagonals bisect each other, so the midpoints of $\vec{AC}$ and $\vec{BD}$ are equal.

\end{frame}

\begin{frame}{Equation}
\textbf{The  midpoint of $\vec{A}$ and $\vec{B}$ is}

\begin{align}
    \vec{M_{AB}} = \frac{\vec{A}+\vec{B}}{2}
\end{align}
\end{frame}


\begin{frame}{Theoretical Solution}
Midpoint of $\vec{AC}$:

\begin{align}
    \vec{M_{AC}} = \frac{\vec{A}+\vec{C}}{2}
\end{align}

Midpoint of $\vec{BD}$:

\begin{align}
    \vec{M_{BD}} = \frac{\vec{B}+\vec{D}}{2}
\end{align}
\end{frame}


\begin{frame}{Theoretical Solution}
As $\vec{M_{AC}}=\vec{M_{BD}}$:

\begin{align}
     \vec{A} + \vec{C} = \vec{B} + \vec{D}  \\
     \vec{D} = \vec{A} + \vec{C} - \vec{B}  \\
     \vec{D} = \myvec{3\\-1\\2} + \myvec{-1\\1\\2}  - \myvec{1\\-2\\4}  \\
     \vec{D} = \myvec{3+(-1)-1 \\ -1+1-(-2) \\ 2+2-4} = \myvec{1\\2\\0}
\end{align}
The fourth vertex is $\vec{D}\myvec{1\\2\\0}$.
\end{frame}


\begin{frame}[fragile]
    \frametitle{main C Code}

    \begin{lstlisting}
#include <stdio.h>

int main() {
    double A[3] = {3, -1, 2};
    double B[3] = {1, -2, 4};
    double C[3] = {-1, 1, 2};
    double D[3];

    D[0] = A[0] + C[0] - B[0];
    D[1] = A[1] + C[1] - B[1];
    D[2] = A[2] + C[2] - B[2];

    FILE *fp = fopen("coords.dat", "w");
    if (fp == NULL) {
        printf("Error opening file!\n");
        return 1;
    }
    \end{lstlisting}
\end{frame}

\begin{frame}[fragile]
    \frametitle{main C Code}

    \begin{lstlisting}
   fprintf(fp, "%lf %lf %lf\n", A[0], A[1], A[2]);
    fprintf(fp, "%lf %lf %lf\n", B[0], B[1], B[2]);
    fprintf(fp, "%lf %lf %lf\n", C[0], C[1], C[2]);
    fprintf(fp, "%lf %lf %lf\n", D[0], D[1], D[2]);

    fclose(fp);

    printf("Fourth vertex D: (%.2lf, %.2lf, %.2lf)\n", D[0], D[1], D[2]);

    return 0;
}
    \end{lstlisting}
\end{frame}

\begin{frame}[fragile]
    \frametitle{C Code}

    \begin{lstlisting}
#include <stdio.h>

void find_fourth_vertex(double A[3], double B[3], double C[3], double D[3]) {
    D[0] = A[0] + C[0] - B[0];
    D[1] = A[1] + C[1] - B[1];
    D[2] = A[2] + C[2] - B[2];
}
    \end{lstlisting}
\end{frame}
\begin{frame}[fragile]
    \frametitle{Python Code}
    \begin{lstlisting}
from ctypes import CDLL, c_double, POINTER
import numpy as np
import matplotlib.pyplot as plt

# Load the shared library
lib = CDLL("./libvertex.so")

# Define argument and return types
lib.find_fourth_vertex.argtypes = [POINTER(c_double), POINTER(c_double), POINTER(c_double), POINTER(c_double)]

# Define points
A = (c_double * 3)(3, -1, 2)
B = (c_double * 3)(1, -2, 4)
C = (c_double * 3)(-1, 1, 2)
D = (c_double * 3)()
    \end{lstlisting}
\end{frame}

\begin{frame}[fragile]
    \frametitle{Python Code}
    \begin{lstlisting}
# Call the C function
lib.find_fourth_vertex(A, B, C, D)

# Convert to Python list
D_point = [D[i] for i in range(3)]
print(f"Fourth vertex D: {D_point}")

#  Read coordinates from .dat file (generated by main C code)
coords = np.loadtxt("coords.dat")

#  Plot the parallelogram
fig = plt.figure()
ax = fig.add_subplot(111, projection='3d')

# Points
x = coords[:, 0]
y = coords[:, 1]
z = coords[:, 2]
    \end{lstlisting}
\end{frame}

\begin{frame}[fragile]
    \frametitle{Python Code}
    \begin{lstlisting}
# Plot parallelogram edges
ax.plot([x[0], x[1]], [y[0], y[1]], [z[0], z[1]], 'r-')
ax.plot([x[1], x[2]], [y[1], y[2]], [z[1], z[2]], 'r-')
ax.plot([x[2], x[3]], [y[2], y[3]], [z[2], z[3]], 'r-')
ax.plot([x[3], x[0]], [y[3], y[0]], [z[3], z[0]], 'r-')

# Plot diagonals
ax.plot([x[0], x[2]], [y[0], y[2]], [z[0], z[2]], 'b--')
ax.plot([x[1], x[3]], [y[1], y[3]], [z[1], z[3]], 'b--')

# Labels
ax.scatter(x, y, z, color='black')
for i, txt in enumerate(['A', 'B', 'C', 'D']):
    ax.text(x[i], y[i], z[i], txt)
plt.savefig("/home/gara-varun-kumar/ee1030-2025/ai25btech11011/matgeo/1.2.17/figs/Fig 1.png")
plt.show()
    \end{lstlisting}
\end{frame}

\begin{frame}{Plot}
    \centering
    \includegraphics[width=\columnwidth, height=0.8\textheight, keepaspectratio]{figs/Fig 1.png}     
\end{frame}

\end{document}
