\let\negmedspace\undefined
\let\negthickspace\undefined
\documentclass[journal]{IEEEtran}


\setlength{\headheight}{1cm} % Set the height of the header box
\setlength{\headsep}{0mm}     % Set the distance between the header box and the top of the text
 \usepackage[a4paper,margin=10mm, onecolumn]{geometry}
\usepackage{gvv-book}
\usepackage{gvv}
\usepackage{cite}
\usepackage{amsmath,amssymb,amsfonts,amsthm}
\usepackage{algorithmic}
\usepackage{graphicx}
\usepackage{textcomp}
\usepackage{xcolor}
\usepackage{txfonts}
\usepackage{listings}
\usepackage{enumitem}
\usepackage{mathtools}
\usepackage{gensymb}
\usepackage{comment}
\usepackage[breaklinks=true]{hyperref}
\usepackage{tkz-euclide} 
\usepackage{listings}                                       
\def\inputGnumericTable{}                                
\usepackage[latin1]{inputenc}                                
\usepackage{color}                                            
\usepackage{array}                                            
\usepackage{longtable}                                       
\usepackage{calc}                                             
\usepackage{multirow}                                         
\usepackage{hhline}                                           
\usepackage{ifthen}                                           
\usepackage{lscape}
\usepackage{circuitikz}
\tikzstyle{block} = [rectangle, draw, fill=blue!20, 
    text width=4em, text centered, rounded corners, minimum height=3em]
\tikzstyle{sum} = [draw, fill=blue!10, circle, minimum size=1cm, node distance=1.5cm]
\tikzstyle{input} = [coordinate]
\tikzstyle{output} = [coordinate]

\begin{document}

\bibliographystyle{IEEEtran}
\vspace{3cm}

\title{1.2.17}
\author{AI25BTECH11011-VARUN}
 \maketitle
% \newpage
% \bigskip
{\let\newpage\relax\maketitle}

\renewcommand{\thefigure}{\theenumi}
\renewcommand{\thetable}{\theenumi}
\setlength{\intextsep}{10pt} % Space between text and floats


\numberwithin{equation}{enumi}
\numberwithin{figure}{enumi}
\renewcommand{\thetable}{\theenumi}
\textbf{Question}:\\
Three vertices of a parallelogram ABCD are A(3,-1,2),B(1,-2,4),C(-1,1,2). Find the coordinates of the fourth vertex.

\textbf{Solution}:\\
Let the vertices of parallelogram ABCD be
$\Vec{A}\myvec{3\\-1\\2}$, $\Vec{ B}\myvec{1\\-2\\4}$, $\Vec{C}\myvec{-1\\1\\2}$.
In any parallelogram, the diagonals bisect each other, so the midpoints of $\vec{AC}$ and $\vec{BD}$ are equal.

\textbf{The  midpoint of $\vec{A}$ and $\vec{B}$ is}

\begin{align}
    \vec{M_{AB}} = \frac{\vec{A}+\vec{B}}{2}
\end{align}

Midpoint of $\vec{AC}$:

\begin{align}
    \vec{M_{AC}} = \frac{\vec{A}+\vec{C}}{2}
\end{align}

Midpoint of $\vec{BD}$:

\begin{align}
    \vec{M_{BD}} = \frac{\vec{B}+\vec{D}}{2}
\end{align}

As $\vec{M_{AC}}=\vec{M_{BD}}$:

\begin{align}
     \vec{A} + \vec{C} = \vec{B} + \vec{D}  \\
     \vec{D} = \vec{A} + \vec{C} - \vec{B}  \\
     \vec{D} = \myvec{3\\-1\\2} + \myvec{-1\\1\\2}  - \myvec{1\\-2\\4}  \\
     \vec{D} = \myvec{3+(-1)-1 \\ -1+1-(-2) \\ 2+2-4}  \\
     \vec{D} = \myvec{1\\2\\0}
\end{align}

The fourth vertex is $\vec{D}\myvec{1\\2\\0}$.

\begin{figure}[H]
    \centering
    \includegraphics[width=0.8\textwidth]{figs/Fig 1.png}
    \caption{}
    \label{fig:Parallelogram}
\end{figure}

\end{document}
