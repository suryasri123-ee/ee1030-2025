\let\negmedspace\undefined
\let\negthickspace\undefined
\documentclass[journal]{IEEEtran}
\usepackage[a5paper, margin=10mm, onecolumn]{geometry}
%\usepackage{lmodern} % Ensure lmodern is loaded for pdflatex
\usepackage{tfrupee} % Include tfrupee package

\setlength{\headheight}{1cm} % Set the height of the header box
\setlength{\headsep}{0mm}     % Set the distance between the header box and the top of the text

\usepackage{gvv-book}
\usepackage{gvv}
\usepackage{cite}
\usepackage{amsmath,amssymb,amsfonts,amsthm}
\usepackage{algorithmic}
\usepackage{graphicx}
\usepackage{textcomp}
\usepackage{xcolor}
\usepackage{txfonts}
\usepackage{listings}
\usepackage{enumitem}
\usepackage{mathtools}
\usepackage{gensymb}
\usepackage{comment}
%\usepackage{multiclo}
\usepackage[breaklinks=true]{hyperref}
\usepackage{tkz-euclide} 
\usepackage{listings}
% \usepackage{gvv} 
\graphicspath{ {./figs/} }

\begin{document}

\title{
ME: MECHANICAL ENGINEERING}
\author{AI25BTECH11011}
\maketitle
\renewcommand{\thefigure}{\theenumi}
\renewcommand{\thetable}{\theenumi}

\begin{enumerate}

\item He did not manage to fix the car himself, so he ------------ in the garage.

\begin{enumerate}
    \item got it fixed
    \item getting it fixed
    \item gets fixed
    \item got fixed
\end{enumerate}
\hfill (GATE ME 2023)

\item Planting : Seed :: Raising : ------------

(By word meaning)

\begin{enumerate}
    \item Child
    \item Temperature
    \item Height
    \item Lift
\end{enumerate}
\hfill (GATE ME 2023)

\item A certain country has 504 universities and 25951 colleges. These are categorised into Grades I, II, and III as shown in the given pie charts. What is the percentage, correct to one decimal place, of higher education institutions (colleges and universities) that fall into Grade III?

\begin{figure}[H]
    \centering
    \includegraphics[width=0.8\textwidth]{Fig 1.png}
    \caption{}
    \label{fig:question3}
\end{figure}

\begin{enumerate}
    \item 22.7
    \item 23.7
    \item 15.0
    \item 66.8
\end{enumerate}
\hfill (GATE ME 2023)

\item The minute-hand and second-hand of a clock cross each other ------------ times between 09:15:00 AM and 09:45:00 AM on a day.

\begin{enumerate}
    \item 30
    \item 15
    \item 29
    \item 31
\end{enumerate}
\hfill (GATE ME 2023)

\item The symbols $ \bigcirc, \ast, \triangle $, and $ \square $ are to be filled, one in each box, as shown below. The rules for filling in the four symbols are as follows.

1) Every row and every column must contain each of the four symbols.

2) Every 2×2 square delineated by bold lines must contain each of the four symbols.

Which symbol will occupy the box marked with ‘$ ? $’ in the partially filled figure?

\begin{figure}[H]
    \centering
    \includegraphics[width=0.3\textwidth]{Fig 2.png}
    \caption{}
    \label{fig:question5}
\end{figure}

\begin{enumerate}
    \item $ \bigcirc $
    \item $ \ast $
    \item $ \triangle $
    \item $ \square $
\end{enumerate}
\hfill (GATE ME 2023)

\item In a recently held parent-teacher meeting, the teachers had very few complaints about Ravi. After all, Ravi was a hardworking and kind student. Incidentally, almost all of Ravi’s friends at school were hardworking and kind too. But the teachers drew attention to Ravi’s complete lack of interest in sports. The teachers believed that, along with some of his friends who showed similar disinterest in sports, Ravi needed to engage in some sports for his overall development.

Based only on the information provided above, which one of the following statements can be logically inferred with certainty?

\begin{enumerate}
    \item All of Ravi’s friends are hardworking and kind.
    \item No one who is not a friend of Ravi is hardworking and kind.
    \item None of Ravi’s friends are interested in sports.
    \item Some of Ravi’s friends are hardworking and kind.
\end{enumerate}
\hfill (GATE ME 2023)

\item Consider the following inequalities

$ p^2 - 4q < 4 $

$ 3p + 2q < 6 $

where $ p $ and $ q $ are positive integers.

The value of $ (p + q) $ is ------------.

\begin{enumerate}
    \item 2
    \item 1
    \item 3
    \item 4
\end{enumerate}
\hfill (GATE ME 2023)

\item Which one of the sentence sequences in the given options creates a coherent narrative? 

(i) I could not bring myself to knock. 

(ii) There was a murmur of unfamiliar voices coming from the big drawing room and the door was firmly shut. 

(iii) The passage was dark for a bit, but then it suddenly opened into a bright kitchen. 

(iv) I decided I would rather wander down the passage.

\begin{enumerate}
    \item (iv), (i), (iii), (ii)
    \item (iii), (i), (ii), (iv)
    \item (ii), (i), (iv), (iii)
    \item (i), (iii), (ii), (iv)
\end{enumerate}
\hfill (GATE ME 2023)

\item How many pairs of sets (S,T) are possible among the subsets of \{1, 2, 3, 4, 5, 6\} that satisfy the condition that S is a subset of T?

\begin{enumerate}
    \item 729
    \item 728
    \item 665
    \item 664
\end{enumerate}
\hfill (GATE ME 2023)

\item An opaque pyramid (shown below), with a square base and isosceles faces, is suspended in the path of a parallel beam of light, such that its shadow is cast on a screen oriented perpendicular to the direction of the light beam. The pyramid can be reoriented in any direction within the light beam. Under these conditions, which one of the shadows \textbf{P},\textbf{Q}, \textbf{R}, and \textbf{S} is NOT possible?
\begin{figure}[H]
\centering
\includegraphics[width=0.6\textwidth]{Fig 3.png}
\caption{}
\label{fig:question10}
\end{figure}

\begin{enumerate}
    \item \textbf{P}
    \item \textbf{Q}
    \item \textbf{R}
    \item \textbf{S}
\end{enumerate}
\hfill (GATE ME 2023)

\item A machine produces a defective component with a probability of 0.015. The number of defective components in a packed box containing 200 components produced by the machine follows a Poisson distribution. The mean and the variance of the distribution are

\begin{enumerate}
    \item 3 and 3, respectively
    \item $ \sqrt{3} $ and $ \sqrt{3} $, respectively
    \item 0.015 and 0.015, respectively
    \item 3 and 9, respectively
\end{enumerate}
\hfill (GATE ME 2023)

\item The figure shows the plot of a function over the interval [-4, 4]. Which one of the options given CORRECTLY identifies the function?
\begin{figure}[H]
\centering
\includegraphics[width=0.6\textwidth]{Fig 4.png}
\caption{}
\label{fig:question12}
\end{figure}

\begin{enumerate}
    \item $ |2 - x| $
    \item $ |2 - |x|| $
    \item $ |2 + |x|| $
    \item $ 2 - |x| $
\end{enumerate}
\hfill (GATE ME 2023)

\item With reference to the Economic Order Quantity (EOQ) model, which one of the options given is correct?
\begin{figure}[H]
\centering
\includegraphics[width=0.6\textwidth]{Fig 5.png}
\caption{}
\label{fig:question13}
\end{figure}

\begin{enumerate}
    \item Curve P1: Total cost, Curve P2: Holding cost, Curve P3: Setup cost, and Curve P4: Production cost.
    \item Curve P1: Holding cost, Curve P2: Setup cost, Curve P3: Production cost, and Curve P4: Total cost.
    \item Curve P1: Production cost, Curve P2: Holding cost, Curve P3: Total cost, and Curve P4: Setup cost.
    \item Curve P1: Total cost, Curve P2: Production cost, Curve P3: Holding cost, and Curve P4: Setup cost.
\end{enumerate}
\hfill (GATE ME 2023)

\item Which one of the options given represents the feasible region of the linear programming model:

\begin{center}
Maximize $ 45X_1 + 60X_2 $

$ X_1 \leq 45 $

$ X_2 \leq 50 $

$ 10X_1 + 10X_2 \geq 600 $

$ 25X_1 + 5X_2 \leq 750 $
\end{center}

\begin{figure}[H]
\centering
\includegraphics[width=0.8\textwidth]{Fig 6.png}
\caption{}
\label{fig:question14}
\end{figure}

\begin{enumerate}
    \item Region P
    \item Region Q
    \item Region R
    \item Region S
\end{enumerate}
\hfill (GATE ME 2023)

\item A cuboidal part has to be accurately positioned first, arresting six degrees of freedom and then clamped in a fixture, to be used for machining. Locating pins in the form of cylinders with hemi-spherical tips are to be placed on the fixture for positioning. Four different configurations of locating pins are proposed as shown. Which one of the options given is correct?

\begin{figure}[H]
\centering
\includegraphics[width=0.7\textwidth]{Fig 7.png}
\caption{}
\label{fig:question15}
\end{figure}

\begin{enumerate}
    \item Configuration P1 arrests 6 degrees of freedom, while Configurations P2 and P4 are over-constrained and Configuration P3 is under-constrained.
    \item Configuration P2 arrests 6 degrees of freedom, while Configurations P1 and P3 are over-constrained and Configuration P4 is under-constrained.
    \item Configuration P3 arrests 6 degrees of freedom, while Configurations P2 and P4 are over-constrained and Configuration P1 is under-constrained.
    \item Configuration P4 arrests 6 degrees of freedom, while Configurations P1 and P3 are over-constrained and Configuration P2 is under-constrained.
\end{enumerate}
\hfill (GATE ME 2023)

\item The effective stiffness of a cantilever beam of length $ L $ and flexural rigidity $ EI $ subjected to a transverse tip load W is

\begin{figure}[H]
\includegraphics[width=0.6\textwidth]{Fig 8.png}
\caption{}
\label{fig:question16}
\end{figure}

\begin{enumerate}
    \item $ \frac{3EI}{L^3} $
    \item $ \frac{2EI}{L^3} $
    \item $ \frac{L^3}{2EI} $
    \item $ \frac{L^3}{3EI} $
\end{enumerate}
\hfill (GATE ME 2023)

\item The options show frames consisting of rigid bars connected by pin joints. Which one of the frames is non-rigid?

\begin{enumerate}
    \item \includegraphics[width=0.2\textwidth]{Fig 9.png}
    \item \includegraphics[width=0.2\textwidth]{Fig 10.png}
    \item \includegraphics[width=0.2\textwidth]{Fig 11.png}
    \item \includegraphics[width=0.2\textwidth]{Fig 12.png}
\end{enumerate}
\hfill (GATE ME 2023)

\item The S-N curve from a fatigue test for steel is shown. Which one of the options gives the endurance limit?
\begin{figure}[H]
\centering
\includegraphics[width=0.5\textwidth]{Fig 13.png}
\caption{}
\label{fig:question18}
\end{figure}

\begin{enumerate}
    \item $ S_{ut} $
    \item $ S_2 $
    \item $ S_3 $
    \item $ S_4 $
\end{enumerate}
\hfill (GATE ME 2023)

\item Air (density = 1.2 kg/m³, kinematic viscosity = 1.5×10$^{-5}$ m²/s) flows over a flat plate with a free-stream velocity of 2 m/s. The wall shear stress at a location 15 mm from the leading edge is $ \tau_w $. What is the wall shear stress at a location 30 mm from the leading edge?

\begin{enumerate}
    \item $ \tau_w / 2 $
    \item $ \sqrt{2} \tau_w $
    \item $ 2 \tau_w $
    \item $ \tau_w / \sqrt{2} $
\end{enumerate}
\hfill (GATE ME 2023)

\item Consider an isentropic flow of air (ratio of specific heats = 1.4) through a duct as shown in the figure. The variations in the flow across the cross-section are negligible. The flow conditions at Location 1 are given as follows: $ P_1 = 100 $ kPa, $ \rho_1 = 1.2 $ kg/m$^3$, $ u_1 = 400 $ m/s. The duct cross-sectional area at Location 2 is given by A$_2$ = 2A$_1$, where A$_1$ denotes the duct cross-sectional area at Location 1. Which one of the given statements about the velocity $ u_2 $ and pressure $ P_2 $ at Location 2 is TRUE?
\begin{figure}[H]
\centering
\includegraphics[width=0.5\textwidth]{Fig 14.png}
\caption{}
\label{fig:question20}
\end{figure}

\begin{enumerate}
    \item $ u_2 < u_1 , P_2 < P_1 $
    \item $ u_2 < u_1 , P_2 > P_1 $
    \item $ u_2 > u_1 , P_2 < P_1 $
    \item $ u_2 > u_1 , P_2 > P_1 $
\end{enumerate}
\hfill (GATE ME 2023)

\item Consider incompressible laminar flow of a constant property Newtonian fluid in an isothermal circular tube. The flow is steady with fully-developed temperature and velocity profiles. The Nusselt number for this flow depends on

\begin{enumerate}
    \item neither the Reynolds number nor the Prandtl number
    \item both the Reynolds and Prandtl numbers
    \item the Reynolds number but not the Prandtl number
    \item the Prandtl number but not the Reynolds number
\end{enumerate}
\hfill (GATE ME 2023)

\item A heat engine extracts heat ($Q_H$) from a thermal reservoir at a temperature of 1000 K and rejects heat ($Q_L$) to a thermal reservoir at a temperature of 100 K, while producing work ($W$). Which one of the combinations of $[Q_H, Q_L \text{ and } W]$ given is allowed?

\begin{enumerate}
    \item $Q_H = 2000  J,  Q_L = 500  J,  W = 1000  J$
    \item $Q_H = 2000  J,  Q_L = 750  J,  W = 1250  J$
    \item $Q_H = 6000  J,  Q_L = 500  J,  W = 5500  J$
    \item $Q_H = 6000  J,  Q_L = 600  J,  W = 5500  J$
\end{enumerate}
\hfill (GATE ME 2023)

\item Two surfaces P and Q are to be joined together. In which of the given joining operation(s), there is no melting of the two surfaces P and Q for creating the joint?

\begin{enumerate}
    \item Arc welding
    \item Brazing
    \item Adhesive bonding
    \item Spot welding
\end{enumerate}
\hfill (GATE ME 2023)

\item A beam is undergoing pure bending as shown in the figure. The stress $ \sigma_y $ - strain $ \varepsilon $ curve for the material is also given. The yield stress of the material is $ \sigma_y $. Which of the option(s) given represent(s) the bending stress distribution at cross-section AA after plastic yielding?
\begin{figure}[H]
\centering
\includegraphics[width=0.7\textwidth]{Fig 15.png}
\caption{}
\label{fig:question24}
\end{figure}

\begin{enumerate}
    \item \includegraphics[width=0.2\textwidth]{Fig 16.png}
    \item \includegraphics[width=0.2\textwidth]{Fig 17.png}
    \item \includegraphics[width=0.2\textwidth]{Fig 18.png}
    \item \includegraphics[width=0.2\textwidth]{Fig 19.png}
\end{enumerate}
\hfill (GATE ME 2023)

\item In a metal casting process to manufacture parts, both patterns and moulds provide shape by dictating where the material should or should not go. Which of the option(s) given correctly describe(s) the mould and the pattern?

\begin{enumerate}
    \item Mould walls indicate boundaries within which the molten part material is allowed, while pattern walls indicate boundaries of regions where mould material is not allowed.
    \item Moulds can be used to make patterns.
    \item Pattern walls indicate boundaries within which the molten part material is allowed, while mould walls indicate boundaries of regions where mould material is not allowed.
    \item Patterns can be used to make moulds.
\end{enumerate}
\hfill (GATE ME 2023)

\item The principal stresses at a point P in a solid are 70 MPa, –70 MPa and 0. The yield stress of the material is 100 MPa. Which prediction(s) about material failure at P is/are CORRECT?

\begin{enumerate}
    \item Maximum normal stress theory predicts that the material fails
    \item Maximum shear stress theory predicts that the material fails
    \item Maximum normal stress theory predicts that the material does not fail
    \item Maximum shear stress theory predicts that the material does not fail
\end{enumerate}
\hfill (GATE ME 2023)

\item Which of the plot(s) shown is/are valid Mohr’s circle representations of a plane stress state in a material? (The center of each circle is indicated by O.)
\begin{figure}[H]
\centering
\includegraphics[width=0.8\textwidth]{Fig 20.png}
\caption{}
\label{fig:question27}
\end{figure}

\begin{enumerate}
    \item M1
    \item M2
    \item M3
    \item M4
\end{enumerate}
\hfill (GATE ME 2023)

\item Consider a laterally insulated rod of length L and constant thermal conductivity. Assuming one-dimensional heat conduction in the rod, which of the following steady-state temperature profile(s) can occur without internal heat generation?

\begin{enumerate}
    \item \includegraphics[width=0.2\textwidth]{Fig 21.png}
    \item \includegraphics[width=0.2\textwidth]{Fig 22.png}
    \item \includegraphics[width=0.2\textwidth]{Fig 23.png}
    \item \includegraphics[width=0.2\textwidth]{Fig 24.png}
\end{enumerate}
\hfill (GATE ME 2023)

\item Two meshing spur gears 1 and 2 with diametral pitch of 8 teeth per mm and an angular velocity ratio $ |\omega_2|/|\omega_1| = \frac{1}{4} $, have their centers 30 mm apart. The number of teeth on the driver (gear 1) is ------------. (Answer in integer)

\begin{figure}[H]
\centering
\includegraphics[width=0.3\textwidth]{Fig 25.png}
\caption{}
\label{fig:question29}
\end{figure}
\hfill (GATE ME 2023)

\item The figure shows a block of mass $ m = 20  \text{kg} $ attached to a pair of identical linear springs, each having a spring constant $ k = 1000  \text{N/m} $. The block oscillates on a frictionless horizontal surface. Assuming free vibrations, the time taken by the block to complete ten oscillations is ------------ seconds. (Rounded off to two decimal places) Take $ \pi = 3.14 $.
\begin{figure}[H]
\centering
\includegraphics[width=0.6\textwidth]{Fig 26.png}
\caption{}
\label{fig:question30}
\end{figure}
\hfill (GATE ME 2023)

\item A vector field
\begin{center}
$ B(x, y, z) = x \hat{i} + y \hat{j} - 2z \hat{k} $
\end{center}
is defined over a conical region having height $ h = 2 $, base radius $ r = 3 $ and axis along $ z $, as shown in the figure. The base of the cone lies in the $ x-y $ plane and is centered at the origin.

If $ n $ denotes the unit outward normal to the curved surface $ S $ of the cone, the value of the integral
\begin{center}
$ \int_{S} B \cdot n  dS $
\end{center}
equals ------------. (Answer in integer)
\begin{figure}[H]
\centering
\includegraphics[width=0.4\textwidth]{Fig 27.png}
\caption{}
\label{fig:question31}
\end{figure}
\hfill (GATE ME 2023)

\item A linear transformation maps a point $(x, y)$ in the plane to the point $(\hat{x}, \hat{y})$ according to the rule
\begin{center}
$ \hat{x} = 3y, \quad \hat{y} = 2x. $
\end{center}
Then, the disc $x^2 + y^2 \leq 1$ gets transformed to a region with an area equal to ------------. (Rounded off to two decimals)

Use $ \pi = 3.14 $.
\hfill (GATE ME 2023)

\item The value of $k$ that makes the complex-valued function
\begin{center}
$ f(z) = e^{-kz} (\cos 2y - i \sin 2y) $
\end{center}
analytic, where $z = x + iy$, is ------------. 

(Answer in integer)
\hfill (GATE ME 2023)

\item The braking system shown in the figure uses a belt to slow down a pulley rotating in the clockwise direction by the application of a force P. The belt wraps around the pulley over an angle $\alpha = 270$ degrees. The coefficient of friction between the belt and the pulley is 0.3. The influence of centrifugal forces on the belt is negligible. During braking, the ratio of the tensions $T_1$ to $T_2$ in the belt is equal to ------------. (Rounded off to two decimal places) Take $ \pi = 3.14 $.
\begin{figure}[H]
\centering
\includegraphics[width=0.6\textwidth]{Fig 28.png}
\caption{}
\label{fig:question34}
\end{figure}
\hfill (GATE ME 2023)

\item Consider a counter-flow heat exchanger with the inlet temperatures of two fluids (1 and 2) being $T_{1, in} = 300  K$ and $T_{2, in} = 350  K$. The heat capacity rates of the two fluids are $C_1 = 1000  W/K$ and $C_2 = 400  W/K$, and the effectiveness of the heat exchanger is 0.5. The actual heat transfer rate is ------------ kW.

(Answer in integer)
\hfill (GATE ME 2023)

\item Which one of the options given is the inverse Laplace transform of $ \frac{1}{s^3 - s} $? $ u(t) $ denotes the unit-step function.

\begin{enumerate}
    \item $ \left( -1 + \frac{1}{2} e^{-t} + \frac{1}{2} e^{t} \right) u(t) $
    \item $ \left( \frac{1}{3} e^{-t} - e^{t} \right) u(t) $
    \item $ \left( -1 + \frac{1}{2} e^{-(t-1)} + \frac{1}{2} e^{(t-1)} \right) u(t-1) $
    \item $ \left( -1 - \frac{1}{2} e^{-(t-1)} - \frac{1}{2} e^{(t-1)} \right) u(t-1) $
\end{enumerate}
\hfill (GATE ME 2023)

\item A spherical ball weighing 2 kg is dropped from a height of 4.9 m onto an immovable rigid block as shown in the figure. If the collision is perfectly elastic, what is the momentum vector of the ball (in kg m/s) just after impact? Take the acceleration due to gravity to be $ g = 9.8  \text{m/s}^2 $. Options have been rounded off to one decimal place.
\begin{figure}[H]
\centering
\includegraphics[width=0.6\textwidth]{Fig 29.png}
\caption{}
\label{fig:question37}
\end{figure}

\begin{enumerate}
    \item 19.6 $\hat{i}$
    \item 19.6 $\hat{j}$
    \item 17.0 $\hat{i} + 9.8 \hat{j}$
    \item 9.8 $\hat{i} + 17.0 \hat{j}$
\end{enumerate}
\hfill (GATE ME 2023)

\item The figure shows a wheel rolling without slipping on a horizontal plane with angular velocity $\omega_1$. A rigid bar PQ is pinned to the wheel at P while the end Q slides on the floor. 

What is the angular velocity $\omega_2$ of the bar PQ?
\begin{figure}[H]
\centering
\includegraphics[width=0.6\textwidth]{Fig 30.png}
\caption{}
\label{fig:question38}
\end{figure}

\begin{enumerate}
    \item $ \omega_2 = 2\omega_1 $
    \item $ \omega_2 = \omega_1 $
    \item $ \omega_2 = 0.5\omega_1 $
    \item $ \omega_2 = 0.25\omega_1 $
\end{enumerate}
\hfill (GATE ME 2023)

\item A beam of length $ L $ is loaded in the $ xy $ –plane by a uniformly distributed load, and by a concentrated tip load parallel to the $ z $ –axis, as shown in the figure. The resulting bending moment distributions about the $ y $ and the $ z $ axes are denoted by $ M_y $ and $ M_z $, respectively. Which one of the options given depicts qualitatively CORRECT variations of $ M_y $ and $ M_z $ along the length of the beam?
\begin{figure}[H]
\centering
\includegraphics[width=0.6\textwidth]{Fig 31.png}
\caption{}
\label{fig:question39}
\end{figure}

\begin{enumerate}
    \item \includegraphics[width=0.4\textwidth]{Fig 32.png}
    \item \includegraphics[width=0.4\textwidth]{Fig 33.png}
    \item \includegraphics[width=0.4\textwidth]{Fig 34.png}
    \item \includegraphics[width=0.4\textwidth]{Fig 35.png}
\end{enumerate}
\hfill (GATE ME 2023)

\item The figure shows a thin-walled open-top cylindrical vessel of radius $ r $ and wall thickness $ t $. The vessel is held along the brim and contains a constant-density liquid to height $ h $ from the base. Neglect atmospheric pressure, the weight of the vessel and bending stresses in the vessel walls. Which one of the plots depicts qualitatively CORRECT dependence of the magnitudes of axial wall stress ($\sigma_1$) and circumferential wall stress ($\sigma_2$) on $ y $?
\begin{figure}[H]
\centering
\includegraphics[width=0.4\textwidth]{Fig 36.png}
\caption{}
\label{fig:question40}
\end{figure}

\begin{enumerate}
    \item \includegraphics[width=0.2\textwidth]{Fig 37.png}
    \item \includegraphics[width=0.2\textwidth]{Fig 38.png}
    \item \includegraphics[width=0.2\textwidth]{Fig 39.png}
    \item \includegraphics[width=0.2\textwidth]{Fig 40.png}
\end{enumerate}
\hfill (GATE ME 2023)

\item Which one of the following statements is FALSE?

\begin{enumerate}
    \item For an ideal gas, the enthalpy is independent of pressure.
    \item For a real gas going through an adiabatic reversible process, the process equation is given by $ PV^\gamma = $ constant, where $ P $ is the pressure, $ V $ is the volume and $ \gamma $ is the ratio of the specific heats of the gas at constant pressure and constant volume.
    \item For an ideal gas undergoing a reversible polytropic process $ PV^{1.5} = $ constant, the equation connecting the pressure, volume and temperature of the gas at any point along the process is $ \frac{P}{R} = \frac{mT}{V} $, where $ R $ is the gas constant and $ m $ is the mass of the gas.
    \item Any real gas behaves as an ideal gas at sufficiently low pressure or sufficiently high temperature.
\end{enumerate}
\hfill (GATE ME 2023)

\item Consider a fully adiabatic piston-cylinder arrangement as shown in the figure. The piston is massless and cross-sectional area of the cylinder is $ A $. The fluid inside the cylinder is air (considered as a perfect gas), with $ \gamma $ being the ratio of the specific heat at constant pressure to the specific heat at constant volume for air. The piston is initially located at a position $ L_1 $. The initial pressure of the air inside the cylinder is $ P_1 \gg P_0 $, where $ P_0 $ is the atmospheric pressure. The stop $ S_I $ is instantaneously removed and the piston moves to the position $ L_2 $, where the equilibrium pressure of air inside the cylinder is $ P_2 \gg P_0 $. What is the work done by the piston on the atmosphere during this process?
\begin{figure}[H]
\centering
\includegraphics[width=0.6\textwidth]{Fig 41.png}
\caption{}
\label{fig:question42}
\end{figure}

\begin{enumerate}
    \item 0
    \item $ P_0 A(L_2 - L_1) $
    \item $ P_1 A L_1 \ln \frac{L_1}{L_2} $
    \item $ \frac{(P_2 L_2 - P_1 L_1) A}{(1 - \gamma)} $
\end{enumerate}
\hfill (GATE ME 2023)

\item A cylindrical rod of length h and diameter d is placed inside a cubic enclosure of side length L. S denotes the inner surface of the cube. The view-factor $ F_{S-S} $ is

\begin{enumerate}
    \item 0
    \item 1
    \item $ \frac{(\pi d h + \pi d^2/2)}{6L^2} $
    \item $ 1 - \frac{(\pi d h + \pi d^2/2)}{6L^2} $
\end{enumerate}
\hfill (GATE ME 2023)

\item In an ideal orthogonal cutting experiment (see figure), the cutting speed V is 1 m/s, the rake angle of the tool $\alpha = 5^\circ$, and the shear angle, $\phi$, is known to be 45°. Applying the ideal orthogonal cutting model, consider two shear planes PQ and RS close to each other. As they approach the thin shear zone (shown as a thick line in the figure), plane RS gets sheared with respect to PQ (point R1 shears to R2, and S1 shears to S2). Assuming that the perpendicular distance between PQ and RS is $\delta = 25 \mu m$, what is the value of shear strain rate (in s$^{-1}$) that the material undergoes at the shear zone?
\begin{figure}[H]
\centering
\includegraphics[width=0.8\textwidth]{Fig 42.png}
\caption{}
\label{fig:question44}
\end{figure}

\begin{enumerate}
    \item $ 1.84 \times 10^4 $
    \item $ 5.20 \times 10^4 $
    \item $ 0.71 \times 10^4 $
    \item $ 1.30 \times 10^4 $
\end{enumerate}
\hfill (GATE ME 2023)

\item A CNC machine has one of its linear positioning axes as shown in the figure, consisting of a motor rotating a lead screw, which in turn moves a nut horizontally on which a table is mounted. The motor moves in discrete rotational steps of 50 steps per revolution. The pitch of the screw is 5 mm and the total horizontal traverse length of the table is 100 mm. What is the total number of controllable locations at which the table can be positioned on this axis?
\begin{figure}[H]
\centering
\includegraphics[width=0.8\textwidth]{Fig 43.png}
\caption{}
\label{fig:question45}
\end{figure}

\begin{enumerate}
    \item 5000
    \item 2
    \item 1000
    \item 200
\end{enumerate}
\hfill (GATE ME 2023)

\item Cylindrical bars P and Q have identical lengths and radii, but are composed of different linear elastic materials. The Young’s modulus and coefficient of thermal expansion of Q are twice the corresponding values of P. Assume the bars to be perfectly bonded at the interface, and their weights to be negligible. 

The bars are held between rigid supports as shown in the figure and the temperature is raised by $\Delta T$. Assume that the stress in each bar is homogeneous and uniaxial. Denote the magnitudes of stress in P and Q by $\sigma_1$ and $\sigma_2$, respectively.

Which of the statement(s) given is/are CORRECT?
\begin{figure}[H]
\centering
\includegraphics[width=0.5\textwidth]{Fig 44.png}
\caption{}
\label{fig:question46}
\end{figure}

\begin{enumerate}
    \item The interface between P and Q moves to the left after heating
    \item The interface between P and Q moves to the right after heating
    \item $ \sigma_1 < \sigma_2 $
    \item $ \sigma_1 = \sigma_2 $
\end{enumerate}
\hfill (GATE ME 2023)

\item A very large metal plate of thickness $ d $ and thermal conductivity $ k $ is cooled by a stream of air at temperature $ T_\infty = 300 $ K with a heat transfer coefficient $ h $, as shown in the figure. The centerline temperature of the plate is $ T_P $. In which of the following case(s) can the lumped parameter model be used to study the heat transfer in the metal plate?
\begin{figure}[H]
\centering
\includegraphics[width=0.6\textwidth]{Fig 45.png}
\caption{}
\label{fig:question47}
\end{figure}

\begin{enumerate}
    \item $ h = 10  \text{Wm}^{-2}\text{K}^{-1}, k = 100  \text{Wm}^{-1}\text{K}^{-1}, d = 1  \text{mm}, T_P = 350  \text{K} $
    \item $ h = 100  \text{Wm}^{-2}\text{K}^{-1}, k = 100  \text{Wm}^{-1}\text{K}^{-1}, d = 1  \text{m}, T_P = 325  \text{K} $
    \item $ h = 100  \text{Wm}^{-2}\text{K}^{-1}, k = 1000  \text{Wm}^{-1}\text{K}^{-1}, d = 1  \text{mm}, T_P = 325  \text{K} $
    \item $ h = 1000  \text{Wm}^{-2}\text{K}^{-1}, k = 1  \text{Wm}^{-1}\text{K}^{-1}, d = 1  \text{m}, T_P = 350  \text{K} $
\end{enumerate}
\hfill (GATE ME 2023)

\item The smallest perimeter that a rectangle with area of 4 square units can have is ------------ units. 

(Answer in integer)

\hfill (GATE ME 2023)

\item Consider the second-order linear ordinary differential equation
\begin{center}
$ x^2 \frac{d^2 y}{dx^2} + x \frac{dy}{dx} - y = 0, \quad x \geq 1 $
\end{center}
with the initial conditions
\begin{center}
$ y(x = 1) = 6, \quad \left. \frac{dy}{dx} \right|_{x=1} = 2. $
\end{center}
The value of $ y $ at $ x = 2 $ equals ------------. 

(Answer in integer)

\hfill (GATE ME 2023)

\item The initial value problem 
\begin{center}
$ \frac{dy}{dt} + 2y = 0, \quad y(0) = 1 $
\end{center}
is solved numerically using the forward Euler’s method with a constant and positive time step of $\Delta t$. Let $y_n$ represent the numerical solution obtained after $n$ steps. The condition $|y_{n+1}| \leq |y_n|$ is satisfied if and only if $\Delta t$ does not exceed ------------. 

(Answer in integer)
\hfill (GATE ME 2023)

\item The atomic radius of a hypothetical face-centered cubic (FCC) metal is $(\sqrt{2}/10)$ nm. The atomic weight of the metal is 24.092 g/mol. Taking Avogadro’s number to be 6.023$\times 10^{23}$ atoms/mol, the density of the metal is ------------ kg/m$^3$. 

(Answer in integer)
\hfill (GATE ME 2023)

\item A steel sample with 1.5 wt.\% carbon (no other alloying elements present) is slowly cooled from 1100 °C to just below the eutectoid temperature (723 °C). A part of the iron-cementite phase diagram is shown in the figure. The ratio of the pro-eutectoid cementite content to the total cementite content in the microstructure that develops just below the eutectoid temperature is ------------. (Rounded off to two decimal places)
\begin{figure}[H]
\centering
\includegraphics[width=0.7\textwidth]{Fig 46.png}
\caption{}
\label{fig:question52}
\end{figure}
\hfill (GATE ME 2023)

\item A part, produced in high volumes, is dimensioned as shown. The machining process making this part is known to be statistically in control based on sampling data. The sampling data shows that D1 follows a normal distribution with a mean of 20 mm and a standard deviation of 0.3 mm, while D2 follows a normal distribution with a mean of 35 mm and a standard deviation of 0.4 mm. An inspection of dimension C is carried out in a sufficiently large number of parts. To be considered under six-sigma process control, the upper limit of dimension C should be ------------ mm. (Rounded off to one decimal place)
\begin{figure}[H]
\centering
\includegraphics[width=0.5\textwidth]{Fig 47.png}
\caption{}
\label{fig:question53}
\end{figure}
\hfill (GATE ME 2023)

\item A coordinate measuring machine (CMM) is used to determine the distance between Surface SP and Surface SQ of an approximately cuboidal shaped part. Surface SP is declared as the datum as per the engineering drawing used for manufacturing this part. The CMM is used to measure four points P1, P2, P3, P4 on Surface SP, and four points Q1, Q2, Q3, Q4 on Surface SQ as shown. A regression procedure is used to fit the necessary planes.

The distance between the two fitted planes is ------------ mm. 

(Answer in integer)

\begin{figure}[H]
\centering
\includegraphics[width=0.8\textwidth]{Fig 48.png}
\caption{}
\label{fig:question54}
\end{figure}
\hfill (GATE ME 2023)

\item A solid part (see figure) of polymer material is to be fabricated by additive manufacturing (AM) in square-shaped layers starting from the bottom of the part working upwards. The nozzle diameter of the AM machine is a/10 mm and the nozzle follows a linear serpentine path parallel to the sides of the square layers with a feed rate of a/5 mm/min.

Ignore any tool path motions other than those involved in adding material, and any other delays between layers or the serpentine scan lines.

The time taken to fabricate this part is ------------ minutes.

(Answer in integer)

\begin{figure}[H]
\centering
\includegraphics[width=0.8\textwidth]{Fig 49.png}
\caption{}
\label{fig:question55}
\end{figure}
\hfill (GATE ME 2023)

\item An optical flat is used to measure the height difference between a reference slip gauge A and a slip gauge B. Upon viewing via the optical flat using a monochromatic light of wavelength 0.5 µm, 12 fringes were observed over a length of 15 mm of gauge B. If the gauges are placed 45 mm apart, the height difference of the gauges is ------------ µm.

(Answer in integer)

\begin{figure}[H]
\centering
\includegraphics[width=0.5\textwidth]{Fig 50.png}
\caption{}
\label{fig:question56}
\end{figure}
\hfill (GATE ME 2023)

\item Ignoring the small elastic region, the true stress ($\sigma$) – true strain ($\varepsilon$) variation of a material beyond yielding follows the equation $\sigma = 400\varepsilon^{0.3}$ MPa. The engineering ultimate tensile strength value of this material is ------------ MPa.

(Rounded off to one decimal place)
\hfill (GATE ME 2023)

\item The area moment of inertia about the y-axis of a linearly tapered section shown in the figure is ------------ m$^4$. 

(Answer in integer)

\begin{figure}[H]
\centering
\includegraphics[width=0.6\textwidth]{Fig 51.png}
\caption{}
\label{fig:question58}
\end{figure}
\hfill (GATE ME 2023)

\item A cylindrical bar has a length $ L = 5  m $ and cross section area $ S = 10  m^2 $. The bar is made of a linear elastic material with a density $ \rho = 2700  \text{kg/m}^3 $ and Young’s modulus $ E = 70  \text{GPa} $. The bar is suspended as shown in the figure and is in a state of uniaxial tension due to its self-weight. 

The elastic strain energy stored in the bar equals ------------ J. (Rounded off to two decimal places) 

Take the acceleration due to gravity as $ g = 9.8  \text{m/s}^2 $.

\begin{figure}[H]
\centering
\includegraphics[width=0.3\textwidth]{Fig 52.png}
\caption{}
\label{fig:question59}
\end{figure}
\hfill (GATE ME 2023)

\item A cylindrical transmission shaft of length 1.5 m and diameter 100 mm is made of a linear elastic material with a shear modulus of 80 GPa. While operating at 500 rpm, the angle of twist across its length is found to be 0.5 degrees. 

The power transmitted by the shaft at this speed is ------------ kW. (Rounded off to two decimal places) 

Take $ \pi = 3.14 $.
\hfill (GATE ME 2023)

\item Consider a mixture of two ideal gases, $ X $ and $ Y $, with molar masses $ \Bar{M_X} = 10  \text{kg/kmol} $ and $\Bar{ M_Y} = 20  \text{kg/kmol} $, respectively, in a container. The total pressure in the container is 100 kPa, the total volume of the container is 10 m$^3$ and the temperature of the contents of the container is 300 K. If the mass of gas-$ X $ in the container is 2 kg, then the mass of gas-$ Y $ in the container is ------------ kg. (Rounded off to one decimal place)

Assume that the universal gas constant is 8314 J kmol$^{-1}$K$^{-1}$.
\hfill (GATE ME 2023)

\item The velocity field of a certain two-dimensional flow is given by

$ V(x,y) = k(x \hat{i} - y \hat{j}) $

where $ k = 2  \text{s}^{-1} $. The coordinates $ x $ and $ y $ are in meters. Assume gravitational effects to be negligible. 

If the density of the fluid is 1000 kg/m$^3$ and the pressure at the origin is 100 kPa, the pressure at the location (2 m, 2 m) is ------------ kPa. 

(Answer in integer)
\hfill (GATE ME 2023)

\item Consider a unidirectional fluid flow with the velocity field given by 
\begin{center}
$ \Vec{V}(x,y,z,t) = u(x,t)  \hat{i} $
\end{center}
where $ u(0,t) = 1 $. If the spatially homogeneous density field varies with time $ t $ as 
\begin{center}
$ \rho(t) = 1 + 0.2e^{-t} $
\end{center}
the value of $ u(2,1) $ is ------------. (Rounded off to two decimal places) 

Assume all quantities to be dimensionless.
\hfill (GATE ME 2023)

\item The figure shows two fluids held by a hinged gate. The atmospheric pressure is $ P_a = 100  \text{kPa} $. The moment per unit width about the base of the hinge is ------------ kNm/m. (Rounded off to one decimal place) 

Take the acceleration due to gravity to be $ g = 9.8  \text{m/s}^2 $.

\begin{figure}[H]
\centering
\includegraphics[width=0.5\textwidth]{Fig 53.png}
\caption{}
\label{fig:question64}
\end{figure}
\hfill (GATE ME 2023)

\item An explosion at time $ t = 0 $ releases energy $ E $ at the origin in a space filled with a gas of density $ \rho $. Subsequently, a hemispherical blast wave propagates radially outwards as shown in the figure. 

Let $ R $ denote the radius of the front of the hemispherical blast wave. The radius $ R $ follows the relationship $ R = k  t^a E^b \rho^c $, where $ k $ is a dimensionless constant. The value of exponent $ a $ is ------------. 

(Rounded off to one decimal place)

\begin{figure}[H]
\centering
\includegraphics[width=0.4\textwidth]{Fig 54.png}
\caption{}
\label{fig:question65}
\end{figure}
\hfill (GATE ME 2023)

\end{enumerate}
\end{document}