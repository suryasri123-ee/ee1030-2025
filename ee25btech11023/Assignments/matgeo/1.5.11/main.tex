\let\negmedspace\undefined
\let\negthickspace\undefined
\documentclass[journal]{IEEEtran}
\usepackage[a5paper, margin=10mm, onecolumn]{geometry}
%\usepackage{lmodern} % Ensure lmodern is loaded for pdflatex
\usepackage{tfrupee} % Include tfrupee package

\setlength{\headheight}{1cm} % Set the height of the header box
\setlength{\headsep}{0mm}     % Set the distance between the header box and the top of the text

\usepackage{gvv-book}
\usepackage{gvv}
\usepackage{cite}
\usepackage{amsmath,amssymb,amsfonts,amsthm}
\usepackage{algorithmic}
\usepackage{graphicx}
\usepackage{textcomp}
\usepackage{xcolor}
\usepackage{txfonts}
\usepackage{listings}
\usepackage{enumitem}
\usepackage{mathtools}
\usepackage{gensymb}
\usepackage{comment}
\usepackage[breaklinks=true]{hyperref}
\usepackage{tkz-euclide} 
\usepackage{listings}
% \usepackage{gvv}                                        
\def\inputGnumericTable{}                                 
\usepackage[latin1]{inputenc}                                
\usepackage{color}                                            
\usepackage{array}                                            
\usepackage{longtable}                                       
\usepackage{calc}                                             
\usepackage{multirow}                                         
\usepackage{hhline}                                           
\usepackage{ifthen}                                           
\usepackage{lscape}
\begin{document}

\bibliographystyle{IEEEtran}

\title{1.5.11}
\author{EE25BTECH11023 - Venkata Sai}
% \maketitle
% \newpage
% \bigskip
{\let\newpage\relax\maketitle}

\renewcommand{\thefigure}{\theenumi}
\renewcommand{\thetable}{\theenumi}
\setlength{\intextsep}{10pt} % Space between text and floats


\numberwithin{equation}{enumi}
\numberwithin{figure}{enumi}
\renewcommand{\thetable}{\theenumi}


\textbf{Question}:\newline
The point \textbf{R} divides the line segment AB, where \textbf{A}$\brak{-4,0}$ and \textbf{B}$\brak{0,6}$ such that
AR = $\frac{3}{4}$AB. Find the coordinates of \textbf{R}. 
\\
\textbf{Solution: }
\begin{table}[h!]    
  \centering
  \begin{center}
    \begin{tabular}{|c|c|} 
        \hline
            \textbf{Variable}  & \textbf{Formula} \\ 
        \hline
            $a$   & $a = \myvec{4 \\ -1 \\ 1}$ \\ 
        \hline
            $b$   &  $b = \myvec{2 \\ -2 \\ 1}$\\ 
        \hline
           \end{tabular}
\end{center}  

  \caption{Variables Used}
\end{table}
\begin{align}
\vec{AR}=\frac{3}{4}\vec{AB} \implies \frac{\vec{AR}}{\vec{RB}}=3\\
\vec{R}=\frac{k(\vec{B})+(\vec{A})}{k+1}=\myvec{x\\y}\\
\end{align}
Here according to problem value of k is 3\\
\begin{align}
\vec{R}=\frac{3\vec{B}+\vec{A}}{4}=\frac{3\myvec{0\\6}+\myvec{-4\\0}}{4}=\frac{\myvec{-4\\18}}{4}\\
\end{align}
\begin{align}
\vec{R}=\myvec{-1\\\frac{9}{2}}
\end{align}
Hence the coordinates of $\vec{R}$ are $\brak{-1,\frac{9}{2}}$
\begin{figure}[h!]
   \centering
   \includegraphics[width=0.7\columnwidth]{figs/Fig1.png}
   \caption{Stem Plot of y\brak{n}}
   \label{stemplot}
\end{figure}
\end{document}  
