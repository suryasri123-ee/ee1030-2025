\documentclass[journal,12pt,onecolumn]{IEEEtran}
\usepackage{cite}
\usepackage{graphicx}
\usepackage{amsmath,amssymb,amsfonts,amsthm}
\usepackage{algorithmic}
\usepackage{graphicx}
\usepackage{textcomp}
\usepackage{xcolor}
\usepackage{txfonts}
\usepackage{listings}
\usepackage{enumitem}
\usepackage{mathtools}
\usepackage{gensymb}
\usepackage{comment}
\usepackage[breaklinks=true]{hyperref}
\usepackage{tkz-euclide} 
\usepackage{listings}
\usepackage{gvv}                                        
%\def\inputGnumericTable{}                                 
\usepackage[latin1]{inputenc} 
\usetikzlibrary{arrows.meta, positioning}
\usepackage{xparse}
\usepackage{color}                                            
\usepackage{array}                                            
\usepackage{longtable}                                       
\usepackage{calc}                                             
\usepackage{multirow}
\usepackage{multicol}
\usepackage{hhline}                                           
\usepackage{ifthen}                                           
\usepackage{lscape}
\usepackage{tabularx}
\usepackage{array}
\usepackage{float}
\newtheorem{theorem}{Theorem}[section]
\newtheorem{problem}{Problem}
\newtheorem{proposition}{Proposition}[section]
\newtheorem{lemma}{Lemma}[section]
\newtheorem{corollary}[theorem]{Corollary}
\newtheorem{example}{Example}[section]
\newtheorem{definition}[problem]{Definition}
\newcommand{\BEQA}{\begin{eqnarray}}
\newcommand{\EEQA}{\end{eqnarray}}
\usepackage{float}
%\newcommand{\define}{\stackrel{\triangle}{=}}
\theoremstyle{remark}
\usepackage{circuitikz}
\usepackage{tikz}
\graphicspath{{figs/}}
\usepackage[top=1in,bottom=1in,left=1in,right=1in]{geometry}
\pagestyle{empty}
\setlength{\parindent}{0pt}

\title{PI: PRODUCTION AND INDUSTRIAL ENGINEERING}
\author{EE25BTECH11023-Venkata Sai}
\begin{document}

{\centering
  \maketitle
  \textit{\large Duration:} 3 Hours \hfill \textit{\large Maximum Marks:} 100
}

\textbf{Read the following instructions carefully.}
\begin{enumerate}[leftmargin=0.5cm,label=\arabic*.]
 \setlength\itemsep{0.6em}
\item This question paper contains 16 printed pages including pages for rough work. Please check all pages and report discrepancy, if any.
\item Write your registration number, your name and name of the examination centre at the specified locations on the right half of the Optical Response Sheet (ORS).
\item Using HB pencil, darken the appropriate bubble under each digit of your registration number and the letters corresponding to your paper code.
\item All questions in this paper are of objective type.
\item Questions must be answered on Optical Response Sheet (ORS) by darkening the appropriate bubble (marked A, B, C, D) using HB pencil against the question number on the left hand side of the ORS. Each question has only one correct answer. In case you wish to change an answer, erase the old answer completely. More than one answer bubbled against a question will be treated as an incorrect response.
\item There are a total of 60 questions carrying 100 marks. Questions 1 through 20 are 1-mark questions, questions 21 through 60 are 2-mark questions.
\item Questions 51 through 56 (3 pairs) are common data questions and question pairs (57, 58) and (59, 60) are linked answer questions. The answer to the second question of the above 2 pairs depends on the answer to the first question of the pair. If the first question in the linked pair is wrongly answered or is un-attempted, then the answer to the second question in the pair will not be evaluated.
\item Un-attempted questions will carry zero marks.
\item Wrong answers will carry NEGATIVE marks. For Q.1 to Q.20, ½ mark will be deducted for each wrong answer. For Q. 21 to Q. 56, ½ mark will be deducted for each wrong answer. The question pairs (Q.57, Q.58), and (Q.59, Q.60) are questions with linked answers. There will be negative marks only for wrong answer to the first question of the linked answer question pair i.e. for Q.57 and Q.59, ½ mark will be deducted for each wrong answer. There is no negative marking for Q.58 and Q.60.
\item Calculator (without data connectivity) is allowed in the examination hall.
\item Charts, graph sheets or tables are NOT allowed in the examination hall.
\item Rough work can be done on the question paper itself. Additionally, blank pages are given at the end of the question paper for rough work.
\end{enumerate}

\newpage
 
\textbf{Q. 1 - Q. 20 carry one mark each.}

\begin{enumerate}[label=Q.\arabic*, leftmargin=*]

\item The homogeneous part of the differential equation
$\frac{d^2 y}{dx^2} + p \frac{dy}{dx} + q y = r$
has real distinct roots if (p, q and r are constants)
\begin{multicols}{2}
\begin{tabular}[t]{p{0.8\linewidth} p{0.9\linewidth}}
(A) $p^2 - 4q > 0$ & (B) $p^2 - 4q < 0$ \\
(C) $p^2 - 4q = 0$ & (D) $p^2 - 4q = r$ \\
\end{tabular}
\end{multicols}
\hfill (GATE PI 2009) 
\item The total derivative of the function $xy$ is
\begin{multicols}{2}
\begin{tabular}[t]{p{0.8\linewidth} p{0.9\linewidth}}
(A) $x dy + ydx$ & (B) $xdx + ydy$ \\
(C) $dx + dy$ & (D) $dx dy$ \\
\end{tabular}
\end{multicols}
\hfill (GATE PI 2009) 
\item A helical compression spring has: \( d = \) wire diameter, \( D = \) mean coil diameter, \( E = \) Young's modulus, \( G = \) modulus of rigidity and \( N_a = \) number of active coils. The spring stiffness is
\begin{multicols}{2}
\begin{tabular}[t]{p{0.8\linewidth} p{0.9\linewidth}}
(A) $\frac{d E}{8 D^{3} N_a}$ & (B) $\frac{d G}{8 D^{3} N_a}$ \\
(C) $\frac{d^{3} E}{8 D N_a}$ & (D) $\frac{d^{3}}{8 D N_a}$ \\
\end{tabular}
\end{multicols}
\hfill (GATE PI 2009)
\item Which of the following processes is NOT executed by an ideal Rankine cycle with no superheat?
\begin{enumerate}[label=(\Alph*)]
\item Isentropic expansion
\item Isentropic compression
\item Constant temperature heat addition
\item Constant temperature heat rejection
\end{enumerate}
\hfill (GATE PI 2009)
\item During the numerical solution of a first order differential equation using the Euler (also known as Euler Cauchy) method with step size \(h\), the local truncation error is of the order of

(A) $h^2$ \hfill (B) $h^3$ \hfill(C) $h^4$ \hfill (D) $h^5$ \hfill
\hfill (GATE PI 2009)
\item For a granted patent to last for 20 years, the patent must be
\begin{multicols}{2}
\begin{tabular}[t]{p{0.8\linewidth} p{0.9\linewidth}}
(A) owned by the inventor & (B) renewed and maintained \\
(C) novel & (D) non-obvious \\
\end{tabular}
\end{multicols}
\hfill (GATE PI 2009)
\item As per Kendall's notation in M/G/c queuing system, the number of arrivals in a fixed time follows
\begin{multicols}{2}
\begin{tabular}[t]{p{1.4\linewidth} p{1\linewidth}}
(A) Beta distribution & (B) Normal distribution \\
(C) Poisson distribution & (D) Uniform distribution \\
\end{tabular}
\end{multicols}
\hfill (GATE PI 2009)
\item Which of the following forecasting models explicitly accounts for seasonality of demand?
\begin{multicols}{2}
\begin{tabular}[t]{p{0.9\linewidth} p{0.9\linewidth}}
(A) Simple moving average  model & (B) Simple exponential smoothing model \\
(C) Holt's model & (D) Winter's model \\
\end{tabular}
\end{multicols}
\hfill (GATE PI 2009)
\item A typical Fe-C alloy containing greater than 0.8\% C is known as
\begin{multicols}{2}
\begin{tabular}[t]{p{0.8\linewidth} p{0.9\linewidth}}
(A) Eutectoid steel & (B) Hypoeutectoid steel \\
(C) Mild steel & (D) Hypereutectoid steel \\
\end{tabular}
\end{multicols}
\hfill (GATE PI 2009)
\item The capacity of a material to absorb energy when deformed elastically, and to release it back when unloaded is termed as
\begin{multicols}{2}
\begin{tabular}[t]{p{0.8\linewidth} p{0.9\linewidth}}
(A) toughness & (B) resilience \\
(C) ductility & (D) malleability \\
\end{tabular}
\end{multicols}
\hfill (GATE PI 2009)
\item The product of the complex numbers \((3 - i2)\) and \((3 + i4)\) results in

(A) $(1 + i^6)$ \hfill (B) $(9-i^8)$ \hfill (C) $(9+i^8)$ \hfill (D) $(17 + i^6)$ \\\

\hfill (GATE PI 2009)
\item The value of the determinant
$
\begin{vmatrix}
4 & 1 & 1 \\
2 & 1 & 3 \\
1 & 3 & 2
\end{vmatrix}
$
is

(A) -28 \hfill (B) -24 \hfill(C) 32 \hfill (D) 36 \\

\hfill (GATE PI 2009)

\item If module and number of teeth of a spur gear with involute profile are 3 mm and 23 respectively, then the pitch diameter (in mm) of the spur gear is

(A) 7.67 \hfill (B) 15.34 \hfill (C) 34.50 \hfill (D) 69.00  \\

\hfill (GATE PI 2009)
\item Hot chamber die casting process is NOT suited for
\begin{multicols}{2}
\begin{tabular}[t]{p{0.8\linewidth} p{0.9\linewidth}}
(A) Lead and its alloys & (B) Zinc and its alloys \\
(C) Tin and its alloys & (D) Aluminum and its alloys \\
\end{tabular}
\end{multicols}
\hfill (GATE PI 2009)
\item The total angular movement (in degrees) of a lead-screw with a pitch of 5.0 mm to drive the work-table by a distance of 200 mm in a NC machine is

(A) 14400 \hfill(B) 28800\hfill  (C) 57600 \hfill (D) 72000 \\
 
\hfill (GATE PI 2009)
\item Anisotropy in rolled components is caused by
\begin{multicols}{2}
\begin{tabular}[t]{p{0.8\linewidth} p{0.9\linewidth}}
(A) change in dimensions & (B) scale formation \\
(C) closure of defects & (D) grain orientation \\
\end{tabular}
\end{multicols}
\hfill (GATE PI 2009)
\item Which of the following processes is used to manufacture products with controlled porosity?
\begin{multicols}{2}
\begin{tabular}[t]{p{0.8\linewidth} p{0.9\linewidth}}
(A) Casting & (B) Welding \\
(C) Forming & (D) Powder metallurgy \\
\end{tabular}
\end{multicols}
\hfill (GATE PI 2009)
\item Which of the following powders should be fed for effective oxy-fuel cutting of stainless steel?

(A) Steel \hfill (B) Aluminum\hfill (C) Copper \hfill (D) Ceramic 
\hfill (GATE PI 2009)
\item An autocollimator is used to

\begin{enumerate}[label=(\Alph*)]
\item measure small angular displacements on flat surfaces
\item compare known and unknown dimensions
\item measure the flatness error
\item measure roundness error between centers
\end{enumerate}
\hfill (GATE PI 2009)
\item Diamond cutting tools are not recommended for machining of ferrous metals due to

\begin{enumerate}[label=(\Alph*)]
\item high tool hardness
\item high thermal conductivity of work material
\item poor tool toughness
\item chemical affinity of tool material with iron
\end{enumerate}
\hfill (GATE PI 2009)
\item The value of $x_3$ obtained by solving the following system of linear equations is
$$x + 2x_2 - 2x_3 = 4$$ 
$$2x + x_2 + x_3 = -2 $$
$$-x + x_2 - x_3 = 2$$
\begin{multicols}{2}
\begin{tabular}[t]{p{0.8\linewidth} p{0.9\linewidth}}
(A) -12 & (B) -2 \\
(C) 0 & (D) 12 \\
\end{tabular}
\end{multicols} 
\hfill (GATE PI 2009)
\item The displacement and acceleration of a cam follower mechanism are plotted in the following figures:
\begin{figure}[h]
    \centering
    \includegraphics[height=11em,width=1\linewidth]{figs/1.png}
    \label{fig:placeholder}
\end{figure} 
The nature of the displacement curve is:
\begin{multicols}{2}
\begin{tabular}[t]{p{0.8\linewidth} p{0.9\linewidth}}
(A) Cubic & (B) Quadratic \\
(C) Simple harmonic & (D) Linear 
\end{tabular}
\end{multicols}
\hfill (GATE PI 2009)
\item The solution of the differential equation
$
\frac{d^2 r}{dx^2} = 0
$
with boundary conditions: (i) $\frac{dy}{dx} = 1$ at $x = 0$, (ii) $\frac{dy}{dx} = 1$ at x=1 is
\begin{enumerate}[label=(\Alph*)]
\item $y = 1$ 
\item $y = x$ 
\item $y = x + C$, where $C$ is an arbitrary constant  
\item $y = C_1 x + C_2$, where $C_1, C_2$ are arbitrary constants 

\end{enumerate}
\hfill (GATE PI 2009)
\item The line integral of the vector function $\mathbf{F} = 2x + x^2 \mathbf{\hat{j}}$ along the x-axis from $x=1$ to $x=2$ is

(A)0\hfill(B)2.33\hfill(C)3\hfill(D)5.33 \\

\hfill (GATE PI 2009)
\item Using direct extrusion process, a round billet of 100 mm length and 50 mm diameter is extruded. Considering an ideal deformation process (no friction and no redundant work), extrusion ratio 4, and average flow stress of material 300 MPa, the pressure (in MPa) on the ram will be

(A)416\hfill(B)624\hfill(C)700\hfill(D)832\\

\hfill (GATE PI 2009)
\item A friction clutch is designed to transmit 15 horsepower at 1500 rpm. The torque (in N·m) experienced by the clutch is
\begin{multicols}{2}
\begin{tabular}[t]{p{0.8\linewidth} p{0.9\linewidth}}
(A) 1.19 & (B) 7.46 \\
(C) 71.24 & (D) 447.61 \\
\end{tabular}
\end{multicols}

\hfill (GATE PI 2009)
\item A manufacturer has set up an assembly line where first, Task I is performed in Workstation 1 for 0.3 minutes; then Task II is performed in Workstation 2 for 0.4 minutes; and finally Task III is performed in Workstation 3 for 0.3 minutes. The efficiency (in \%) of this assembly line setup is
\begin{multicols}{2}
\begin{tabular}[t]{p{0.8\linewidth} p{0.9\linewidth}}
(A) 33.33 & (B) 64.33 \\
(C) 75.33 & (D) 83.33 \\
\end{tabular}
\end{multicols}
\hfill (GATE PI 2009)

\item A biaxial stress element is subjected to tensile and shear stresses as shown in the figure. If $\sigma_1 = 40$ MPa, $\sigma_y = 20$ MPa and \(T_{xy} = T_{yx} = 15\) MPa. The principal normal stresses (in MPa) are:
\newpage
\begin{figure}[h]
    \centering
    \includegraphics[width=0.3\linewidth]{figs/2.png}
    \label{fig:placeholder}
\end{figure} 

\begin{multicols}{2}
\begin{tabular}[t]{p{0.45\linewidth} p{0.45\linewidth}}
(A) 5 and 55 \\
(B) 10 and 30 \\
(C) 12 and 48 \\
(D) 20 and 40 \\
\end{tabular}
\end{multicols}
\hfill (GATE PI 2009)
\item The area under the curve shown, between \(x=1\) and \(x=3\), to be evaluated using the trapezoidal rule. The following points on the curve are given:
\begin{figure}[h]
    \centering
    \includegraphics[width=0.3\linewidth]{figs/3.png}
    \label{fig:placeholder}
\end{figure} 
\begin{center}
\begin{tabular}{ccc}
Point & \(X\) coordinate (m) & \(Y\) coordinate (m) \\
1 & 1 & 1 \\
2 & 2 & 4 \\
3 & 3 & 9
\end{tabular}
\end{center}
The evaluated area (in m\(^2\)) will be
\begin{multicols}{2}
\begin{tabular}[t]{p{0.8\linewidth} p{0.9\linewidth}}
(A) 7 & (B) 8.67 \\
(C) 9 & (D) 18 \\
\end{tabular}
\end{multicols}
\hfill (GATE PI 2009)
\item The pressure drop for laminar flow of a liquid in a smooth pipe at normal temperature and pressure is
\begin{multicols}{2}
\begin{tabular}[t]{p{0.9\linewidth} p{0.9\linewidth}}
(A) directly proportional to density & (B) inversely proportional to density \\
(C) independent of density & (D) proportional to density\(^{0.75}\) \\
\end{tabular}
\end{multicols}
\hfill (GATE PI 2009)
\item A titanium sheet of 5.0 mm thickness is cut by wire-cut EDM process using a wire of 1.0 mm diameter. A uniform spark gap of 0.5 mm on both sides of the wire is maintained during cutting operation. If the feed rate of the wire into the sheet is 20 mm/min, the material removal rate (in mm$^3$/min) will be
\begin{multicols}{2}
\begin{tabular}[t]{p{0.8\linewidth} p{0.9\linewidth}}
(A) 150 & (B) 200 \\
(C) 300 & (D) 400 \\
\end{tabular}
\end{multicols}
\hfill (GATE PI 2009)
\item Autogenous gas tungsten arc welding of a steel plate is carried out with welding current of 500 A, voltage of 20 V, and weld speed of 20 mm/min. Consider the heat transfer efficiency from the arc to the weld pool as 90\%. The heat input per unit length (in kJ/mm) is

(A) 0.25 \hfill (B) 0.35 \hfill(C) 0.45 \hfill (D) 0.55 \\

\hfill (GATE PI 2009)
\item Consider steady flow of water in a situation where two pipe lines (Pipe 1 and Pipe 2) combine into a single pipeline (Pipe 3) as shown in the figure. The cross-sectional areas of all three pipelines are constant. The following data is given : \\
\begin{tabular}{c c c}
Pipe number & Area(m$^2$) & Velocity(m/s) \\
   1  & 1 & 1 \\
   2  & 2 & 2\\
   3 & 2.5 & ? \\
   \end{tabular} \\
Assuming water properties and velocities to be uniform across the cross sections of the inlets and the outlet, the exit velocity (in m/s) in pipe 3 is
\begin{multicols}{2}
\begin{tabular}[t]{p{0.8\linewidth} p{0.9\linewidth}}
(A) 1 & (B) 1.5 \\
(C) 2 & (D) 2.5 \\
\end{tabular}
\end{multicols}
\hfill (GATE PI 2009)
\item Match the following: \\
\noindent
\begin{minipage}[t]{0.45\textwidth}
\textbf{Group I (Layout types)}\\[0.5em]
P. Process layout \\
Q. Product flow layout \\
R. Fixed position layout \\
S. Cellular layout
\end{minipage}
\hfill
\begin{minipage}[t]{0.9\textwidth}
\textbf{Group II (Layout characteristics)}\\[0.5em]
1. Inflexible to significant changes in product design \\
2. Distinct part families and expanded worker training \\
3. Low equipment utilization and high skill requirement \\
4. Large work-in-process and increased material handling
\end{minipage}


\begin{multicols}{2}
\begin{enumerate}[label=(\Alph*)]
    \item P-4, Q-1, R-3, S-2
    \item P-4, Q-3, R-2, S-1
    \item P-2, Q-1, R-4, S-3
    \item P-1, Q-4, R-3, S-2
\end{enumerate}
\end{multicols}
\hfill (GATE PI 2009)
\item Consider the joint probability mass function of random variables X and Y as shown in the table below: \\
For instance, $P\{X=1, Y=2\} = 0.3$
$$
\begin{array}{|c|c|c}
\hline
& X=1 & X=2 \\ \hline
Y=1 & 0.2 & 0.3 \\ \hline
Y=2 & 0.3 & 0.1 \\  \hline
Y=3 & 0.1 & \\
\end{array}
$$
The value of $P\{X=2|Y=2\}$ is

(A) 0.10 \hfill (B) 0.25 \hfill C) 0.40 \hfill (D) 0.75 \\

\hfill (GATE PI 2009)
\item A grocery store faces a demand of 50 units of soap per day. The store orders soap periodically. It costs Rs. 100 to initiate a purchase order. It costs Rs. 0.04 per soap per day to store the soap. The lead time between placing and receiving the order is 4 days. The optimal inventory policy for ordering soap is to
\begin{enumerate}[label=(\Alph*)]
\item order 500 units when inventory drops to 200 units 
\item order 500 units when inventory drops to 100 units
\item order 1000 units when inventory drops to 200 units
\item order 1000 units when inventory drops to 100 units
\end{enumerate}
\hfill (GATE PI 2009)
\item A disk of 200 mm diameter is blanked from a strip of an aluminum alloy of thickness 3.2 mm. The material shear strength to fracture is 150 MPa. The blanking force (in kN) is

(A) 291 \hfill (B) 301 \hfill(C) 311 \hfill (D) 321  \\

\hfill (GATE PI 2009)
\noindent 
\item Match the following:

\noindent
\begin{minipage}[t]{0.45\textwidth}
\textbf{Group I (Product)}\\[0.5em]
P. Refrigerator liners \\
Q. Composite pressure vessels \\
R. Hollow parts of thermoset plastics \\
S. Rubber sheets
\end{minipage}
\hfill
\begin{minipage}[t]{0.45\textwidth}
\textbf{Group II (Manufacturing process)}\\[0.5em]
1. Filament winding \\
2. Thermoforming \\
3. Calendering \\
4. Rotational moulding
\end{minipage}


\begin{multicols}{2}
\begin{enumerate}[label=(\Alph*)]
    \item P-2, Q-1, R-4, S-3
    \item P-1, Q-2, R-3, S-4
    \item P-1, Q-4, R-2, S-3
    \item P-2, Q-4, R-1, S-3
\end{enumerate}
\end{multicols}
\hfill (GATE PI 2009)
\noindent 
\item Match the following:

\noindent
\begin{minipage}[t]{0.45\textwidth}
\textbf{Group I (Device)}\\[0.5em]
P. Jig \\
Q. Fixture \\
R. Clamp \\
S. Locator
\end{minipage}
\hfill
\begin{minipage}[t]{0.45\textwidth}
\textbf{Group II (Function)}\\[0.5em]
1. helps to place the workpiece in the same position cycle after cycle \\
2. holds the workpiece only \\
3. holds and positions the workpiece \\
4. holds and positions the workpiece and guides the cutting tool during a machining operation
\end{minipage}


\begin{multicols}{2}
\begin{enumerate}[label=(\Alph*)]
    \item P-4, Q-3, R-1, S-2
    \item P-1, Q-2, R-3, S-4
    \item P-1, Q-4, R-3, S-2
    \item P-4, Q-3, R-2, S-1
\end{enumerate}
\end{multicols}
\hfill (GATE PI 2009)
\item A spur gear having a pressure angle of 20°, module of 4 mm and 40 teeth is to be inspected for its pitch circle diameter using two rollers (test plug method). If the centres of the rollers lie on the pitch circle, the suitable roller diameter (in mm) and the resulting distance (in mm) between the rollers placed in opposite spaces will respectively be
\begin{multicols}{2}
\begin{tabular}[t]{p{0.8\linewidth} p{0.9\linewidth}}
(A) 2.9 and 82.9 & (B) 2.9 and 165.9 \\
(C) 5.9 and 82.9 & (D) 5.9 and 165.9 \\
\end{tabular}
\end{multicols}
\hfill (GATE PI 2009)
\item A company makes a product using three independent components I, II and III, with reliabilities of 0.80, 0.85 and 0.90 respectively. If the company decides to add one redundant unit of component I to improve reliability, then the reliability of the product is

(A) 0.612 \hfill (B) 0.734 \hfill (C) 0.837 \hfill (D) 0.969 \\

\hfill (GATE PI 2009)
\item Given: \\
Assertion [a] : Managers spend time on job analysis and job rating.\\
Reason [r]: Scientific management of wage structures through job evaluation helps increase productivity.
\begin{enumerate}[label=(\Alph*)]
\item Both [a] and [r] are true and [r] is the correct reason for [a].
\item Both [a] and [r] are true, but [r] is not the correct reason for [a].
\item Both [a] and [r] are false. 
\item \text[a] is true but [r] is false.
\end{enumerate}
\hfill (GATE PI 2009)
\item A spare parts retail shop has sales of Rs. 4,00,000 and a profit of Rs. 50,000 for a product, in its first quarter. The profit volume (PV) ratio is 25\%. The margin of safety = profit / PV ratio. The break even point of sales (in Rs.) is
\begin{multicols}{2}
\begin{tabular}[t]{p{0.8\linewidth} p{0.9\linewidth}}
(A) 20,000 & (B) 40,000 \\
(C) 2,00,000 & (D) 4,00,000 \\
\end{tabular}
\end{multicols}
\hfill (GATE PI 2009)
\item The following information relates to worker's payment in a company: \\
\centerline{Standard production of a worker = 12 jobs per hour} \\
\centerline{Standard job rate = Rs. 3.00 per job}\\
\centerline{Pay for production less than standard = 85\% of standard job rate}\\
\centerline{Pay for production more than standard = 120\% of standard job rate} \\
Three workers produce at the rate of 11, 13 and 15 jobs per hour. The total pay for three workers per hour based on differential wage incentive scheme is
\begin{multicols}{2}
\begin{tabular}[t]{p{0.8\linewidth} p{0.9\linewidth}}
(A) Rs. 117.00 & (B) Rs. 128.85 \\
(C) Rs. 1404.00 & (D) Rs. 1546.20 \\
\end{tabular}
\end{multicols}
\hfill (GATE PI 2009)
\item{Match the following:}

\noindent
\begin{minipage}[t]{0.45\textwidth}
\textbf{Group I (Protection type)}\\[0.5em]
P. Patent \\
Q. Trademark \\
R. Copyright \\
S. Industrial design
\end{minipage}
\hfill
\begin{minipage}[t]{0.8\textwidth}
\textbf{Group II (Example in the Indian context)}\\[0.5em]
1. Manual of a product \\
2. Appearance of an MP3 player \\
3. Logo of a company \\
4. Microprocessor
\end{minipage}

\begin{multicols}{2}
\begin{enumerate}[label=(\Alph*)]
    \item P-2, Q-4, R-3, S-1
    \item P-4, Q-1, R-3, S-2
    \item P-2, Q-3, R-4, S-1
    \item P-4, Q-3, R-1, S-2
\end{enumerate}
\end{multicols}
\hfill (GATE PI 2009)
\item{Match the following:}

\noindent
\begin{minipage}[t]{0.5\textwidth}
\textbf{Group I (Design aspect)}\\[0.5em]
P. Form design \\
Q. Concurrent engineering \\
R. Value analysis \\
S. Product life cycle
\end{minipage}
\hfill
\begin{minipage}[t]{0.8\textwidth}
\textbf{Group II (Description)}\\[0.5em]
1. Introduction, growth, maturity and decline \\
2. Determines cost of each function of the design \\
3. Integration of product design and manufacturing \\
4. Appearance, shape, colour and size of product
\end{minipage}

\begin{multicols}{2}
\begin{enumerate}[label=(\Alph*)]
    \item P-4, Q-1, R-2, S-3
    \item P-3, Q-2, R-4, S-1
    \item P-4, Q-3, R-2, S-1
    \item P-4, Q-2, R-3, S-1
\end{enumerate}
\end{multicols}
\hfill (GATE PI 2009)
\item In an orthogonal machining operation, the tool life obtained is 10 min at a cutting speed of 100 m/min, while at 75 m/min cutting speed, the tool life is 30 min. The value of index $n$ in the Taylor's tool life equation is
\begin{multicols}{2}
\begin{tabular}[t]{p{0.8\linewidth} p{0.9\linewidth}}
(A) 0.262 & (B) 0.323 \\
(C) 0.423 & (D) 0.521 \\
\end{tabular}
\end{multicols}
\hfill (GATE PI 2009)
\item A solid cylinder of diameter D and height equal to D, and a solid cube of side L are being sand cast by using the same material. Assuming there is no superheat in both cases, the ratio of solidification time of the cylinder to that of the cube is
\begin{multicols}{2}
\begin{tabular}[t]{p{0.8\linewidth} p{0.9\linewidth}}
(A) $(L/D)^2$& (B) $(2L/D)^2$ \\
(C) $(2D/L)^2$& (D) $(D/L)^2$ \\
\end{tabular}
\end{multicols}
\hfill (GATE PI 2009)
\item Following are some possible characteristics of a pile of powder mixture: \\
P. Low inter-particle friction \\
Q. High inter-particle friction \\
R. Low porosity \\
S. High porosity \\
If the angle of repose for a pile of powder mixture is low, it will exhibit
\begin{multicols}{2}
\begin{tabular}[t]{p{0.8\linewidth} p{0.9\linewidth}}
(A) P and R & (B) P and S \\
(C) Q and S & (D) Q and R \\
\end{tabular}
\end{multicols}
\hfill (GATE PI 2009)
\item Match the following:

\noindent
\begin{minipage}[t]{0.45\textwidth}
\textbf{Group I}\\[0.5em]
P. Relational DBMS \\
Q. Primary key \\
R. Retrieving data \\
S. Boolean search
\end{minipage}
\hfill
\begin{minipage}[t]{0.45\textwidth}
\textbf{Group II}\\[0.5em]
1. SQL \\
2. AND, OR \\
3. Tables, columns and rows \\
4. Columns that uniquely identify a row
\end{minipage}

\begin{multicols}{2}
\begin{enumerate}[label=(\Alph*)]
    \item P-3, Q-4, R-2, S-1
    \item P-3, Q-1, R-4, S-2
    \item P-3, Q-4, R-1, S-2
    \item P-4, Q-1, R-2, S-3 
\end{enumerate}
\end{multicols}
\hfill (GATE PI 2009)

\textbf{\large{Common Data Questions}}

\textbf{Common Data for Questions 51 and 52:}

Consider the Linear Programming Problem (LPP)\\
Maximize  $ z = 4x_1 + 3x_2 + 2x_3 $ \\
Subject to:

\centerline{ $2x_1 + x_2 + 2x_3 \leq 50\  \text{(constraint 1)}\ $ } 
\centerline {$ x_1 + x_2 + x_3 \leq 30\  \text{(constraint 2)} $}
\centerline{$x_1, x_2, x_3 \geq 0 $}

The associated simplex tableau at optimality is shown below, where \(s_1\) and \(s_2\) represent the slacks for constraints 1 and 2 respectively.

$$\begin{array}{c|ccccc|c}
\text{} & x_1 & x_2 & x_3 & s_1 & s_2 & \text{RHS} \\ \hline
z\text{-row} & 0 & 0 & 2 & 1 & 2 & 110 \\ \hline
x_1 & 1 & 0 & 1 & 1 & -1 & 20 \\
x_2 & 0 & 1 & 0 & -1 & 2 & 10 \\
\end{array}
$$
\end{enumerate}
\begin{enumerate}[label=Q.\arabic*, leftmargin=*]
\setcounter{enumi}{50}

\item Basic variables in the optimal solution are
\begin{multicols}{2}
\begin{tabular}[t]{p{0.8\linewidth} p{0.9\linewidth}}
(A) $s_1\ \text{and}\ s_2$ & (B) $x_1\ \text{and}\ x_2$ \\
(C) $x_1, x_2\ \text{and}\ x_3$ & (D) $x_3, s_1\ \text{and}\ s_2$ \\
\end{tabular}
\end{multicols}
\hfill (GATE PI 2009)
\item Suppose that in the LPP given, the right hand side of constraint 1 changes from 50 to 40. The new objective value is
\begin{multicols}{2}
\begin{tabular}[t]{p{0.8\linewidth} p{0.9\linewidth}}
(A) 90 & (B) 100 \\
(C) 110 & (D) 120 \\
\end{tabular}
\end{multicols}
\end{enumerate}
\hfill (GATE PI 2009) \\
\textbf{Common Data for Questions 53 and 54:}


In acceptance sampling, the probability distribution of the number of defectives X in a sample can be approximated as a Poisson distribution,\\
$$\text{Prob}\{ X = k \} = \frac{(np)^k e^{-np}}{k!}\ k=0,1,2,...$$
where $n$ is the sample size and $p$ is the actual proportion or percent of defective items in a batch.\\

A company receives a shipment batch of $N = 2000$ items. The sampling plan followed by the company is to sample $n = 50$ items from the batch and accept the batch if the number of defective items is 2 or less. Let the Acceptable Quality Level (AQL) be 0.02 and the Lot Tolerance Percent Defective (LTPD) be 0.05.

\begin{enumerate}[label=Q.\arabic*), leftmargin=*]
\setcounter{enumi}{52}

\item[Q.53] The probability of incorrectly rejecting a good batch or the Producer's risk is
\begin{multicols}{2}
\begin{tabular}[t]{p{0.8\linewidth} p{0.9\linewidth}}
(A) 0.0805 & (B) 0.3678 \\
(C) 0.5437 & (D) 0.9195 \\
\end{tabular}
\end{multicols}
\hfill (GATE PI 2009)
\item[Q.54] The probability of incorrectly accepting a bad batch or the Consumer's risk is
\begin{multicols}{2}
\begin{tabular}[t]{p{0.8\linewidth} p{0.9\linewidth}}
(A) 0.0805 & (B) 0.3678 \\
(C) 0.5437 & (D) 0.9195 \\
\end{tabular}
\end{multicols}
\end{enumerate}
\hfill (GATE PI 2009) \\
\textbf{Common Data for Questions 55 and 56:}

An orthogonal turning operation is carried out at 20 m/min cutting speed, using a cutting tool of rake angle 15°. The chip thickness is 0.4 mm and the uncut chip thickness is 0.2 mm.

\begin{enumerate}[label=Q.\arabic*, leftmargin=*]
\setcounter{enumi}{54}

\item[Q.55] The shear plane angle (in degrees) is
\begin{multicols}{2}
\begin{tabular}[t]{p{0.8\linewidth} p{0.9\linewidth}}
(A) 26.8 & (B) 27.8 \\
(C) 28.8 & (D) 29.8 \\
\end{tabular}
\end{multicols}
\hfill (GATE PI 2009)
\item[Q.56] The chip velocity (in m/min) is
\begin{multicols}{2}
\begin{tabular}[t]{p{0.8\linewidth} p{0.9\linewidth}}
(A) 8 & (B) 10 \\
(C) 12 & (D) 14 \\
\end{tabular}
\end{multicols}
\end{enumerate}
\hfill (GATE PI 2009) \\
\textbf{\large{Linked Answer Equations}}

\textbf{Statement for linked Answer Questions 57 and 58}

 Four jobs need to be processed sequentially on two machines, first on Machine M and then on Machine N. Each machine can process only one job at a time. The processing times (in minutes) are given in the table below:

\begin{enumerate}[label=Q.\arabic*, leftmargin=*]
\setcounter{enumi}{56}
\item The optimal sequence of jobs that will minimize makespan (total time required to complete all jobs) is
\begin{multicols}{2}
\begin{tabular}[t]{p{0.8\linewidth} p{0.9\linewidth}}
(A) I - II - III - IV & (B) III - II - I - IV \\
(C) IV - III - I - II & (D) III - I - IV - II \\
\end{tabular}
\end{multicols}
\hfill (GATE PI 2009)
\item When the jobs are processed based on the optimal sequence that minimizes makespan, the total idle time (in minutes) on Machine N is
\begin{multicols}{2}
\begin{tabular}[t]{p{0.8\linewidth} p{0.9\linewidth}}
(A) 1 & (B) 3 \\
(C) 4 & (D) 6 \\
\end{tabular}
\end{multicols}
\end{enumerate}
\hfill (GATE PI 2009)

\textbf{Statement for Linked Answer Questions 59 and 60:} \\
Resistance spot welding of two steel sheets is carried out in lap joint configuration by using a welding current of 3 kA and a weld time of 0.2 s. A molten weld nugget of volume 20 mm$^3$ is obtained. The effective contact resistance is 200 $\mu\Omega$ (micro-ohms). The material properties of steel are given as: (i)latent heat of melting: 1400 kJ/kg,(ii) density: 8000 kg/m$^3$),(iii) melting temperature: 1520°C,(iv) specific heat: 0.5 kJ/kg°C.

The ambient temperature is 20$\degree$C.

\begin{enumerate}[label=Q.\arabic*), leftmargin=*, resume]

\item Heat (in Joules) used for producing weld nugget will be (assuming 100\%heat transfer efficiency)
\begin{multicols}{2}
\begin{tabular}[t]{p{0.8\linewidth} p{0.9\linewidth}}
(A) 324 & (B) 334 \\
(C) 344 & (D) 354 \\
\end{tabular}
\end{multicols}
\hfill (GATE PI 2009)
\item Heat (in Joules) dissipated to the base metal will be (neglecting all other heat losses)
\begin{multicols}{2}
\begin{tabular}[t]{p{0.8\linewidth} p{0.9\linewidth}}
(A) 10 & (B) 16 \\
(C) 22 & (D) 32 \\
\end{tabular}
\end{multicols}
\hfill (GATE PI 2009)

\end{enumerate}
\end{document}
