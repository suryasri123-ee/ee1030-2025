\documentclass[journal]{IEEEtran}
\usepackage[a5paper, margin=10mm]{geometry}
%\usepackage{lmodern} % Ensure lmodern is loaded for pdflatex
\usepackage{tfrupee} % Include tfrupee package


\setlength{\headheight}{1cm} % Set the height of the header box
\setlength{\headsep}{0mm}     % Set the distance between the header box and the top of the text


%\usepackage[a5paper, top=10mm, bottom=10mm, left=10mm, right=10mm]{geometry}

%
\setlength{\intextsep}{10pt} % Space between text and floats

\makeindex


\usepackage{cite}
\usepackage{amsmath,amssymb,amsfonts,amsthm}
\usepackage{algorithmic}
\usepackage{graphicx}
\usepackage{textcomp}
\usepackage{xcolor}
\usepackage{txfonts}
\usepackage{listings}
\usepackage{enumitem}
\usepackage{mathtools}
\usepackage{gensymb}
\usepackage{comment}
\usepackage[breaklinks=true]{hyperref}
\usepackage{tkz-euclide} 
\usepackage{listings}
\usepackage{multicol}
\usepackage{xparse}
\usepackage{gvv}
%\def\inputGnumericTable{}                                 
\usepackage[latin1]{inputenc}                                
\usepackage{color}                                            
\usepackage{array}                                            
\usepackage{longtable}                                       
\usepackage{calc}                                             
\usepackage{multirow}                                         
\usepackage{hhline}                                           
\usepackage{ifthen}                                               
\usepackage{lscape}
\usepackage{tabularx}
\usepackage{array}
\usepackage{float}
\usepackage{ar}
\usepackage[version=4]{mhchem}


\newtheorem{theorem}{Theorem}[section]
\newtheorem{problem}{Problem}
\newtheorem{proposition}{Proposition}[section]
\newtheorem{lemma}{Lemma}[section]
\newtheorem{corollary}[theorem]{Corollary}
\newtheorem{example}{Example}[section]
\newtheorem{definition}[problem]{Definition}
\newcommand{\BEQA}{\begin{eqnarray}}
\newcommand{\EEQA}{\end{eqnarray}}

\theoremstyle{remark}


\begin{document}
\bibliographystyle{IEEEtran}
\onecolumn

\title{2.2.24}
\author{Jnanesh Sathisha Karmar- EE25BTECH11029}
\maketitle


\renewcommand{\thefigure}{\theenumi}
\renewcommand{\thetable}{\theenumi}
\textbf{Question}Show that the points $\brak{1, 7}$, $\brak{4, 2}$, $\brak{-1, -1}$ and $\brak{-4, 4}$ are the vertices of a square.

\textbf{Solution}
Given details:
\begin{align}
    \vec{A}=\myvec{1\\7}  \vec{B}=\myvec{4\\2} \vec{C}=\myvec{-1\\-1} \vec{D}=\myvec{-4\\4}
\end{align}
For the points $\vec{ABCD}$ to represent a square:
\begin{align}
    \norm{AB}=\norm{BC}=\norm{CD}=\norm{DA}\\
    \angle{BAD}=\angle{ABC}=\angle{DCA}=\angle{ADC}=90\degree{}
\end{align}
Find the sides
\begin{align}
\vec{AB}=\vec{B}-\vec{A}=\myvec{3\\-5} \  \vec{BC}=\vec{C}-\vec{B}=\myvec{-5\\-3} \\
\vec{CD}=\vec{D}-\vec{C}=\myvec{-3\\5} \ 
\vec{DA}=\vec{A}-\vec{D}=\myvec{5\\3}
\end{align}
Check side lengths
\begin{align}
    \norm{AB}=\sqrt{\vec{AB}^T\vec{AB}}=\sqrt{3^2+\brak{-5}^2}=\sqrt{34}\\
    \norm{BC}=\sqrt{\vec{BC}^T\vec{BC}}=\sqrt{\brak{-5}^2+\brak{-3}^2}=\sqrt{34}\\
    \norm{CD}=\sqrt{\vec{CD}^T\vec{CD}}=\sqrt{\brak{-3}^2+5^2}=\sqrt{34}\\
    \norm{DA}=\sqrt{\vec{DA}^T\vec{DA}}=\sqrt{5^2+3^2}=\sqrt{34}
\end{align}
Therefore all the sides are of equal length
\begin{align}
    \norm{AB}=\norm{BC}=\norm{CD}=\norm{DA}
\end{align}
Condition for right angle:
For two sides to be angled at $90\degree{}$ the Dot product between the 2 side vectors should be 0
\begin{align}
    \vec{AB^TBC}=\brak{3}\brak{5}+\brak{-5}\brak{-3}=-15+15=0
\end{align}
Therefore the sides are perpendicular to each other.

Since all the sides are equal and one the angles is 90\degree{} , all the points represent a square.
\begin{figure}[H]
    \centering
    \includegraphics[width=1\columnwidth]{figs/sqaure.png}
    \caption{Sqaure}
    \label{fig:placeholder_1}
\end{figure}
\end{document}