% Preamble templated from Dhawal24112006/EE1030
\documentclass{beamer}
\mode<presentation>
\usepackage{amsmath}
\usepackage{amssymb}
%\usepackage{advdate}
\usepackage{adjustbox}
\usepackage{subcaption}
\usepackage{enumitem}
\usepackage{multicol}
\usepackage{mathtools}
\usepackage{listings}
\usepackage{url}
\def\UrlBreaks{\do\/\do-}
\usetheme{Boadilla}
\usecolortheme{lily}
\setbeamertemplate{footline}
{
  \leavevmode%
  \hbox{%
  \begin{beamercolorbox}[wd=\paperwidth,ht=2.25ex,dp=1ex,right]{author in head/foot}%
    \insertframenumber{} / \inserttotalframenumber\hspace*{2ex}
  \end{beamercolorbox}}%
  \vskip0pt%
}
\setbeamertemplate{navigation symbols}{}

\providecommand{\nCr}[2]{\,^{#1}C_{#2}} % nCr
\providecommand{\nPr}[2]{\,^{#1}P_{#2}} % nPr
\providecommand{\mbf}{\mathbf}
\providecommand{\pr}[1]{\ensuremath{\Pr\left(#1\right)}}
\providecommand{\qfunc}[1]{\ensuremath{Q\left(#1\right)}}
\providecommand{\sbrak}[1]{\ensuremath{{}\left[#1\right]}}
\providecommand{\lsbrak}[1]{\ensuremath{{}\left[#1\right.}}
\providecommand{\rsbrak}[1]{\ensuremath{{}\left.#1\right]}}
\providecommand{\brak}[1]{\ensuremath{\left(#1\right)}}
\providecommand{\lbrak}[1]{\ensuremath{\left(#1\right.}}
\providecommand{\rbrak}[1]{\ensuremath{\left.#1\right)}}
\providecommand{\cbrak}[1]{\ensuremath{\left\{#1\right\}}}
\providecommand{\lcbrak}[1]{\ensuremath{\left\{#1\right.}}
\providecommand{\rcbrak}[1]{\ensuremath{\left.#1\right\}}}
\theoremstyle{remark}
\newtheorem{rem}{Remark}
\newcommand{\sgn}{\mathop{\mathrm{sgn}}}
\providecommand{\abs}[1]{\left\vert#1\right\vert}
\providecommand{\res}[1]{\Res\displaylimits_{#1}}
\providecommand{\norm}[1]{\lVert#1\rVert}
\providecommand{\mtx}[1]{\mathbf{#1}}
\providecommand{\mean}[1]{E\left[ #1 \right]}
\providecommand{\fourier}{\overset{\mathcal{F}}{ \rightleftharpoons}}
%\providecommand{\hilbert}{\overset{\mathcal{H}}{ \rightleftharpoons}}
\providecommand{\system}{\overset{\mathcal{H}}{ \longleftrightarrow}}
	%\newcommand{\solution}[2]{\textbf{Solution:}{#1}}
%\newcommand{\solution}{\noindent \textbf{Solution: }}
\providecommand{\dec}[2]{\ensuremath{\overset{#1}{\underset{#2}{\gtrless}}}}
\newcommand{\myvec}[1]{\ensuremath{\begin{pmatrix}#1\end{pmatrix}}}
\let\vec\mathbf

\lstset{
%language=C,
frame=single,
breaklines=true,
columns=fullflexible
}

\numberwithin{equation}{section}

\title{MATGEO Presentation: 1.6.5}
\author{Subhodeep Chakraborty \\ ee25btech11055,\\IIT Hyderabad.}

\date{\today}
\begin{document}

\begin{frame}
\titlepage
\end{frame}

\section*{Outline}
\begin{frame}
\tableofcontents
\end{frame}

\section{Problem}
\begin{frame}
\frametitle{Problem Statement}

 Show that the points \brak{2, 3, 4}, \brak{-1, -2, 1}, \brak{5, 8, 7} are collinear.

\end{frame}

\section{Solution}
\subsection{Theory}
\begin{frame}
 \frametitle{Theory}
 For three points $\vec{A}$, $\vec{B}$ and $\vec{C}$ to be collinear:
\begin{align}
    \label{eq:theory}
    \text{rank}\myvec{\vec{B-A} & \vec{C-A}} = \text{rank}\myvec{\vec{B-A} & \vec{C-A}}^\top = 1
\end{align}
Given:
\begin{align}
 \vec{A} &\equiv \brak{2, 3, 4} \\
 \vec{B} &\equiv \brak{-1, -2, 1} \\
 \vec{C} &\equiv \brak{5, 8, 7}
\end{align}
\end{frame}

\subsection{Calculations}
\begin{frame}{Calculations}
The transpose of the collinearity matrix can be expressed as:
\begin{align*}
 \myvec{\vec{B-A} & \vec{C-A}}^\top = \myvec{-3 & -5 & -3 \\ 3 & 5 & 3} \\
 \xleftrightarrow[]{R_2 = R_2 + R_1}
 \myvec{-3 & -5 & -3 \\ 0 & 0 & 0}
\end{align*}
which has rank 1. Using \ref{eq:theory}, we conclude that the given points are collinear.
\end{frame}

\subsection{Plot}
\begin{frame}{Plot}
 \begin{figure}[H]
    \centering
    \includegraphics[width=\columnwidth]{../figs/plot.png}
    \caption*{}
    \label{fig:plot}
\end{figure}
\end{frame}

\section{C Code}
\begin{frame}[fragile]{C code for generating points on line}
\begin{lstlisting}[language=C]
void point_gen(const double* P1, const double* P2, double t, double* result_point) {
    result_point[0] = P1[0] + t * (P2[0] - P1[0]);
    result_point[1] = P1[1] + t * (P2[1] - P1[1]);
    result_point[2] = P1[2] + t * (P2[2] - P1[2]);
}
\end{lstlisting}
\end{frame}
\section{Python Code}
\subsection{Using C shared objects}
\begin{frame}[fragile]{Python code for plotting using C}
\begin{lstlisting}[language=Python]
import ctypes
import numpy as np
import matplotlib.pyplot as plt
from mpl_toolkits.mplot3d import Axes3D

lib = ctypes.CDLL("./line.so")

get_point = lib.point_gen
get_point.argtypes = [
    ctypes.POINTER(ctypes.c_double),  # P1
    ctypes.POINTER(ctypes.c_double),  # P2
    ctypes.c_double,  # t
    ctypes.POINTER(ctypes.c_double),  # result_point
]
get_point.restype = None
\end{lstlisting}
\end{frame}

\begin{frame}[fragile]{Python code for plotting using C}
\begin{lstlisting}[language=Python]
DoubleArray3 = ctypes.c_double * 3
P1_arr = DoubleArray3(-1, -2, 1)
P2_arr = DoubleArray3(5, 8, 7)

t_values = np.linspace(0, 1, 100)
line_points_x, line_points_y, line_points_z = [], [], []

for t in t_values:
    result_arr = DoubleArray3()

    get_point(P1_arr, P2_arr, t, result_arr)

    line_points_x.append(result_arr[0])
    line_points_y.append(result_arr[1])
    line_points_z.append(result_arr[2])

original_points = np.array([[-1, -2, 1], [2, 3, 4], [5, 8, 7]])

\end{lstlisting}
\end{frame}

\begin{frame}[fragile]{Python code for plotting using C}
\begin{lstlisting}[language=Python]
fig = plt.figure(figsize=(8, 6))
ax = fig.add_subplot(111, projection="3d")

ax.scatter(
    original_points[:, 0],
    original_points[:, 1],
    original_points[:, 2],
    color="red",
    s=50,
    label="Given Points",
)

\end{lstlisting}
\end{frame}

\begin{frame}[fragile]{Python code for plotting using C}
\begin{lstlisting}[language=Python]
ax.plot(
    line_points_x,
    line_points_y,
    line_points_z,
    color="blue",
    label="Line Segment",
)

ax.set_xlabel("X Axis")
ax.set_ylabel("Y Axis")
ax.set_zlabel("Z Axis")
ax.set_title("1.6.5")
ax.legend()
ax.grid(True)

plt.savefig("../figs/plot.png")
plt.show()
\end{lstlisting}
\end{frame}

\subsection{Plot}
\begin{frame}{Plot}
 \begin{figure}[H]
    \centering
    \includegraphics[width=\columnwidth]{../figs/plot.png}
    \caption*{}
    \label{fig:plot_c}
\end{figure}
\end{frame}
\subsection{In Pure Python}
\begin{frame}[fragile]{Pure Python Code for Plotting}
\begin{lstlisting}[language=Python]
import numpy as np
import matplotlib.pyplot as plt
from mpl_toolkits.mplot3d import Axes3D

points = np.array([[-1, -2, 1], [2, 3, 4], [5, 8, 7]])

x_coords = points[:, 0]
y_coords = points[:, 1]
z_coords = points[:, 2]

fig = plt.figure(figsize=(8, 6))
ax = fig.add_subplot(111, projection="3d")

ax.scatter(x_coords, y_coords, z_coords, color="red", s=50, label="Given Points")
ax.plot(x_coords, y_coords, z_coords, color="blue", label="Line Segment")

\end{lstlisting}
\end{frame}

\begin{frame}[fragile]{Pure Python Code for Plotting}
\begin{lstlisting}[language=Python]

ax.set_xlabel("X Axis")
ax.set_ylabel("Y Axis")
ax.set_zlabel("Z Axis")
ax.set_title("1.6.5")
ax.legend()
ax.grid(True)

plt.savefig("../figs/python.png")
plt.show()
\end{lstlisting}
\end{frame}

\subsection{Plot}
\begin{frame}{Pure Python Plot}
 \begin{figure}[H]
    \centering
    \includegraphics[width=\columnwidth]{../figs/python.png}
    \caption*{}
    \label{fig:plot_p}
\end{figure}
\end{frame}
\end{document}
