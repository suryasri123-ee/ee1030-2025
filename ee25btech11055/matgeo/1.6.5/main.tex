% Preamble templated from Mihir-Divyansh/Course-Setup
%iffalse
\let\negmedspace\undefined
\let\negthickspace\undefined
\documentclass[journal,12pt,onecolumn]{IEEEtran}
\usepackage{cite}
\usepackage{amsmath,amssymb,amsfonts,amsthm}
\usepackage{algorithmic}
\usepackage{graphicx}
\usepackage{textcomp}
\usepackage{xcolor}
\usepackage{txfonts}
\usepackage{listings}
\usepackage{enumitem}
\usepackage{mathtools}
\usepackage{gensymb}
\usepackage{comment}
\usepackage[breaklinks=true]{hyperref}
\usepackage{tkz-euclide}
\usepackage{listings}
\usepackage{gvv}
%\def\inputGnumericTable{}
\usepackage[latin1]{inputenc}
\usepackage{color}
\usepackage{array}
\usepackage{longtable}
\usepackage{calc}
\usepackage{multirow}
\usepackage{hhline}
\usepackage{ifthen}
\usepackage{lscape}
\usepackage{tabularx}
\usepackage{array}
\usepackage{float}
\usepackage{caption}
\usepackage{multicol}

\newtheorem{theorem}{Theorem}[section]
\newtheorem{problem}{Problem}
\newtheorem{proposition}{Proposition}[section]
\newtheorem{lemma}{Lemma}[section]
\newtheorem{corollary}[theorem]{Corollary}
\newtheorem{example}{Example}[section]
\newtheorem{definition}[problem]{Definition}
\newcommand{\BEQA}{\begin{eqnarray}}
\newcommand{\EEQA}{\end{eqnarray}}
%\newcommand{\define}{\stackrel{\triangle}{=}}
\theoremstyle{remark}
%\newtheorem{rem}{Remark}

% Marks the beginning of the document
\begin{document}
\bibliographystyle{IEEEtran}
\vspace{3cm}

\title{Assignment 1: 1.6.5}
\author{EE25BTECH11055 - Subhodeep Chakraborty}
\maketitle
\hrulefill
\bigskip

\renewcommand{\thefigure}{\theenumi}
\renewcommand{\thetable}{\theenumi}

\textbf{Question:}\par
Show that the points \brak{2, 3, 4}, \brak{-1, -2, 1}, \brak{5, 8, 7} are collinear.\par
\textbf{Solution:}\par
For three points $\vec{A}$, $\vec{B}$ and $\vec{C}$ to be collinear:
\begin{align}
    \label{eq:theory}
    \text{rank}\myvec{\vec{B-A} & \vec{C-A}} = \text{rank}\myvec{\vec{B-A} & \vec{C-A}}^\top = 1
\end{align}
Given:
\begin{align}
 \vec{A} &\equiv \brak{2, 3, 4} \\
 \vec{B} &\equiv \brak{-1, -2, 1} \\
 \vec{C} &\equiv \brak{5, 8, 7}
\end{align}
The transpose of the collinearity matrix can be expressed as:
\begin{align}
 \myvec{\vec{B-A} & \vec{C-A}}^\top = \myvec{-3 & -5 & -3 \\ 3 & 5 & 3} \\
 \xleftrightarrow[]{R_2 = R_2 + R_1}
 \myvec{-3 & -5 & -3 \\ 0 & 0 & 0}
\end{align}
which has rank 1. Using \ref{eq:theory}, we conclude that the given points are collinear.
\begin{figure}[H]
    \centering
    \includegraphics{figs/plot.png}
    \caption*{}
    \label{fig:plot}
\end{figure}
\end{document}
