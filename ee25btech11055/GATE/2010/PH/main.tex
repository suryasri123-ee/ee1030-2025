\documentclass{exam}
\usepackage{amsmath}
\usepackage{amssymb}
\usepackage{graphicx}
\usepackage{enumitem}
\usepackage{float}
\usepackage{multicol}
\usepackage{gvv}

\begin{document}

\begin{center}
	\textbf{GATE 2010 PH: PHYSICS} \\
	\textbf{EE25BTECH11055: Subhodeep Chakraborty} \\
	\textbf{Assignment 1}
\end{center}

\noindent\textbf{Duration:} Three Hours \hfill \textbf{Maximum Marks:} 100

\hrulefill
% \setlist[enumerate]{label=(\Aplh*)}

\subsection*{Q.1 - Q.25 carry one mark each.}

\begin{questions}
\question Consider an anti-symmetric tensor $P_{U}$ with the indices i and j running from 1 to 5. The number of independent components of the tensor is \hfill\brak{\text{GATE PH 2010}}


\begin{oneparchoices}
	\choice 3 \choice 10 \choice 9 \choice 6
\end{oneparchoices}

\question The value of the integral $\oint\limits_{C}\frac{e^{z}\sin(z)}{z^{2}}dz$ where the contour C is the unit circle: $|z-2|=1$, is\hfill\brak{\text{GATE PH 2010}}


\begin{oneparchoices}
	\choice 2$\pi i$ \choice 4$\pi i$ \choice $\pi i$ \choice 0
\end{oneparchoices}

\question The eigenvalues of the matrix $\begin{pmatrix} 2 & 3 & 0 \\ 3 & 2 & 0 \\ 0 & 0 & 1 \end{pmatrix}$ are\hfill\brak{\text{GATE PH 2010}}


\begin{oneparchoices}
	\choice 5, 2, -2 \choice -5, -1, -1 \choice 5, 1, -1 \choice -5, 1, 1
\end{oneparchoices}

\question If $f(x)=\begin{cases}0& \text{for } x<3, \\ x-3& \text{for } x\ge3, \end{cases}$ then the Laplace transform of $f(x)$ is\hfill\brak{\text{GATE PH 2010}}

\begin{oneparchoices}
	\choice $s^{-2}e^{3s}$ \choice $s^{2}e^{-3s}$ \choice $s^{-2}$ \choice $s^{-2}e^{-3s}$
\end{oneparchoices}

\question The valence electrons do not directly determine the following property of a metal.\hfill\brak{\text{GATE PH 2010}}

\begin{choices}
\begin{multicols}{2}
\choice Electrical conductivity
\choice Thermal conductivity
\choice Shear modulus
\choice Metallic lustre
\end{multicols}
\end{choices}

\question Consider X-ray diffraction from a crystal with a face-centered-cubic (fcc) lattice. The lattice plane for which there is NO diffraction peak is\hfill\brak{\text{GATE PH 2010}}

\begin{oneparchoices}
	\choice (2, 1, 2) \choice (1, 1, 1) \choice (2, 0, 0) \choice (3, 1, 1)
\end{oneparchoices}

\question The Hall coefficient, $R_H$, of sodium depends on\hfill\brak{\text{GATE PH 2010}}

\begin{choices}
\choice The effective charge carrier mass and carrier density
\choice The charge carrier density and relaxation time
\choice The charge carrier density only
\choice The effective charge carrier mass
\end{choices}

\question The Bloch theorem states that within a crystal, the wavefunction, $\psi(\vec{r})$, of an electron has the form\hfill\brak{\text{GATE PH 2010}}

\begin{choices}
	\choice $\psi(\vec{r})=u(\vec{r})e^{i\vec{k}\cdot\vec{r}}$ where $u(\vec{r})$ is an arbitrary function and $\vec{k}$ is an arbitrary vector
	\choice $\psi(\vec{r})=u(\vec{r})e^{i\vec{G}\cdot\vec{r}}$ where $u(\vec{r})$ is an arbitrary function and $\vec{G}$ is a reciprocal lattice vector
	\choice $\psi(\vec{r})=u(\vec{r})e^{i\vec{G}\cdot\vec{r}}$ where $u(\vec{r})=u(\vec{r}+\vec{\Lambda})$, $\vec{\Lambda}$ is a lattice vector and $\vec{G}$ is a reciprocal lattice vector
	\choice $\psi(\vec{r})=u(\vec{r})e^{i\vec{k}\cdot\vec{r}}$ where $u(\vec{r})=u(\vec{r}+\vec{\Lambda})$, $\vec{\Lambda}$ is a lattice vector and $\vec{k}$ is an arbitrary vector
\end{choices}

\question In an experiment involving a ferromagnetic medium, the following observations were made. Which one of the plots does NOT correctly represent the property of the medium? ($T_C$ is the Curie temperature)\hfill\brak{\text{GATE PH 2010}}
\begin{figure}[H]
	\centering
	\includegraphics[width=0.45\textwidth]{pics/9a.png}
	\includegraphics[width=0.45\textwidth]{pics/9b.png}
	\includegraphics[width=0.45\textwidth]{pics/9c.png}
	\includegraphics[width=0.45\textwidth]{pics/9d.png}
\end{figure}

\question The thermal conductivity of a given material reduces when it undergoes a transition from its normal state to the superconducting state. The reason is:\hfill\brak{\text{GATE PH 2010}}

\begin{choices}
	\choice The Cooper pairs cannot transfer energy to the lattice
	\choice Upon the formation of Cooper pairs, the lattice becomes less efficient in heat transfer
	\choice The electrons in the normal state lose their ability to transfer heat because of their coupling to the Cooper pairs
	\choice The heat capacity increases on transition to the superconducting state leading to a reduction in thermal conductivity
\end{choices}

\question The basic process underlying the neutron $\beta$-decay is\hfill\brak{\text{GATE PH 2010}}

\begin{oneparchoices}
	\choice $d \rightarrow u + e^{-} + \overline{v}_{e}$
	\choice $d \rightarrow u + e^{-}$
	\choice $s \rightarrow u + e^{-} + \overline{v}_{e}$
	\choice $u \rightarrow d + e^{+} + \overline{v}_{e}$
\end{oneparchoices}

\question In the nuclear shell model the spin parity of ${}^{15}N$ is given by\hfill\brak{\text{GATE PH 2010}}

\begin{oneparchoices}
	\choice $\frac{1}{2}^{-}$ \choice $\frac{1}{2}^{+}$ \choice $\frac{3}{2}^{-}$ \choice $\frac{3}{2}^{+}$
\end{oneparchoices}

\question Match the reactions on the left with the associated interactions on the right.\hfill\brak{\text{GATE PH 2010}}

\begin{multicols}{2}
\begin{enumerate}[label=(\arabic*)]
		\item $\pi^{+} \rightarrow \mu^{+} + \nu_{\mu}$
		\item $\pi^{0} \rightarrow \gamma + \gamma$
		\item $\pi^{0} + n \rightarrow \pi^{-} + p$
	\end{enumerate}
\columnbreak
\begin{enumerate}[label=(\roman*)]
		\item Strong
		\item Electromagnetic
		\item Weak
	\end{enumerate}
\end{multicols}

\begin{oneparchoices}
	\choice (1, iii), (2, ii), (3, i) \choice (1, i), (2, ii), (3, iii)
	\choice (1, ii), (2, i), (3, iii) \choice (1, iii), (2, i), (3, ii)
\end{oneparchoices}

\question To detect trace amounts of a gaseous species in a mixture of gases, the preferred probing tool is\hfill\brak{\text{GATE PH 2010}}

\begin{choices}
	\begin{multicols}{2}
\choice Ionization spectroscopy with X-rays \choice NMR spectroscopy
	\choice ESR spectroscopy \choice Laser spectroscopy
	\end{multicols}
\end{choices}

\question A collection of N atoms is exposed to a strong resonant electromagnetic radiation with $N_g$ atoms in the ground state and $N_e$ atoms in the excited state, such that $N_g+N_e=N$. This collection of two-level atoms will have the following population distribution:\hfill\brak{\text{GATE PH 2010}}

\begin{oneparchoices}
	\choice $N_g \ll N_e$ \choice $N_g \gg N_e$ \choice $N_g \approx N_e = N/2$ \choice $N_g - N_e \approx N/2$
\end{oneparchoices}

\question Two states of an atom have definite parities. An electric dipole transition between these states is\hfill\brak{\text{GATE PH 2010}}

\begin{choices}
	\choice Allowed if both the states have even parity
	\choice Allowed if both the states have odd parity
	\choice Allowed if the two states have opposite parities
	\choice Not allowed unless a static electric field is applied
\end{choices}

\question The spectrum of radiation emitted by a black body at a temperature 1000 K peaks in the\hfill\brak{\text{GATE PH 2010}}

\begin{choices}
	\begin{multicols}{2}
	 \choice Visible range of frequencies \choice Infrared range of frequencies
	\choice Ultraviolet range of frequencies \choice Microwave range of frequencies
	\end{multicols}
\end{choices}

\question An insulating sphere of radius a carries a charge density $\rho(\vec{r})=\rho_{0}(a^{2}-r^{2})\cos\theta; r<a$. The leading order term for the electric field at a distance d, far away from the charge distribution, is proportional to\hfill\brak{\text{GATE PH 2010}}

\begin{oneparchoices}
	\choice $d^{-1}$ \choice $d^{-2}$ \choice $d^{-3}$ \choice $d^{-4}$
\end{oneparchoices}

\question The voltage resolution of a 12-bit digital to analog converter (DAC), whose output varies from -10 V to +10 V is, approximately\hfill\brak{\text{GATE PH 2010}}

\begin{oneparchoices}
	\choice 1 mV \choice 5 mV \choice 20 mV \choice 100 mV
\end{oneparchoices}

\question In one of the following circuits, negative feedback does not operate for a negative input. Which one is it? The opamps are running from $\pm$15 V supplies.\hfill\brak{\text{GATE PH 2010}}
\begin{figure}[H]
	\centering
	\includegraphics[width=0.9\textwidth]{pics/20.png}
\end{figure}

\question A system of N non-interacting classical point particles is constrained to move on the two-dimensional surface of a sphere. The internal energy of the system is\hfill\brak{\text{GATE PH 2010}}

\begin{oneparchoices}
	\choice $\frac{3}{2}Nk_{B}T$ \choice $\frac{1}{2}Nk_{B}T$ \choice $Nk_{B}T$ \choice $\frac{5}{2}Nk_{B}T$
\end{oneparchoices}

\question Which of the following atoms cannot exhibit Bose-Einstein condensation, even in principle?\hfill\brak{\text{GATE PH 2010}}

\begin{oneparchoices}
	\choice $^{1}H_{1}$ \choice $^{4}He_{2}$ \choice $^{23}Na_{11}$ \choice $^{40}K_{19}$
\end{oneparchoices}

\question For the set of all Lorentz transformations with velocities along the x-axis, consider the two statements given below:
	\begin{itemize}
		\item[P:] If L is a Lorentz transformation then, $L^{-1}$ is also a Lorentz transformation.
		\item[Q:] If $L_1$ and $L_2$ are Lorentz transformations then, $L_1L_2$ is necessarily a Lorentz transformation.
	\end{itemize}
	Choose the correct option.\hfill\brak{\text{GATE PH 2010}}

\begin{choices}
\begin{multicols}{2}
	\choice P is true and Q is false. \choice Both P and Q are true.
	\choice Both P and Q are false. \choice P is false and Q is true.
	\end{multicols}
\end{choices}

\question Which of the following is an allowed wavefunction for a particle in a bound state? N is a constant and $\alpha, \beta > 0$.\hfill\brak{\text{GATE PH 2010}}

\begin{choices}
	\begin{multicols}{2}
	 \choice $\psi=N\frac{e^{-\alpha r}}{r^{3}}$
	\choice $\psi=N(1-e^{-\alpha r})$
	\choice $\psi=Ne^{-\alpha x}e^{-\beta(x^{2}+y^{2}+z^{2})}$
	\choice $\psi=\begin{cases} \text{non-zero constant} & \text{if } r<R \\ 0 & \text{if } r>R \end{cases}$
	\end{multicols}
\end{choices}

\question A particle is confined within a spherical region of radius one femtometer ($10^{-15}$m). Its momentum can be expected to be about\hfill\brak{\text{GATE PH 2010}}

\begin{oneparchoices}
	\choice $20 \frac{\text{keV}}{c}$ \choice $200 \frac{\text{keV}}{c}$ \choice $200 \frac{\text{MeV}}{c}$ \choice $2 \frac{\text{GeV}}{c}$
\end{oneparchoices}

\subsection*{Q.26 - Q.55 carry two marks each.}

\question For the complex function, $f(z)=\frac{e^{\sqrt{z}}-e^{-\sqrt{z}}}{\sin(\sqrt{z})},$ which of the following statements is correct?\hfill\brak{\text{GATE PH 2010}}

\begin{choices}
	\choice $z=0$ is a branch point \choice $z=0$ is a pole of order one
	\choice $z=0$ is a removable singularity \choice $z=0$ is an essential singularity
\end{choices}

\question The solution of the differential equation for $y(t): \frac{d^{2}y}{dt^{2}}-y=2\cosh(t)$, subject to the initial conditions $y(0)=0$ and $\frac{dy}{dt}|_{t=0}=0,$ is\hfill\brak{\text{GATE PH 2010}}

\begin{choices}
	\begin{multicols}{2}
	 \choice $\frac{1}{2}\cosh(t)+t\sinh(t)$ \choice $-\sinh(t)+t\cosh(t)$
	\choice $t\cosh(t)$ \choice $t\sinh(t)$
	\end{multicols}
\end{choices}

\question Given the recurrence relation for the Legendre polynomials $$(2n+1) x P_n(x)=(n+1) P_{n+1}(x)+nP_{n-1}(x)$$, which of the following integrals has a non-zero value?\hfill\brak{\text{GATE PH 2010}}

\begin{choices}
	\begin{multicols}{2}
	 \choice $\int_{-1}^{+1}x^{2}P_{n}(x)P_{n+1}(x)dx$ \choice $\int_{-1}^{+1}x P_{n}(x)P_{n+2}(x)dx$
	\choice $\int_{-1}^{1}x[P_{n}(x)]^{2}dx$ \choice $\int_{-1}^{+1}x^{2}P_{n}(x)P_{n+2}(x)dx$
	\end{multicols}
\end{choices}

\question For a two-dimensional free electron gas, the electronic density n, and the Fermi energy $E_F$, are related by\hfill\brak{\text{GATE PH 2010}}

\begin{choices}
	\begin{multicols}{2}
	 \choice $n=\frac{(2mE_{F})^{3/2}}{3\pi^{2}\hbar^{3}}$ \choice $n=\frac{mE_{F}}{\pi\hbar^{2}}$
	\choice $n=\frac{mE_{F}}{2\pi\hbar^{2}}$ \choice $n=\frac{2^{3/2}(mE_{F})^{1/2}}{\pi\hbar}$
	\end{multicols}
\end{choices}

\question Far away from any of the resonance frequencies of a medium, the real part of the dielectric permittivity is\hfill\brak{\text{GATE PH 2010}}

\begin{choices}
	\begin{multicols}{2}
	 \choice Always independent of frequency \choice Monotonically decreasing with frequency
	\choice Monotonically increasing with frequency \choice A non-monotonic function of frequency
	\end{multicols}
\end{choices}

\question The ground state wavefunction of deuteron is in a superposition of s and d states. Which of the following is NOT true as a consequence?\hfill\brak{\text{GATE PH 2010}}

\begin{choices}
	\choice It has a non-zero quadruple moment \choice The neutron-proton potential is non-central
	\choice The orbital wavefunction is not spherically symmetric \choice The Hamiltonian does not conserve the total angular momentum
\end{choices}

\question The first three energy levels of ${}^{228}Th_{90}$ are shown below
	\begin{center}
	\begin{tabular}{r c l}
		$4^+$ & \rule{3cm}{0.4pt} & 187 keV \\
		$2^+$ & \rule{3cm}{0.4pt} & 57.5 keV \\
		$0^+$ & \rule{3cm}{0.4pt} & 0 keV \\
	\end{tabular}
	\end{center}
The expected spin-parity and energy of the next level are given by\hfill\brak{\text{GATE PH 2010}}

\begin{oneparchoices}
	\choice ($6^+$; 400 keV) \choice ($6^+$; 300 keV) \choice ($2^+$; 400 keV) \choice ($4^+$; 300 keV)
\end{oneparchoices}

\question The quark content of $\Sigma^{+}$, $K^{-}$, $\pi^{-}$ and p is indicated: $|\Sigma^{+}\rangle=|uus\rangle; |K^{-}\rangle=|s\bar{u}\rangle; |\pi^{-}\rangle=|\bar{u}d\rangle; |p\rangle=|uud\rangle$. In the process, $\pi^{-} + p \rightarrow K^{-} + \Sigma^{+}$, considering strong interactions only, which of the following statements is true?\hfill\brak{\text{GATE PH 2010}}

\begin{choices}
	\choice The process is allowed because $\Delta S = 0$
	\choice The process is allowed because $\Delta I_3 = 0$
	\choice The process is not allowed because $\Delta S \neq 0$ and $\Delta I_3 \neq 0$
	\choice The process is not allowed because the baryon number is violated
\end{choices}

\question The three principal moments of inertia of a methanol (CH$_3$OH) molecule have the property $I_x = I_y = I$ and $I_z \neq I$. The rotational energy eigenvalues are\hfill\brak{\text{GATE PH 2010}}

\begin{oneparchoices}
	\choice $\frac{\hbar^{2}}{2I}l(l+1)+\frac{\hbar^{2}m_{l}^{2}}{2}(\frac{1}{I_{z}}-\frac{1}{I})$
	\choice $\frac{\hbar^{2}}{2I}l(l+1)$
	\choice $\frac{\hbar^{2}m_{l}^{2}}{2}(\frac{1}{I_{z}}-\frac{1}{I})$
	\choice $\frac{\hbar^{2}}{2I}l(l+1)+\frac{\hbar^{2}m_{l}^{2}}{2}(\frac{1}{I_{z}}+\frac{1}{I})$
\end{oneparchoices}

\question A particle of mass m is confined in the potential $V(x)=\begin{cases} \frac{1}{2}m\omega^{2}x^{2} & \text{for } x>0, \\ \infty & \text{for } x \le 0. \end{cases}$
	Let the wavefunction of the particle be given by $\psi(x)=-\frac{1}{\sqrt{5}}\psi_{0}+\frac{2}{\sqrt{5}}\psi_{1}$, where $\psi_0$ and $\psi_1$ are the eigenfunctions of the ground state and the first excited state respectively. The expectation value of the energy is\hfill\brak{\text{GATE PH 2010}}
	\begin{center}
		\includegraphics[width=0.2\textwidth]{pics/35.png}
	\end{center}


\begin{oneparchoices}
	\choice $\frac{31}{10}\hbar\omega$ \choice $\frac{25}{10}\hbar\omega$ \choice $\frac{13}{10}\hbar\omega$ \choice $\frac{11}{10}\hbar\omega$
\end{oneparchoices}

\question Match the typical spectra of stable molecules with the corresponding wave-number range\hfill\brak{\text{GATE PH 2010}}

\begin{multicols}{2}
 \begin{enumerate}
  \item Electronic Spectra
  \item Rotational Spectra
  \item Molecular dissociation
 \end{enumerate}
\columnbreak
\begin{itemize}
 \item [i.] $10^6 cm^{-1}$ and above
 \item [ii.] $10^5 - 10^6 cm^{-1}$
 \item [iii.] $10^0 - 10^2 cm^{-1}$
\end{itemize}


\end{multicols}


\begin{choices}
	\begin{multicols}{2}
	 \choice 1-ii, 2-i, 3-iii \choice 1-ii, 2-iii, 3-i \choice 1-iii, 2-ii, 3-i \choice 1-i, 2-ii, 3-iii
	\end{multicols}
\end{choices}

\question Consider the operations $P:\vec{r}\rightarrow-\vec{r}$ (parity) and $T:t\rightarrow-t$ (time-reversal). For the electric and magnetic fields $\vec{E}$ and $\vec{B}$, which of the following set of transformations is correct?\hfill\brak{\text{GATE PH 2010}}

\begin{choices}
	\begin{multicols}{2}
	 \choice $P:\vec{E}\rightarrow-\vec{E},\vec{B}\rightarrow\vec{B};\\ T:\vec{E}\rightarrow\vec{E},\vec{B}\rightarrow-\vec{B}$
	\choice $P:\vec{E}\rightarrow\vec{E},\vec{B}\rightarrow-\vec{B};\\ T:\vec{E}\rightarrow\vec{E},\vec{B}\rightarrow\vec{B}$
	\choice $P:\vec{E}\rightarrow-\vec{E},\vec{B}\rightarrow\vec{B};\\ T:\vec{E}\rightarrow-\vec{E},\vec{B}\rightarrow-\vec{B}$
	\choice $P:\vec{E}\rightarrow\vec{E},\vec{B}\rightarrow-\vec{B};\\ T:\vec{E}\rightarrow-\vec{E},\vec{B}\rightarrow\vec{B}$
	\end{multicols}
\end{choices}

\question Two magnetic dipoles of magnitude m each are placed in a plane as shown. The energy of interaction is given by\hfill\brak{\text{GATE PH 2010}}
	\begin{center}
		\includegraphics[width=0.1\textwidth]{pics/38.png}
	\end{center}


\begin{oneparchoices}
	\choice Zero \choice $\frac{\mu_{0}}{4\pi}\frac{m^{2}}{d^{3}}$ \choice $\frac{3\mu_{0}}{2\pi}\frac{m^{2}}{d^{3}}$ \choice $-\frac{3\mu_{0}}{8\pi}\frac{m^{2}}{d^{3}}$
\end{oneparchoices}

\question Consider a conducting loop of radius a and total loop resistance R placed in a region with a magnetic field B thereby enclosing a flux $\phi_0$. The loop is connected to an electronic circuit as shown, the capacitor being initially uncharged.
	\begin{center}
		\includegraphics[width=0.8\textwidth]{pics/39.png}
	\end{center}
	If the loop is pulled out of the region of the magnetic field at a constant speed u, the final output voltage $V_{out}$ is independent of\hfill\brak{\text{GATE PH 2010}}


\begin{oneparchoices}
	\choice $\phi_0$ \choice u \choice R \choice C
\end{oneparchoices}

\question The figure shows a constant current source charging a capacitor that is initially uncharged.
	\begin{center}
		\includegraphics[width=0.3\textwidth]{pics/40.png}
	\end{center}
	If the switch is closed at $t=0$, which of the following plots depicts correctly the output voltage of the circuit as a function of time?\hfill\brak{\text{GATE PH 2010}}
	\begin{figure}[H]
		\centering
		\includegraphics[width=0.45\textwidth]{pics/40a.png}
		\includegraphics[width=0.45\textwidth]{pics/40b.png}
		\includegraphics[width=0.45\textwidth]{pics/40c.png}
		\includegraphics[width=0.45\textwidth]{pics/40d.png}
	\end{figure}


\question For any set of inputs, A and B, the following circuits give the same output, Q, except one. Which one is it?\hfill\brak{\text{GATE PH 2010}}

\begin{oneparchoices}
	\choice \includegraphics[width=0.51\textwidth]{pics/41a.png}
	\choice \includegraphics[width=0.51\textwidth]{pics/41b.png}
	\choice \includegraphics[width=0.51\textwidth]{pics/41c.png}
	\choice \includegraphics[width=0.51\textwidth]{pics/41d.png}
\end{oneparchoices}

\question CO$_2$ molecule has the first few energy levels uniformly separated by approximately 2.5 meV. At a temperature of 300 K, the ratio of the number of molecules in the 4\textsuperscript{th} excited state to the number in the 2\textsuperscript{nd} excited state is about\hfill\brak{\text{GATE PH 2010}}

\begin{oneparchoices}
	\choice 0.5 \choice 0.6 \choice 0.8 \choice 0.9
\end{oneparchoices}

\question Which among the following sets of Maxwell relations is correct? (U- internal energy, H- enthalpy, A- Helmholtz free energy and G- Gibbs free energy)\hfill\brak{\text{GATE PH 2010}}

\begin{choices}
	\begin{multicols}{2}
	 \choice $T = \left(\frac{\partial U}{\partial V}\right)_{S}$ and $P = -\left(\frac{\partial U}{\partial S}\right)_{V}$
	\choice $V = \left(\frac{\partial H}{\partial P}\right)_{S}$ and $T = \left(\frac{\partial H}{\partial S}\right)_{P}$
	\choice $P = -\left(\frac{\partial G}{\partial V}\right)_{T}$ and $V = -\left(\frac{\partial G}{\partial P}\right)_{S}$
	\choice $P = -\left(\frac{\partial A}{\partial S}\right)_{T}$ and $S = -\left(\frac{\partial A}{\partial P}\right)_{V}$
	\end{multicols}
\end{choices}

\question For a spin-s particle, in the eigen basis of $\hat{S}^2,S$, the expectation value $\langle s m | S_x^2 | s m \rangle$ is\hfill\brak{\text{GATE PH 2010}}

\begin{choices}
	\begin{multicols}{2}
	 \choice $\frac{\hbar^2[s(s+1)-m^2]}{2}$ \choice $\hbar^2[s(s+1)-2m^2]$
	\choice $\hbar^2[s(s+1)-m^2]$ \choice $\hbar^2 m^2$
	\end{multicols}
\end{choices}

\question A particle is placed in a region with the potential $V(x)=\frac{1}{2}kx^2-\frac{1}{4}\lambda x^3$, where $k, \lambda > 0$. Then,\hfill\brak{\text{GATE PH 2010}}

\begin{choices}
	\choice $x=0$ and $x=\pm\sqrt{\frac{k}{\lambda}}$ are points of stable equilibrium
	\choice $x=0$ is a point of stable equilibrium and $x=\pm\sqrt{\frac{k}{\lambda}}$ is a point of unstable equilibrium
	\choice $x=0$ and $x=\pm\sqrt{\frac{k}{\lambda}}$ are points of unstable equilibrium
	\choice There are no points of stable or unstable equilibrium
\end{choices}

\question A $\pi^0$ meson at rest decays into two photons, which move along the x-axis. They are both detected simultaneously after a time, $t=10$ s. In an inertial frame moving with a velocity $V=0.6c$ in the direction of one of the photons, the time interval between the two detections is\hfill\brak{\text{GATE PH 2010}}

\begin{oneparchoices}
	\choice 15 s \choice 0 s \choice 10 s \choice 20 s
\end{oneparchoices}

\question A particle of mass m is confined in an infinite potential well: $$V(x) = \begin{cases} 0 & \text{if } 0<x<L, \\ \infty & \text{otherwise} \end{cases}$$. It is subjected to a perturbing potential $V_p(x) = V_0 \sin\left(\frac{2\pi x}{L}\right)$ within the well. Let $E^{(1)}$ and $E^{(2)}$ be the corrections to the ground state energy in the first and second order in $V_0$, respectively. Which of the following are true?\hfill\brak{\text{GATE PH 2010}}
	\begin{center}
		\includegraphics[width=0.4\textwidth]{pics/47.png}
	\end{center}


\begin{choices}
	\begin{multicols}{2}
	 \choice $E^{(1)}=0; \quad E^{(2)}<0$ \choice $E^{(1)}>0; \quad E^{(2)}=0$
	\choice $E^{(1)}=0; \quad E^{(2)}$ depends on the sign of $V_0$ \choice $E^{(1)}<0; \quad E^{(2)}<0$
	\end{multicols}
\end{choices}

\subsection*{Common Data Questions}
\textbf{Common Data for Questions 48 and 49:}
\par\noindent In the presence of a weak magnetic field, atomic hydrogen undergoes the transition:
\[ ^2P_{3/2} \rightarrow ^2S_{1/2} \]
by emission of radiation.

\question The number of distinct spectral lines that are observed in the resultant Zeeman spectrum is\par\hfill\brak{\text{GATE PH 2010}}

\begin{oneparchoices}
	\choice 2 \choice 3 \choice 4 \choice 6
\end{oneparchoices}

\question The spectral line corresponding to the transition $$^2P_{3/2} \left(m_j = +\frac{1}{2}\right) \rightarrow ^2S_{1/2} \left(m_j = -\frac{1}{2}\right)$$ is observed along the direction of the applied magnetic field. The emitted electromagnetic field is:\hfill\brak{\text{GATE PH 2010}}

\begin{choices}
	\begin{multicols}{2}
	 \choice Circularly polarized \choice Linearly polarized \choice Unpolarized \choice Not emitted along the magnetic field direction
	\end{multicols}
\end{choices}

\textbf{Common Data for Questions 50 and 51:}
\par\noindent The partition function for a gas of photons is given by
\[ \ln Z = \frac{\pi^2 V(k_B T)^3}{45 \hbar^3 c^3} \]

\question The specific heat of the photon gas varies with temperature as\hfill\brak{\text{GATE PH 2010}}
\begin{figure}[H]
	\centering
	\includegraphics[width=0.35\textwidth]{pics/50a.png}
	\includegraphics[width=0.35\textwidth]{pics/50b.png}
	\includegraphics[width=0.35\textwidth]{pics/50c.png}
	\includegraphics[width=0.35\textwidth]{pics/50d.png}
\end{figure}

\question The pressure of the photon gas is\hfill\brak{\text{GATE PH 2010}}

\begin{oneparchoices}
	\choice $\frac{\pi^2(k_B T)^4}{15 \hbar^3 c^3}$ \choice $\frac{\pi^2(k_B T)^4}{8 \hbar^3 c^3}$
	\choice $\frac{\pi^2(k_B T)^4}{45 \hbar^3 c^3}$ \choice $\frac{\pi(k_B T)^4}{45 \hbar^3 c^3}$
\end{oneparchoices}

\subsection*{Linked Answer Questions}
\textbf{Statement for Linked Answer Questions 52 and 53:}
\par\noindent Consider the propagation of electromagnetic waves in a linear, homogeneous and isotropic material medium with electric permittivity $\epsilon$ and magnetic permeability $\mu$.

\question For a plane wave of angular frequency $\omega$ and propagation vector $\vec{k}$ propagating in the medium Maxwell's equations reduce to\hfill\brak{\text{GATE PH 2010}}

\begin{choices}
	\choice $\vec{k} \cdot \vec{E} = 0; \quad \vec{k} \cdot \vec{H} = 0; \quad \vec{k} \times \vec{E} = \omega\epsilon\vec{H}; \quad \vec{k} \times \vec{H} = -\omega\mu\vec{E}$
	\choice $\vec{k} \cdot \vec{E} = 0; \quad \vec{k} \cdot \vec{H} = 0; \quad \vec{k} \times \vec{E} = -\omega\epsilon\vec{H}; \quad \vec{k} \times \vec{H} = \omega\mu\vec{E}$
	\choice $\vec{k} \cdot \vec{E} = 0; \quad \vec{k} \cdot \vec{H} = 0; \quad \vec{k} \times \vec{E} = -\omega\mu\vec{H}; \quad \vec{k} \times \vec{H} = \omega\epsilon\vec{E}$
	\choice $\vec{k} \cdot \vec{E} = 0; \quad \vec{k} \cdot \vec{H} = 0; \quad \vec{k} \times \vec{E} = \omega\mu\vec{H}; \quad \vec{k} \times \vec{H} = -\omega\epsilon\vec{E}$
\end{choices}

\question If $\epsilon$ and $\mu$ assume negative values in a certain frequency range, then the directions of the propagation vector $\vec{k}$ and the Poynting vector $\vec{S}$ in that frequency range are related as\hfill\brak{\text{GATE PH 2010}}

\begin{choices}
	\choice $\vec{k}$ and $\vec{S}$ are parallel \choice $\vec{k}$ and $\vec{S}$ are anti-parallel
	\choice $\vec{k}$ and $\vec{S}$ are perpendicular to each other
	\choice $\vec{k}$ and $\vec{S}$ make an angle that depends on the magnitude of $|\epsilon|$ and $|\mu|$
\end{choices}

\textbf{Statement for Linked Answer Questions 54 and 55:}
\par\noindent The Lagrangian for a simple pendulum is given by:
\[ L = \frac{1}{2}ml^2\dot{\theta}^2 - mgl(1-\cos\theta) \]

\question Hamilton's equations are then given by\hfill\brak{\text{GATE PH 2010}}

\begin{choices}
	\begin{multicols}{2}
	 \choice $p_{\theta} = -mgl\sin\theta; \quad \dot{\theta} = \frac{p_{\theta}}{ml^2}$
	\choice $p_{\theta} = mgl\sin\theta; \quad \dot{\theta} = \frac{p_{\theta}}{ml^2}$
	\choice $p_{\theta} = -m\ddot{\theta}; \quad \dot{\theta} = \frac{p_{\theta}}{m}$
	\choice $\dot{p_{\theta}} = -\left(\frac{g}{l}\right)\theta; \quad \dot{\theta} = \frac{p_{\theta}}{ml}$
	\end{multicols}
\end{choices}

\question The Poisson bracket between $\theta$ and $\dot{\theta}$ is\hfill\brak{\text{GATE PH 2010}}

\begin{oneparchoices}
	\choice $\{\theta, \dot{\theta}\} = 1$ \choice $\{\theta, \dot{\theta}\} = \frac{1}{ml^2}$
	\choice $\{\theta, \dot{\theta}\} = \frac{1}{m}$ \choice $\{\theta, \dot{\theta}\} = \frac{g}{l}$
\end{oneparchoices}
\newpage
\section*{General Aptitude (GA) Questions}
\subsection*{Q.56 - Q.60 carry one mark each.}

\question \textit{Choose the most appropriate word from the options given below to complete the following sentence.}\hfill\brak{\text{GATE PH 2010}}\par
	\textbf{\noindent His rather casual remarks on politics \rule{2cm}{0.4pt} his lack of seriousness about the subject.}

\begin{oneparchoices}
	\choice masked \choice belied \choice betrayed \choice suppressed
\end{oneparchoices}

\question Which of the following options is the closest in meaning to the word below:\hfill\brak{\text{GATE PH 2010}}\par\noindent \textbf{Circuitous}

\begin{oneparchoices}
	\choice cyclic \choice indirect \choice confusing \choice crooked
\end{oneparchoices}

\question \textit{Choose the most appropriate word from the options given below to complete the following sentence:}\hfill\brak{\text{GATE PH 2010}}
	\par\noindent \textbf{If we manage to \rule{2cm}{0.4pt} our natural resources, we would leave a better planet for our children.}

\begin{oneparchoices}
	\choice uphold \choice restrain \choice cherish \choice conserve
\end{oneparchoices}

\question 25 persons are in a room. 15 of them play hockey, 17 of them play football and 10 of them play both hockey and football. Then the number of persons playing neither hockey nor football is:\hfill\brak{\text{GATE PH 2010}}

\begin{oneparchoices}
	\choice 2 \choice 17 \choice 13 \choice 3
\end{oneparchoices}

\question \textit{The question below consists of a pair of related words followed by four pairs of words. Select the pair that best expresses the relation in the original pair.}\hfill\brak{\text{GATE PH 2010}}
	\par\noindent\textbf{Unemployed : Worker}

\begin{oneparchoices}
	\choice fallow : land \choice unaware : sleeper \choice wit : jester \choice renovated : house
\end{oneparchoices}

\subsection*{Q.61 - Q.65 carry two marks each.}

\question If $137 + 276 = 435$ how much is $731 + 672$?\hfill\brak{\text{GATE PH 2010}}

\begin{oneparchoices}
	\choice 534 \choice 1403 \choice 1623 \choice 1513
\end{oneparchoices}

\question Hari (H), Gita (G), Irfan (I) and Saira (S) are siblings (i.e. brothers and sisters). All were born on 1\textsuperscript{st} January. The age difference between any two successive siblings (that is born one after another) is less than 3 years. Given the following facts:

\begin{enumerate}
		\item Hari's age + Gita's age $>$ Irfan's age + Saira's age.
		\item The age difference between Gita and Saira is 1 year. However, Gita is not the oldest and Saira is not the youngest.
		\item There are no twins.
	\end{enumerate}
	In what order were they born (oldest first)?\hfill\brak{\text{GATE PH 2010}}


\begin{oneparchoices}
	\choice HSIG \choice SGHI \choice IGSH \choice IHSG
\end{oneparchoices}

\question \textbf{Modern warfare has changed from large scale clashes of armies to suppression of civilian populations. Chemical agents that do their work silently appear to be suited to such warfare; and regretfully, there are people in military establishments who think that chemical agents are useful tools for their cause.}
	\par\noindent \textit{Which of the following statements best sums up the meaning of the above passage:}\hfill\brak{\text{GATE PH 2010}}

\begin{oneparchoices}
	\choice Modern warfare has resulted in civil strife.
	\choice Chemical agents are useful in modern warfare.
	\choice Use of chemical agents in warfare would be undesirable.
	\choice People in military establishments like to use chemical agents in war.
\end{oneparchoices}

\question 5 skilled workers can build a wall in 20 days; 8 semi-skilled workers can build a wall in 25 days; 10 unskilled workers can build a wall in 30 days. If a team has 2 skilled, 6 semi-skilled and 5 unskilled workers, how long will it take to build the wall?\hfill\brak{\text{GATE PH 2010}}

\begin{oneparchoices}
	\choice 20 days \choice 18 days \choice 16 days \choice 15 days
\end{oneparchoices}

\question Given digits 2, 2, 3, 3, 3, 4, 4, 4, 4 how many distinct 4 digit numbers greater than 3000 can be formed?\hfill\brak{\text{GATE PH 2010}}

\begin{oneparchoices}
	\choice 50 \choice 51 \choice 52 \choice 54
\end{oneparchoices}
\end{questions}

\begin{center}
	\hrulefill\\
	\textbf{END OF THE QUESTION PAPER}
\end{center}

\end{document}
