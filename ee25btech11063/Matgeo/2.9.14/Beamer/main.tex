\documentclass{beamer}
\mode<presentation>
\usepackage{amsmath,amssymb,mathtools}
\usepackage{textcomp}
\usepackage{gensymb}
\usepackage{adjustbox}
\usepackage{subcaption}
\usepackage{enumitem}
\usepackage{multicol}
\usepackage{listings}
\usepackage{url}
\usepackage{graphicx} % <-- needed for images
\def\UrlBreaks{\do\/\do-}

\usetheme{Boadilla}
\usecolortheme{lily}
\setbeamertemplate{footline}{
  \leavevmode%
  \hbox{%
  \begin{beamercolorbox}[wd=\paperwidth,ht=2ex,dp=1ex,right]{author in head/foot}%
    \insertframenumber{} / \inserttotalframenumber\hspace*{2ex}
  \end{beamercolorbox}}%
  \vskip0pt%
}
\setbeamertemplate{navigation symbols}{}

\lstset{
  frame=single,
  breaklines=true,
  columns=fullflexible,
  basicstyle=\ttfamily\tiny   % tiny font so code fits
}

\numberwithin{equation}{section}

% ---- your macros ----
\providecommand{\nCr}[2]{\,^{#1}C_{#2}}
\providecommand{\nPr}[2]{\,^{#1}P_{#2}}
\providecommand{\mbf}{\mathbf}
\providecommand{\pr}[1]{\ensuremath{\Pr\left(#1\right)}}
\providecommand{\qfunc}[1]{\ensuremath{Q\left(#1\right)}}
\providecommand{\sbrak}[1]{\ensuremath{{}\left[#1\right]}}
\providecommand{\lsbrak}[1]{\ensuremath{{}\left[#1\right.}}
\providecommand{\rsbrak}[1]{\ensuremath{\left.#1\right]}}
\providecommand{\brak}[1]{\ensuremath{\left(#1\right)}}
\providecommand{\lbrak}[1]{\ensuremath{\left(#1\right.}}
\providecommand{\rbrak}[1]{\ensuremath{\left.#1\right)}}
\providecommand{\cbrak}[1]{\ensuremath{\left\{#1\right\}}}
\providecommand{\lcbrak}[1]{\ensuremath{\left\{#1\right.}}
\providecommand{\rcbrak}[1]{\ensuremath{\left.#1\right\}}}
\theoremstyle{remark}
\newtheorem{rem}{Remark}
\newcommand{\sgn}{\mathop{\mathrm{sgn}}}
\providecommand{\abs}[1]{\left\vert#1\right\vert}
\providecommand{\res}[1]{\Res\displaylimits_{#1}}
\providecommand{\norm}[1]{\lVert#1\rVert}
\providecommand{\mtx}[1]{\mathbf{#1}}
\providecommand{\mean}[1]{E\left[ #1 \right]}
\providecommand{\fourier}{\overset{\mathcal{F}}{ \rightleftharpoons}}
\providecommand{\system}{\overset{\mathcal{H}}{ \longleftrightarrow}}
\providecommand{\dec}[2]{\ensuremath{\overset{#1}{\underset{#2}{\gtrless}}}}
\newcommand{\myvec}[1]{\ensuremath{\begin{pmatrix}#1\end{pmatrix}}}
\let\vec\mathbf

\title{Matgeo Presentation - Problem 2.9.14}
\author{ee25btech11063 - Vejith}

\begin{document}


\frame{\titlepage}
\begin{frame}{Question}
The two adjacent sides of a parallelogram are represented by $2\hat{i} + 4\hat{j} + -5\hat{k}$\hspace{0.3cm}and\hspace{0.3cm}$\hat{i} + 2\hat{j} + 3\hat{k}$.find the unit vectors parallel  to its diagonals.using diagonal vectors find the area of parallelogram
\end{frame}

\begin{frame}{Description}
\textbf{Solution: }\\
\begin{table}[h!]    
  \centering
  \begin{tabular}{c|c c c c c|c}
    & $x_1$ & $x_2$ & $x_3$ & $s_1$ & $s_2$ & RHS \\ \hline
    z\--row & 0 & 0 & 2 & 1 & 2 & 110 \\ \hline
$x_1$ & 1 & 0 & 1 & 1 & -1 & 20 \\
$x_2$ & 0 & 1 & 0 & -1 & 2 & 10 \\
\end{tabular}


  \caption{Variables Used}
  \label{}
\end{table}
\end{frame}

\begin{frame}{Solution}
The diagonals of the parallelogram are given by \\
\begin{align}
    \Vec{a}+\Vec{b}=\myvec{3\\6\\-2} \hspace{1.5cm} \text{and}\hspace{1.5cm} \Vec{a}-\Vec{b}=\myvec{1\\2\\-8}\\
    \notag
    \end{align}
 The corresponding unit vectors parallel to diagonals are\\
 \begin{align}
 \frac{\Vec{a}+\Vec{b}}{\norm{\Vec{a}+\Vec{b}}}=\myvec{\frac{3}{7}\\ \frac{6}{7}\\\frac{-2}{7}}\hspace{1.5cm} \text{and}\hspace{1.5cm} \frac{\Vec{a}-\Vec{b}}{\norm{\Vec{a}-\Vec{b}}}=\myvec{\frac{1}{\sqrt{69}}\\ \frac{2}{\sqrt{69}}\\\frac{-8}{\sqrt{69}}}\\
 \notag
 \end{align}
 If $\Vec{d1}$ and $\Vec{d}2$ are the diagonals of a parallelogram then area of parallelogram is =$\frac{1}{2}$$\norm{\Vec{d1}\times\Vec{d2}}$\\
 \end{frame}
 \begin{frame}{Conclusion}
 \begin{align}
 \rightarrow \text{area of parallelogram}=\frac{1}{2}\norm{(\Vec{a}+\Vec{b})\times(\Vec{a}-\Vec{b})}
\implies \text{area}=\frac{1}{2}\norm{\myvec{-44\\22\\0}}\\=\norm{\myvec{-22\\11\\0}}
= \sqrt{605}=24.59\end{align}
\end{frame}

\begin{frame}{Plot}
  \begin{figure}[H]
    \centering
    \includegraphics[width=1.0\columnwidth]{figs/01.png}
    \label{fig-1}
\end{figure}  
\end{frame}

% --------- CODE APPENDIX ---------
\section*{Appendix: Code}

% C program
\begin{frame}[fragile]{C Code: code.c}
\begin{lstlisting}[language=C]
#include <stdio.h>
#include <math.h>

/* magnitude of a 3D vector */
double magnitude(const double v[3]) {
    return sqrt(v[0]*v[0] + v[1]*v[1] + v[2]*v[2]);}
/* normalize a 3D vector into unit; if zero vector, sets unit to 0,0,0 */
void normalize(const double v[3], double unit[3]) {
    double mag = magnitude(v);
    if (mag == 0.0) {
        unit[0] = unit[1] = unit[2] = 0.0;
        return;}
    unit[0] = v[0] / mag;
    unit[1] = v[1] / mag;
    unit[2] = v[2] / mag;}
int main(void) {
    FILE *fp = fopen("plgm.dat", "w");
    if (fp == NULL) {
        perror("fopen");
        return 1;
    }
/* Given adjacent sides */
    double a[3] = {2.0, 4.0, -5.0};
    double b[3] = {1.0, 2.0, 3.0};

    /* Diagonals */
    double d1[3], d2[3];
    \end{lstlisting}
\end{frame}

\begin{frame}[fragile]{C Code: code.c}
\begin{lstlisting}[language=C]
    for (int i = 0; i < 3; ++i) {
        d1[i] = a[i] + b[i]; /* first diagonal */
        d2[i] = a[i] - b[i]; /* second diagonal */
    }

    /* Unit vectors parallel to diagonals */
    double u1[3], u2[3];
    normalize(d1, u1);
    normalize(d2, u2);
  /* Cross product of diagonals and area = 0.5 * |d1 x d2| */
    double cross[3];
    cross[0] = d1[1]*d2[2] - d1[2]*d2[1];
    cross[1] = d1[2]*d2[0] - d1[0]*d2[2];
    cross[2] = d1[0]*d2[1] - d1[1]*d2[0];
    double area = 0.5 * magnitude(cross);

    /* Write results to plgm.dat */
    fprintf(fp, "Adjacent sides:\n");
    fprintf(fp, "a = (%.2f, %.2f, %.2f)\n", a[0], a[1], a[2]);
    fprintf(fp, "b = (%.2f, %.2f, %.2f)\n\n", b[0], b[1], b[2]);

    fprintf(fp, "Diagonals:\n");
    fprintf(fp, "d1 = (%.2f, %.2f, %.2f)\n", d1[0], d1[1], d1[2]);
    fprintf(fp, "d2 = (%.2f, %.2f, %.2f)\n\n", d2[0], d2[1], d2[2]);

    fprintf(fp, "Unit vectors parallel to diagonals:\n");
    fprintf(fp, "u1 = (%.6f, %.6f, %.6f)\n", u1[0], u1[1], u1[2]);
    fprintf(fp, "u2 = (%.6f, %.6f, %.6f)\n\n", u2[0], u2[1], u2[2]);
    fprintf(fp, "Area of parallelogram = %.6f\n", area);
    fclose(fp);
    printf("plgm.dat written successfully.\n");
    return 0;
}
\end{lstlisting}
\end{frame}

\begin{frame}[fragile]{Python: plot.py}
\begin{lstlisting}[language=Python]
   import numpy as np
import matplotlib.pyplot as plt
from mpl_toolkits.mplot3d.art3d import Poly3DCollection

# Given adjacent sides
a = np.array([2, 4, -5])
b = np.array([1, 2, 3])

# Diagonals
d1 = a + b
d2 = a - b

# Define parallelogram vertices
O = np.array([0, 0, 0])  # Origin
A = a
B = b
C = a + b  # Opposite vertex

# Setup 3D plot
fig = plt.figure(figsize=(8, 6))
ax = fig.add_subplot(111, projection="3d")

# Draw the parallelogram surface
verts = [[O, A, C, B]]
ax.add_collection3d(Poly3DCollection(verts, alpha=0.3, facecolor="cyan"))

# Plot vectors for sides
ax.quiver(0, 0, 0, a[0], a[1], a[2], color="r", label="a (side)", linewidth=2)
ax.quiver(0, 0, 0, b[0], b[1], b[2], color="g", label="b (side)", linewidth=2)

# Plot diagonals
ax.quiver(0, 0, 0, d1[0], d1[1], d1[2], color="b", linestyle="dashed", label="d1 = a+b")
ax.quiver(0, 0, 0, d2[0], d2[1], d2[2], color="m", linestyle="dashed", label="d2 = a-b")
\end{lstlisting}
\end{frame}   

\begin{frame}[fragile]{Python: plot.py}
\begin{lstlisting}[language=Python]
# Set labels
ax.set_xlabel("X")
ax.set_ylabel("Y")
ax.set_zlabel("Z")
ax.set_title("Parallelogram with Sides and Diagonals")

# Auto scale
max_range = np.array([a, b, d1, d2]).max() - np.array([a, b, d1, d2]).min()
Xb = np.array([O[0], A[0], B[0], C[0], d1[0], d2[0]])
Yb = np.array([O[1], A[1], B[1], C[1], d1[1], d2[1]])
Zb = np.array([O[2], A[2], B[2], C[2], d1[2], d2[2]])

ax.set_xlim([Xb.min()-1, Xb.max()+1])
ax.set_ylim([Yb.min()-1, Yb.max()+1])
ax.set_zlim([Zb.min()-1, Zb.max()+1])

ax.legend()

# Save figure
plt.savefig("parallelogram.png", dpi=300)
plt.show()
 \end{lstlisting}
\end{frame}   
\end{document}
