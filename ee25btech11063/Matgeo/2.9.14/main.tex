\let\negmedspace\undefined
\let\negthickspace\undefined
\documentclass[journal]{IEEEtran}
\usepackage[a4paper, margin=10mm, onecolumn]{geometry}
%\usepackage{lmodern} % Ensure lmodern is loaded for pdflatex
\usepackage{tfrupee} % Include tfrupee package

\setlength{\headheight}{1cm} % Set the height of the header box
\setlength{\headsep}{0mm}  % Set the distance between the header box and the top of the text

\usepackage{gvv-book}
\usepackage{gvv}
\usepackage{cite}
\usepackage{amsmath,amssymb,amsfonts,amsthm}
\usepackage{algorithmic}
\usepackage{graphicx}
\usepackage{float}
\usepackage{textcomp}
\usepackage{xcolor}
\usepackage{txfonts}
\usepackage{listings}
\usepackage{enumitem}
\usepackage{mathtools}
\usepackage{gensymb}
\usepackage{comment}
\usepackage[breaklinks=true]{hyperref}
\usepackage{tkz-euclide} 
\usepackage{listings}
% \usepackage{gvv}                                        
\def\inputGnumericTable{}                                 
\usepackage[latin1]{inputenc}                                
\usepackage{color}                                            
\usepackage{array}                                            
\usepackage{longtable}                                       
\usepackage{calc}                                             
\usepackage{multirow}                                         
\usepackage{hhline}                                           
\usepackage{ifthen}                                           
\usepackage{lscape}
\usepackage{tikz}
\usetikzlibrary{patterns}

\begin{document}

\bibliographystyle{IEEEtran}
\vspace{3cm}

\title{2.2.5.31}
\author{ee25btech11063-vejith}

\maketitle
% \maketitle
% \newpage
% \bigskip
{\let\newpage\relax\maketitle}
\renewcommand{\thefigure}{\theenumi}
\renewcommand{\thetable}{\theenumi}
\setlength{\intextsep}{10pt} % Space between text and floats
\textbf{Question}:\\
The two adjacent sides of a parallelogram are represented by $2\hat{i} + 4\hat{j} + -5\hat{k}$\hspace{0.3cm}and\hspace{0.3cm}$\hat{i} + 2\hat{j} + 3\hat{k}$.find the unit vectors parallel  to its diagonals.using diagonal vectors find the area of parallelogram\\
\textbf{Solution}\\
 \begin{table}[h!]    
  \centering
  \begin{tabular}{c|c c c c c|c}
    & $x_1$ & $x_2$ & $x_3$ & $s_1$ & $s_2$ & RHS \\ \hline
    z\--row & 0 & 0 & 2 & 1 & 2 & 110 \\ \hline
$x_1$ & 1 & 0 & 1 & 1 & -1 & 20 \\
$x_2$ & 0 & 1 & 0 & -1 & 2 & 10 \\
\end{tabular}


  \caption{Variables Used}
  \label{}
\end{table}\\
The diagonals of the parallelogram are given by \\
\begin{align}
    \Vec{a}+\Vec{b}=\myvec{3\\6\\-2} \hspace{1.5cm} \text{and}\hspace{1.5cm} \Vec{a}-\Vec{b}=\myvec{1\\2\\-8}\\
    \notag
    \end{align}
 The corresponding unit vectors parallel to diagonals are\\
 \begin{align}
 \frac{\Vec{a}+\Vec{b}}{\norm{\Vec{a}+\Vec{b}}}=\myvec{\frac{3}{7}\\ \frac{6}{7}\\\frac{-2}{7}}\hspace{1.5cm} \text{and}\hspace{1.5cm} \frac{\Vec{a}-\Vec{b}}{\norm{\Vec{a}-\Vec{b}}}=\myvec{\frac{1}{\sqrt{69}}\\ \frac{2}{\sqrt{69}}\\\frac{-8}{\sqrt{69}}}\\
 \notag
 \end{align}
 If $\Vec{d1}$ and $\Vec{d}2$ are the diagonals of a parallelogram then area of parallelogram is =$\frac{1}{2}$$\norm{\Vec{d1}\times\Vec{d2}}$\\
 \begin{align}
\rightarrow \text{area of parallelogram}=\frac{1}{2}\norm{(\Vec{a}+\Vec{b})\times(\Vec{a}-\Vec{b})}
\implies \text{area}=\frac{1}{2}\norm{\myvec{-44\\22\\0}}=\norm{\myvec{-22\\11\\0}}
= \sqrt{605}=24.59\end{align}
\begin{figure}[H]
    \centering
    \includegraphics[width=0.66\columnwidth]{figs/01.png}
    \label{fig-1}
\end{figure}
   
\end{document}
