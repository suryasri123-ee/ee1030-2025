\let\negmedspace\undefined
\let\negthickspace\undefined
\documentclass[journal]{IEEEtran}
\usepackage[a5paper, margin=10mm, onecolumn]{geometry}
%\usepackage{lmodern} % Ensure lmodern is loaded for pdflatex

\setlength{\headheight}{1cm} % Set the height of the header box
\setlength{\headsep}{0mm}     % Set the distance between the header box and the top of the text

\usepackage{gvv-book}
\usepackage{gvv}
\usepackage{cite}
\usepackage{amsmath,amssymb,amsfonts,amsthm}
\usepackage{algorithmic}
\usepackage{graphicx}
\usepackage{textcomp}
\usepackage{xcolor}
\usepackage{txfonts}
\usepackage{listings}
\usepackage{enumitem}
\usepackage{mathtools}
\usepackage{gensymb}
\usepackage{comment}
\usepackage[breaklinks=true]{hyperref}
\usepackage{tkz-euclide} 
\usepackage{listings}
\def\inputGnumericTable{}                                 
\usepackage[latin1]{inputenc}                                
\usepackage{color}                                            
\usepackage{array}                                            
\usepackage{longtable}                                       
\usepackage{calc}                                             
\usepackage{multirow}                                         
\usepackage{hhline}                                           
\usepackage{ifthen}                                           
\usepackage{lscape}
\begin{document}

\bibliographystyle{IEEEtran}
\vspace{3cm}

\title{1.11.12}
\author{AI25BTECH11006 - Nikhila}
% \maketitle
% \newpage
% \bigskip
{\let\newpage\relax\maketitle}


\renewcommand{\thefigure}{\theenumi}
\renewcommand{\thetable}{\theenumi}
\setlength{\intextsep}{10pt} % Space between text and floats


\numberwithin{equation}{enumi}
\numberwithin{figure}{enumi}
\renewcommand{\thetable}{\theenumi}


\textbf{Question}:\\

Find the direction cosines of the line joining the points $\vec{P}(4,3,-5)$ and $\vec{Q}(-2,1,-8)$.\\

\textbf{Solution: }\\

The vector components of the given points are $\vec{P}$ \myvec{4 \\ 3 \\ -5}, $\vec{Q}$ \myvec{-2 \\ 1 \\ -8}. 

\vspace{2em}

The unit vector in the direction of $\vec{PQ}$ is given as
 \begin{align*}
\frac{\vec{Q}-\vec{P}}{\lVert \vec{Q}-\vec{P} \rVert}
\end{align*}


\begin{align}
    \vec{Q}-\vec{P} = \myvec{-6 \\ -2 \\ -3 } 
\end{align}

\begin{align}
    \lVert \vec{Q}-\vec{P} \rVert = \sqrt{(-6)^2 + (-2)^2 + (-3)^2} = 7
\end{align}

\begin{align}
    \frac{\vec{Q}-\vec{P}}{\lVert \vec{Q}-\vec{P} \rVert} = \frac{1}{7} \myvec{-6 \\ -2 \\ -3}
\end{align}

and the elements of the above vector are the direction cosines .
\vspace{2em}

\begin{figure}[h!]
   \centering
   \includegraphics[width=1\linewidth]{fig1.png}
   \caption{}
   \label{stemplot}
\end{figure}


\end{document}
