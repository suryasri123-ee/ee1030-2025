  \let\negmedspace\undefined
\let\negthickspace\undefined
\documentclass[journal]{IEEEtran}
\usepackage[a5paper, margin=10mm, onecolumn]{geometry}
\usepackage{lmodern} 
\usepackage{tfrupee} 

\setlength{\headheight}{1cm} % Set the height of the header box
\setlength{\headsep}{0mm}     % Set the distance between the header box and the top of the text

\usepackage{gvv-book}
\usepackage{gvv}
\usepackage{cite}
\usepackage{amsmath,amssymb,amsfonts,amsthm}
\usepackage{algorithmic}
\usepackage{graphicx}
\usepackage{textcomp}
\usepackage{xcolor}
\usepackage{txfonts}
\usepackage{listings}
\usepackage{enumitem}
\usepackage{mathtools}
\usepackage{gensymb}
\usepackage{comment}
\usepackage[breaklinks=true]{hyperref}
\usepackage{tkz-euclide} 
\usepackage{listings}                                      
\def\inputGnumericTable{}                                 
\usepackage[latin1]{inputenc}                                
\usepackage{color}                                            
\usepackage{array}                                            
\usepackage{longtable}
\usepackage{multicol}
\usepackage{calc}                                             
\usepackage{multirow}                                         
\usepackage{hhline}                                           
\usepackage{ifthen}                                           
\usepackage{lscape}
\begin{document}

\bibliographystyle{IEEEtran}
\vspace{3cm}

\title{1.5.19}
\author {EE25BTECH11031 - Sai Sreevallabh}
% \maketitle
% \newpage
% \bigskip
{\let\newpage\relax\maketitle}

\renewcommand{\thefigure}{\theenumi}
\renewcommand{\thetable}{\theenumi}
\setlength{\intextsep}{10pt} % Space between text and floats


\numberwithin{equation}{enumi}
\numberwithin{figure}{enumi}
\renewcommand{\thetable}{\theenumi}

\textbf{Question: }\\

Find the ratio in which the segment joining the points $\brak{1,3}$ and $\brak{4,5}$ is divided by the X-axis. Also find the coordinates of this point on the X-axis.\\ 

\textbf{Solution: }\\

Given points are\\
\begin{align}
    \vec{A}=\myvec{1\\3} \ \text{and}\  \vec{B}=\myvec{4\\5}
\end{align}

Let $\vec{P}$ be a point on the x-axis. We can assume it to be\\

\begin{align}
    \vec{P}=\myvec{x\\0}
\end{align}

$\vec{A}$, $\vec{B}$ and $\vec{P}$ are collinear. 

\begin{align}
    \vec{P}-\vec{A}=\myvec{x-1 \\ -3} \ , \ 
    \vec{B}-\vec{A}=\myvec{3\\2}  
\end{align}

\begin{align}
    \myvec{\vec{P}-\vec{A} & \vec{B}-\vec{A}}^T =& \myvec{x-1 & 3\\ -3 & 2}^T\\
    =& \myvec{x-1 & -3 \\ 3 & 2}
\end{align}

Converting into echelon form using row operations

\begin{align}
    \myvec{x-1 & -3 \\ 3 & 2}\ \xleftrightarrow[]{R_2 \to R_2-\frac{3}{x-1}R_1} \  \myvec{x-1 & -3 \\ 0 & \frac{2x+7}{x-1}}
\end{align}

Since the points are collinear, we can say that the rank of the matrix is $1$ i.e. 

\begin{align}
    &\frac{2x+7}{x-1} = 0\\
    \implies& x=-\frac{7}{2}
\end{align}

Let $\vec{P}$ divide the line joining points $\vec{A}$ and $\vec{B}$ in the ratio $k:1$. 

\begin{align}
    \vec{P}=\frac{k\vec{B}+\vec{A}}{k+1}
\end{align}

\begin{align}
    k\brak{\vec{P}- \vec{B}} = \vec{A}-\vec{P}
\end{align}

\begin{align}
    k=\frac{\brak{\vec{P}-\vec{B}}^T\brak{\vec{A}-\vec{P}}}{||\brak{\vec{P}-\vec{B}}||^2}
\end{align}

\begin{align}
      k=\frac{\myvec{x-4 & -5}\myvec{1-x\\3}}{{\norm{\myvec{x-4 \\ -5}}}^2}
\end{align}

Substituting the value of $x$ as $-\frac{7}{2}$, we get the value of k as

\begin{align}
    k=-\frac{3}{5}
\end{align}

Therefore\\

The point $\vec{P}\myvec{-\frac{7}{2}\\0}$ on the X-axis divides the line segment in the ratio $-3:5$ i.e. externally in the ratio $3:5$.

\begin{figure}
    \centering
    \includegraphics[width=0.8\columnwidth]{Figs/plot(py).png}
    \caption{Caption}
    \label{fig:placeholder}
\end{figure}

\end{document}
