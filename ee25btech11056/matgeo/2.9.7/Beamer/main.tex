\documentclass{beamer}
\mode<presentation>
\usepackage{amsmath,amssymb,mathtools}
\usepackage{textcomp}
\usepackage{gensymb}
\usepackage{adjustbox}
\usepackage{subcaption}
\usepackage{enumitem}
\usepackage{multicol}
\usepackage{listings}
\usepackage{url}
\usepackage{graphicx} % <-- needed for images
\def\UrlBreaks{\do\/\do-}

\usetheme{Boadilla}
\usecolortheme{lily}
\setbeamertemplate{footline}{
  \leavevmode%
  \hbox{%
  \begin{beamercolorbox}[wd=\paperwidth,ht=2ex,dp=1ex,right]{author in head/foot}%
    \insertframenumber{} / \inserttotalframenumber\hspace*{2ex}
  \end{beamercolorbox}}%
  \vskip0pt%
}
\setbeamertemplate{navigation symbols}{}

\lstset{
  frame=single,
  breaklines=true,
  columns=fullflexible,
  basicstyle=\ttfamily\tiny   % tiny font so code fits
}

\numberwithin{equation}{section}

% ---- your macros ----
\providecommand{\nCr}[2]{\,^{#1}C_{#2}}
\providecommand{\nPr}[2]{\,^{#1}P_{#2}}
\providecommand{\mbf}{\mathbf}
\providecommand{\pr}[1]{\ensuremath{\Pr\left(#1\right)}}
\providecommand{\qfunc}[1]{\ensuremath{Q\left(#1\right)}}
\providecommand{\sbrak}[1]{\ensuremath{{}\left[#1\right]}}
\providecommand{\lsbrak}[1]{\ensuremath{{}\left[#1\right.}}
\providecommand{\rsbrak}[1]{\ensuremath{\left.#1\right]}}
\providecommand{\brak}[1]{\ensuremath{\left(#1\right)}}
\providecommand{\lbrak}[1]{\ensuremath{\left(#1\right.}}
\providecommand{\rbrak}[1]{\ensuremath{\left.#1\right)}}
\providecommand{\cbrak}[1]{\ensuremath{\left\{#1\right\}}}
\providecommand{\lcbrak}[1]{\ensuremath{\left\{#1\right.}}
\providecommand{\rcbrak}[1]{\ensuremath{\left.#1\right\}}}
\theoremstyle{remark}
\newtheorem{rem}{Remark}
\newcommand{\sgn}{\mathop{\mathrm{sgn}}}
\providecommand{\abs}[1]{\left\vert#1\right\vert}
\providecommand{\res}[1]{\Res\displaylimits_{#1}}
\providecommand{\norm}[1]{\lVert#1\rVert}
\providecommand{\mtx}[1]{\mathbf{#1}}
\providecommand{\mean}[1]{E\left[ #1 \right]}
\providecommand{\fourier}{\overset{\mathcal{F}}{ \rightleftharpoons}}
\providecommand{\system}{\overset{\mathcal{H}}{ \longleftrightarrow}}
\providecommand{\dec}[2]{\ensuremath{\overset{#1}{\underset{#2}{\gtrless}}}}
\newcommand{\myvec}[1]{\ensuremath{\begin{pmatrix}#1\end{pmatrix}}}
\newcommand{\mydet}[1]{\ensuremath{\begin{vmatrix}#1\end{vmatrix}}}
\let\vec\mathbf
% ---------------------

\title{Matgeo Presentation - Problem 2.9.7}
\author{ee25btech11056 - Suraj.N}

\begin{document}

\begin{frame}
  \titlepage
\end{frame}

\begin{frame}{Problem Statement}
\textbf{Question} :  

\begin{align*}
\vec{a}=2\hat{i}+\hat{j}+3\hat{k},\ \vec{b}=-\hat{i}+2\hat{j}+\hat{k},\ \vec{c}=3\hat{i}+\hat{j}+2\hat{k}
\end{align*}

\begin{center}
then find \(\vec{a}\cdot(\vec{b}\times\vec{c})\)
\end{center}

\begin{table}[h!]
  \centering
  \begin{center}
    \begin{tabular}{|c|c|} 
        \hline
            \textbf{Variable}  & \textbf{Formula} \\ 
        \hline
            $a$   & $a = \myvec{4 \\ -1 \\ 1}$ \\ 
        \hline
            $b$   &  $b = \myvec{2 \\ -2 \\ 1}$\\ 
        \hline
           \end{tabular}
\end{center}  

  \caption*{Table : vectors}
  \label{2.9.7}
\end{table}

\end{frame}

\begin{frame}{Solution}

\begin{align*}
\vec{b} \times \vec{c} = 
\myvec{
|\vec{B_{23}} & \vec{C_{23}}| \\
|\vec{B_{31}} & \vec{C_{31}}| \\
|\vec{B_{12}} & \vec{C_{12}}|
}
\end{align*}

\begin{align*}
\mydet{\vec{B_{23}} & \vec{C_{23}}} = \mydet{2 & 1\\1 & 2} = 3
\end{align*}

\begin{align*}
\mydet{\vec{B_{31}} & \vec{C_{31}}} = \mydet{1 & 2\\-1 & 3} = 5
\end{align*}

\begin{align*}
\mydet{\vec{B_{12}} & \vec{C_{12}}} = \mydet{-1 & 3\\2 & 1} = -7
\end{align*}

\begin{align*}
  \vec{b} \times \vec{c} = \myvec{3\\5\\-7}
\end{align*}


\end{frame}

\begin{frame}
\begin{center}
the value of \(\vec{a}\cdot(\vec{b}\times\vec{c})\) = $\vec{a}^\top(\vec{b}\times\vec{c})$ = \myvec{2 & 1 & 3}\myvec{3\\5\\-7}\\
= (2)(3) + (1)(5) + (3)(-7) 
= 6 + 5 - 21\\
= -10

\textbf{Final Answer} : \(\vec{a}\cdot(\vec{b}\times\vec{c})\)  = -10 
\end{center}

\end{frame}

\begin{frame}{Plot}
\begin{figure}[h!]
  \centering
  \includegraphics[width=0.6\columnwidth]{figs/vectors.png} 
   \caption*{Fig : Vectors}
  \label{Fig1}
\end{figure}

\end{frame}

\section*{Appendix: Code}

% C program
\begin{frame}[fragile]{C Code: points.c}
\begin{lstlisting}[language=C]

#include <stdio.h>

// Function to compute scalar triple product: a · (b × c)
double triple_product(double a[3], double b[3], double c[3]) {
  double cross[3];

  // Cross product b × c
  cross[0] = b[1] * c[2] - b[2] * c[1];
  cross[1] = b[2] * c[0] - b[0] * c[2];
  cross[2] = b[0] * c[1] - b[1] * c[0];

  // Dot product a · (b × c)
  double result = a[0] * cross[0] + a[1] * cross[1] + a[2] * cross[2];
  return result;
}

\end{lstlisting}
\end{frame}

% Python calling C
\begin{frame}[fragile]{Python: call\_c.py}
\begin{lstlisting}[language=Python]

import ctypes
import numpy as np
import sys
import matplotlib.pyplot as plt

# Load shared library
lib = ctypes.CDLL("./libpoints.so")
lib.triple_product.restype = ctypes.c_double
lib.triple_product.argtypes = [
    ctypes.POINTER(ctypes.c_double),
    ctypes.POINTER(ctypes.c_double),
    ctypes.POINTER(ctypes.c_double)
]

# Default vectors (if no args provided)
a = np.array([2.0, 1.0, 3.0], dtype=np.double)
b = np.array([-1.0, 2.0, 1.0], dtype=np.double)
c = np.array([3.0, 1.0, 2.0], dtype=np.double)

# If user provides command-line arguments → override
if len(sys.argv) == 10:  # script + 9 numbers
    a = np.array([float(sys.argv[1]), float(sys.argv[2]), float(sys.argv[3])], dtype=np.double)
    b = np.array([float(sys.argv[4]), float(sys.argv[5]), float(sys.argv[6])], dtype=np.double)
    c = np.array([float(sys.argv[7]), float(sys.argv[8]), float(sys.argv[9])], dtype=np.double)

# Call C function
res = lib.triple_product(a.ctypes.data_as(ctypes.POINTER(ctypes.c_double)),
                         b.ctypes.data_as(ctypes.POINTER(ctypes.c_double)),
                         c.ctypes.data_as(ctypes.POINTER(ctypes.c_double)))

print(f"Scalar Triple Product (from C): {res}")

\end{lstlisting}
\end{frame}

\begin{frame}[fragile]{Python: call\_c.py}
\begin{lstlisting}[language=Python]


# ---- Visualization using matplotlib ----
cross_bc = np.cross(b, c)

fig = plt.figure()
ax = fig.add_subplot(111, projection='3d')

ax.quiver(0, 0, 0, a[0], a[1], a[2], color='blue', label='a', arrow_length_ratio=0.1)
ax.quiver(0, 0, 0, b[0], b[1], b[2], color='red', label='b', arrow_length_ratio=0.1)
ax.quiver(0, 0, 0, c[0], c[1], c[2], color='green', label='c', arrow_length_ratio=0.1)
ax.quiver(0, 0, 0, cross_bc[0], cross_bc[1], cross_bc[2],
          color='purple', label='b × c', arrow_length_ratio=0.1)

ax.set_title(f"Scalar Triple Product = {res:.2f}")
ax.set_xlabel("X-axis")
ax.set_ylabel("Y-axis")
ax.set_zlabel("Z-axis")
ax.legend()

plt.savefig("vectors.png", dpi=300, bbox_inches="tight")
plt.show()

\end{lstlisting}
\end{frame}

\begin{frame}[fragile]{Python: plot.py}
\begin{lstlisting}[language=Python]
import numpy as np
import matplotlib.pyplot as plt

# Vectors
a = np.array([2, 1, 3])
b = np.array([-1, 2, 1])
c = np.array([3, 1, 2])

# Scalar triple product
res = np.dot(a, np.cross(b, c))
print(f"Scalar Triple Product (using NumPy): {res}")

# Cross product b × c
cross_bc = np.cross(b, c)

# Plot
fig = plt.figure()
ax = fig.add_subplot(111, projection='3d')

ax.quiver(0, 0, 0, a[0], a[1], a[2], color='blue', label='a', arrow_length_ratio=0.1)
ax.quiver(0, 0, 0, b[0], b[1], b[2], color='red', label='b', arrow_length_ratio=0.1)
ax.quiver(0, 0, 0, c[0], c[1], c[2], color='green', label='c', arrow_length_ratio=0.1)
ax.quiver(0, 0, 0, cross_bc[0], cross_bc[1], cross_bc[2],
          color='purple', label='b × c', arrow_length_ratio=0.1)

ax.set_title(f"Scalar Triple Product = {res:.2f}")
ax.set_xlabel("X-axis")
ax.set_ylabel("Y-axis")
ax.set_zlabel("Z-axis")
ax.legend()

plt.savefig("vectors.png", dpi=300, bbox_inches="tight")
plt.show()

\end{lstlisting}
\end{frame}


\end{document}
