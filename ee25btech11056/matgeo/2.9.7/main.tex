\let\negmedspace\undefined
\let\negthickspace\undefined
\documentclass[journal,12pt,onecolumn]{IEEEtran}
\usepackage{cite}
\usepackage{amsmath,amssymb,amsfonts,amsthm}
\usepackage{algorithmic}
\usepackage{graphicx}
\graphicspath{{./figs/}}
\usepackage{textcomp}
\usepackage{xcolor}
\usepackage{txfonts}
\usepackage{listings}
\usepackage{enumitem}
\usepackage{mathtools}
\usepackage{gensymb}
\usepackage{comment}
\usepackage{caption}
\usepackage[breaklinks=true]{hyperref}
\usepackage{tkz-euclide} 
\usepackage{listings}
\usepackage{gvv}                                        
%\def\inputGnumericTable{}                                 
\usepackage[latin1]{inputenc}     
\usepackage{xparse}
\usepackage{color}                                            
\usepackage{array}
\usepackage{longtable}                                       
\usepackage{calc}                                             
\usepackage{multirow}
\usepackage{multicol}
\usepackage{hhline}                                           
\usepackage{ifthen}                                           
\usepackage{lscape}
\usepackage{tabularx}
\usepackage{array}
\usepackage{float}
\newtheorem{theorem}{Theorem}[section]
\newtheorem{problem}{Problem}
\newtheorem{proposition}{Proposition}[section]
\newtheorem{lemma}{Lemma}[section]
\newtheorem{corollary}[theorem]{Corollary}
\newtheorem{example}{Example}[section]
\newtheorem{definition}[problem]{Definition}
\newcommand{\BEQA}{\begin{eqnarray}}
\newcommand{\EEQA}{\end{eqnarray}}
\newcommand{\define}{\stackrel{\triangle}{=}}
\theoremstyle{remark}
\newtheorem{rem}{Remark}

\begin{document}

\title{2.9.7}
\author{ee25btech11056 - Suraj.N}
\maketitle
\renewcommand{\thefigure}{\theenumi}
\renewcommand{\thetable}{\theenumi}

\textbf{Question} :  

\begin{align*}
\vec{a}=2\hat{i}+\hat{j}+3\hat{k},\ \vec{b}=-\hat{i}+2\hat{j}+\hat{k},\ \vec{c}=3\hat{i}+\hat{j}+2\hat{k}
\end{align*}

\begin{center}
then find \(\vec{a}\cdot(\vec{b}\times\vec{c})\)
\end{center}

\begin{table}[h!]
  \centering
  \begin{center}
    \begin{tabular}{|c|c|} 
        \hline
            \textbf{Variable}  & \textbf{Formula} \\ 
        \hline
            $a$   & $a = \myvec{4 \\ -1 \\ 1}$ \\ 
        \hline
            $b$   &  $b = \myvec{2 \\ -2 \\ 1}$\\ 
        \hline
           \end{tabular}
\end{center}  

  \caption*{Table : vectors}
  \label{2.9.7}
\end{table}

\textbf{Solution} :

\begin{align*}
\vec{b} \times \vec{c} = 
\myvec{
|\vec{B_{23}} & \vec{C_{23}}| \\
|\vec{B_{31}} & \vec{C_{31}}| \\
|\vec{B_{12}} & \vec{C_{12}}|
}
\end{align*}

\begin{align*}
\mydet{\vec{B_{23}} & \vec{C_{23}}} = \mydet{2 & 1\\1 & 2} = 3
\end{align*}

\begin{align*}
\mydet{\vec{B_{31}} & \vec{C_{31}}} = \mydet{1 & 2\\-1 & 3} = 5
\end{align*}

\begin{align*}
\mydet{\vec{B_{12}} & \vec{C_{12}}} = \mydet{-1 & 3\\2 & 1} = -7
\end{align*}

\begin{align*}
  \vec{b} \times \vec{c} = \myvec{3\\5\\-7}
\end{align*}

\begin{center}
the value of \(\vec{a}\cdot(\vec{b}\times\vec{c})\) = $\vec{a}^\top(\vec{b}\times\vec{c})$ = \myvec{2 & 1 & 3}\myvec{3\\5\\-7}\\
= (2)(3) + (1)(5) + (3)(-7) 
= 6 + 5 - 21\\
= -10

\textbf{Final Answer} : \(\vec{a}\cdot(\vec{b}\times\vec{c})\)  = -10 
\end{center}

\pagebreak

\begin{figure}[h!]
  \centering
  \includegraphics[width=0.7\columnwidth]{figs/vectors.png} 
   \caption*{Fig : Vectors}
  \label{Fig1}
\end{figure}



\end{document}
