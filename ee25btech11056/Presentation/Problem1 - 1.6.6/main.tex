\documentclass{beamer}
\mode<presentation>
\usepackage{amsmath}
\usepackage{amssymb}
%\usepackage{advdate}
\usepackage{textcomp}
\usepackage{gensymb}
\usepackage{adjustbox}
\usepackage{subcaption}
\usepackage{enumitem}
\usepackage{multicol}
\usepackage{gensymb}
\usepackage{mathtools}
\usepackage{listings}
\usepackage{url}
\def\UrlBreaks{\do\/\do-}
\usetheme{Boadilla}
\usecolortheme{lily}
\setbeamertemplate{footline}
{
  \leavevmode%
  \hbox{%
  \begin{beamercolorbox}[wd=\paperwidth,ht=2ex,dp=1ex,right]{author in head/foot}%
    \insertframenumber{} / \inserttotalframenumber\hspace*{2ex} 
  \end{beamercolorbox}}%
  \vskip0pt%
}
\setbeamertemplate{navigation symbols}{}

\providecommand{\nCr}[2]{\,^{#1}C_{#2}} % nCr
\providecommand{\nPr}[2]{\,^{#1}P_{#2}} % nPr
\providecommand{\mbf}{\mathbf}
\providecommand{\pr}[1]{\ensuremath{\Pr\left(#1\right)}}
\providecommand{\qfunc}[1]{\ensuremath{Q\left(#1\right)}}
\providecommand{\sbrak}[1]{\ensuremath{{}\left[#1\right]}}
\providecommand{\lsbrak}[1]{\ensuremath{{}\left[#1\right.}}
\providecommand{\rsbrak}[1]{\ensuremath{{}\left.#1\right]}}
\providecommand{\brak}[1]{\ensuremath{\left(#1\right)}}
\providecommand{\lbrak}[1]{\ensuremath{\left(#1\right.}}
\providecommand{\rbrak}[1]{\ensuremath{\left.#1\right)}}
\providecommand{\cbrak}[1]{\ensuremath{\left\{#1\right\}}}
\providecommand{\lcbrak}[1]{\ensuremath{\left\{#1\right.}}
\providecommand{\rcbrak}[1]{\ensuremath{\left.#1\right\}}}
\theoremstyle{remark}
\newtheorem{rem}{Remark}
\newcommand{\sgn}{\mathop{\mathrm{sgn}}}
\providecommand{\abs}[1]{\left\vert#1\right\vert}
\providecommand{\res}[1]{\Res\displaylimits_{#1}} 
\providecommand{\norm}[1]{\lVert#1\rVert}
\providecommand{\mtx}[1]{\mathbf{#1}}
\providecommand{\mean}[1]{E\left[ #1 \right]}
\providecommand{\fourier}{\overset{\mathcal{F}}{ \rightleftharpoons}}
%\providecommand{\hilbert}{\overset{\mathcal{H}}{ \rightleftharpoons}}
\providecommand{\system}{\overset{\mathcal{H}}{ \longleftrightarrow}}
	%\newcommand{\solution}[2]{\textbf{Solution:}{#1}}
%\newcommand{\solution}{\noindent \textbf{Solution: }}
\providecommand{\dec}[2]{\ensuremath{\overset{#1}{\underset{#2}{\gtrless}}}}
\newcommand{\myvec}[1]{\ensuremath{\begin{pmatrix}#1\end{pmatrix}}}
\let\vec\mathbf

\lstset{
%language=C,
frame=single, 
breaklines=true,
columns=fullflexible
}

\numberwithin{equation}{section}

\title{Matgeo Presentation - Problem 1.6.6}
\author{ee25btech11056 - Suraj.N}

\begin{document}

\begin{frame}
  \titlepage
\end{frame}

% Problem Statement
\begin{frame}{Problem Statement}

\begin{itemize}
  \item In each of the following, find the value of $k$ for which the points are collinear:

\begin{itemize}
\item $(7,-2),\ (5,1),\ (3,k)$
\item $(8,1),\ (k,-4),\ (2,-5)$
\end{itemize}

\end{itemize}

\end{frame}

% Method Slide
\begin{frame}{Method}
\textbf{Condition for Collinearity:}
\begin{itemize}
\item Three points $A,B,C$ are collinear iff vectors 
\[
\vec{B}-\vec{A}, \quad \vec{C}-\vec{A}
\]
are linearly dependent.
\item Equivalently, the \textit{collinearity matrix}
\[
  M = \myvec{ \vec{B}-\vec{A} & \vec{C}-\vec{A} }^\top
\]
must satisfy $\operatorname{rank}(M)=1$.
\end{itemize}
\end{frame}

% Part (a) Setup
\begin{frame}{Part (a) Setup}
Let
\[
A=\myvec{7\\-2}, \quad B=\myvec{5\\1}, \quad C=\myvec{3\\k}.
\]
\[
\vec{B}-\vec{A} = \myvec{5-7\\ 1-(-2)} = \myvec{-2\\3},
\]
\[
\vec{C}-\vec{A} = \myvec{3-7\\ k-(-2)} = \myvec{-4\\k+2}.
\]
Thus,
\[
M = \myvec{-2 & 3\\ -4 & k+2}.
\]
\end{frame}

% Part (a) Row Reduction
\begin{frame}{Part (a) Row Reduction}
Apply row transformation:
\[
R_2 = R_2 - 2R_1 \quad \implies \quad
\myvec{-2 & 3\\ 0 & k-4}.
\]
\[
\text{For collinearity: } k-4=0 \quad \implies \quad k=\boxed{4}.
\]
\end{frame}

% Part (b) Setup
\begin{frame}{Part (b) Setup}
Let
\[
A=\myvec{8\\1}, \quad B=\myvec{k\\-4}, \quad C=\myvec{2\\-5}.
\]
\[
\vec{B}-\vec{A} = \myvec{k-8\\ -5}, \quad
\vec{C}-\vec{A} = \myvec{-6\\ -6}.
\]
Thus,
\[
M = \myvec{k-8 & -5\\ -6 & -6}.
\]
\end{frame}

% Part (b) Row Reduction
\begin{frame}{Part (b) Row Reduction}
Row operation:
\[
R_2 = (k-8)R_2 + 6R_1
\]
\[
\implies \quad \myvec{k-8 & -5\\ 0 & 18-6k}.
\]
\[
\text{For collinearity: } 18-6k=0 \quad \implies \quad k=\boxed{3}.
\]
\end{frame}

% Conclusion
\begin{frame}{Final Answer}
\begin{itemize}
\item (a) $k=4$
\item (b) $k=3$
\end{itemize}
\end{frame}


\section{Plot of line1}

\begin{frame}
\frametitle{Plot of line1}

\begin{figure}[h!]
  \centering
  \includegraphics[width=0.5\columnwidth]{figs/fig_a.png} 
   \caption*{Fig 1 : Line through the given points}
  \label{Fig1}
\end{figure}

\end{frame}

\section{Plot of line2}

\begin{frame}
\frametitle{Plot of line2}

\begin{figure}[h!]
  \centering
  \includegraphics[width=0.5\columnwidth]{figs/fig_b.png} 
   \caption*{Fig 2 : Line through the given points}
  \label{Fig2}
\end{figure}

\end{frame}





\end{document}
