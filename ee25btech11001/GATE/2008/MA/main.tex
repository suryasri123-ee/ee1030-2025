\documentclass[journal,12pt,onecolumn]{IEEEtran}

\usepackage{cite}
\usepackage{graphicx}
\usepackage{amsmath,amssymb,amsfonts,amsthm}
\usepackage{algorithmic}
\usepackage{textcomp}
\usepackage{xcolor}
\usepackage{txfonts}
\usepackage{listings}
\usepackage{enumitem}
\usepackage{mathtools}
\usepackage{gensymb}
\usepackage{comment}
\usepackage[breaklinks=true]{hyperref}
\usepackage{tkz-euclide} 
\usepackage[latin1]{inputenc} 
\usetikzlibrary{arrows.meta, positioning}
\usepackage{color}                                            
\usepackage{array}                                            
\usepackage{longtable}                                       
\usepackage{calc}                                             
\usepackage{multirow}
\usepackage{multicol}
\usepackage{hhline}                                           
\usepackage{ifthen}                                           
\usepackage{lscape}
\usepackage{tabularx}
\usepackage{float}
\usepackage{gvv}
\usepackage{enumitem}



\newtheorem{theorem}{Theorem}[section]
\newtheorem{problem}{Problem}
\newtheorem{proposition}{Proposition}[section]
\newtheorem{lemma}{Lemma}[section]
\newtheorem{corollary}[theorem]{Corollary}
\newtheorem{example}{Example}[section]
\newtheorem{definition}[problem]{Definition}

\newcommand{\BEQA}{\begin{eqnarray}}
\newcommand{\EEQA}{\end{eqnarray}}

\theoremstyle{remark}
\usepackage{tikz}

\title{MA: MATHEMATICS}
\author{EE25BTECH11001 - Aarush Dilawri}

\begin{document}
\maketitle

\begin{enumerate}
\item Consider the subspace $W = \{ [a_{ij}] ; a_{ij} = 0 \text{ if $i$ is even} \}$ of all $10\times 10$ real matrices. Then the dimension of $W$ is  
\\[-0.3em]\makebox[\textwidth][r]{\textit{[GATE EE 2025]}}



\begin{multicols}{4}
\begin{enumerate}[label=(\Alph*)]
\item 25
\item 50
\item 75
\item 100
\end{enumerate}
\end{multicols}

\item Let $S$ be the open unit disk and $f : S \to \mathbb{C}$ be a real-valued analytic function with $f(0) = 1$. Then the set $\{ z \in S : f(z) \neq 1 \}$ is  
\\[-0.3em]\makebox[\textwidth][r]{\textit{[GATE EE 2025]}}

\begin{multicols}{4}
\begin{enumerate}[label=(\Alph*)]
\item empty
\item nonempty finite
\item countably infinite
\item uncountable
\end{enumerate}
\end{multicols}

\item Let $E = \{ (x,y) \in \mathbb{R}^2 : 0 \le x \le 1, 0 \le y \le x \}$. Then  
\[
\iint_E (x + y) \, dx \, dy
\]
is equal to 
\\[-0.3em]\makebox[\textwidth][r]{\textit{[GATE EE 2025]}}

\begin{multicols}{4}
\begin{enumerate}[label=(\Alph*)]
\item $-1$
\item $0$
\item $\frac12$
\item $1$
\end{enumerate}
\end{multicols}

\item For $(x,y) \in \mathbb{R}^2$, let  
\[
f(x,y) = 
\begin{cases}
\dfrac{2xy}{x^2 + y^2} & \text{if } (x,y) \neq (0,0),\\[4pt]
0 & \text{if } (x,y) = (0,0).
\end{cases}
\]
Then
\\[-0.3em]\makebox[\textwidth][r]{\textit{[GATE EE 2025]}}

\begin{multicols}{2}
\begin{enumerate}[label=(\Alph*)]
\item $f_x$ and $f_y$ exist at $(0,0)$, and $f$ is continuous at $(0,0)$
\item $f_x$ and $f_y$ exist at $(0,0)$, and $f$ is discontinuous at $(0,0)$
\item $f_x$ and $f_y$ do not exist at $(0,0)$, and $f$ is continuous at $(0,0)$
\item $f_x$ and $f_y$ do not exist at $(0,0)$, and $f$ is discontinuous at $(0,0)$
\end{enumerate}
\end{multicols}

\item Let $y$ be a solution of $y' = e^{x^2} - 1$ on $[0,1]$ which satisfies $y(0) = 0$. Then 
\\[-0.3em]\makebox[\textwidth][r]{\textit{[GATE EE 2025]}}

\begin{multicols}{2}
\begin{enumerate}[label=(\Alph*)]
\item $y(x) > 0$ for $x > 0$
\item $y(x) < 0$ for $x > 0$
\item $y$ changes sign in $[0,1]$
\item $y \equiv 0$ for $x > 0$
\end{enumerate}
\end{multicols}

\item For the equation $x(x-1)y'' + \sin(x)y' + 2x(x-1)y = 0$, consider the statements:  
P: $x=0$ is a regular singular point.  
Q: $x=1$ is a regular singular point.  
Then 
\\[-0.3em]\makebox[\textwidth][r]{\textit{[GATE EE 2025]}}

\begin{multicols}{2}
\begin{enumerate}[label=(\Alph*)]
\item both P and Q are true
\item P is false but Q is true
\item P is true but Q is false
\item both P and Q are false
\end{enumerate}
\end{multicols}

\item Let $G = \mathbb{R} \setminus \{ 0 \}$ and $H = \{ -1, 1 \}$ be groups under multiplication. Then the map $\varphi : G \to H$ defined by $\varphi(x) = \frac{x}{|x|}$ is 
\\[-0.3em]\makebox[\textwidth][r]{\textit{[GATE EE 2025]}}

\begin{multicols}{2}
\begin{enumerate}[label=(\Alph*)]
\item not a homomorphism
\item a one-one homomorphism, which is not onto
\item an onto homomorphism, which is not one-one
\item an isomorphism
\end{enumerate}
\end{multicols}


\item The number of maximal ideals in $\mathbb{Z}_{27}$ is
\\[-0.3em]\makebox[\textwidth][r]{\textit{[GATE EE 2025]}}

\begin{enumerate}
\item 0
\item 1
\item 2
\item 3
\end{enumerate}


\item For $1 \le p \le \infty$, let $\| \cdot \|_p$ denote the $p$-norm on $\mathbb{R}^2$. If $\| \cdot \|_p$ satisfies the parallelogram law, then $p$ is equal to
\\[-0.3em]\makebox[\textwidth][r]{\textit{[GATE EE 2025]}}

\begin{enumerate}
\item 1
\item 2
\item 3
\item $\infty$
\end{enumerate}

% Q10
\item Consider the initial value problem $\frac{dy}{dx} = f(x,y)$, $y(x_0) = y_0$. The aim is to compute the value of $y_1 = y(x_1)$, where $x_1 = x_0 + h$ $(h > 0)$. At $x = x_1$, if the value of $y_1$ is equated to the corresponding value of the straight line passing through $(x_0, y_0)$ and having the slope equal to the slope of the curve $y(x)$ at $x = x_0$, then the method is called
\\[-0.3em]\makebox[\textwidth][r]{\textit{[GATE EE 2025]}}

\begin{enumerate}
\item Euler's method
\item Improved Euler's method
\item Backward Euler's method
\item Taylor series method of order 2
\end{enumerate}


\item The solution of $xu_x + yu_y = 0$ is of the form
\\[-0.3em]\makebox[\textwidth][r]{\textit{[GATE EE 2025]}}

\begin{enumerate}
\item $f\left(\frac{y}{x}\right)$
\item $f(x+y)$
\item $f(x-y)$
\item $f(xy)$
\end{enumerate}


\item If the partial differential equation $(x-1)^2 u_{xx} - (y-2)^2 u_{yy} + 2xu_x + 2yu_y + 2xyu = 0$ is parabolic in $S \subset \mathbb{R}^2$ but not in $\mathbb{R}^2 \setminus S$, then $S$ is
\\[-0.3em]\makebox[\textwidth][r]{\textit{[GATE EE 2025]}}

\begin{enumerate}
\item $\{(x,y) \in \mathbb{R}^2 : x=1 \text{ or } y=2\}$
\item $\{(x,y) \in \mathbb{R}^2 : x=1 \text{ and } y=2\}$
\item $\{(x,y) \in \mathbb{R}^2 : x=1\}$
\item $\{(x,y) \in \mathbb{R}^2 : y=2\}$
\end{enumerate}


\item Let $E$ be a connected subset of $\mathbb{R}$ with at least two elements. Then the number of elements in $E$ is
\\[-0.3em]\makebox[\textwidth][r]{\textit{[GATE EE 2025]}}

\begin{enumerate}
\item exactly two
\item more than two but finite
\item countably infinite
\item uncountable
\end{enumerate}


\item Let $X$ be a non-empty set. Let ${T}_1$ and ${T}_2$ be two topologies on $X$ such that ${T}_1$ is strictly contained in ${T}_2$. If $I:(X, {T}_3) \to (X, {T}_3)$ is the identity map, then
\\[-0.3em]\makebox[\textwidth][r]{\textit{[GATE EE 2025]}}

\begin{enumerate}
\item both $I$ and $I^{-1}$ are continuous
\item both $I$ and $I^{-1}$ are not continuous
\item $I$ is continuous but $I^{-1}$ is not continuous
\item $I$ is not continuous but $I^{-1}$ is continuous
\end{enumerate}


\item Let $X_1, X_2, \dots, X_{10}$ be a random sample from $N(80, 3^2)$ distribution. Define
\[
S = \sum_{i=1}^{10} U_i \quad \text{and} \quad T = \sum_{i=1}^{10} \left( U_i - \frac{S}{10} \right)^2,
\]
where $U_i = \frac{X_i - 80}{3}$, $i=1,2,\dots,10$. Then the value of $E(ST)$ is equal to
\\[-0.3em]\makebox[\textwidth][r]{\textit{[GATE EE 2025]}}

\begin{enumerate}
\item 0
\item 1
\item 10
\item $\frac{80}{3}$
\end{enumerate}

% Q16
\item Two (indistinguishable) fair coins are tossed simultaneously. Given that ONE of them lands up head, the probability of the OTHER to land up tail is equal to
\\[-0.3em]\makebox[\textwidth][r]{\textit{[GATE EE 2025]}}

\begin{enumerate}
\item $\frac{1}{3}$
\item $\frac{1}{2}$
\item $\frac{2}{3}$
\item $\frac{3}{4}$
\end{enumerate}

% Q17
\item Let $c_{ij} \ge 2$ be the cost of the $(i,j)^{\text{th}}$ cell of an assignment problem. If a new cost matrix is generated by the elements $c'_{ij} = \frac{1}{2} c_{ij} + 1$, then
\\[-0.3em]\makebox[\textwidth][r]{\textit{[GATE EE 2025]}}

\begin{enumerate}
\item optimal assignment plan remains unchanged and cost of assignment decreases
\item optimal assignment plan changes and cost of assignment decreases
\item optimal assignment plan remains unchanged and cost of assignment increases
\item optimal assignment plan changes and cost of assignment increases
\end{enumerate}

% Q18
\item Let a primal linear programming problem admit an optimal solution. Then the corresponding dual problem
\\[-0.3em]\makebox[\textwidth][r]{\textit{[GATE EE 2025]}}

\begin{enumerate}
\item does not have a feasible solution
\item has a feasible solution but does not have any optimal solution
\item does not have a convex feasible region
\item has an optimal solution
\end{enumerate}

% Q19
\item In any system of particles, suppose we do not assume that the internal forces come in pairs. Then the fact that the sum of internal forces is zero follows from
\\[-0.3em]\makebox[\textwidth][r]{\textit{[GATE EE 2025]}}

\begin{enumerate}
\item Newton's second law
\item conservation of angular momentum
\item conservation of energy
\item principle of virtual displacement
\end{enumerate}

% Q20
\item Let $q_1, q_2, \dots, q_n$ be the generalized coordinates and $\dot{q}_1, \dot{q}_2, \dots, \dot{q}_n$ be the generalized velocities in a conservative force field. If under a transformation $\varphi$, the new coordinate system has the generalized coordinates $Q_1, Q_2, \dots, Q_n$ and velocities $\dot{Q}_1, \dot{Q}_2, \dots, \dot{Q}_n$, then the equation
\[
\frac{\partial L}{\partial q_i} - \frac{d}{dt} \left( \frac{\partial L}{\partial \dot{q}_i} \right)
\]
takes the form
\\[-0.3em]\makebox[\textwidth][r]{\textit{[GATE EE 2025]}}

\begin{enumerate}
\item $\varphi \frac{\partial L}{\partial Q_i} - \frac{d}{dt} \left( \frac{\partial L}{\partial \dot{Q}_i} \right)$
\item $\varphi \frac{\partial L}{\partial Q_i} + \frac{d}{dt} \left( \frac{\partial L}{\partial \dot{Q}_i} \right)$
\item $- \varphi \frac{\partial L}{\partial Q_i} - \frac{d}{dt} \left( \frac{\partial L}{\partial \dot{Q}_i} \right)$
\item $\frac{\partial L}{\partial Q_i} - \frac{d}{dt} \left( \frac{\partial L}{\partial \dot{Q}_i} \right)$
\end{enumerate}

% Q21
\item Let $T: \mathbb{R}^4 \to \mathbb{R}^4$ be the linear map satisfying
\[
T(e_1) = e_2, \quad T(e_2) = e_3, \quad T(e_3) = 0, \quad T(e_4) = e_3,
\]
where $\{ e_1, e_2, e_3, e_4 \}$ is the standard basis of $\mathbb{R}^4$. Then
\\[-0.3em]\makebox[\textwidth][r]{\textit{[GATE EE 2025]}}

\begin{enumerate}
\item $T$ is idempotent
\item $T$ is invertible
\item $\mathrm{Rank}\,T = 3$
\item $T$ is nilpotent
\end{enumerate}




% ----- Continue from Q22 -----
% ----- Continue from Q22 -----
\item Let
\[
M=\myvec{
1 & 1 & 2\\
0 & 1 & 1\\
0 & 1 & 1
}, \qquad
V=\{Mx : x\in\mathbb{R}^3\}.
\]
Then an orthonormal basis for \(V\) is
\\[-0.3em]\makebox[\textwidth][r]{\textit{[GATE EE 2025]}}

\begin{enumerate}[label=(\Alph*)]
\begin{multicols}{2}
\item \(\{\myvec{1\\0\\0},\myvec{0\\\frac{2}{\sqrt{5}}\\\frac{1}{\sqrt{5}}}\}\)
\item \(\{\myvec{1\\0\\0},\myvec{0\\\frac{1}{\sqrt{2}}\\\frac{1}{\sqrt{2}}}\}\)
\item \(\{\myvec{\frac{1}{\sqrt{3}}\\\frac{1}{\sqrt{3}}\\\frac{1}{\sqrt{3}}},\myvec{\frac{2}{\sqrt{6}}\\\frac{1}{\sqrt{6}}\\\frac{1}{\sqrt{6}}}\}\)
\item \(\{\myvec{1\\0\\0},\myvec{0\\0\\1}\}\)
\end{multicols}
\end{enumerate}

\item For any \(n\in\mathbb{N}\), let \(P_n\) denote the vector space of all real polynomials of degree at most \(n-1\).  
Define the linear map \(T:P_n\to P_{n+1}\) by
\[
T(p)(x)=p(x)-\int_0^x p(t)\,dt.
\]
Then \(\dim(\ker T)\) is
\\[-0.3em]\makebox[\textwidth][r]{\textit{[GATE EE 2025]}}

\begin{enumerate}[label=(\Alph*)]
\item \(0\)
\item \(1\)
\item \(n\)
\item \(n+1\)
\end{enumerate}

\item Let
\[
M=\myvec{
1 & 0 & 0\\
0 & \cos\theta & -\sin\theta\\
0 & \sin\theta & \cos\theta
},\qquad 0<\theta<\tfrac{\pi}{2},
\]
and let \(V=\{u\in\mathbb{R}^3:Mu=u\}\). Then \(\dim(V)\) is
\\[-0.3em]\makebox[\textwidth][r]{\textit{[GATE EE 2025]}}

\begin{enumerate}[label=(\Alph*)]
\item \(0\)
\item \(1\)
\item \(2\)
\item \(3\)
\end{enumerate}

\item The number of linearly independent eigenvectors of the matrix
\[
\myvec{
2 & 2 & 0 & 0\\
2 & 1 & 0 & 0\\
0 & 0 & 3 & 0\\
0 & 0 & 1 & 4
}
\]
is
\\[-0.3em]\makebox[\textwidth][r]{\textit{[GATE EE 2025]}}

\begin{enumerate}[label=(\Alph*)]
\item \(1\)
\item \(2\)
\item \(3\)
\item \(4\)
\end{enumerate}

\item Let \(f\) be a bilinear transformation mapping \(-1\mapsto 1\), \(i\mapsto 0\), and \(-i\mapsto 0\).  
Then \(f(i)\) is equal to
\\[-0.3em]\makebox[\textwidth][r]{\textit{[GATE EE 2025]}}

\begin{enumerate}[label=(\Alph*)]
\item \(-2\)
\item \(-1\)
\item \(i\)
\item \(-i\)
\end{enumerate}


% ----- 5th Page -----

\item Which one of the following does NOT hold for all continuous functions 
$f:\sbrak{-\pi,\pi} \to \mathbb{C}$?
\\[-0.3em]\makebox[\textwidth][r]{\textit{[GATE EE 2025]}}

\begin{enumerate}[label=(\Alph*)]
\item If $f(-t) = f(t)$ for each $t \in \sbrak{-\pi,\pi}$, then 
$\int_{-\pi}^\pi f(t)dt = 2\int_{0}^\pi f(t)dt$

\item If $f(-t) = -f(t)$ for each $t \in \sbrak{-\pi,\pi}$, then 
$\int_{-\pi}^\pi f(t)dt = 0$

\item $\int_{-\pi}^\pi f(-t)dt = -\int_{-\pi}^\pi f(t)dt$

\item There is an $\alpha$ with $-\pi < \alpha < \pi$ such that 
$\int_{-\pi}^\pi f(t)dt = 2\pi f(\alpha)$
\end{enumerate}

\item Let $S$ be the positively oriented circle given by $\abs{z - 3i} = 2$. 
Then the value of $\int_{S} \frac{dz}{z^2 + 4}$ is
\\[-0.3em]\makebox[\textwidth][r]{\textit{[GATE EE 2025]}}

\begin{enumerate}[label=(\Alph*)]
\item $-\frac{\pi}{2}$
\item $\frac{\pi}{2}$
\item $-\frac{i\pi}{2}$
\item $\frac{i\pi}{2}$
\end{enumerate}

\item Let $T$ be the closed unit disk and $\partial T$ be the unit circle. 
Then which one of the following holds for every analytic function 
$f:T \to \mathbb{C}$?
\\[-0.3em]\makebox[\textwidth][r]{\textit{[GATE EE 2025]}}

\begin{enumerate}[label=(\Alph*)]
\item $\abs{f}$ attains its minimum and its maximum on $\partial T$
\item $\abs{f}$ attains its minimum on $\partial T$ but need not attain its maximum on $\partial T$
\item $\abs{f}$ attains its maximum on $\partial T$ but need not attain its minimum on $\partial T$
\item $\abs{f}$ need not attain its maximum on $\partial T$ and also it need not attain its minimum on $\partial T$
\end{enumerate}

\item Let $S$ be the disk $\abs{z} < 3$ in the complex plane and let 
$f:S \to \mathbb{C}$ be an analytic function such that 
\[
f\brak{1+\frac{\sqrt{2}}{n}i} = \frac{2}{n^2}
\]
for each natural number $n$. Then $f(\sqrt{2}i)$ is equal to
\\[-0.3em]\makebox[\textwidth][r]{\textit{[GATE EE 2025]}}

\begin{enumerate}[label=(\Alph*)]
\item $3 - 2\sqrt{2}$
\item $3 + 2\sqrt{2}$
\item $2 - 3\sqrt{2}$
\item $2 + 3\sqrt{2}$
\end{enumerate}

\item Which one of the following statements holds?
\\[-0.3em]\makebox[\textwidth][r]{\textit{[GATE EE 2025]}}

\begin{enumerate}[label=(\Alph*)]
\item The series $\sum_{n=0}^\infty x^n$ converges for each $x \in \sbrak{-1,1}$
\item The series $\sum_{n=0}^\infty x^n$ converges uniformly in $\brak{-1,1}$
\item The series $\sum_{n=1}^\infty \frac{x^n}{n^2}$ converges for each $x \in \sbrak{-1,1}$
\item The series $\sum_{n=1}^\infty \frac{x^n}{n}$ converges uniformly in $\brak{-1,1}$
\end{enumerate}

% ----- 6th Page -----

\item For $x \in \sbrak{-\pi,\pi}$, let
\[
f(x) = (\pi + x)(\pi - x), \quad 
g(x) =
\begin{cases}
\cos\brak{\frac{1}{x}}, & x \neq 0, \\
0, & x = 0.
\end{cases}
\]
Consider the statements:

P: The Fourier series of $f$ converges uniformly to $f$ on $\sbrak{-\pi,\pi}$.  
Q: The Fourier series of $g$ converges uniformly to $g$ on $\sbrak{-\pi,\pi}$.

Then
\\[-0.3em]\makebox[\textwidth][r]{\textit{[GATE EE 2025]}}

\begin{enumerate}[label=(\Alph*)]
\item P and Q are true
\item P is true but Q is false
\item P is false but Q is true
\item Both P and Q are false
\end{enumerate}

\item Let $W = \cbrak{(x,y,z) \in \mathbb{R}^3 : 1 \le x^2 + y^2 + z^2 \le 4}$  
and $F:W \to \mathbb{R}^3$ be defined by
\[
F(x,y,z) = \frac{(x,y,z)}{\sbrak{x^2 + y^2 + z^2}^{3/2}}
\]
for $(x,y,z) \in W$. If $\partial W$ denotes the boundary of $W$ oriented by the outward normal $n$ to $W$, then
\[
\iint_{\partial W} F \cdot n \, dS
\]
is equal to
\\[-0.3em]\makebox[\textwidth][r]{\textit{[GATE EE 2025]}}

\begin{enumerate}[label=(\Alph*)]
\item $0$
\item $4\pi$
\item $8\pi$
\item $12\pi$
\end{enumerate}

\item For each $n \in \mathbb{N}$, let $f_n : \sbrak{0,1} \to \mathbb{R}$ be a measurable function such that  
$\abs{f_n(t)} \le \frac{1}{\sqrt{t}}$ for all $t \in (0,1]$.  
Let $f : \sbrak{0,1} \to \mathbb{R}$ be defined by  
$f(t) = 1$ if $t$ is irrational and $f(t) = -1$ if $t$ is rational.  
Assume that $f_n(t) \to f(t)$ as $n \to \infty$ for all $t \in \sbrak{0,1}$.  
Then
\\[-0.3em]\makebox[\textwidth][r]{\textit{[GATE EE 2025]}}

\begin{enumerate}[label=(\Alph*)]
\item $f$ is not measurable
\item $f$ is measurable and $\int_{\sbrak{0,1}} f_n \, d\mu \to 1$ as $n \to \infty$
\item $f$ is measurable and $\int_{\sbrak{0,1}} f_n \, d\mu \to 0$ as $n \to \infty$
\item $f$ is measurable and $\int_{\sbrak{0,1}} f_n \, d\mu \to -1$ as $n \to \infty$
\end{enumerate}

\item Let $y_1$ and $y_2$ be two linearly independent solutions of
\[
y'' + (\sin x) y = 0, \quad 0 \le x \le 1.
\]
Let $g(x) = W(y_1,y_2)(x)$ be the Wronskian of $y_1$ and $y_2$. Then
\\[-0.3em]\makebox[\textwidth][r]{\textit{[GATE EE 2025]}}

\begin{enumerate}[label=(\Alph*)]
\item $g' > 0$ on $\sbrak{0,1}$
\item $g' < 0$ on $\sbrak{0,1}$
\item $g'$ vanishes at only one point of $\sbrak{0,1}$
\item $g'$ vanishes at all points of $\sbrak{0,1}$
\end{enumerate}

\item One particular solution of $y''' - y'' + y' + y = e^x$ is a constant multiple of
\\[-0.3em]\makebox[\textwidth][r]{\textit{[GATE EE 2025]}}

\begin{enumerate}[label=(\Alph*)]
\item $x e^x$
\item $x e^{-x}$
\item $x^2 e^x$
\item $x^2 e^{-x}$
\end{enumerate}
% ----- 7th Page -----

\item Let $a,b \in \mathbb{R}$. Let $y = (y_1, y_2)^\top$ be a solution of the system
\[
y_1' = y_2, \quad y_2' = a y_1 + b y_2.
\]
Every solution $y(x) \to 0$ as $x \to \infty$ if
\\[-0.3em]\makebox[\textwidth][r]{\textit{[GATE EE 2025]}}

\begin{enumerate}[label=(\Alph*)]
\item $a < 0, \ b < 0$
\item $a < 0, \ b > 0$
\item $a > 0, \ b > 0$
\item $a > 0, \ b < 0$
\end{enumerate}

\item Let $G$ be a group of order $45$. Let $H$ be a $3$-Sylow subgroup of $G$ and $K$ be a $5$-Sylow subgroup of $G$. Then
\\[-0.3em]\makebox[\textwidth][r]{\textit{[GATE EE 2025]}}

\begin{enumerate}[label=(\Alph*)]
\item Both $H$ and $K$ are normal in $G$
\item $H$ is normal in $G$ but $K$ is not normal in $G$
\item $H$ is not normal in $G$ but $K$ is normal in $G$
\item Both $H$ and $K$ are not normal in $G$
\end{enumerate}

\item The ring $\mathbb{Z}[\sqrt{-11}]$ is
\\[-0.3em]\makebox[\textwidth][r]{\textit{[GATE EE 2025]}}

\begin{enumerate}[label=(\Alph*)]
\item A Euclidean Domain
\item A Principal Ideal Domain, but not a Euclidean Domain
\item A Unique Factorization Domain, but not a Principal Ideal Domain
\item Not a Unique Factorization Domain
\end{enumerate}

\item Let $R$ be a Principal Ideal Domain and $a,b$ be any two non-unit elements of $R$. Then the ideal generated by $a$ and $b$ is also generated by
\\[-0.3em]\makebox[\textwidth][r]{\textit{[GATE EE 2025]}}

\begin{enumerate}[label=(\Alph*)]
\item $a + b$
\item $ab$
\item $\gcd(a,b)$
\item $\mathrm{lcm}(a,b)$
\end{enumerate}

\item Consider the action of $S_4$, the symmetric group of order $4$, on $\mathbb{Z}[x_1,x_2,x_3,x_4]$ given by
\[
\sigma \cdot p(x_1,x_2,x_3,x_4) = p\brak{x_{\sigma^{-1}(1)}, x_{\sigma^{-1}(2)}, x_{\sigma^{-1}(3)}, x_{\sigma^{-1}(4)}}
\]
for $\sigma \in S_4$.  

Let $H \subseteq S_4$ denote the cyclic subgroup generated by $(1\ 4\ 2\ 3)$. Then the cardinality of the orbit  
$O_H(x_1x_3 + x_2x_4)$ of $H$ on the polynomial $x_1x_3 + x_2x_4$ is
\\[-0.3em]\makebox[\textwidth][r]{\textit{[GATE EE 2025]}}

\begin{enumerate}[label=(\Alph*)]
\item $1$
\item $2$
\item $3$
\item $4$
\end{enumerate}

\item Let $f:\ell^2 \to \mathbb{R}$ be defined by
\[
f(x_1,x_2,\dots) = \sum_{n=1}^\infty \frac{x_n}{2^{n/2}}
\]
for $(x_1,x_2,\dots) \in \ell^2$. Then $\norm{f}$ is equal to
\\[-0.3em]\makebox[\textwidth][r]{\textit{[GATE EE 2025]}}

\begin{enumerate}[label=(\Alph*)]
\item $\frac{1}{2}$
\item $1$
\item $2$
\item $\frac{1}{\sqrt{2} - 1}$
\end{enumerate}

\item Consider $\mathbb{R}^3$ with norm $\norm{\cdot}$ and the linear transformation $T : \mathbb{R}^3 \to \mathbb{R}^3$ defined by the $3\times3$ matrix
\[
\myvec{
1 & 1 & 3 \\
2 & 2 & 2 \\
1 & 3 & -3
}
\]
Then the operator norm $\norm{T}$ of $T$ is equal to
\\[-0.3em]\makebox[\textwidth][r]{\textit{[GATE EE 2025]}}

\begin{enumerate}[label=(\Alph*)]
\item $6$
\item $7$
\item $8$
\item $\sqrt{42}$
\end{enumerate}
% ----- 8th Page -----

\item Consider $\mathbb{R}^2$ with norm $\norm{\cdot}$, and let 
\[
Y = \{(y_1,y_2) \in \mathbb{R}^2 : y_1 + y_2 = 0\}.
\]
If $g : Y \to \mathbb{R}$ is defined by $g(y_1,y_2) = y_2$ for $(y_1,y_2) \in Y$, then  
\\[-0.3em]\makebox[\textwidth][r]{\textit{[GATE EE 2025]}}

\begin{enumerate}[label=(\Alph*)]
\item $g$ has no Hahnanach extension to $\mathbb{R}^2$
\item $g$ has a unique HahnBanach extension to $\mathbb{R}^2$
\item Every linear functional $f : \mathbb{R}^2 \to \mathbb{R}$ satisfying $f(-1,1) = 1$ is a HahnBanach extension of $g$ to $\mathbb{R}^2$
\item The functionals $f_1, f_2 : \mathbb{R}^2 \to \mathbb{R}$ given by $f_1(x_1,x_2) = x_2$ and $f_2(x_1,x_2) = -x_1$ are both HahnBanach extensions of $g$ to $\mathbb{R}^2$
\end{enumerate}

\item Let $X$ be a Banach space and $Y$ be a normed linear space. Consider a sequence $(F_n)$ of bounded linear maps from $X$ to $Y$ such that for each fixed $x \in X$, the sequence $(F_n(x))$ is bounded in $Y$. Then
\\[-0.3em]\makebox[\textwidth][r]{\textit{[GATE EE 2025]}}

\begin{enumerate}[label=(\Alph*)]
\item For each fixed $x \in X$, the sequence $(F_n(x))$ is convergent in $Y$
\item For each fixed $n \in \mathbb{N}$, the set $\{F_n(x) : x \in X\}$ is bounded in $Y$
\item The sequence $(\norm{F_n})$ is bounded in $\mathbb{R}$
\item The sequence $(F_n)$ is uniformly bounded on $X$
\end{enumerate}

\item Let $H = L^2\sbrak{[0,\pi]}$ with the usual inner product. For $n \in \mathbb{N}$, let
\[
u_n(t) = \frac{\sqrt{2}}{\sqrt{\pi}} \sin nt, \quad t \in [0,\pi], 
\]
and $E = \{u_n : n \in \mathbb{N}\}$. Then
\\[-0.3em]\makebox[\textwidth][r]{\textit{[GATE EE 2025]}}

\begin{enumerate}[label=(\Alph*)]
\item $E$ is not a linearly independent subset of $H$
\item $E$ is a linearly independent subset of $H$, but is not an orthonormal subset of $H$
\item $E$ is an orthonormal subset of $H$, but is not an orthonormal basis for $H$
\item $E$ is an orthonormal basis for $H$
\end{enumerate}

\item Let $X = \mathbb{R}$ and let $\mathfrak{T} = \{U \subset X : X \setminus U \text{ is finite}\} \cup \{\phi, X\}$. The sequence
\[
1, \frac{1}{2}, \frac{1}{3}, \dots, \frac{1}{n}, \dots
\]
in $(X, \mathfrak{T})$  
\\[-0.3em]\makebox[\textwidth][r]{\textit{[GATE EE 2025]}}

\begin{enumerate}[label=(\Alph*)]
\item Converges to $0$ and not to any other point of $X$
\item Does not converge to $0$
\item Converges to each point of $X$
\item Is not convergent in $X$
\end{enumerate}

\item Let $E = \{(x,y) \in \mathbb{R}^2 : \abs{x} \leq 1, \ \abs{y} \leq 1\}$. Define $f : E \to \mathbb{R}$ by
\[
f(x,y) = \frac{x+y}{1+x^2+y^2}.
\]
Then the range of $f$ is a
\\[-0.3em]\makebox[\textwidth][r]{\textit{[GATE EE 2025]}}

\begin{enumerate}[label=(\Alph*)]
\item Connected open set
\item Connected closed set
\item Bounded open set
\item Closed and unbounded set
\end{enumerate}

\item Let $X = \{1,2,3\}$ and ${T} = \{\phi, \{1\}, \{2\}, \{1,2\}, \{2,3\}, \{1,2,3\}\}$.  
The topological space $\brak{X, {T}}$ is said to have the property $P$ if for any two proper disjoint closed subsets $Y$ and $Z$ of $X$, there exist disjoint open sets $U, V$ such that $Y \subseteq U$ and $Z \subseteq V$.  
Then the topological space $\brak{X, {T}}$  
\\[-0.3em]\makebox[\textwidth][r]{\textit{[GATE EE 2025]}}

\begin{enumerate}[label=(\Alph*)]
\item is $T_1$ and satisfies $P$
\item is $T_1$ and does not satisfy $P$
\item is not $T_1$ and satisfies $P$
\item is not $T_1$ and does not satisfy $P$
\end{enumerate}

\item Which one of the following subsets of $\mathbb{R}$ (with the usual metric) is NOT complete?  
\\[-0.3em]\makebox[\textwidth][r]{\textit{[GATE EE 2025]}}

\begin{enumerate}[label=(\Alph*)]
\item $\sbrak{1,2} \cup \sbrak{3,4}$
\item $\sbrak{0,\infty}$
\item $\sbrak{0,1}$
\item $\{0\} \cup \cbrak{\frac{1}{n} : n \in \mathbb{N}}$
\end{enumerate}

\item Consider the function  
\[
f(x) =
\begin{cases}
k\brak{x - \lfloor x \rfloor}, & 0 \le x < 2 \\
0, & \text{otherwise}
\end{cases}
\]
where $\lfloor x \rfloor$ is the integral part of $x$.  
The value of $k$ for which the above function is a probability density function of some random variable is  
\\[-0.3em]\makebox[\textwidth][r]{\textit{[GATE EE 2025]}}

\begin{enumerate}[label=(\Alph*)]
\item $\frac14$
\item $\frac12$
\item $1$
\item $2$
\end{enumerate}

\item For two random variables $X$ and $Y$, the regression lines are given by  
\[
Y = 5X - 15 \quad \text{and} \quad Y = 10X - 35
\]
Then the regression coefficient of $X$ on $Y$ is  
\\[-0.3em]\makebox[\textwidth][r]{\textit{[GATE EE 2025]}}

\begin{enumerate}[label=(\Alph*)]
\item $0.1$
\item $0.2$
\item $5$
\item $10$
\end{enumerate}

\item In an examination there are $80$ questions each having four choices. Exactly one of these four choices is correct and the other three are wrong. A student is awarded $1$ mark for each correct answer, and $-0.25$ for each wrong answer. If a student ticks the answer of each question randomly, then the expected value of his/her total marks in the examination is  
\\[-0.3em]\makebox[\textwidth][r]{\textit{[GATE EE 2025]}}

\begin{enumerate}[label=(\Alph*)]
\item $-15$
\item $0$
\item $5$
\item $20$
\end{enumerate}

\item Let $X_1, X_2, \dots, X_n$ be a random sample from the uniform distribution on $\sbrak{0, \theta}$. Then the maximum likelihood estimator (MLE) of $\theta$ based on the above random sample is  
\\[-0.3em]\makebox[\textwidth][r]{\textit{[GATE EE 2025]}}

\begin{enumerate}[label=(\Alph*)]
\item $\frac{2}{n} \sum_{i=1}^n X_i$
\item $\frac{1}{n} \sum_{i=1}^n X_i$
\item $\min \cbrak{X_1, X_2, \dots, X_n}$
\item $\max \cbrak{X_1, X_2, \dots, X_n}$
\end{enumerate}

\item The cost matrix of a transportation problem is given by  
\[
\myvec{
4 & 1 & 2 & 3 \\
3 & 2 & 3 & 2 \\
2 & 2 & 1 & 4
}
\]
The following are the values of variables in a feasible solution:  
\[
x_{12} = 6,\quad x_{13} = 2,\quad x_{24} = 6,\quad x_{31} = 4,\quad x_{33} = 6
\]
Then which of the following is correct?  
\\[-0.3em]\makebox[\textwidth][r]{\textit{[GATE EE 2025]}}

\begin{enumerate}[label=(\Alph*)]
\item The solution is degenerate and basic
\item The solution is non-degenerate and basic
\item The solution is degenerate and non-basic
\item The solution is non-degenerate and non-basic
\end{enumerate}

\item The maximum value of $z = 3x_1 - x_2$ subject to $2x_1 - x_2 \leq 1$, $x_1 \leq 3$ and $x_1, x_2 \geq 0$ is  
\\[-0.3em]\makebox[\textwidth][r]{\textit{[GATE EE 2025]}}

\begin{enumerate}[label=(\Alph*)]
\item $0$
\item $4$
\item $6$
\item $9$
\end{enumerate}

\item Consider the problem of maximizing $z = 2x_1 + 3x_2 - 4x_3 + x_4$ subject to  
\begin{align*}
x_1 + x_2 + x_3 &= 2, \\
x_2 + x_4 &= 3, \\
2x_1 + 3x_2 + 2x_5 - x_6 &= 0, \\
x_1, x_2, x_3, x_4 &\ge 0.
\end{align*}
Then  
\\[-0.3em]\makebox[\textwidth][r]{\textit{[GATE EE 2025]}}

\begin{enumerate}[label=(\Alph*)]
\item $(1,0,1,4)$ is a basic feasible solution but $(2,0,0,4)$ is not
\item $(1,0,1,4)$ is not a basic feasible solution but $(2,0,0,4)$ is
\item Neither $(1,0,1,4)$ nor $(2,0,0,4)$ are basic feasible solutions
\item Both $(1,0,1,4)$ and $(2,0,0,4)$ are basic feasible solutions
\end{enumerate}

\item In the closed system of a simple harmonic motion of a pendulum, let $H$ denote the Hamiltonian and $E$ be the total energy. Then  
\\[-0.3em]\makebox[\textwidth][r]{\textit{[GATE EE 2025]}}

\begin{enumerate}[label=(\Alph*)]
\item $H$ is a constant and $H = E$
\item $H$ is a constant but $H \ne E$
\item $H$ is not constant but $H = E$
\item $H$ is not constant and $H \ne E$
\end{enumerate}

\item The possible values of $\alpha$ for which the variational problem  
\[
J\big[ y(x) \big] = \int_0^1 \sbrak{ \brak{3y'}^2 + 2x^\alpha y^2 } \, dx, \quad y(\alpha) = 1
\]
has extremals are  
\\[-0.3em]\makebox[\textwidth][r]{\textit{[GATE EE 2025]}}

\begin{enumerate}[label=(\Alph*)]
\item $-1, 0$
\item $0, 1$
\item $-1, 1$
\item $-1, 0, 1$
\end{enumerate}

\item The functional  
\[
\int_0^1 \sbrak{ y'^2 + x^4 } \, dx, \quad y(1) = 1
\]
achieves its  
\\[-0.3em]\makebox[\textwidth][r]{\textit{[GATE EE 2025]}}

\begin{enumerate}[label=(\Alph*)]
\item Weak maximum on all its extremals
\item Weak minimum on all its extremals
\item Weak maximum on some, but not on all, of its extremals
\item Weak minimum on some, but not on all, of its extremals
\end{enumerate}

\end{enumerate}

\begin{enumerate}
\setcounter{enumi}{60}

\item The integral equation  
\[
x(t) = \sin t + \lambda \int_0^t \sbrak{ \brak{s^3 + e^{s^2}} x(s) } \, ds, \quad 0 \le t \le 1,\ \lambda \in \mathbb{R},\ \lambda \ne 0
\]
has a solution for  
\\[-0.3em]\makebox[\textwidth][r]{\textit{[GATE EE 2025]}}

\begin{multicols}{2}
\begin{enumerate}[label=(\Alph*)]
\item All non-zero values of $\lambda$
\item No value of $\lambda$
\item Only countably many positive values of $\lambda$
\item Only countably many negative values of $\lambda$
\end{enumerate}
\end{multicols}

\item The integral equation  
\[
x(t) - \int_0^t \sbrak{\cos t \ \sec s \ x(s)} \, ds = \sinh t, \quad 0 \le t \le 1
\]
has  
\\[-0.3em]\makebox[\textwidth][r]{\textit{[GATE EE 2025]}}

\begin{multicols}{2}
\begin{enumerate}[label=(\Alph*)]
\item No solution
\item A unique solution
\item More than one but finitely many solutions
\item Infinitely many solutions
\end{enumerate}
\end{multicols}

\item If $y_{i+1} = y_i + h\,\varphi\big( f, x_i, y_i, h \big),\ i = 1, 2, \dots$, where  
\[
\varphi(f, x, y, h) = a f(x, y) + b f\big( x + h, y + h f(x, y) \big),
\]
is a second-order accurate scheme to solve the initial value problem  
\[
\frac{dy}{dx} = f(x, y), \quad y(x_0) = y_0,
\]
then $a$ and $b$, respectively, are  
\\[-0.3em]\makebox[\textwidth][r]{\textit{[GATE EE 2025]}}

\begin{multicols}{2}
\begin{enumerate}[label=(\Alph*)]
\item $\frac{h}{2},\ \frac{h}{2}$
\item $1,\ -1$
\item $\frac{1}{2},\ \frac{1}{2}$
\item $h,\ -h$
\end{enumerate}
\end{multicols}

\item If a quadrature formula  
\[
\frac{3}{2} f\!\left( -\frac{1}{3} \right) + K\,f\!\left( \frac{1}{3} \right) + \frac{1}{2} f(1)
\]
that approximates $\int_0^1 f(x)\, dx$ is found to be exact for quadratic polynomials, then the value of $K$ is  
\\[-0.3em]\makebox[\textwidth][r]{\textit{[GATE EE 2025]}}

\begin{multicols}{2}
\begin{enumerate}[label=(\Alph*)]
\item $2$
\item $1$
\item $0$
\item $-1$
\end{enumerate}
\end{multicols}

\item If  
\[
\myvec{
1 & 4 & 3 \\
1 & 2 & 0 \\
5 & 8 & a
}
\myvec{
l_{11} & 0 & 0 \\
l_{21} & l_{22} & 0 \\
l_{31} & l_{32} & l_{33}
}
\myvec{
u_{11} & u_{12} & u_{13} \\
0 & u_{22} & u_{23} \\
0 & 0 & u_{33}
}
=
\myvec{
1 & * & * \\
0 & -53 & 0 \\
0 & 0 & 0
}
\]
then the value of $a$ is  
\\[-0.3em]\makebox[\textwidth][r]{\textit{[GATE EE 2025]}}

\begin{multicols}{2}
\begin{enumerate}[label=(\Alph*)]
\item $-2$
\item $-1$
\item $1$
\item $2$
\end{enumerate}
\end{multicols}

\item Using the least squares method, if a curve $y = ax^2 + bx + c$ is fitted to the collinear data points $(1, 2)$, $(1, 1)$, $(3, 5)$ and $(7, 13)$, then the triplet $(a, b, c)$ is equal to  
\\[-0.3em]\makebox[\textwidth][r]{\textit{[GATE EE 2025]}}

\begin{multicols}{2}
\begin{enumerate}[label=(\Alph*)]
\item $(-1, 2, 0)$
\item $(0, 2, -1)$
\item $(2, -1, 0)$
\item $(0, 1, 2)$
\end{enumerate}
\end{multicols}

\item A quadratic polynomial $p(x)$ is constructed by interpolating the data points $(0, 1)$, $(1, e)$ and $(2, e^2)$. If $\sqrt{e}$ is approximated by using $p(x)$, then its approximate value is  
\\[-0.3em]\makebox[\textwidth][r]{\textit{[GATE EE 2025]}}

\begin{multicols}{2}
\begin{enumerate}[label=(\Alph*)]
\item $\frac{1}{8} \big( 3 + 6e - e^2 \big)$
\item $\frac{1}{8} \big( 3 - 6e + 2e^2 \big)$
\item $\frac{1}{8} \big( 3 - 6e - e^2 \big)$
\item $\frac{1}{8} \big( 3 + 6e - 2e^2 \big)$
\end{enumerate}
\end{multicols}

\end{enumerate}
\begin{enumerate}
\setcounter{enumi}{67}

\item The characteristic curve of \( 2y u_{tt} + (2x + y^2) u_x = 0 \) passing through \((0,0)\) is
\\[-0.3em]\makebox[\textwidth][r]{\textit{[GATE EE 2025]}}

\begin{multicols}{2}
\begin{enumerate}[label=(\Alph*)]
\item \( y^2 = 2\left( e^x + x - 1 \right) \)
\item \( y^2 = 2\left( e^x - x + 1 \right) \)
\item \( y^2 = 2\left( e^{-x} - x - 1 \right) \)
\item \( y^2 = 2\left( e^{-x} + x + 1 \right) \)
\end{enumerate}
\end{multicols}

\item The initial value problem \( u_t + u_y = 1, \ u(x,s) = \sin s, \ 0 \le s \le 1 \), has
\\[-0.3em]\makebox[\textwidth][r]{\textit{[GATE EE 2025]}}

\begin{multicols}{2}
\begin{enumerate}[label=(\Alph*)]
\item Two solutions
\item A unique solution
\item No solution
\item Infinitely many solutions
\end{enumerate}
\end{multicols}

\item Let \( u(x,t) \) be the solution of \( u_{tt} - u_{xx} = 1, \ x \in \mathbb{R}, \ t > 0 \), with  
\( u(x,0) = 0, \ u_t(x,0) = 0, \ x \in \mathbb{R} \).  
Then \( u\left( \frac12, \frac12 \right) \) is equal to
\\[-0.3em]\makebox[\textwidth][r]{\textit{[GATE EE 2025]}}

\begin{multicols}{2}
\begin{enumerate}[label=(\Alph*)]
\item \( \frac18 \)
\item \( -\frac18 \)
\item \( \frac14 \)
\item \( -\frac14 \)
\end{enumerate}
\end{multicols}




Let \( X = C([0,1]) \) with sup norm \( \|\cdot\| \).

\item Let \( S = \cbrak{ x \in X : \|x\|_{\infty} \le 1 } \). Then
\\[-0.3em]\makebox[\textwidth][r]{\textit{[GATE EE 2025]}}

\begin{multicols}{2}
\begin{enumerate}[label=(\Alph*)]
\item \( S \) is convex and compact
\item \( S \) is not convex but compact
\item \( S \) is convex but not compact
\item \( S \) is neither convex nor compact
\end{enumerate}
\end{multicols}

\item Which one of the following is true?
\\[-0.3em]\makebox[\textwidth][r]{\textit{[GATE EE 2025]}}

\begin{multicols}{2}
\begin{enumerate}[label=(\Alph*)]
\item \( C^{\infty}([0,1]) \) is dense in \( X \)
\item \( X \) is dense in \( L^{\infty}([0,1]) \)
\item \( X \) has a countable basis
\item There is a sequence in \( X \) which is uniformly Cauchy on \([0,1]\) but does not converge uniformly on \([0,1]\)
\end{enumerate}
\end{multicols}

\item Let \( I = \cbrak{ x \in X : x(0) = 0 } \). Then
\\[-0.3em]\makebox[\textwidth][r]{\textit{[GATE EE 2025]}}

\begin{multicols}{2}
\begin{enumerate}[label=(\Alph*)]
\item \( I \) is not an ideal of \( X \)
\item \( I \) is an ideal, but not a prime ideal of \( X \)
\item \( I \) is a prime ideal, but not a maximal ideal of \( X \)
\item \( I \) is a maximal ideal of \( X \)
\end{enumerate}
\end{multicols}


\item Let \( X = C^1([0,1]) \) and \( Y = C([0,1]) \), both with the sup norm. Define \( F: X \to Y \) by \( F(x) = x + x' \) and  
\( f(x) = x(1) + x'(1) \) for \( x \in X \).

 Then
\\[-0.3em]\makebox[\textwidth][r]{\textit{[GATE EE 2025]}}

\begin{multicols}{2}
\begin{enumerate}[label=(\Alph*)]
\item \( F \) and \( f \) are continuous
\item \( F \) is continuous and \( f \) is discontinuous
\item \( F \) is discontinuous and \( f \) is continuous
\item \( F \) and \( f \) are discontinuous
\end{enumerate}
\end{multicols}





\item Then
\\[-0.3em]\makebox[\textwidth][r]{\textit{[GATE EE 2025]}}

\begin{multicols}{2}
\begin{enumerate}[label=(\Alph*)]
\item \(F\) and \(f\) are closed maps
\item \(F\) is a closed map and \(f\) is not a closed map
\item \(F\) is not a closed map and \(f\) is a closed map
\item Neither \(F\) nor \(f\) is a closed map
\end{enumerate}
\end{multicols}

\textbf{Linked Answer Questions: Q.76 to Q.85 carry two marks each.}


\item Let  
\[
N = \myvec{\frac{3}{5} & -\frac{4}{5} & 0 \\ \frac{4}{5} & \frac{3}{5} & 0 \\ 0 & 0 & 1}
\]

 Then \(N\) is
\\[-0.3em]\makebox[\textwidth][r]{\textit{[GATE EE 2025]}}

\begin{multicols}{2}
\begin{enumerate}[label=(\Alph*)]
\item Non-invertible
\item Skew-symmetric
\item Symmetric
\item Orthogonal
\end{enumerate}
\end{multicols}

\item If \(M\) is any \(3 \times 3\) real matrix, then \(\mathrm{trace}(NMN^T)\) is equal to
\\[-0.3em]\makebox[\textwidth][r]{\textit{[GATE EE 2025]}}

\begin{multicols}{2}
\begin{enumerate}[label=(\Alph*)]
\item \(\left[ \mathrm{trace}(N) \right]^2 \mathrm{trace}(M)\)
\item \(2\,\mathrm{trace}(N) + \mathrm{trace}(M)\)
\item \(\mathrm{trace}(M)\)
\item \(\left[ \mathrm{trace}(N) \right]^2 + \mathrm{trace}(M)\)
\end{enumerate}
\end{multicols}


\item Let \( f(z) = \frac{\cos z - \frac{\sin z}{z}}{\,}\) for non-zero \(z \in \mathbb{C}\) and \(f(0) = 0\).  
Also, let \( g(z) = \sinh z \) for \( z \in \mathbb{C} \).

Then \(f(z)\) has a zero at \(z=0\) of order
\\[-0.3em]\makebox[\textwidth][r]{\textit{[GATE EE 2025]}}

\begin{multicols}{2}
\begin{enumerate}[label=(\Alph*)]
\item \(0\)
\item \(1\)
\item \(2\)
\item Greater than \(2\)
\end{enumerate}
\end{multicols}

\item Then \(\frac{g(z)}{z f(z)}\) has a pole at \(z=0\) of order
\\[-0.3em]\makebox[\textwidth][r]{\textit{[GATE EE 2025]}}

\begin{multicols}{2}
\begin{enumerate}[label=(\Alph*)]
\item \(1\)
\item \(2\)
\item \(3\)
\item Greater than \(3\)
\end{enumerate}
\end{multicols}


\item Let \(n \geq 3\) be an integer. Let \(y\) be the polynomial solution of  
\[
(1 - x^2) y'' - 2x y' + n(n-1) y = 0
\]  
satisfying \(y(1) = 1\).

Then the degree of \(y\) is
\\[-0.3em]\makebox[\textwidth][r]{\textit{[GATE EE 2025]}}

\begin{multicols}{2}
\begin{enumerate}[label=(\Alph*)]
\item \(n\)
\item \(n-1\)
\item Less than \(n-1\)
\item Greater than \(n+1\)
\end{enumerate}
\end{multicols}

\item If \(I = \int_{-1}^{1} y(x)x^{n-3} \, dx\) and \(J = \int_{-1}^{1} y(x)x^n \, dx\), then
\\[-0.3em]\makebox[\textwidth][r]{\textit{[GATE EE 2025]}}

\begin{multicols}{2}
\begin{enumerate}[label=(\Alph*)]
\item \(I \neq 0, J \neq 0\)
\item \(I \neq 0, J = 0\)
\item \(I = 0, J \neq 0\)
\item \(I = 0, J = 0\)
\end{enumerate}
\end{multicols}



\textbf{Statement for Linked Answer Questions 82 \& 83:}  

 Consider the boundary value problem  
\[
u_{xx} + u_{yy} = 0,\quad x \in (0, \pi), \; y \in (0, \pi),
\]
\[
u(x,0) = u(x,\pi) = u(0,y) = 0.
\]

\item Any solution of this boundary value problem is of the form
\\[-0.3em]\makebox[\textwidth][r]{\textit{[GATE EE 2025]}}

\begin{multicols}{2}
\begin{enumerate}[label=(\Alph*)]
\item \(\sum_{n=1}^{\infty} a_n \sinh nx \sin ny\)
\item \(\sum_{n=1}^{\infty} a_n \cosh nx \sin ny\)
\item \(\sum_{n=1}^{\infty} a_n \sinh nx \cos ny\)
\item \(\sum_{n=1}^{\infty} a_n \cosh nx \cos ny\)
\end{enumerate}
\end{multicols}

\item If an additional boundary condition \(u_x(\pi, y) = \sin y\) is satisfied, then \(u(x, \pi/2)\) is equal to
\\[-0.3em]\makebox[\textwidth][r]{\textit{[GATE EE 2025]}}

\begin{multicols}{2}
\begin{enumerate}[label=(\Alph*)]
\item \(\frac{\pi}{2} \frac{(e^x - e^{-x})(e^x + e^{-x})}{\phantom{.}}\)
\item \(\frac{\pi (e^x + e^{-x})}{(e^x - e^{-x})}\)
\item \(\frac{\pi (e^x - e^{-x})}{(e^x + e^{-x})}\)
\item \(\frac{\pi}{2} (e^x + e^{-x})(e^x + e^{-x})\)
\end{enumerate}
\end{multicols}

\textbf{Statement for Linked Answer Questions 84 \& 85:}  
Let a random variable \(X\) follow the exponential distribution with mean \(2\). Define  
\[
Y = \left[ X - 2 \,\middle|\, X > 2 \right].
\]

\item The value of \(P(Y \ge t)\) is
\\[-0.3em]\makebox[\textwidth][r]{\textit{[GATE EE 2025]}}

\begin{multicols}{2}
\begin{enumerate}[label=(\Alph*)]
\item \(e^{-t/2}\)
\item \(e^{-2t}\)
\item \(\frac{1}{2} e^{-t/2}\)
\item \(\frac{1}{2} e^{-t}\)
\end{enumerate}
\end{multicols}

\item The value of \(E(Y)\) is equal to
\\[-0.3em]\makebox[\textwidth][r]{\textit{[GATE EE 2025]}}

\begin{multicols}{2}
\begin{enumerate}[label=(\Alph*)]
\item \(\frac{1}{4}\)
\item \(\frac{1}{2}\)
\item \(1\)
\item \(2\)
\end{enumerate}
\end{multicols}

\end{enumerate}







\end{document}