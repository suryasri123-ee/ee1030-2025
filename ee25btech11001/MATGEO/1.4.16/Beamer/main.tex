\documentclass{beamer}
\usepackage[utf8]{inputenc}

\usetheme{Madrid}
\usecolortheme{default}
\usepackage{amsmath,amssymb,amsfonts,amsthm}
\usepackage{txfonts}
\usepackage{tkz-euclide}
\usepackage{listings}
\usepackage{adjustbox}
\usepackage{array}
\usepackage{tabularx}
\usepackage{gvv}
\usepackage{lmodern}
\usepackage{circuitikz}
\usepackage{tikz}
\usepackage{graphicx}

\setbeamertemplate{page number in head/foot}[totalframenumber]

\title{1.4.16}
\date{August 29, 2025}
\author{EE25BTECH11001 - Aarush Dilawri}

\begin{document}

\frame{\titlepage}

% ---- Question ----
\begin{frame}{Question}
Find the coordinates of the points which trisect the line segment joining the points 
\[
\vec{P}(4,2,-6) \quad \text{and} \quad \vec{Q}(10,-16,6).
\]
\end{frame}

% ---- Vectors ----
\begin{frame}{Vectors}
Let the vectors be
\begin{align}
    \vec{P} &= \myvec{4\\2\\-6}, \\
    \vec{Q} &= \myvec{10\\-16\\6}.
\end{align}
We want to find the points which divide $PQ$ in the ratio $2:1$ and $1:2$.
\end{frame}

% ---- Section Formula ----
\begin{frame}{Section Formula}
\textbf{Section formula:}  
If a point divides the line joining $\vec{A}$ and $\vec{B}$ in the ratio $k:1$, then
\[
\vec{P} = \frac{k\vec{B} + \vec{A}}{k+1}.
\]
\end{frame}

% ---- First Trisection Point ----
\begin{frame}{First Trisection Point}
Using section formula for ratio $2:1$,
\begin{align}
    \vec{S} &= \frac{2\vec{Q} + \vec{P}}{3} \\
    &= \frac{\myvec{20\\-32\\12} + \myvec{4\\2\\-6}}{3} \\
    &= \frac{\myvec{24\\-30\\6}}{3} \\
    &= \myvec{8\\-10\\2}.
\end{align}
\end{frame}

% ---- Second Trisection Point ----
\begin{frame}{Second Trisection Point}
Using section formula for ratio $1:2$,
\begin{align}
    \vec{R} &= \frac{\vec{Q} + 2\vec{P}}{3} \\
    &= \frac{\myvec{10\\-16\\6} + \myvec{8\\4\\-12}}{3} \\
    &= \frac{\myvec{18\\-12\\-6}}{3} \\
    &= \myvec{6\\-4\\-2}.
\end{align}
\end{frame}

% ---- Final Answer ----
\begin{frame}{Final Answer}
Therefore, the points of trisection of $PQ$ are
\[
\vec{S} = \myvec{8\\-10\\2}, \quad \vec{R} = \myvec{6\\-4\\-2}.
\]
\end{frame}

% ---- Plot ----
\begin{frame}{Plot}
\centering
\includegraphics[width=0.8\linewidth]{figs/fig.png}
\end{frame}

\end{document}
