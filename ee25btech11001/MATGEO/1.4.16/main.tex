\let\negmedspace\undefined
\let\negthickspace\undefined
\documentclass[journal]{IEEEtran}
\usepackage[a5paper, margin=10mm, onecolumn]{geometry}
%\usepackage{lmodern} % Ensure lmodern is loaded for pdflatex
\usepackage{tfrupee} % Include tfrupee package

\setlength{\headheight}{1cm} % Set the height of the header box
\setlength{\headsep}{0mm}     % Set the distance between the header box and the top of the text

\usepackage{gvv-book}
\usepackage{gvv}
\usepackage{cite}
\usepackage{amsmath,amssymb,amsfonts,amsthm}
\usepackage{algorithmic}
\usepackage{graphicx}
\usepackage{textcomp}
\usepackage{xcolor}
\usepackage{txfonts}
\usepackage{listings}
\usepackage{enumitem}
\usepackage{mathtools}
\usepackage{gensymb}
\usepackage{comment}
\usepackage[breaklinks=true]{hyperref}
\usepackage{tkz-euclide} 
\usepackage{listings}
% \usepackage{gvv}                                        
\def\inputGnumericTable{}                                 
\usepackage[latin1]{inputenc}                                
\usepackage{color}                                            
\usepackage{array}                                            
\usepackage{longtable}                                       
\usepackage{calc}                                             
\usepackage{multirow}                                         
\usepackage{hhline}                                           
\usepackage{ifthen}                                           
\usepackage{lscape}
\begin{document}

\bibliographystyle{IEEEtran}
\vspace{3cm}

\title{1.4.16}
\author{EE25BTECH11001 - Aarush Dilawri}
% \maketitle
% \newpage
% \bigskip
{\let\newpage\relax\maketitle}

\renewcommand{\thefigure}{\theenumi}
\renewcommand{\thetable}{\theenumi}
\setlength{\intextsep}{10pt} % Space between text and floats
\textbf{Question}:\\
Find the coordinates of the points which trisect the line segment joining the points $\vec{P}(4,2,-6)$ and $\vec{Q}(10,-16,6)$.\\


\textbf{Solution}:\\
Let the vector $\vec{P}$ be 
\begin{align*}
    \vec{P}=\begin{myvec}{4\\2\\-6}\end{myvec} \;
\end{align*}

Let the vector $\vec{Q}$ be 
\begin{align*}
    \vec{Q}=\begin{myvec}{10\\-16\\6}\end{myvec} \;
\end{align*}

Using Section formula, we have to find the coordinates of the points which divide the line segment $PQ$ in the ratio $2:1$ and $1:2$.

Section formula for a vector $\vec{P}$ which divides the line formed by vectors $\vec{S}$ and $\vec{R}$ in the ratio k:1 is given by
\begin{align*}
    \vec{P}=\frac{k\vec{R}+\vec{S}}{k+1}
\end{align*}

Let the vector which divides $PQ$ in the ratio $2:1$ be $\vec{S}$ and the vector which divides $PQ$ in the ratio $1:2$ be $\vec{R}$.
Using section formula,\\
\begin{align*}
    \vec{S} = \frac{2\myvec{10\\-16\\6}+\myvec{4\\2\\-6}}{3}\\
    \implies \vec{S} = \frac{\myvec{20\\-32\\12}+\myvec{4\\2\\-6}}{3}\\
    \implies \vec{S} = \frac{\myvec{24\\-30\\6}}{3} 
    \implies \vec{S} = \myvec{8\\-10\\2}\\
\end{align*}
Similarly,
\begin{align*}
    \vec{R} = \frac{\myvec{10\\-16\\6}+2\myvec{4\\2\\-6}}{3}\\
    \implies \vec{R} = \frac{\myvec{10\\-16\\6}+\myvec{8\\4\\-12}}{3}\\
    \implies \vec{R} = \frac{\myvec{18\\-12\\-6}}{3} 
    \implies \vec{R} = \myvec{6\\-4\\-2}\\
\end{align*}

Therefore, the points of trisection of $PQ$ are\\
\begin{align*}
    \vec{S}=\myvec{8\\-10\\2} ,\vec{R}=\myvec{6\\-4\\-2}
\end{align*}

\bigskip

See Fig. 0 ,
\begin{figure}[H]
\begin{center}
\includegraphics[width=0.6\columnwidth]{figs/fig.png}
\end{center}
\caption{}
\label{fig:Fig1}
\end{figure}
\end{document}