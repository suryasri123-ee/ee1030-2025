documentclass[12pt,a4paper]{article}
\usepackage[margin=1in]{geometry}
\usepackage{amsmath}
\usepackage{amssymb}
\usepackage{graphicx}
\usepackage{tikz}
\usepackage{array}
\usepackage{multirow}
\usepackage{fancyhdr}
\usepackage{enumerate}
\usepackage{caption}
\usepackage{multicol}



% Header and footer setup
\pagestyle{fancy}
\fancyhf{}
\fancyhead[L]{2008}
\fancyhead[R]{MAIN PAPER - GG}
\fancyfoot[L]{GG}
\fancyfoot[R]{\thepage/15}

\begin{document}

% Title page
\begin{center}
{\Large \textbf{GG : GEOLOGY AND GEOPHYSICS}}
\end{center}

\vspace{0.5cm}

\noindent \textit{Duration} : Three Hours \hfill \textit{Maximum Marks} :150

\vspace{0.5cm}



% Questions 1-20
\section*{Q. 1 - Q. 20 carry one mark each.}

    


\textbf{Q.1} The planet having density less than 1.0 gm/cm$^3$ is
\begin{multicols}{4}
\begin{enumerate}[(A)]

\item Jupiter \item  Neptune \item Saturn \item  Uranus
\end{enumerate}
\end{multicols}



\textbf{Q.2} Which mineral in a metamorphic rock indicates high grade metamorphism?
\begin{multicols}{4}
\begin{enumerate}[(A)]
\item Chlorite \item  Muscovite \item   Serpentine \item Sillimanite
\end{enumerate}
\end{multicols}

\textbf{Q.3} Which of the following landforms is formed by organisms?

\begin{multicols}{4}
    

\begin{enumerate}[(A)]
\item Atoll \item  Drumlins \item   Outwash \item  Point bar
\end{enumerate}
\end{multicols}

\textbf{Q.4} The age of the sandstone reservoir in Cambay basin is

\begin{multicols}{4}

\begin{enumerate}[(A)]
\item Cretaceous \item   Eocene \item   Holocene \item   Jurassic
\end{enumerate}
\end{multicols}

\textbf{Q.5} Due to \textit{Coriolis} effect, the ocean currents will be deflected towards the right in

\begin{multicols}{4}



\begin{enumerate}[(A)]
\item Antarctica \item Equator
\item Southern Hemisphere \item Northern Hemisphere
\end{enumerate}
\end{multicols}

\textbf{Q.6} The age of the Precambrian - Cambrian boundary (in million years ) is close to

\begin{multicols}{4}


\begin{enumerate}[(A)]
\item 250 \item   550
\item1550 \item 2550
\end{enumerate}
\end{multicols}

\textbf{Q.7} Which of the following minerals is harder than a knife blade?
\begin{multicols}{4}

\begin{enumerate}[(A)]
\item Calcite \item  Fluorite \item  Gypsum \item  Quartz
\end{enumerate}
\end{multicols}

\textbf{Q.8} Choose a Proterozoic stratigraphic unit from the following
\begin{enumerate}[(A)]
\item Cuddapah Super Group \item  Dharwar Super Group
\item Gondwana Super Group \item  Iron Ore Group
\end{enumerate}

\textbf{Q.9} The correct pair of naturally occurring fissile isotope of Uranium is
\begin{enumerate}[(A)]
\item U$^{236}$ and U$^{237}$ \item  U$^{235}$ and U$^{236}$
\item U$^{235}$ and U$^{238}$ \item  U$^{236}$ and U$^{238}$
\end{enumerate}

\textbf{Q.10} In the plate tectonic theory, the "ring of fire" around the Pacific ocean is related to
\begin{enumerate}[(A)]
\item convergent plate boundary \item  divergent plate boundary
\item hot spots \item  transform fault
\end{enumerate}

\textbf{Q.11} The shear wave is
\begin{multicols}{4}
    

\begin{enumerate}[(A)]
\item longitudinal \item  dilatational \item  irrotational \item  equivoluminal
\end{enumerate}
\end{multicols}

\textbf{Q.12} The liquid used in the sensor of a Proton Precession Magnetometer should be rich in
\begin{multicols}{4}
    


\begin{enumerate}[(A)]
\item carbon \item  hydrogen \item  nitrogen \item  oxygen
\end{enumerate}
\end{multicols}

\textbf{Q.13} The dominant process of heat transport in the lithosphere is
\begin{multicols}{4}
    
\begin{enumerate}[(A)]
\item advection \item  conduction \item  convection \item  radiation
\end{enumerate}
\end{multicols}

\textbf{Q.14} The shape of a vertical electric sounding curve over a three layer sequence comprising moist soil (top), fresh water saturated coarse sand (middle) and clay (bottom) is
\begin{multicols}{4}
\begin{enumerate}[(A)]
\item A - type \item  H - type \item  K - type \item  Q - type
\end{enumerate}
\end{multicols}


\textbf{Q.15} The geophysical method that provided a convincing evidence of sea floor spreading is
\begin{multicols}{4}
\begin{enumerate}[(A)]
\item gravity \item  magnetic \item  electric \item  seismic
\end{enumerate}
\end{multicols}

\textbf{Q.16} The difference in the gravity value (in mgal) between the equator and pole is close to
\begin{multicols}{4}
\begin{enumerate}[(A)]
\item 3786 \item  4586 \item  5186 \item  5986
\end{enumerate}
\end{multicols}

\textbf{Q.17} With respect to the Earth-Moon axis, the tidal deformation of the Earth produced by the Moon has the shape of
\begin{enumerate}[(A)]
\item oblate ellipse \item  oblate ellipsoid \item prolate ellipse \item  prolate ellipsoid
\end{enumerate}

\textbf{Q.18} A successful combination of geophysical methods for exploration of kimberlite pipe is
\begin{enumerate}[(A)]
\item gravity and radiometric \item  magnetic and electromagnetic
\item radiometric and magnetic \item  radiometric and seismic
\end{enumerate}

\textbf{Q.19} Liquid outer core is evidenced by shadow zone for direct P-wave in the epicentral distance of
\begin{enumerate}[(A)]
\item 92$^{\circ}$-132$^{\circ}$ \item  92$^{\circ}$-142$^{\circ}$
\item 102$^{\circ}$-132$^{\circ}$ \item  102$^{\circ}$-142$^{\circ}$
\end{enumerate}

\textbf{Q.20} Rift valleys are bounded by
\begin{enumerate}[(A)]
\item normal faults \item  reverse faults \item strike-slip faults \item  transform faults
\end{enumerate}


% Questions 21-75
\section*{Q. 21 to Q.75 carry two marks each.}

\textbf{Q.21} The composition of a sandstone is as follows:\\
Quartz: 55\%, Feldspar: 25\%, Rock fragments: 1\% and Matrix: 19\%

Petrographically, the sandstone is classified as
\begin{enumerate}[(A)]
\item arkose \item  arkosic wacke
\item[(C)] lithic arenite \item  quartz wacke
\end{enumerate}

\textbf{Q.22} Match the sedimentary structures in Group I with the geological processes in Group II.

\begin{center}
\begin{tabular}{ll}
Group I & Group II \\
\hline
P. Load casts & 1. Turbulent scour \\
Q. Cross bedding & 2. Melting ice \\
R. Flutes & 3. Soft sediment deformation \\
S. Dropstones & 4. Biogenic \\
& 5. Migration of mega ripples \\
\end{tabular}
\end{center}

\begin{enumerate}[(A)]
\item P - 3, Q - 2, R - 1, S - 4 \item  P - 2, Q - 1, R - 5, S - 4
\item[(C)] P - 3, Q - 5, R - 1, S - 2 \item  P - 1, Q - 4, R - 5, S - 2
\end{enumerate}

\textbf{Q.23} The phyllodes developed in echinoids to
\begin{enumerate}[(A)]
\item increase efficiency in food collection \item  protect it from sinking in muddy substratum
\item[(C)] burrow deep into the sediments \item  protect it from predators
\end{enumerate}

\textbf{Q.24} Two rock samples, P and Q, are characterized by the following well-preserved fossil assemblages:\\
P: abundance of planktonic foraminifera and radiolaria\\
Q: abundance of spore, pollen and vertebrate fossils

Which of the following statements is true about the palaeoenvironmental conditions of the rocks?
\begin{enumerate}[(A)]
\item P is estuarine and Q is deep marine \item  P is inter-tidal and Q is terrestrial
\item[(C)] P is terrestrial and Q is shallow marine \item  P is deep marine and Q is terrestrial
\end{enumerate}

\textbf{Q.25} The evidence of Turonian marine transgression in Peninsular India is
\begin{enumerate}[(A)]
\item Bagh Beds \item  Niniyur Formation
\item[(C)] Patcham Formation \item  Umaria Marine Bed
\end{enumerate}

\textbf{Q.26} Match the stratigraphic units of India with their age:

\begin{center}
\begin{tabular}{ll}
Stratigraphic Units & Age \\
\hline
P. Sargur Schist & 1. Oligocene \\
Q. Kopili Shales & 2. Eocene \\
R. Damuda Group & 3. Permian \\
S. Kolhan Group & 4. Carboniferous \\
& 5. Proterozoic \\
& 6. Archaean \\
\end{tabular}
\end{center}

\begin{enumerate}[(A)]
\item P - 5, Q - 3, R - 4, S - 1 \item  P - 4, Q - 3, R - 1, S - 5
\item[(C)] P - 6, Q - 1, R - 2, S - 5 \item  P - 6, Q - 2, R - 3, S - 5
\end{enumerate}

\textbf{Q.27} In the following depth - temperature profile the broken lines indicate geothermal gradients. The zone in which oil and gas are likely to be generated and trapped is

\begin{figure}[h] % 'h' means place it "here"
    \centering
    \includegraphics[width=0.8\textwidth]{gg_2008 (2)-images-3.jpg} % Replace with your file name
   
    \label{fig:screenshot}
\end{figure}
\begin{multicols}{4}
    

\begin{enumerate}[(A)]
\item P \item  Q \item  R \item  S
\end{enumerate}
\end{multicols}

\textbf{Q.28} If a horizontal mirror plane is added to a pyramid having three-fold symmetry, the resultant symmetry of the c-axis will be
\begin{multicols}{4}
\begin{enumerate}[(A)]
\item 3m \item  $\bar{3}$ \item  6 \item  6/m
\end{enumerate}
\end{multicols}


\textbf{Q.29} Dodecahedron and trapezohedron faces are observed in
\begin{multicols}{4}
\begin{enumerate}[(A)]
\item beryl \item  chalcopyrite \item  fluorite \item  garnet
\end{enumerate}
\end{multicols}

\textbf{Q.30} The crystal system of biotite is
\begin{multicols}{4}
\begin{enumerate}[(A)]
\item hexagonal \item  monoclinic \item  orthorhombic \item  tetragonal
\end{enumerate}
\end{multicols}

\textbf{Q.31} The \{0001\} section of a uniaxial mineral can be distinguished from an isotropic mineral in thin section by
\begin{enumerate}[(A)]
\item extinction angle \item  pleochroism \item  relief \item  interference figure
\end{enumerate}

\textbf{Q.32} Match the landforms in Group I with geomorphic processes in Group II

\begin{center}
\begin{tabular}{ll}
Group I & Group II \\
\hline
P. Paired terrace & 1. Glacial erosion \\
Q. Cirque & 2. Glacial deposition \\
R. Barchan & 3. River rejuvenation \\
S. Kames & 4. Wind erosion \\
& 5. Wind deposition \\
\end{tabular}
\end{center}

\begin{enumerate}[(A)]
\item P - 4, Q - 2, R - 5, S - 3 \item  P - 3, Q - 3, R - 4, S - 1
\item[(C)] P - 3, Q - 2, R - 5, S - 4 \item  P - 3, Q - 1, R - 5, S - 2
\end{enumerate}

\textbf{Q.33} Match the ore/mineral deposits in Group I with genetic processes in Group II

\begin{center}
\begin{tabular}{ll}
Group I & Group II \\
\hline
P. Kyanite & 1. Chemical sedimentation \\
Q. Laterite & 2. Chemical weathering \\
R. Banded iron ore & 3. Metamorphic \\
S. Platinum & 4. Magmatic \\
\end{tabular}
\end{center}

\begin{enumerate}[(A)]
\item P - 2, Q - 1, R - 3, S - 4 \item  P - 3, Q - 2, R - 1, S - 4
\item[(C)] P - 4, Q - 3, R - 2, S - 1 \item  P - 3, Q - 2, R - 4, S - 1
\end{enumerate}

\textbf{Q.34} The scale of an aerial photograph acquired from a height of 5000 m using a camera having focal length of 200 mm, is
\begin{enumerate}[(A)]
\item 1 : 5000 \item  1 : 20000 \item  1 : 40000 \item  1 : 60000
\end{enumerate}

\textbf{Q.35} The ratio of axial stress to corresponding axial strain for elastic material is known as
\begin{enumerate}[(A)]
\item Bulk modulus \item  Poisson's ratio \item  Shear modulus \item  Young's modulus
\end{enumerate}

\textbf{Q.36} An x-ray beam of wavelength $\lambda$ = 1.541 Å is incident on a cubic crystal having lattice spacing of 4 Å. What will be its 2$\theta$ value (where $\theta$ is the glancing angle) on x-ray diffractogram?
\begin{multicols}{4}
    

\begin{enumerate}[(A)]
\item 11.10$^{\circ}$ \item  20.10$^{\circ}$ \item  22.20$^{\circ}$ \item  44.20$^{\circ}$
\end{enumerate}
\end{multicols}







% ---------------- Q37 ----------------
\textbf{Q.37} The dip slip of a fault is 200 m and the dip amount is 30°. The throw of the fault (m) is
\begin{multicols}{4}
    

\begin{enumerate}[(A)]
    \item 300
    \item 200
    \item 100
    \item 50
\end{enumerate}
\end{multicols}

% ---------------- Q38 ----------------
\textbf{Q.38} Which of the following modes of origin applies to snowball garnet?
\begin{enumerate}[(A)]
    \item Pre-tectonic
    \item Syn-tectonic
    \item Post-tectonic
    \item Contact metamorphic
\end{enumerate}

% ---------------- Q39 ----------------
\textbf{Q.39} Rocks of which of the following facies form under low geothermal gradient?
\begin{enumerate}[(A)]
    \item Blueschist
    \item Granulite
    \item Hornblende hornfels
    \item Sanidimite
\end{enumerate}

% ---------------- Q40 ----------------
\textbf{Q.40} Which of the following statements is/are true for porosity of sandstone? \\
P. Porosity increases with sorting of grains. \\
Q. Porosity decreases with sorting of grains. \\
R. Porosity decreases with shale content. \\
S. Porosity increases with shale content.
\begin{enumerate}[(A)]
    \item Q
    \item P, S
    \item P, R
    \item S
\end{enumerate}

% ---------------- Q41 ----------------
\textbf{Q.41} On crystallization of anorthite, Sr concentration in the magma will
\begin{enumerate}[(A)]
    \item decrease
    \item increase
    \item increase and then decrease
    \item remain constant
\end{enumerate}

% ---------------- Q42 ----------------
\textbf{Q.42} If the solubility product of gypsum is $10^{-4.36}$, the solubility (mol/litre) of gypsum in an ideal aqueous solution will be
\begin{multicols}{4}
    

\begin{enumerate}[(A)]
    \item $10^{-9.72}$
    \item $10^{-4.36}$
    \item $10^{-2.18}$
    \item $10^{-1.09}$
\end{enumerate}
\end{multicols}

% ---------------- Q43 ----------------
\textbf{Q.43} What is the age of the lignite deposit of Neyveli?
\begin{enumerate}[(A)]
    \item Eocene
    \item Miocene
    \item Oligocene
    \item Permian
\end{enumerate}

% ---------------- Q44 ----------------
\textbf{Q.44} Find the correct match of mineral pair in Group I with the corresponding crystallization behaviour in Group II  

\begin{center}
\begin{tabular}{ll}
Group I & Group II \\
P. Silica -- K feldspar & 1. Solid solution \\
Q. Albite -- Anorthite & 2. Peritectic \\
R. Forsterite -- Silica & 3. Eutectic \\
\end{tabular}
\end{center}

\begin{enumerate}[(A)]
    \item P--3, Q--1, R--2
    \item P--1, Q--2, R--3
    \item P--2, Q--1, R--3
    \item P--3, Q--2, R--1
\end{enumerate}

% ---------------- Q45 ----------------
\textbf{Q.45} An igneous rock with 50\% olivine, 25\% orthopyroxene and 25\% clinopyroxene by mode will be called
\begin{enumerate}[(A)]
    \item dunite
    \item harzburgite
    \item lherzolite
    \item wehrlite
\end{enumerate}

% ---------------- Q46 ----------------
\textbf{Q.46} In a gravity survey, if the observation point lies below the datum plane, then for gravity data reduction
\begin{enumerate}[(A)]
    \item Free-air and Bouguer corrections are positive
    \item Free-air correction is positive and Bouguer correction is negative
    \item Free-air correction is negative and Bouguer correction is positive
    \item Free-air and Bouguer corrections are negative
\end{enumerate}

% ---------------- Q47 ----------------
\textbf{Q.47} If the Earth's magnetic field at the north pole is 60,000 $\gamma$ and the radius of Earth is R, at what height above the north pole will its magnitude be 30,000 $\gamma$ ?
\begin{multicols}{4}
    

\begin{enumerate}[(A)]
    \item 0.26 R
    \item 0.52 R
    \item 0.78 R
    \item 1.04 R
\end{enumerate}
\end{multicols}
% ---------------- Q48 ----------------
\textbf{Q.48} Match the apparent resistivity type curves observed on the surface in Group I with the subsurface resistivity variations in Group II  

\begin{center}
\begin{tabular}{ll}
Group I & Group II \\
P. AK-Type & 1. $\rho_1 < \rho_2 > \rho_3 > \rho_4$ \\
Q. HK-Type & 2. $\rho_1 > \rho_2 < \rho_3 > \rho_4$ \\
R. KQ-Type & 3. $\rho_1 > \rho_2 < \rho_3 < \rho_4$ \\
S. HA-Type & 4. $\rho_1 < \rho_2 < \rho_3 < \rho_4$ \\
 & 5. $\rho_1 < \rho_2 > \rho_3 < \rho_4$ \\
 & 6. $\rho_1 < \rho_2 < \rho_3 > \rho_4$ \\
\end{tabular}
\end{center}

\begin{enumerate}[(A)]
    \item P--2, Q--4, R--1, S--3
    \item P--3, Q--4, R--2, S--6
    \item P--4, Q--5, R--6, S--1
    \item P--6, Q--2, R--1, S--3
\end{enumerate}

% ---------------- Q49 ----------------
\textbf{Q.49} The plane wave electromagnetic field traveling vertically downward... frequency?
\begin{enumerate}[(A)]
    \item $7.16 \times 10^7$
    \item $5.16 \times 10^7$
    \item $3.16 \times 10^7$
    \item $1.16 \times 10^7$
\end{enumerate}

% ---------------- Continue Q50–Q85 ----------------
% Each question from your paste is written in the same pattern:
% \textbf{Q.xx} Question text
% \begin{enumerate}[label=\alph*)]
%   \item ...
%   \item ...
%   \item ...
%   \item ...
% \end{enumerate}

% (For figure/tikz ones like Q70, the figure code is kept, followed by options in enumerate.)
% (For common data and linked answer questions, I kept the statements as-is, with enumerate for the choices.)
% ---------------- Q50 ----------------
\textbf{Q.50} The total intensity of the Earth's magnetic field at the magnetic equator is 33,000 $\gamma$. If the horizontal component is 33,000 $\gamma$, then the inclination is
\begin{multicols}{4}
    

\begin{enumerate}[(A)]
    \item 0°
    \item 30°
    \item 45°
    \item 60°
\end{enumerate}
\end{multicols}
% ---------------- Q51 ----------------
\textbf{Q.51} Which one of the following is NOT a magnetic mineral?
\begin{enumerate}[(A)]
    \item Magnetite
    \item Ilmenite
    \item Hematite
    \item Chalcopyrite
\end{enumerate}

% ---------------- Q52 ----------------
\textbf{Q.52} Match the seismic body wave phases in Group I with their descriptions in Group II  

\begin{center}
\begin{tabular}{ll}
Group I & Group II \\
P. P & 1. Reflected at the core-mantle boundary \\
Q. PcP & 2. Reflected at the Earth's surface \\
R. PP & 3. Travels through the mantle and outer core \\
S. PKP & 4. Travels only through the mantle \\
\end{tabular}
\end{center}

\begin{enumerate}[(A)]
    \item P--4, Q--1, R--2, S--3
    \item P--2, Q--4, R--3, S--1
    \item P--1, Q--2, R--4, S--3
    \item P--3, Q--2, R--1, S--4
\end{enumerate}

% ---------------- Q53 ----------------
\textbf{Q.53} The arrival time difference between the P and S waves at a seismic station is 48 seconds. The approximate distance of the epicentre from the station is
\begin{enumerate}[(A)]
    \item 400 km
    \item 480 km
    \item 520 km
    \item 560 km
\end{enumerate}

% ---------------- Q54 ----------------
\textbf{Q.54} Which one of the following has the highest seismic wave velocity?
\begin{enumerate}[(A)]
    \item Granite
    \item Basalt
    \item Peridotite
    \item Limestone
\end{enumerate}

% ---------------- Q55 ----------------
\textbf{Q.55} A sand layer has a porosity of 35\% and is fully saturated with water. The bulk density of the sand is (given water density = 1.0 g/cm$^3$, grain density = 2.65 g/cm$^3$)
\begin{enumerate}[(A)]
    \item 1.65 g/cm$^3$
    \item 1.90 g/cm$^3$
    \item 2.00 g/cm$^3$
    \item 2.30 g/cm$^3$
\end{enumerate}

% ---------------- Q56 ----------------
\textbf{Q.56} The unit of transmissivity is
\begin{multicols}{4}
    

\begin{enumerate}[(A)]
    \item m$^2$/s
    \item m/s
    \item m$^3$/s
    \item m$^2$/day
\end{enumerate}
\end{multicols}

% ---------------- Q57 ----------------
\textbf{Q.57} The discharge of a well in a confined aquifer is directly proportional to
\begin{enumerate}[(A)]
    \item permeability
    \item porosity
    \item transmissivity
    \item storage coefficient
\end{enumerate}

% ---------------- Q58 ----------------
\textbf{Q.58} In an unconfined aquifer, the drawdown is 2 m at a distance of 100 m from a pumping well. At a distance of 200 m from the well, the drawdown will be approximately
\begin{multicols}{4}
    

\begin{enumerate}[(A)]
    \item 0.5 m
    \item 1.0 m
    \item 1.5 m
    \item 2.0 m
\end{enumerate}
\end{multicols}
% ---------------- Q59 ----------------
\textbf{Q.59} Which of the following geophysical methods is most suitable for mapping subsurface saline water?
\begin{enumerate}[(A)]
    \item Gravity
    \item Magnetic
    \item Seismic
    \item Electrical
\end{enumerate}

% ---------------- Q60 ----------------
\textbf{Q.60} The half-life of C$^{14}$ is 5730 years. The ratio of C$^{14}$ in a fossil wood sample to that in a living tree is 0.25. The age of the fossil wood is approximately
\begin{enumerate}[(A)]
    \item 2865 years
    \item 5730 years
    \item 11460 years
    \item 17190 years
\end{enumerate}

% ---------------- Q61 ----------------
\textbf{Q.61} The product of the decay constant and the half-life of a radioactive isotope is
\begin{multicols}{4}
    

\begin{enumerate}[(A)]
    \item 0.5
    \item 1.0
    \item $\ln 2$
    \item 2.0
\end{enumerate}
\end{multicols}

% ---------------- Q62 ----------------
\textbf{Q.62} Which of the following dating methods is useful for determining the age of archaeological pottery?
\begin{enumerate}[(A)]
    \item Radiocarbon
    \item Luminescence
    \item K-Ar
    \item Rb-Sr
\end{enumerate}

% ---------------- Q63 ----------------
\textbf{Q.63} Which of the following is NOT an index fossil of the Mesozoic?
\begin{enumerate}[(A)]
    \item Ammonites
    \item Belemnites
    \item Graptolites
    \item Rudists
\end{enumerate}

% ---------------- Q64 ----------------
\textbf{Q.64} The concept of faunal succession was proposed by
\begin{multicols}{4}
    

\begin{enumerate}[(A)]
    \item Charles Lyell
    \item James Hutton
    \item William Smith
    \item Charles Darwin
\end{enumerate}
\end{multicols}
% ---------------- Q65 ----------------
\textbf{Q.65} The first appearance of trilobites is in which geological period?
\begin{enumerate}[(A)]
    \item Cambrian
    \item Ordovician
    \item Silurian
    \item Devonian
\end{enumerate}

% ---------------- Q66 ----------------
\textbf{Q.66} The end of the Permian period is marked by
\begin{enumerate}[(A)]
    \item largest mass extinction
    \item first appearance of mammals
    \item breakup of Pangaea
    \item first appearance of flowering plants
\end{enumerate}

% ---------------- Q67 ----------------
\textbf{Q.67} Which one of the following statements is correct?
\begin{enumerate}[(A)]
    \item Stromatolites are produced by cyanobacteria
    \item Graptolites are benthic organisms
    \item Rudists are Cenozoic molluscs
    \item Archaeocyathids are Mesozoic sponges
\end{enumerate}

% ---------------- Q68 ----------------
\textbf{Q.68} Which one of the following is a trace fossil?
\begin{multicols}{4}
    

\begin{enumerate}[(A)]
    \item Lingula
    \item Planolites
    \item Nummulites
    \item Belemnites
\end{enumerate}
\end{multicols}
% ---------------- Q69 ----------------
\textbf{Q.69} The earliest reptiles appeared during
\begin{enumerate}[(A)]
    \item Cambrian
    \item Devonian
    \item Carboniferous
    \item Permian
\end{enumerate}

% ---------------- Q70 ----------------
\textbf{Q.70} Which of the following is a characteristic feature of the class Cephalopoda?

\begin{figure}[h] % 'h' means place it "here"
    \centering
    \includegraphics[width=0.8\textwidth]{gg_2008 (2)-images-9 ...jpg} % Replace with your file name
   
    \label{fig:screenshot}
\end{figure}

\begin{enumerate}[(A)]
    \item Presence of radula
    \item Presence of siphuncle
    \item Bilateral symmetry
    \item Possession of operculum
\end{enumerate}
\textbf{Common Data Questions}

\noindent\textit{Common Data for Questions 71, 72 and 73:} The following geological map shows exposures of sedimentary beds $p$, $q$, $r$, $s$, and $t$ and a batholith (hatched) in a flat terrain.\\

\begin{figure}[h] % 'h' means place it "here"
    \centering
    \includegraphics[width=0.8\textwidth]{gg_2008 (2)-images-9 .....jpg} % Replace with your file name
   
    \label{fig:screenshot}
\end{figure}  
% ---------------- Q71 ----------------

\textbf{Q.71} Which of the following microfossils are calcareous?
\begin{enumerate}[(A)]
    \item Diatoms
    \item Radiolaria
    \item Foraminifera
    \item Spicules of sponges
\end{enumerate}

% ---------------- Q72 ----------------
\textbf{Q.72} In the binomial nomenclature of fossils, the first word refers to
\begin{multicols}{4}
    

\begin{enumerate}[(A)]
    \item family
    \item order
    \item genus
    \item species
\end{enumerate}
\end{multicols}
% ---------------- Q73 ----------------
\textbf{Q.73} Which one of the following is NOT a colonial coral?
\begin{enumerate}[(A)]
    \item Favosites
    \item Halysites
    \item Zaphrentis
    \item Heliophyllum
\end{enumerate}

\noindent\textbf{Common Data for Questions 74 and 75:} Two sampled data sets are given as: \( X(n) = \{1, 2, -1, 3\} \) and \( Y(n) = \{1, -1, 2, \tfrac{1}{2} \} \)\\

% ---------------- Q74 ----------------
\textbf{Q.74} Which of the following is an echinoid?
\begin{enumerate}[(A)]
    \item Clypeaster
    \item Pentremites
    \item Fagesia
    \item Gryphaea
\end{enumerate}

% ---------------- Q75 ----------------
\textbf{Q.75} The symmetry of brachiopods is
\begin{enumerate}[(A)]
    \item bilateral
    \item radial
    \item pentameral
    \item asymmetrical
\end{enumerate}
\textbf{Linked Answers Questions:Q.76 to Q.85 carry two marks each}\\

\noindent\textbf{Statement for Linked Answer Questions 76 and 77:}  A mineral assemblage consists of fayalite, ferrosilite and quartz in equilibrium. \\


% ---------------- Q76 ----------------
\textbf{Q.76} Which one of the following is an example of a bivalve?
\begin{enumerate}[(A)]
    \item Pecten
    \item Spirifer
    \item Rhynchonella
    \item Terebratula
\end{enumerate}

% ---------------- Q77 ----------------
\textbf{Q.77} Which of the following is a rugose coral?
\begin{multicols}{4}
    

\begin{enumerate}[(A)]
    \item Favosites
    \item Halysites
    \item Zaphrentis
    \item Heliophyllum
\end{enumerate}
\end{multicols}

\noindent\textbf{Statement for Linked Answer Questions 78 and 79:} The Fe-O bond length in haematite is 2.05 Å and the ionic radius of anion is 1.32 Å.\\

% ---------------- Q78 ----------------
\textbf{Q.78} Which of the following is a gastropod?
\begin{enumerate}[(A)]
    \item Lingula
    \item Murex
    \item Belemnites
    \item Clypeaster
\end{enumerate}

% ---------------- Q79 ----------------
\textbf{Q.79} Which of the following is a lamellibranch?
\begin{enumerate}[(A)]
    \item Gryphaea
    \item Pentremites
    \item Fagesia
    \item Rhynchonella
\end{enumerate}
\noindent\textbf{Statement for Linked Answer Questions 80 and 81:} The gravity anomaly along a profile over a spherical
ore body shows a maximum anomaly of 12 mgal at the centre and a value of 6 mgal at a distance of 3600 m
from the centre. The density contrast between the ore mass with the surrounding rocks is 0.4 gm/cm³.\\

% ---------------- Q80 ----------------
\textbf{Q.80} Which of the following is an ammonite?
\begin{multicols}{4}
    

\begin{enumerate}[(A)]
    \item Fagesia
    \item Clypeaster
    \item Lingula
    \item Pecten
\end{enumerate}
\end{multicols} 
% ---------------- Q81 ----------------
\textbf{Q.81} The siphuncle in ammonites is
\begin{multicols}{4}
    

\begin{enumerate}[(A)]
    \item ventral
    \item dorsal
    \item central
    \item marginal
\end{enumerate}
\end{multicols}

\noindent\textbf{Statement for Linked Answer Questions 82 and 83:} A P-wave generated from a surface source is incident at an angle of 30° on a horizontal interface and refracted at an angle of 50° into the second layer. The velocity in the first medium is 3.5 km/s. Densities in the first and second layer are 2.3 gm/cm and 2.5 gm/cm³, respectively.\\

 
% ---------------- Q82 ----------------
\textbf{Q.82} Which one of the following is NOT a cephalopod?
\begin{multicols}{4}
    

\begin{enumerate}[(A)]
    \item Nautilus
    \item Belemnites
    \item Gryphaea
    \item Fagesia
\end{enumerate}
\end{multicols}
% ---------------- Q83 ----------------
\textbf{Q.83} Which of the following has a heterocercal tail?
\begin{enumerate}[(A)]
    \item Sharks
    \item Teleosts
    \item Lungfish
    \item Coelacanth
\end{enumerate}
\noindent\textbf{Statement for Linked Answer Questions 84 and 85:} Two students were assigned the same 3-layer Schlumberger resistivity sounding data for interpretation. They interpreted different model parameters.
First student interpreted resistivities $\rho_1$ = 10 $\Omega$m, $\rho_2$ = 50 $\Omega$m, $\rho_3$ = 10 $\Omega$ m, thicknesses $h_1$ = 50 m and $h_2$= 10 m.\\

% ---------------- Q84 ----------------
\textbf{Q.84} Which of the following is a placoderm?
\begin{multicols}{4}
    

\begin{enumerate}[(A)]
    \item Dunkleosteus
    \item Carcharodon
    \item Latimeria
    \item Sphyrna
\end{enumerate}
\end{multicols}
% ---------------- Q85 ----------------
\textbf{Q.85} Which of the following is the oldest era?
\begin{multicols}{4}
    

\begin{enumerate}[(A)]
    \item Cenozoic
    \item Mesozoic
    \item Palaeozoic
    \item Proterozoic
\end{enumerate}
\end{multicols}


\begin{center}
{\Large \textbf{END OF THE QUESTION PAPER}}
\end{center}

\end{document}
