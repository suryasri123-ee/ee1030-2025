\let\negmedspace\undefined
\let\negthickspace\undefined
\documentclass[journal]{IEEEtran}
\usepackage[a5paper, margin=10mm, onecolumn]{geometry}
\usepackage{lmodern} 
\usepackage{tfrupee} 
\setlength{\headheight}{1cm}
\setlength{\headsep}{0mm}   

\usepackage{gvv-book}
\usepackage{gvv}
\usepackage{cite}
\usepackage{amsmath,amssymb,amsfonts,amsthm}
\usepackage{algorithmic}
\usepackage{graphicx}
\usepackage{textcomp}
\usepackage{xcolor}
\usepackage{txfonts}
\usepackage{listings}
\usepackage{enumitem}
\usepackage{mathtools}
\usepackage{gensymb}
\usepackage{comment}
\usepackage[breaklinks=true]{hyperref}
\usepackage{tkz-euclide} 
\usepackage{listings}                             
\def\inputGnumericTable{}                                 
\usepackage[latin1]{inputenc}                                
\usepackage{color}                                            
\usepackage{array}                                            
\usepackage{longtable}                                       
\usepackage{calc}                                             
\usepackage{multirow}                                         
\usepackage{hhline}                                           
\usepackage{ifthen}                                           
\usepackage{lscape}
\usepackage{xparse}

\bibliographystyle{IEEEtran}

\title{1.6.9}
\author{EE25BTECH11059 - Vaishnavi Ramkrishna Anantheertha}

\begin{document}
\maketitle

\renewcommand{\thefigure}{\theenumi}
\renewcommand{\thetable}{\theenumi}

\numberwithin{equation}{enumi}
\numberwithin{figure}{enumi} 

\textbf{Question}:\\
if three points \myvec{h \\ 0},\myvec{a \\ b},\myvec{0 \\ k} lie on a line,show that
\[
\frac{a}{h} + \frac{b}{k} = 1
\]
\\
\textbf{Solution: }
\begin{table}[H]    
  \centering
  \begin{center}
    \begin{tabular}{|c|c|} 
        \hline
            \textbf{Variable}  & \textbf{Formula} \\ 
        \hline
            $a$   & $a = \myvec{4 \\ -1 \\ 1}$ \\ 
        \hline
            $b$   &  $b = \myvec{2 \\ -2 \\ 1}$\\ 
        \hline
           \end{tabular}
\end{center}  

  \caption{Variables Used}
  \label{tab:1.6.9}
\end{table}
If the rank of the Collinearity matrix is $1$, then the points are collinear\\
   The Collinearity matrix is given by\\
\begin{align}
    \myvec{
    \vec{C}-\vec{A} & \vec{B}-\vec{A}
  }^T = \myvec{
    a-h & b
    \\
    -h & k
    }\\
  \xleftrightarrow[]{R_1 \rightarrow {\frac{R_1}{a-h}}}
 \myvec{
    1 & \frac{b}{a-h}
    \\
    -h & k
    }\\
     \xleftrightarrow[]{R_2 \rightarrow {\frac{R_2}{-h}}}
 \myvec{
     1 & \frac{b}{a-h}
    \\
     1 & \frac{-k}{h}
    }\\
       \xleftrightarrow[]{R_2 \rightarrow R_2- R_1}
 \myvec{
    1 & \frac{b}{a-h}
    \\
    0 & \frac{-k}{h}-\frac{b}{a-h}
      } 
\end{align}

since the rank of matrix=1
\begin{align}
    \frac{-k}{h} - \frac{b}{a - h} &= 0 \label{eq1} \\
    \Rightarrow -bh - ka + kh &= 0 \label{eq2} \\
    \text{(Dividing Eq.(0.6) by $kh$)} 
    \Rightarrow \frac{a}{h} + \frac{b}{k} &= 1 \label{eq3}
\end{align}
Refer to Fig

\begin{figure}[H]
\begin{center}
\includegraphics[width=0.6\columnwidth]{figs/graph1.png}
\end{center}
\caption{}
\label{fig:Fig}
\end{figure}
\end{document}  