\documentclass{beamer}
\usepackage[utf8]{inputenc}

\usetheme{Madrid}
\usecolortheme{default}
\usepackage{amsmath,amssymb,amsfonts,amsthm}
\usepackage{txfonts}
\usepackage{tkz-euclide}
\usepackage{listings}
\usepackage{adjustbox}
\usepackage{array}
\usepackage{tabularx}
\usepackage{gvv}
\usepackage{lmodern}
\usepackage{circuitikz}
\usepackage{tikz}
\usepackage{graphicx}
\usepackage{mathtools}
\setbeamertemplate{page number in head/foot}[totalframenumber]

\usepackage{tcolorbox}
\tcbuselibrary{minted,breakable,xparse,skins}



\definecolor{bg}{gray}{0.95}
\DeclareTCBListing{mintedbox}{O{}m!O{}}{%
  breakable=true,
  listing engine=minted,
  listing only,
  minted language=#2,
  minted style=default,
  minted options={%
    linenos,
    gobble=0,
    breaklines=true,
    breakafter=,,
    fontsize=\small,
    numbersep=8pt,
    #1},
  boxsep=0pt,
  left skip=0pt,
  right skip=0pt,
  left=25pt,
  right=0pt,
  top=3pt,
  bottom=3pt,
  arc=5pt,
  leftrule=0pt,
  rightrule=0pt,
  bottomrule=2pt,
  toprule=2pt,
  colback=bg,
  colframe=orange!70,
  enhanced,
  overlay={%
    \begin{tcbclipinterior}
    \fill[orange!20!white] (frame.south west) rectangle ([xshift=20pt]frame.north west);
    \end{tcbclipinterior}},
  #3,
}
\lstset{
    language=C,
    basicstyle=\ttfamily\small,
    keywordstyle=\color{blue},
    stringstyle=\color{orange},
    commentstyle=\color{green!60!black},
    numbers=left,
    numberstyle=\tiny\color{gray},
    breaklines=true,
    showstringspaces=false,
}
%This block of code defines the information to appear in the
%Title page
\title %optional
{1.6.9}
%\subtitle{A short story}

\author % (optional)
{Vaishnavi - EE25BTECH11059}



\begin{document}


\frame{\titlepage}
\begin{frame}{Question}
if three points \myvec{h \\ 0},\myvec{a \\ b},\myvec{0 \\ k} lie on a line,show that
\[
\frac{a}{h} + \frac{b}{k} = 1
\]
\\
 

\end{frame}
\begin{frame}{allowframebreaks}
\frametitle{Solution}
\begin{table}[H]    
  \centering
  \begin{center}
    \begin{tabular}{|c|c|} 
        \hline
            \textbf{Variable}  & \textbf{Formula} \\ 
        \hline
            $a$   & $a = \myvec{4 \\ -1 \\ 1}$ \\ 
        \hline
            $b$   &  $b = \myvec{2 \\ -2 \\ 1}$\\ 
        \hline
           \end{tabular}
\end{center}  

  \caption{Variables Used}
  \label{tab:1.6.9}
\end{table}

\end{frame}
\begin{frame}{Solutions}

If the rank of the Collinearity matrix is $1$, then the points are collinear\\
   The Collinearity matrix is given by\\
\begin{align}
    \myvec{
    \vec{C}-\vec{A} & \vec{B}-\vec{A}
  }^T = \myvec{
    a-h & b
    \\
    -h & k
    }\\
  \xleftrightarrow[]{R_1 \rightarrow {\frac{R_1}{a-h}}}
 \myvec{
    1 & \frac{b}{a-h}
    \\
    -h & k
    }\\
     \xleftrightarrow[]{R_2 \rightarrow {\frac{R_2}{-h}}}
 \myvec{
     1 & \frac{b}{a-h}
    \\
     1 & \frac{-k}{h}
    }\\
       \xleftrightarrow[]{R_2 \rightarrow R_2- R_1}
 \myvec{
    1 & \frac{b}{a-h}
    \\
    0 & \frac{-k}{h}-\frac{b}{a-h}
      } 
\end{align}

\end{frame}


\begin{frame}{Solution}
    
since the rank of matrix=1
\begin{align}
    \frac{-k}{h} - \frac{b}{a - h} &= 0 \label{eq1} \\
    \Rightarrow -bh - ka + kh &= 0 \label{eq2} \\
    \text{(Dividing Eq.(6) by $kh$)} 
    \Rightarrow \frac{a}{h} + \frac{b}{k} &= 1 \label{eq3}
\end{align}
\end{frame}

\begin{frame}{solution}
    \frametitle{Graph}
  Refer to Fig.

\begin{figure}[H]
\begin{center}
\includegraphics[width=0.6\columnwidth]{../figs/graph1.png}
\end{center}
\caption{}
\label{fig:Fig}
\end{figure}
\end{frame}

\begin{frame}[fragile]
    \frametitle{Python Code}
    \begin{lstlisting}
 # Plotting points A(1, -2, -8), B(5, 0, -2), and C(11, 3, 7)
 
import numpy as np
import matplotlib.pyplot as plt
from mpl_toolkits.mplot3d import Axes3D

# Define the points as numpy arrays
A = np.array([1, -2, -8])
B = np.array([5, 0, -2])
C = np.array([11, 3, 7])
\end{lstlisting}
\end{frame}

\begin{frame}[fragile]
    \frametitle{Python Code}

    \begin{lstlisting}
import matplotlib
import matplotlib.pyplot as plt
import numpy as np
from mpl_toolkits.mplot3d import Axes3D

# Use LaTeX-compatible font settings
matplotlib.use('pgf')
plt.rcParams.update({
    "text.usetex": True,
    "font.family": "lmodern",
    "font.size": 11,
})

# Define points A and B
h, k = 5, 7
A = np.array([h, 0, 0])
B = np.array([0, k, 0])

# Compute a third point C that lies on the line AB using parameter t
t = 2  # You can change this to move along the line
C = (1 - t) * A + t * B

# Stack points for plotting
points = np.vstack([A, B, C])


    \end{lstlisting}
\end{frame}

\begin{frame}[fragile]
    \frametitle{Python Code}

    \begin{lstlisting}
 # 3D Plot
fig = plt.figure(figsize=(8, 6))
ax = fig.add_subplot(111, projection='3d')

# Plot the points
ax.scatter(points[:,0], points[:,1], points[:,2], color=['red', 'green', 'blue'], s=100)
ax.plot(points[:,0], points[:,1], points[:,2], color='purple', label='Line through A, B, C')

# Annotate the points
ax.text(*A, ' A', color='red')
ax.text(*B, ' B', color='green')
ax.text(*C, ' C', color='blue')




    \end{lstlisting}
\end{frame}

\begin{frame}[fragile]
    \frametitle{Python Code}

    \begin{lstlisting}
# Axis labels
ax.set_xlabel('X-axis')
ax.set_ylabel('Y-axis')
ax.set_zlabel('Z-axis')
ax.set_title(r'Collinear Points $(h,0,0)$, $(0,k,0)$, and $(a,b,c)$')
ax.legend()
ax.grid(True)

# Save the figure
fig.savefig("collinear_3d_plot.png")



\end{lstlisting}
\end{frame}

 



 


\begin{frame}[fragile]
\frametitle{C Code}
\begin{lstlisting}
#include <stdio.h>
#include <stdbool.h>

// Function to check collinearity using the matrix method
bool check_collinearity_matrix(int h, int a, int b, int k) {
    if (h == 0 || k == 0) {
        printf(" Invalid input: h and k must be non-zero.\n");
        return false;
    }

    // Step 1: Construct the matrix from vector differences
    int row1_col1 = a - h;
    int row1_col2 = b;
    int row2_col1 = -h;
    int row2_col2 = k;

    printf("Collinearity matrix before row operations:\n");
    printf("[ %d\t%d ]\n", row1_col1, row1_col2);
    printf("[ %d\t%d ]\n", row2_col1, row2_col2);

    // Step 2: Simulate row operation R1 = R1 - R2 (for illustration)
    int new_r1_col1 = row1_col1 - row2_col1; // a
    int new_r1_col2 = row1_col2 - row2_col2; // b - k
    \end{lstlisting}

\end{frame}

\begin{frame}[fragile]
\frametitle{C Code}
\begin{lstlisting}
    

    


    printf("\nAfter row operation R1 = R1 - R2:\n");
    printf("[ %d\t%d ]\n", new_r1_col1, new_r1_col2);
    printf("[ %d\t%d ]\n", row2_col1, row2_col2);

    // Step 3: Check the condition (a/h + b/k == 1) without floating point
    int lhs = a * k + b * h;
    int rhs = h * k;

    printf("\nChecking condition: (a/h + b/k == 1)\n");
    printf("Computed: (%d * %d + %d * %d) = %d\n", a, k, b, h, lhs);
    printf("Expected: (%d * %d) = %d\n", h, k, rhs);

    return (lhs == rhs);
   }

int main() {
    // Example input values (change as needed)
    int h = 5;
    int a = 2;
    int b = 3;
    int k = 4;
\end{lstlisting}
\end{frame}
\begin{frame}[fragile]
\frametitle{C Code}
\begin{lstlisting}
    



    printf("Checking collinearity for points:\n");
    printf("A = (%d, 0), B = (%d, %d), C = (0, %d)\n\n", h, a, b, k);

    if (check_collinearity_matrix(h, a, b, k)) {
        printf("\nPoints are collinear (Matrix rank = 1 and a/h + b/k = 1).\n");
    } else {
        printf("\n Points are NOT collinear (Condition fails).\n");
    }

    return 0;
}
\end{lstlisting}
    
\end{frame}



\begin{frame}[fragile]
\frametitle{Python and C Code}

\begin{lstlisting}
import subprocess

# Compile the C program
subprocess.run(["gcc", "points.c", "-o", "points"])

# Run the compiled C program
result = subprocess.run(["./points"], capture_output=True, text=True)

# Print the output from the C program (solution)
print(result.stdout) 

\end{lstlisting}

\end{frame}

 





\end{document}