\documentclass{beamer}
\mode<presentation>
\usepackage{amsmath}
\usepackage{amssymb}
\usepackage{bm}
%\usepackage{advdate}
\usepackage{adjustbox}
\usepackage{subcaption}
%\usepackage{enumitem}
\usepackage{enumerate}
\usepackage{multicol}
\usepackage{mathtools}
\usepackage{listings}
\usepackage{url}
\def\UrlBreaks{\do\/\do-}
\usetheme{Boadilla}
\usecolortheme{lily}
\setbeamertemplate{footline}
{
  \leavevmode%
  \hbox{%
  \begin{beamercolorbox}[wd=\paperwidth,ht=2.25ex,dp=1ex,right]{author in head/foot}%
    \insertframenumber{} / \inserttotalframenumber\hspace*{2ex} 
  \end{beamercolorbox}}%
  \vskip0pt%
}
\setbeamertemplate{navigation symbols}{}

\providecommand{\nCr}[2]{\,^{#1}C_{#2}} % nCr
\providecommand{\nPr}[2]{\,^{#1}P_{#2}} % nPr
\providecommand{\mbf}{\mathbf}
\providecommand{\pr}[1]{\ensuremath{\Pr\left(#1\right)}}
\providecommand{\qfunc}[1]{\ensuremath{Q\left(#1\right)}}
\providecommand{\sbrak}[1]{\ensuremath{{}\left[#1\right]}}
\providecommand{\lsbrak}[1]{\ensuremath{{}\left[#1\right.}}
\providecommand{\rsbrak}[1]{\ensuremath{{}\left.#1\right]}}
\providecommand{\brak}[1]{\ensuremath{\left(#1\right)}}
\providecommand{\lbrak}[1]{\ensuremath{\left(#1\right.}}
\providecommand{\rbrak}[1]{\ensuremath{\left.#1\right)}}
\providecommand{\cbrak}[1]{\ensuremath{\left\{#1\right\}}}
\providecommand{\lcbrak}[1]{\ensuremath{\left\{#1\right.}}
\providecommand{\rcbrak}[1]{\ensuremath{\left.#1\right\}}}
\providecommand{\rank}{\text{rank}}
\theoremstyle{remark}
\newtheorem{rem}{Remark}
\newcommand{\sgn}{\mathop{\mathrm{sgn}}}
\providecommand{\abs}[1]{\ensuremath{\left\vert #1 \right\vert}}
\providecommand{\res}[1]{\Res\displaylimits_{#1}} 
\providecommand{\norm}[1]{\lVert#1\rVert}
\providecommand{\mtx}[1]{\mathbf{#1}}
\providecommand{\mean}[1]{\overline{#1}}
\providecommand{\fourier}{\overset{\mathcal{F}}{ \rightleftharpoons}}
%\providecommand{\hilbert}{\overset{\mathcal{H}}{ \rightleftharpoons}}
\providecommand{\system}{\overset{\mathcal{H}}{ \longleftrightarrow}}
	%\newcommand{\solution}[2]{\vec{Solution:}{#1}}
%\newcommand{\solution}{\noindent \vec{Solution: }}
\providecommand{\dec}[2]{\ensuremath{\overset{#1}{\underset{#2}{\gtrless}}}}
\newcommand{\myvec}[1]{\ensuremath{\begin{pmatrix}#1\end{pmatrix}}}
\newenvironment{amatrix}[1]{%
  \left(\begin{array}{@{}*{#1}{c}|c@{}}
}{%
  \end{array}\right)
}
\let\vec\mathbf

\lstset{
%language=C,
frame=single, 
breaklines=true,
columns=fullflexible
}
\usepackage{listings}
\lstset{
  language=Python,
  basicstyle=\ttfamily\small,
  numbers=left,
  numberstyle=\tiny,
  breaklines=true,
  frame=single
}
%\numberwithin{equation}{section}

\title{Matgeo-1.2.27}
\author{AI25BTECH11019-Menavath Sai Sanjana}

\date{}

\begin{document}

\begin{frame}
\titlepage
\end{frame}

\begin{frame}
\frametitle{Question}
Rain is falling vertically with a speed of (30 , \text{m/s}). A woman rides a bicycle with a speed of (10 , \text{m/s}) in the north to south direction. What is the direction in which she should hold her umbrella?
\end{frame}
%
\begin{frame}
\frametitle{Solution}
$
\text{Choose axes: } x\text{ (south, +) , } y\text{ (downward, +).}
$
\begin{align}
\overrightarrow{v}_r&= \myvec{0\\30}\quad\text{(rain velocity: }30\ \text{m/s}\text{ downward)}\\
\overrightarrow{v}_w &= \myvec{10\\0}\quad\text{(woman velocity: }10\ \text{m/s}\text{ south)}\\
\overrightarrow{v}_r/w &= \overrightarrow{v}_r - \overrightarrow{v}_w \\
&= \myvec{0\\30}-\myvec{10\\0} \\
&= \myvec{-10\\30}.
\end{align}
\end{frame}

\begin{frame}
\frametitle{Solution(continuation)}

\text{Horizontal component (north) }=10\ \text{m/s},\qquad \\
\text{Vertical component (down) }=30\ \text{m/s}.\qquad


\vspace{1cm}

$
\tan\theta=\frac{10}{30}=\frac{1}{3}\quad\Rightarrow\quad
\theta=\arctan\!\left(\tfrac{1}{3}\right)\approx18.43^\circ.
$
\vspace{1cm}

\noindent\textbf{Conclusion:} In her frame the rain comes from slightly ahead (from the south and above), so she should tilt the umbrella forward (toward the direction of motion, i.e. south) by $\theta=\arctan(1/3)\approx18.43^\circ$.

\end{frame}

\begin{frame}
\frametitle{Graphical Representation}

\begin{figure}[ht!]
    \centering
    \includegraphics[width=0.65\textwidth]{matgeo-1.2.27.jpeg}
    \caption{}
    \label{fig:1.2.27.jpg}
\end{figure}
\end{frame}

\end{document}

