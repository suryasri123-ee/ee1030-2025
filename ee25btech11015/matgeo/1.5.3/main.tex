\let\negmedspace\undefined
\let\negthickspace\undefined
\documentclass[journal]{IEEEtran}
\usepackage[a5paper, margin=10mm, onecolumn]{geometry}
%\usepackage{lmodern} % Ensure lmodern is loaded for pdflatex
\usepackage{tfrupee} % Include tfrupee package

\setlength{\headheight}{1cm} % Set the height of the header box
\setlength{\headsep}{0mm}     % Set the distance between the header box and the top of the text

\usepackage{gvv-book}
\usepackage{gvv}
\usepackage{cite}
\usepackage{amsmath,amssymb,amsfonts,amsthm}
\usepackage{algorithmic}
\usepackage{graphicx}
\usepackage{textcomp}
\usepackage{xcolor}
\usepackage{txfonts}
\usepackage{listings}
\usepackage{enumitem}
\usepackage{mathtools}
\usepackage{gensymb}
\usepackage{comment}
\usepackage[breaklinks=true]{hyperref}
\usepackage{tkz-euclide} 
\usepackage{listings}
% \usepackage{gvv}                                        
\def\inputGnumericTable{}                                 
\usepackage[latin1]{inputenc}                                
\usepackage{color}                                            
\usepackage{array}                                            
\usepackage{longtable}                                       
\usepackage{calc}                                             
\usepackage{multirow}                                         
\usepackage{hhline}                                           
\usepackage{ifthen}                                           
\usepackage{lscape}
\begin{document}

\bibliographystyle{IEEEtran}
\vspace{3cm}

\title{1.5.3}
\author{EE25BTECH11015 - Bhoomika V}
% \maketitle
% \newpage
% \bigskip
{\let\newpage\relax\maketitle}

\renewcommand{\thefigure}{\theenumi}
\renewcommand{\thetable}{\theenumi}
\setlength{\intextsep}{10pt} % Space between text and floats


\numberwithin{equation}{enumi}
\numberwithin{figure}{enumi}
\renewcommand{\thetable}{\theenumi}
\parindent 0px 
{Question :-} \\ 
In what ratio does the X axis divide line segment joining the points $\vec{A}$(3,6) and $\vec{B}$(-12,-3)? \\ \\
\solution \\
Let $\vec{A}$(3,6) , $\vec{B}$(-12,-3) and the point on X axis be $\vec{X}$(t,0)\\

\begin{table}[H]    
  \centering
  \begin{center}
    \begin{tabular}{|c|c|} 
        \hline
            \textbf{Variable}  & \textbf{Formula} \\ 
        \hline
            $a$   & $a = \myvec{4 \\ -1 \\ 1}$ \\ 
        \hline
            $b$   &  $b = \myvec{2 \\ -2 \\ 1}$\\ 
        \hline
           \end{tabular}
\end{center}  

  \caption{Vectors}
  \label{Answers}
\end{table}

Using the collinearity $\brak{rank}$ test, form the matrix with difference vectors:

\begin{align*}
(\vec{B}-\vec{A} \quad \vec{X}-\vec{A}) &= \myvec{-12-3 & t-3 \\ -3-6 & 0-6} \\
    &= \myvec{-15 & t-2 \\ -9 & -6}.
\end{align*}

The three points are collinear$\iff$this matrix has rank 1 $\brak{its\ rows\ are\ linearly\ dependent}$.\\
\begin{center}
 $R_2 \leftarrow 5R_2 - 3R_1 \implies  \myvec{-45 & 3t-9 \\ 0 & -3t-21 }.$
 \end{center}

 For rank $1$, the second row must be zero:
\begin{center}
$-3t-21= 0 \implies t =-7$
\end{center}

let $\vec{X}$ divide $\vec{A}$ and $\vec{B}$ in the ratio k:1 then \\
\begin{equation}
k = \frac{(\vec{A} - \vec{X})^T (\vec{X} - \vec{B})}{\|\vec{X} - \vec{B}\|^2}
\end{equation}

\begin{center}
$\implies k = 2$
\end{center}
\begin{figure}[H]
\begin{center}
\includegraphics[width=0.6\columnwidth]{Figs/Fig1.png}
\end{center}
\caption{}
\label{fig:Fig.1}
\end{figure}


\end{document}
