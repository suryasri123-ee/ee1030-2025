\documentclass{beamer}
\usepackage[utf8]{inputenc}

\usetheme{Madrid}
\usecolortheme{default}
\usepackage{amsmath,amssymb,amsfonts,amsthm}
\usepackage{txfonts}
\usepackage{tkz-euclide}
\usepackage{listings}
\usepackage{adjustbox}
\usepackage{array}
\usepackage{tabularx}
\usepackage{gvv}
\usepackage{lmodern}
\usepackage{circuitikz}
\usepackage{tikz}
\usepackage{graphicx}

\setbeamertemplate{page number in head/foot}[totalframenumber]

\usepackage{tcolorbox}
\tcbuselibrary{minted,breakable,xparse,skins}



\definecolor{bg}{gray}{0.95}
\DeclareTCBListing{mintedbox}{O{}m!O{}}{%
	breakable=true,
	listing engine=minted,
	listing only,
	minted language=#2,
	minted style=default,
	minted options={%
		linenos,
		gobble=0,
		breaklines=true,
		breakafter=,,
		fontsize=\small,
		numbersep=8pt,
		#1},
	boxsep=0pt,
	left skip=0pt,
	right skip=0pt,
	left=25pt,
	right=0pt,
	top=3pt,
	bottom=3pt,
	arc=5pt,
	leftrule=0pt,
	rightrule=0pt,
	bottomrule=2pt,
	toprule=2pt,
	colback=bg,
	colframe=orange!70,
	enhanced,
	overlay={%
		\begin{tcbclipinterior}
			\fill[orange!20!white] (frame.south west) rectangle ([xshift=20pt]frame.north west);
	\end{tcbclipinterior}},
	#3,
}
\lstset{
	language=C,
	basicstyle=\ttfamily\small,
	keywordstyle=\color{blue},
	stringstyle=\color{orange},
	commentstyle=\color{green!60!black},
	numbers=left,
	numberstyle=\tiny\color{gray},
	breaklines=true,
	showstringspaces=false,
}
%------------------------------------------------------------
%This block of code defines the information to appear in the
%Title page
\title %optional
{1.5.24}
\date{}
%\subtitle{A short story}

\author % (optional)
{M Chanakya Srinivas- EE25BTECH11036}



\begin{document}
	
		\frame{\titlepage}
	\begin{frame}{Question}\textbf{Question}:A line intersects the $Y$-axis and $X$-axis at the points
$P=(0,b)$ and $Q=(c,0)$ respectively. If $(2,-5)$ is the midpoint of
$\overline{PQ}$, then find the coordinates of $P$ and $Q$.
\end{frame}

\begin{frame}{Problem Statement}
A line intersects the $Y$-axis and $X$-axis at points 
\[
P = (0,b), \quad Q = (c,0).
\]
If $(2,-5)$ is the midpoint of $\overline{PQ}$, then find $P$ and $Q$ using a matrix method.
\end{frame}


\begin{frame}{Step 1: Representing the Points as Vectors}
\begin{align*}
\vec{P} &= \myvec{0 \\ b}, \\
\vec{Q} &= \myvec{c \\ 0}, \\
\vec{M} &= \myvec{2 \\ -5}.
\end{align*}

\begin{align*}
\text{Midpoint Formula: } 
\vec{M} &= \tfrac{1}{2}(\vec{P} + \vec{Q}).
\end{align*}
\end{frame}


\begin{frame}{Step 2: Applying the Midpoint Formula}
\begin{align*}
\myvec{2 \\ -5} &= \tfrac{1}{2}\left(\myvec{0 \\ b} + \myvec{c \\ 0}\right) \\[6pt]
&= \tfrac{1}{2}\myvec{c \\ b}
\end{align*}

\begin{align*}
\implies \myvec{4 \\ -10} = \myvec{c \\ b}.
\end{align*}
\end{frame}


\begin{frame}{Step 3: Writing as a Matrix System}
\begin{align*}
c &= 4, \quad b = -10.
\end{align*}

\begin{align*}
\underbrace{\myvec{1 & 0 \\ 0 & 1}}_{A}
\underbrace{\myvec{b \\ c}}_{\vec{x}}
&=
\underbrace{\myvec{-10 \\ 4}}_{\vec{B}}.
\end{align*}
\end{frame}


\begin{frame}{Step 4: Solving the Matrix Equation}
\begin{align*}
\vec{x} &= A^{-1}\vec{B} \\[6pt]
&= I \myvec{-10 \\ 4} \\[6pt]
&= \myvec{-10 \\ 4}.
\end{align*}

\begin{align*}
\therefore \; b = -10, \quad c = 4.
\end{align*}
\end{frame}


\begin{frame}{Final Answer}
\begin{align*}
P &= \myvec{0 \\ -10}, \\
Q &= \myvec{4 \\ 0}.
\end{align*}

\[
\boxed{P=(0,-10),\; Q=(4,0)}
\]
\end{frame}

\begin{frame}[fragile]
\frametitle{C Code - Section formula function}
\begin{lstlisting}
// File: solver.c
void findCoordinates(int mx, int my, int* c_ptr, int* b_ptr) {
    // Calculate c from the x-coordinate of the midpoint
    *c_ptr = 2 * mx;

    // Calculate b from the y-coordinate of the midpoint
    *b_ptr = 2 * my;
}
\end{lstlisting}
\end{frame}

\begin{frame}[fragile]
	\frametitle{Python Code through shared output}
	\begin{lstlisting}
import ctypes
import matplotlib.pyplot as plt
import numpy as np

# --- Part 1: Interfacing with the C Function ---

# Load the shared library
try:
    solver_lib = ctypes.CDLL('./1.5.24.so')
except OSError as e:
    print("Error loading shared library. Did you compile solver.c?")
    print(e)
    exit()

# Define the argument and return types for the C function
# void findCoordinates(int, int, int*, int*)
        \end{lstlisting}
        \end{frame}
        
\begin{frame}[fragile]
	\frametitle{Python Code through shared output}
	\begin{lstlisting}

solver_lib.findCoordinates.argtypes = [
    ctypes.c_int,
    ctypes.c_int,
    ctypes.POINTER(ctypes.c_int),
    ctypes.POINTER(ctypes.c_int)
]
solver_lib.findCoordinates.restype = None
\end{lstlisting}
\end{frame}
\begin{frame}[fragile]
	\frametitle{Python Code through shared output}
	\begin{lstlisting}
# Input: The midpoint coordinates
midpoint_x, midpoint_y = 2, -5

# Create C-type integer variables to store the output from the C function
c_val = ctypes.c_int()
b_val = ctypes.c_int()

# Call the C function, passing the addresses of our output variables
solver_lib.findCoordinates(
    ctypes.c_int(midpoint_x),
    ctypes.c_int(midpoint_y),
    ctypes.byref(c_val),
    ctypes.byref(b_val)
)
\end{lstlisting}
\end{frame}
\begin{frame}[fragile]
	\frametitle{Python Code through shared output}
	\begin{lstlisting}

# Extract the Python values from the C-type objects
c = c_val.value
b = b_val.value

# Define the coordinates of P and Q
P = (0, b)
Q = (c, 0)
M = (midpoint_x, midpoint_y)

print(f"Calculation complete.")
print(f"Coordinates of P = {P}")
print(f"Coordinates of Q = {Q}")
\end{lstlisting}
\end{frame}
\begin{frame}[fragile]
	\frametitle{Python Code through shared output}
	\begin{lstlisting}

# --- Part 2: Plotting the Result ---

# Create arrays for plotting the line
x_points = np.array([P[0], Q[0]])
y_points = np.array([P[1], Q[1]])

# Create the plot
plt.figure(figsize=(8, 7))
plt.plot(x_points, y_points, 'b-', label=f'Line through P and Q') # Line
plt.plot(P[0], P[1], 'go', markersize=10, label=f'P = {P}')       # Point P
plt.plot(Q[0], Q[1], 'ro', markersize=10, label=f'Q = {Q}')       # Point Q
plt.plot(M[0], M[1], 'm*', markersize=12, label=f'Midpoint = {M}')# Midpoint M
\end{lstlisting}
\end{frame}
\begin{frame}[fragile]
	\frametitle{Python Code through shared output}
	\begin{lstlisting}

# Annotate points with their coordinates
plt.text(P[0] + 0.2, P[1], f'P{P}', fontsize=12)
plt.text(Q[0] + 0.2, Q[1], f'Q{Q}', fontsize=12)
plt.text(M[0] + 0.2, M[1], f'M{M}', fontsize=12)

# Formatting the plot
plt.title('Line Intersecting X and Y Axes', fontsize=16)
plt.xlabel('X-axis', fontsize=12)
plt.ylabel('Y-axis', fontsize=12)
plt.axhline(0, color='black', linewidth=0.7) # X-axis
plt.axvline(0, color='black', linewidth=0.7) # Y-axis
plt.grid(True, linestyle='--', alpha=0.6)
plt.legend()
plt.axis('equal') # Ensure the scale is the same on both axes

# Show the plot
plt.show()
\end{lstlisting}
\end{frame}

\begin{frame}[fragile]
	\frametitle{Direct Python Code}
	\begin{lstlisting}
import sys                            # for path to external scripts
import numpy as np
import numpy.linalg as LA
import matplotlib.pyplot as plt
import matplotlib.image as mpimg

# local imports
from libs.line.funcs import *
from libs.triangle.funcs import *
from libs.conics.funcs import circ_gen

# if using termux
import subprocess
import shlex
# end if

# Midpoint given
M = np.array(([2, -5])).reshape(-1,1)
\end{lstlisting}
\end{frame}
\begin{frame}[fragile]
	\frametitle{Direct Python Code}
	\begin{lstlisting}
# Let P = (0,b), Q = (c,0)
# From midpoint formula: (c/2, b/2) = (2, -5)
c = 4
b = -10

# Coordinates
P = np.array(([0,b])).reshape(-1,1)
Q = np.array(([c,0])).reshape(-1,1)

# Generating line PQ
x_PQ = line_gen(P,Q)

# Plotting the line
plt.plot(x_PQ[0,:], x_PQ[1,:], label='$PQ$')

# Plot midpoint
plt.scatter(M[0,:], M[1,:], color='red', label='Midpoint M')
\end{lstlisting}
\end{frame}
\begin{frame}[fragile]
	\frametitle{Direct Python Code}
	\begin{lstlisting}
# Labeling the coordinates
coords = np.block([[P,Q,M]])
vert_labels = ['P','Q','M']
plt.scatter(coords[0,:], coords[1,:])
for i, txt in enumerate(vert_labels):
    plt.annotate(f'{txt}\n({coords[0,i]:.0f}, {coords[1,i]:.0f})',
                 (coords[0,i], coords[1,i]),
                 textcoords="offset points",
                 xytext=(20,-10),
                 ha='center')

# Axis styling
ax = plt.gca()
ax.spines['left'].set_visible(False)
ax.spines['right'].set_visible(False)
ax.spines['top'].set_visible(False)
ax.spines['bottom'].set_visible(False)
\end{lstlisting}
\end{frame}
\begin{frame}[fragile]
	\frametitle{Direct Python Code}
	\begin{lstlisting}
plt.legend(loc='best')
plt.grid()
plt.axis('equal')

outfile = 'chapters/10/7/2/2/figs/fig.pdf'
plt.savefig(outfile)

# Save figure as PNG
outfile = 'chapters/10/7/2/2/figs/fig.png'
plt.savefig(outfile, dpi=300)

# Open image depending on system
try:
    import platform
    import subprocess, shlex

    if "termux" in platform.platform().lower():   # Android Termux
        subprocess.run(shlex.split(f"termux-open {outfile}"))
\end{lstlisting}
\end{frame}
\begin{frame}[fragile]
	\frametitle{Direct Python Code}
	\begin{lstlisting}
    else:                                        # Linux desktop
        subprocess.run(shlex.split(f"xdg-open {outfile}"))
except Exception as e:
    print(f"Could not auto-open file. Saved at {outfile}")

#else
#plt.show()
\end{lstlisting}
\end{frame}
\begin{frame}{Plot by python using shared output from c}
	\begin{center}
		\includegraphics[width=0.8\columnwidth]{figs/Figure_1.png}
	\end{center}
\end{frame}

\begin{frame}{Plot by python only}
	\begin{center}
		\includegraphics[width=0.9\columnwidth]{figs/fig.png}
	\end{center}
\end{frame}
\end{document}