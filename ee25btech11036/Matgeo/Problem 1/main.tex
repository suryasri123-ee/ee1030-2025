
\let\negmedspace\undefined
\let\negthickspace\undefined
\documentclass[journal]{IEEEtran}
\usepackage[a5paper, margin=10mm, onecolumn]{geometry}
\usepackage{lmodern} % Ensure lmodern is loaded for pdflatex

\setlength{\headheight}{1cm} % Set the height of the header box
\setlength{\headsep}{0mm}     % Set the distance between the header box and the top of the text

\usepackage{gvv-book}
\usepackage{gvv}
\usepackage{cite}
\usepackage{amsmath,amssymb,amsfonts,amsthm}
\usepackage{algorithmic}
\usepackage{graphicx}
\graphicspath{{./figs/}}
\usepackage{textcomp}
\usepackage{xcolor}
\usepackage{txfonts}
\usepackage{listings}
\usepackage{enumitem}
\usepackage{mathtools}
\usepackage{gensymb}
\usepackage{comment}
\usepackage[breaklinks=true]{hyperref}
\usepackage{tkz-euclide} 
\usepackage{listings}
\usepackage{gvv}                                        
\def\inputGnumericTable{}                                 
\usepackage[latin1]{inputenc}                                
\usepackage{color}                                            
\usepackage{array}                                            
\usepackage{longtable}                                       
\usepackage{calc}                                             
\usepackage{multirow}                                         
\usepackage{hhline}                                           
\usepackage{ifthen}                                           
\usepackage{lscape}
\usepackage{circuitikz}
\tikzstyle{block} = [rectangle, draw, fill=blue!20, 
text width=4em, text centered, rounded corners, minimum height=3em]
\tikzstyle{sum} = [draw, fill=blue!10, circle, minimum size=1cm, node distance=1.5cm]
\tikzstyle{input} = [coordinate]
\tikzstyle{output} = [coordinate]


\begin{document}
	
	\bibliographystyle{IEEEtran}
	\vspace{3cm}
	
	\title{1.5.24}
	\author{EE25BTECH11036 - M Chanakya Srinivas}
	\maketitle
	% \newpage
	% \bigskip
	{\let\newpage\relax\maketitle}
	
	\renewcommand{\thefigure}{\theenumi}
	\renewcommand{\thetable}{\theenumi}
	\setlength{\intextsep}{10pt} % Space between text and floats
	
	
	\numberwithin{equation}{enumi}
	\numberwithin{figure}{enumi}
	\renewcommand{\thetable}{\theenumi}
\textbf{1.5.24}\quad
A line intersects the $Y$-axis and $X$-axis at the points
$P=(0,b)$ and $Q=(c,0)$ respectively. If $(2,-5)$ is the midpoint of
$\overline{PQ}$, then find the coordinates of $P$ and $Q$.


\textbf{Coordinates: } 
$P = (0,-10), \; Q = (4,0)$

\section*{Matrix Solution}

\begin{enumerate}
\item \textbf{Vector Midpoint Formula}

Let 
\[
\vec{P} = \myvec{0 \\ b}, \quad 
\vec{Q} = \myvec{c \\ 0}, \quad 
\vec{M} = \myvec{2 \\ -5}.
\]

The midpoint formula is
\begin{align*}
\vec{M} &= \tfrac{1}{2}(\vec{P} + \vec{Q}) \\[6pt]
\myvec{2 \\ -5} &= \tfrac{1}{2}\left(\myvec{0 \\ b} + \myvec{c \\ 0}\right) \\[6pt]
&= \tfrac{1}{2}\myvec{c \\ b}.
\end{align*}

Multiplying through by $2$:
\[
\myvec{4 \\ -10} = \myvec{c \\ b}.
\]

---

\item \textbf{System in Matrix Form ($A\vec{x} = \vec{B}$)}

This gives the system
\[
\begin{cases}
b = -10,\\
c = 4.
\end{cases}
\]

Equivalently,
\[
\underbrace{\myvec{1 & 0 \\ 0 & 1}}_{A}
\underbrace{\myvec{b \\ c}}_{\vec{x}}
=
\underbrace{\myvec{-10 \\ 4}}_{\vec{B}}.
\]

---

\item \textbf{Solving the Matrix Equation}

Since $A$ is the identity,
\[
\vec{x} = A^{-1}\vec{B} = I\vec{B} = \myvec{-10 \\ 4}.
\]

So,
\[
b = -10, \quad c = 4.
\]

---

\item \textbf{Final Coordinates}

\[
P = \myvec{0 \\ -10}, 
\qquad 
Q = \myvec{4 \\ 0}.
\]

\end{enumerate}

\begin{figure}[h!]
    \centering
    \includegraphics[width=0.6\columnwidth]{figs/Figure_1.png}
    \caption{Plot using Shared output}
    \label{fig:placeholder}
\end{figure}

\begin{figure}[h!]
    \centering
    \includegraphics[width=0.6\columnwidth]{figs/fig.png}
    \caption{Plot using Python}
    \label{fig:placeholder}
\end{figure}


\end{document}