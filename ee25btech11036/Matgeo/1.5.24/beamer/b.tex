\documentclass{beamer}
\usepackage[utf8]{inputenc}

\usetheme{Madrid}
\usecolortheme{default}
\usepackage{amsmath,amssymb,amsfonts,amsthm}
\usepackage{txfonts}
\usepackage{tkz-euclide}
\usepackage{listings}
\usepackage{adjustbox}
\usepackage{array}
\usepackage{tabularx}
\usepackage{gvv}
\usepackage{lmodern}
\usepackage{circuitikz}
\usepackage{tikz}
\usepackage{graphicx}

\setbeamertemplate{page number in head/foot}[totalframenumber]

\usepackage{tcolorbox}
\tcbuselibrary{minted,breakable,xparse,skins}



\definecolor{bg}{gray}{0.95}
\DeclareTCBListing{mintedbox}{O{}m!O{}}{%
	breakable=true,
	listing engine=minted,
	listing only,
	minted language=#2,
	minted style=default,
	minted options={%
		linenos,
		gobble=0,
		breaklines=true,
		breakafter=,,
		fontsize=\small,
		numbersep=8pt,
		#1},
	boxsep=0pt,
	left skip=0pt,
	right skip=0pt,
	left=25pt,
	right=0pt,
	top=3pt,
	bottom=3pt,
	arc=5pt,
	leftrule=0pt,
	rightrule=0pt,
	bottomrule=2pt,
	toprule=2pt,
	colback=bg,
	colframe=orange!70,
	enhanced,
	overlay={%
		\begin{tcbclipinterior}
			\fill[orange!20!white] (frame.south west) rectangle ([xshift=20pt]frame.north west);
	\end{tcbclipinterior}},
	#3,
}
\lstset{
	language=C,
	basicstyle=\ttfamily\small,
	keywordstyle=\color{blue},
	stringstyle=\color{orange},
	commentstyle=\color{green!60!black},
	numbers=left,
	numberstyle=\tiny\color{gray},
	breaklines=true,
	showstringspaces=false,
}
%------------------------------------------------------------
%This block of code defines the information to appear in the
%Title page
\title %optional
{1.5.24}
\date{}
%\subtitle{A short story}

\author % (optional)
{M Chanakya Srinivas- EE25BTECH11036}





\begin{document}

\begin{frame}
  \titlepage
\end{frame}



%-------------------------------------------------
\begin{frame}{Problem Statement}
A line intersects the $Y$-axis and $X$-axis at the points
\[
P = (0,b), \quad Q = (c,0)
\]
respectively. If $(2,-5)$ is the midpoint of $\overline{PQ}$, then find the coordinates of $P$ and $Q$.
\end{frame}

%-------------------------------------------------
\begin{frame}{Vector Representation}
We represent the points as column vectors:
\[
\vec P = \myvec{0 \\ b}, \quad 
\vec Q = \myvec{c \\ 0}, \quad
\vec M = \myvec{2 \\ -5}.
\]
\end{frame}

%-------------------------------------------------
\begin{frame}{(i) Rank/Collinearity Condition}
Since $P, Q, M$ are collinear,
\begin{align}
\operatorname{rank}\myvec{\vec P - \vec M \quad \vec Q - \vec M}^{\top} = 1.
\end{align}

\begin{align}
\myvec{\vec P - \vec M & \vec Q - \vec M}^{\top} 
&= \myvec{-2 & c-2 \\ b+5 & 5} \\
&\xrightarrow{R_2 \leftarrow -2R_2 - (b+5)R_1}
\myvec{-2 & c-2 \\ 0 & -10 - (b+5)(c-2)}.
\end{align}

For rank $=1$, the last entry must vanish:
\begin{align}
    (b+5)(c-2) = -10.
\end{align}
\end{frame}

%-------------------------------------------------
\begin{frame}{(ii) Midpoint Condition}
The midpoint of $P$ and $Q$ is:
\begin{align}
\vec M &= \frac{\vec P + \vec Q}{2} \\
\myvec{2 \\ -5} &= \frac{1}{2}\myvec{0 \\ b} + \frac{1}{2}\myvec{c \\ 0}
= \frac{1}{2}\myvec{c \\ b}.
\end{align}

Thus,
\begin{align}
\myvec{c \\ b} = \myvec{4 \\ -10}
\quad\Longrightarrow\quad c=4, \; b=-10.
\end{align}


\end{frame}

%-------------------------------------------------
\begin{frame}{Final Answer}
Therefore, the intercept points are:
\[
\vec P = \myvec{0 \\ -10}, 
\qquad 
\vec Q = \myvec{4 \\ 0}.
\]
\end{frame}


\begin{frame}{Illustration}
\centering
\includegraphics[width=0.65\textwidth]{figs/fig.png}

\bigskip
Plot shows \(P(0,-10)\), \(Q(4,0)\) and midpoint \(M(2,-5)\).
\end{frame}




\begin{frame}[fragile]
\frametitle{C Code - Section formula function}
\begin{lstlisting}
#include <stdio.h>

int main() {
 
    printf("Problem 1.5.24:\n");
    printf("A line intersects the Y-axis and X-axis at P=(0,b) and Q=(c,0).\n");
    printf("If (2,-5) is the midpoint of PQ, find P and Q.\n\n");

   
    printf("Step (i): Rank / Collinearity condition\n");
    printf("From collinearity, (b+5)(c-2) = -10\n\n");

 
    printf("Step (ii): Midpoint condition\n");
    int Mx = 2, My = -5;
    int c = 2 * Mx;   // c/2 = 2 -> c = 4
    int b = 2 * My;   // b/2 = -5 -> b = -10
    printf("Midpoint gives: c = %d, b = %d\n\n", c, b);
    \end{lstlisting}
\end{frame}

\begin{frame}[fragile]
\frametitle{C Code - Section formula function}
\begin{lstlisting}

    int Px = 0, Py = b;
    int Qx = c, Qy = 0;
    printf("Final Answer:\n");
    printf("P = (%d, %d)\n", Px, Py);
    printf("Q = (%d, %d)\n", Qx, Qy);

  
    int midx = (Px + Qx) / 2;
    int midy = (Py + Qy) / 2;
    printf("\nVerification:\n");
    printf("Midpoint of P and Q = (%d, %d)\n", midx, midy);

    return 0;
}
\end{lstlisting}
\end{frame}


 \begin{frame}[fragile]
	\frametitle{Python Code through shared output}
	\begin{lstlisting}       



import numpy as np
import matplotlib.pyplot as plt

print("Step 1: Rank condition (collinearity)")
print("Matrix formed from (P-M) and (Q-M):")
print("[[-2, c-2], [b+5, 5]]")

print("Row operation: R2 -> -2*R2 - (b+5)*R1")
print("=> [[-2, c-2], [0, -10 - (b+5)(c-2)]]")

print("For rank=1: (b+5)(c-2) = -10")
\end{lstlisting}
\end{frame}

% --- Step 2 ---
\begin{frame}[fragile]\frametitle{Python Code through shared output}
\begin{lstlisting}
print("Step 2: Midpoint condition")
Mx, My = 2, -5
print("M = ((0+c)/2, (b+0)/2) = (2,-5)")

c = 2*Mx
b = 2*My
print(f"From midpoint: c = {c}, b = {b}")
\end{lstlisting}
\end{frame}


\begin{frame}[fragile]\frametitle{Python Code through shared output}
\begin{lstlisting}
P = (0, b)
Q = (c, 0)

print("Final Answer:")
print(f"P = {P}")
print(f"Q = {Q}")
\end{lstlisting}
\end{frame}


\begin{frame}[fragile]\frametitle{Python Code through shared output}
\begin{lstlisting}
midpoint = ((P[0]+Q[0])/2, (P[1]+Q[1])/2)
print("Verification:")
print(f"Midpoint of P and Q = {midpoint}")
\end{lstlisting}
\end{frame}
\begin{frame}[fragile]\frametitle{Python Code through shared output}
\begin{lstlisting}
# --- Step 5: Plot the graph ---
plt.figure(figsize=(6,6))
plt.axhline(0, color='black', linewidth=0.8)  # X-axis
plt.axvline(0, color='black', linewidth=0.8)  # Y-axis

# Plot line PQ
plt.plot([P[0], Q[0]], [P[1], Q[1]], 'b-', label="Line PQ")

# Mark points
plt.scatter(*P, color='red')
plt.scatter(*Q, color='green')
plt.scatter(*M, color='purple')

plt.text(P[0]-0.5, P[1], f"P({int(P[0])},{int(P[1])})", fontsize=10, color="red")
plt.text(Q[0]+0.2, Q[1], f"Q({int(Q[0])},{int(Q[1])})", fontsize=10, color="green")
plt.text(M[0]+0.2, M[1], f"M({int(M[0])},{int(M[1])})", fontsize=10, color="purple")
\end{lstlisting}
\end{frame}
\begin{frame}[fragile]\frametitle{Python Code through shared output}
\begin{lstlisting}
plt.title("Line PQ with Midpoint M(2,-5)")
plt.grid(True)
plt.legend()
plt.axis("equal")

plt.show()
\end{lstlisting}
\end{frame}
\begin{frame}[fragile]
	\frametitle{Direct Python Code}
	\begin{lstlisting}
import sys                            # for path to external scripts
import numpy as np
import numpy.linalg as LA
import matplotlib.pyplot as plt

# local imports
from libs.line.funcs import *
from libs.triangle.funcs import *
from libs.conics.funcs import circ_gen

# --- Step 1: Rank Matrix condition ---
print("Step 1: Rank condition between b and c")
print("Take M = (2,-5), P = (0,b), Q = (c,0)")

# Construct rank matrix for collinearity of P, Q, M
# Vectors (P-M) and (Q-M) must be linearly dependent => rank = 1
# Matrix form: [[Px-Mx, Qx-Mx], [Py-My, Qy-My]]
# Here M=(2,-5)
M = np.array(([2, -5])).reshape(-1,1)
\end{lstlisting}
\end{frame}
\begin{frame}[fragile]
	\frametitle{Direct Python Code}
	\begin{lstlisting}
b, c = symbols = (None, None)  # placeholders for explanation

print("Matrix formed from (P-M) and (Q-M):")
print("[[-2, c-2], [b+5, 5]]")

print("Row operation: R2 -> -2*R2 - (b+5)*R1")
print("=> [[-2, c-2], [0, -10 - (b+5)(c-2)]]")

print("For rank=1: (b+5)(c-2) = -10   (Relation 1)\n")

# --- Step 2: Midpoint relation ---
print("Step 2: Midpoint relation")
print("Midpoint M = ((0+c)/2, (b+0)/2) = (2, -5)")

c = 2 * M[0,0]
b = 2 * M[1,0]
print(f"c/2 = 2  =>  c = {c}")
print(f"b/2 = -5 =>  b = {b}   (Relation 2)\n")

# --- Step 3: Solve both relations ---
P = np.array(([0,b])).reshape(-1,1)
Q = np.array(([c,0])).reshape(-1,1)
\end{lstlisting}
\end{frame}
\begin{frame}[fragile]
	\frametitle{Direct Python Code}
	\begin{lstlisting}
print("Step 3: Solve")
print(f"Coordinates of P = (0, {b})")
print(f"Coordinates of Q = ({c}, 0)\n")

# --- Step 4: Verification ---
mid = (P+Q)/2
print("Verification: midpoint of P and Q =", mid.ravel())

# --- Step 5: Plotting ---
x_PQ = line_gen(P,Q)
plt.plot(x_PQ[0,:], x_PQ[1,:], label='$PQ$')

# Mark points
coords = np.block([[P,Q,M]])
vert_labels = ['P','Q','M']
plt.scatter(coords[0,:], coords[1,:], color=['green','red','magenta'])
for i, txt in enumerate(vert_labels):
    plt.annotate(f'{txt}\n({coords[0,i]:.0f},{coords[1,i]:.0f})',
                 (coords[0,i], coords[1,i]),
                 textcoords="offset points", xytext=(20,-10), ha='center')
\end{lstlisting}
\end{frame}
\begin{frame}[fragile]
	\frametitle{Direct Python Code}
	\begin{lstlisting}
# Axis styling
ax = plt.gca()
ax.spines['left'].set_visible(False)
ax.spines['right'].set_visible(False)
ax.spines['top'].set_visible(False)
ax.spines['bottom'].set_visible(False)

plt.legend(loc='best')
plt.grid()
plt.axis('equal')

# Save figure as PDF
outfile_pdf = 'chapters/10/7/2/2/figs/fig.pdf'
plt.savefig(outfile_pdf)

# Save figure as PNG
outfile_png = 'chapters/10/7/2/2/figs/fig.png'
plt.savefig(outfile_png, dpi=300)
\end{lstlisting}
\end{frame}
\begin{frame}[fragile]
	\frametitle{Direct Python Code}
	\begin{lstlisting}
# Open image depending on system
try:
    import platform, subprocess, shlex
    if "termux" in platform.platform().lower():   # Android Termux
        subprocess.run(shlex.split(f"termux-open {outfile_png}"))
    else:                                        # Linux desktop
        subprocess.run(shlex.split(f"xdg-open {outfile_png}"))
except Exception as e:
    print(f"Could not auto-open file. Saved at {outfile_png}")
\end{lstlisting}
\end{frame}
\begin{frame}{Plot by python using shared output from c}
	\begin{center}
		\includegraphics[width=0.8\columnwidth]{figs/Figure_1.png}
	\end{center}
\end{frame}

\begin{frame}{Plot by python only}
	\begin{center}
		\includegraphics[width=0.9\columnwidth]{figs/fig.png}
	\end{center}
\end{frame}
\end{document}