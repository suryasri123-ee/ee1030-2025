\documentclass{beamer}
\usepackage[utf8]{inputenc}

\usetheme{Madrid}
\usecolortheme{default}
\usepackage{amsmath,amssymb,amsfonts,amsthm}
\usepackage{txfonts}
\usepackage{tkz-euclide}
\usepackage{listings}
\usepackage{adjustbox}
\usepackage{array}
\usepackage{tabularx}
\usepackage{gvv}
\usepackage{lmodern}
\usepackage{circuitikz}
\usepackage{tikz}
\usepackage{graphicx}

\setbeamertemplate{page number in head/foot}[totalframenumber]

\usepackage{tcolorbox}
\tcbuselibrary{minted,breakable,xparse,skins}



\definecolor{bg}{gray}{0.95}
\DeclareTCBListing{mintedbox}{O{}m!O{}}{%
	breakable=true,
	listing engine=minted,
	listing only,
	minted language=#2,
	minted style=default,
	minted options={%
		linenos,
		gobble=0,
		breaklines=true,
		breakafter=,,
		fontsize=\small,
		numbersep=8pt,
		#1},
	boxsep=0pt,
	left skip=0pt,
	right skip=0pt,
	left=25pt,
	right=0pt,
	top=3pt,
	bottom=3pt,
	arc=5pt,
	leftrule=0pt,
	rightrule=0pt,
	bottomrule=2pt,
	toprule=2pt,
	colback=bg,
	colframe=orange!70,
	enhanced,
	overlay={%
		\begin{tcbclipinterior}
			\fill[orange!20!white] (frame.south west) rectangle ([xshift=20pt]frame.north west);
	\end{tcbclipinterior}},
	#3,
}
\lstset{
	language=C,
	basicstyle=\ttfamily\small,
	keywordstyle=\color{blue},
	stringstyle=\color{orange},
	commentstyle=\color{green!60!black},
	numbers=left,
	numberstyle=\tiny\color{gray},
	breaklines=true,
	showstringspaces=false,
}
%------------------------------------------------------------
%This block of code defines the information to appear in the
%Title page
\title %optional
{1.5.24}
\date{}
%\subtitle{A short story}

\author % (optional)
{M Chanakya Srinivas- EE25BTECH11036}





\begin{document}

\begin{frame}
  \titlepage
\end{frame}

\begin{frame}{Problem statement}
A line intersects the \(Y\)-axis and \(X\)-axis at the points
\[
P=(0,b)\quad\text{and}\quad Q=(c,0)
\]
respectively. If \((2,-5)\) is the midpoint of \(\overline{PQ}\), find the coordinates of \(P\) and \(Q\).
\end{frame}

\begin{frame}{Idea (two relations needed)}
We need two independent relations between \(b\) and \(c\). We'll obtain:
\begin{enumerate}
  \item a relation coming from the \emph{line equation / rank condition}, and
  \item a relation coming from the \emph{midpoint} condition.
\end{enumerate}
Both together determine \(b\) and \(c\).
\end{frame}

\begin{frame}{Line equation and the rank relation}
Consider the general line equation
\[
u x + v y + w = 0,
\]
with constants \(u,v,w\) (not all zero). Since \(P=(0,b)\) lies on the line,
\[
u\cdot 0 + v\cdot b + w = 0 \quad\Rightarrow\quad v b + w = 0. \tag{1}
\]
Since \(Q=(c,0)\) lies on the line,
\[
u\cdot c + v\cdot 0 + w = 0 \quad\Rightarrow\quad u c + w = 0. \tag{2}
\]
Subtract (2) from (1):
\[
v b - u c = 0 \quad\Rightarrow\quad v b = u c. \tag{R}
\]
Equation (R) is the relation between \(b\) and \(c\) arising from the line coefficients.
(It is independent of \(w\).)
\end{frame}

\begin{frame}{Interpretation of the rank relation}
The relation \(v b = u c\) simply says the ratio of the intercepts is tied to the ratio of line coefficients:
\[
\frac{b}{c} = \frac{u}{v}.
\]
This is one constraint linking \(b\) and \(c\). To get their numerical values we need another independent relation — the midpoint condition.
\end{frame}

\begin{frame}{Midpoint condition (second relation)}
The midpoint \(M\) of \(P\) and \(Q\) is
\[
M = \frac{P + Q}{2} = \frac{1}{2}\myvec{0\\b} + \frac{1}{2}\myvec{c\\0} = \frac{1}{2}\myvec{c\\b}.
\]
We are given \(M=\myvec{2\\-5}\). Equate the components:
\[
\frac{c}{2} = 2 \quad\Rightarrow\quad c = 4,
\]
\[
\frac{b}{2} = -5 \quad\Rightarrow\quad b = -10.
\]
These are the numerical values of \(b\) and \(c\).
\end{frame}

\begin{frame}{Consistency with the rank relation}
We found:
\[
b=-10,\qquad c=4.
\]
Plugging into the rank relation \(v b = u c\) gives
\[
v(-10) = u(4) \quad\Rightarrow\quad \frac{u}{v} = -\frac{10}{4} = -\frac{5}{2}.
\]
So any line coefficients \(u,v\) satisfying \(u:v = -5:2\) (and appropriate \(w\)) will give the same intercepts. This confirms the intercepts are consistent with a line having slope \(-b/c = -(-10)/4 = 10/4 = 5/2\) (note sign conventions).
\end{frame}

\begin{frame}{Matrix viewpoint (brief)}
Write the midpoint equations componentwise:
\[
\frac{1}{2}\begin{pmatrix}c\\b\end{pmatrix} = \begin{pmatrix}2\\-5\end{pmatrix}
\quad\Longrightarrow\quad
\begin{pmatrix}c\\b\end{pmatrix} = \begin{pmatrix}4\\-10\end{pmatrix}.
\]
Equivalently as a matrix equation \(I\mathbf{x}=\mathbf{B}\) with
\(\mathbf{x}=\begin{pmatrix}b\\c\end{pmatrix}\) (or ordered as you prefer),
the solution is immediate since \(I^{-1}=I\).
\end{frame}

\begin{frame}{Final answer}
Therefore the intercept points are
\[
\boxed{P=(0,-10),\qquad Q=(4,0).}
\]

\bigskip
(These satisfy the midpoint condition and the linear relation from the line coefficients.)
\end{frame}

\begin{frame}{Illustration}
\centering
\includegraphics[width=0.65\textwidth]{figs/fig.png}

\bigskip
Plot shows \(P(0,-10)\), \(Q(4,0)\) and midpoint \(M(2,-5)\).
\end{frame}




\begin{frame}[fragile]
\frametitle{C Code - Section formula function}
\begin{lstlisting}
#include <stdio.h>

// Function to find coordinates of P(0,b) and Q(c,0)
// given the midpoint (mx, my)
void findCoordinates(int mx, int my, int* c, int* b) {

    printf("Step 1: Rank relation between b and c\n");
    printf("Line: ux + vy + w = 0\n");
    printf("P = (0,b), Q = (c,0) lie on the line\n");
    printf("=> vb + w = 0,   uc + w = 0\n");
    printf("Subtracting: vb - uc = 0 => v b = u c (relation 1)\n\n");

    printf("Step 2: Midpoint relation\n");
    printf("Midpoint M = ( (0+c)/2, (b+0)/2 ) = (%d, %d)\n", mx, my);
    printf("=> c/2 = %d => c = %d\n", mx, 2*mx);
    printf("=> b/2 = %d => b = %d\n\n", my, 2*my);
    \end{lstlisting}
\end{frame}

\begin{frame}[fragile]
\frametitle{C Code - Section formula function}
\begin{lstlisting}
    // Assign values using midpoint
    *c = 2 * mx;
    *b = 2 * my;

    printf("Step 3: Solve both relations\n");
    printf("Coordinates of P = (0,%d)\n", *b);
    printf("Coordinates of Q = (%d,0)\n\n", *c);
}

\end{lstlisting}
\end{frame}


 \begin{frame}[fragile]
	\frametitle{Python Code through shared output}
	\begin{lstlisting}       
import numpy as np
import matplotlib.pyplot as plt

# --- Step 1: Rank relation ---
print("Step 1: Rank relation between b and c")
print("General line: ux + vy + w = 0")
print("Points: P=(0,b), Q=(c,0)")
print("Substitute: vb + w = 0,   uc + w = 0")
print("Subtracting => v b = u c   (relation 1)\n")

# --- Step 2: Midpoint relation ---
print("Step 2: Midpoint relation")
mx, my = 2, -5
print(f"Midpoint M = ((0+c)/2, (b+0)/2) = ({mx}, {my})")

# Solve midpoint equations
c = 2 * mx
b = 2 * my
print(f"c/2 = {mx}  =>  c = {c}")
   \end{lstlisting}
        \end{frame}
         \begin{frame}[fragile]
	\frametitle{Python Code through shared output}
	\begin{lstlisting}    
print(f"b/2 = {my}  =>  b = {b}   (relation 2)\n")

# --- Step 3: Solve ---
P = (0, b)
Q = (c, 0)
M = (mx, my)

print("Step 3: Solve both relations")
print(f"Coordinates of P = {P}")
print(f"Coordinates of Q = {Q}\n")

# --- Step 4: Plot ---
x_points = np.array([P[0], Q[0]])
y_points = np.array([P[1], Q[1]])

plt.figure(figsize=(7,6))
plt.plot(x_points, y_points, 'b-', label="Line PQ")
plt.plot(P[0], P[1], 'go', markersize=10, label=f"P {P}")
plt.plot(Q[0], Q[1], 'ro', markersize=10, label=f"Q {Q}")
 \end{lstlisting}
        \end{frame}
              \begin{frame}[fragile]
	\frametitle{Python Code through shared output}
	\begin{lstlisting}    
plt.plot(M[0], M[1], 'm*', markersize=12, label=f"M {M}")

# Annotate
plt.text(P[0]+0.2, P[1], f"P{P}", fontsize=12)
plt.text(Q[0]+0.2, Q[1], f"Q{Q}", fontsize=12)
plt.text(M[0]+0.2, M[1], f"M{M}", fontsize=12)

# Format
plt.title("Line Intercepting Axes with Midpoint Condition", fontsize=15)
plt.axhline(0, color="black", linewidth=0.7)
plt.axvline(0, color="black", linewidth=0.7)
plt.xlabel("X-axis")
plt.ylabel("Y-axis")
plt.grid(True, linestyle="--", alpha=0.6)
plt.legend()
plt.axis("equal")
plt.show()

\end{lstlisting}
        \end{frame}

\begin{frame}[fragile]
	\frametitle{Direct Python Code}
	\begin{lstlisting}
import sys                            # for path to external scripts
import numpy as np
import numpy.linalg as LA
import matplotlib.pyplot as plt

# local imports
from libs.line.funcs import *
from libs.triangle.funcs import *
from libs.conics.funcs import circ_gen

# --- Step 1: Rank relation ---
print("Step 1: Rank relation between b and c")
print("General line: ux + vy + w = 0")
print("Points: P = (0,b), Q = (c,0)")
print("=> vb + w = 0,   uc + w = 0")
print("Subtracting => v b = u c   (Relation 1)\n")

# --- Step 2: Midpoint relation ---
print("Step 2: Midpoint relation")
\end{lstlisting}
\end{frame}
\begin{frame}[fragile]
	\frametitle{Direct Python Code}
	\begin{lstlisting}

M = np.array(([2, -5])).reshape(-1,1)
print("Midpoint M = ((0+c)/2, (b+0)/2) = (2, -5)")
c = 2 * M[0,0]
b = 2 * M[1,0]
print(f"c/2 = 2  =>  c = {c}")
print(f"b/2 = -5 =>  b = {b}   (Relation 2)\n")

# --- Step 3: Solve both relations ---
P = np.array(([0,b])).reshape(-1,1)
Q = np.array(([c,0])).reshape(-1,1)

print("Step 3: Solve both relations")
print(f"Coordinates of P = (0, {b})")
print(f"Coordinates of Q = ({c}, 0)\n")

# --- Step 4: Plotting ---
x_PQ = line_gen(P,Q)
plt.plot(x_PQ[0,:], x_PQ[1,:], label='$PQ$')
\end{lstlisting}
\end{frame}
\begin{frame}[fragile]
	\frametitle{Direct Python Code}
	\begin{lstlisting}
# Mark points
coords = np.block([[P,Q,M]])
vert_labels = ['P','Q','M']
plt.scatter(coords[0,:], coords[1,:], color=['green','red','magenta'])
for i, txt in enumerate(vert_labels):
    plt.annotate(f'{txt}\n({coords[0,i]:.0f},{coords[1,i]:.0f})',
                 (coords[0,i], coords[1,i]),
                 textcoords="offset points", xytext=(20,-10), ha='center')

# Axis styling
ax = plt.gca()
ax.spines['left'].set_visible(False)
ax.spines['right'].set_visible(False)
ax.spines['top'].set_visible(False)
ax.spines['bottom'].set_visible(False)

plt.legend(loc='best')
plt.grid()
plt.axis('equal')
\end{lstlisting}
\end{frame}
\begin{frame}[fragile]
	\frametitle{Direct Python Code}
	\begin{lstlisting}
# Save figure as PDF
outfile_pdf = 'chapters/10/7/2/2/figs/fig.pdf'
plt.savefig(outfile_pdf)

# Save figure as PNG
outfile_png = 'chapters/10/7/2/2/figs/fig.png'
plt.savefig(outfile_png, dpi=300)

# Open image depending on system
try:
    import platform, subprocess, shlex
    if "termux" in platform.platform().lower():   # Android Termux
        subprocess.run(shlex.split(f"termux-open {outfile_png}"))
    else:                                        # Linux desktop
        subprocess.run(shlex.split(f"xdg-open {outfile_png}"))
except Exception as e:
    print(f"Could not auto-open file. Saved at {outfile_png}")
\end{lstlisting}
\end{frame}
\begin{frame}{Plot by python using shared output from c}
	\begin{center}
		\includegraphics[width=0.8\columnwidth]{figs/Figure_1.png}
	\end{center}
\end{frame}

\begin{frame}{Plot by python only}
	\begin{center}
		\includegraphics[width=0.9\columnwidth]{figs/fig.png}
	\end{center}
\end{frame}
\end{document}