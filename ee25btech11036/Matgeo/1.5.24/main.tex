
\let\negmedspace\undefined
\let\negthickspace\undefined
\documentclass[journal]{IEEEtran}
\usepackage[a5paper, margin=10mm, onecolumn]{geometry}
\usepackage{lmodern} % Ensure lmodern is loaded for pdflatex

\setlength{\headheight}{1cm} % Set the height of the header box
\setlength{\headsep}{0mm}     % Set the distance between the header box and the top of the text

\usepackage{gvv-book}
\usepackage{gvv}
\usepackage{cite}
\usepackage{amsmath,amssymb,amsfonts,amsthm}
\usepackage{algorithmic}
\usepackage{graphicx}
\graphicspath{{./figs/}}
\usepackage{textcomp}
\usepackage{xcolor}
\usepackage{txfonts}
\usepackage{listings}
\usepackage{enumitem}
\usepackage{mathtools}
\usepackage{gensymb}
\usepackage{comment}
\usepackage[breaklinks=true]{hyperref}
\usepackage{tkz-euclide} 
\usepackage{listings}
\usepackage{gvv}                                        
\def\inputGnumericTable{}                                 
\usepackage[latin1]{inputenc}                                
\usepackage{color}                                            
\usepackage{array}                                            
\usepackage{longtable}                                       
\usepackage{calc}                                             
\usepackage{multirow}                                         
\usepackage{hhline}                                           
\usepackage{ifthen}                                           
\usepackage{lscape}
\usepackage{circuitikz}
\tikzstyle{block} = [rectangle, draw, fill=blue!20, 
text width=4em, text centered, rounded corners, minimum height=3em]
\tikzstyle{sum} = [draw, fill=blue!10, circle, minimum size=1cm, node distance=1.5cm]
\tikzstyle{input} = [coordinate]
\tikzstyle{output} = [coordinate]


\begin{document}
	
	\bibliographystyle{IEEEtran}
	\vspace{3cm}
	
	\title{1.5.24}
	\author{EE25BTECH11036 - M Chanakya Srinivas}
	\maketitle
	% \newpage
	% \bigskip
	{\let\newpage\relax\maketitle}
	
	\renewcommand{\thefigure}{\theenumi}
	\renewcommand{\thetable}{\theenumi}
	\setlength{\intextsep}{10pt} % Space between text and floats
	
	
	\numberwithin{equation}{enumi}
	\numberwithin{figure}{enumi}
	\renewcommand{\thetable}{\theenumi}

\textbf{1.5.24}\quad
A line intersects the $Y$-axis and $X$-axis at the points
$P=(0,b)$ and $Q=(c,0)$ respectively. If $(2,-5)$ is the midpoint of
$\overline{PQ}$, then find the coordinates of $P$ and $Q$.

\section*{Matrix Solution}

\begin{enumerate}
\item \textbf{Relation from Rank Condition}

The general equation of a line is
\[
ax + by = 1
\]

Since it intersects the $y$-axis at $P=(0,b)$ and $x$-axis at $Q=(c,0)$,  
we have the relations:
\[
\vec{P} = \myvec{0 \\ b}, \quad \vec{Q} = \myvec{c \\ 0}.
\]

These points must satisfy the line equation, giving the system
\[
\myvec{0 & b \\ c & 0}\vec{x} = \myvec{b \\ c}.
\]

For consistency, the rank condition gives
\[
\frac{0}{b} = \frac{c}{0} \;\;\Rightarrow\;\; bc + 40 = 0
\]
(or the equivalent relation depending on the method).

\item \textbf{Relation from Midpoint}

The midpoint $M$ of $P$ and $Q$ is
\[
M = \frac{P+Q}{2} = \myvec{2 \\ -5}.
\]

So,
\[
\frac{c}{2} = 2 \;\;\Rightarrow\;\; c=4, 
\qquad 
\frac{b}{2} = -5 \;\;\Rightarrow\;\; b=-10.
\]

\item \textbf{Final Coordinates}

\[
P = \myvec{0 \\ -10}, 
\qquad 
Q = \myvec{4 \\ 0}.
\]

\end{enumerate}

\begin{figure}[h!]
    \centering
    \includegraphics[width=0.6\columnwidth]{figs/Figure_1.png}
    \caption{Plot using Shared output}
\end{figure}

\begin{figure}[h!]
    \centering
    \includegraphics[width=0.6\columnwidth]{figs/fig.png}
    \caption{Plot using Python}
\end{figure}

\end{document}


