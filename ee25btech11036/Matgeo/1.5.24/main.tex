
\let\negmedspace\undefined
\let\negthickspace\undefined
\documentclass[journal]{IEEEtran}
\usepackage[a5paper, margin=10mm, onecolumn]{geometry}
\usepackage{lmodern} % Ensure lmodern is loaded for pdflatex

\setlength{\headheight}{1cm} % Set the height of the header box
\setlength{\headsep}{0mm}     % Set the distance between the header box and the top of the text

\usepackage{gvv-book}
\usepackage{gvv}
\usepackage{cite}
\usepackage{amsmath,amssymb,amsfonts,amsthm}
\usepackage{algorithmic}
\usepackage{graphicx}
\graphicspath{{./figs/}}
\usepackage{textcomp}
\usepackage{xcolor}
\usepackage{txfonts}
\usepackage{listings}
\usepackage{enumitem}
\usepackage{mathtools}
\usepackage{gensymb}
\usepackage{comment}
\usepackage[breaklinks=true]{hyperref}
\usepackage{tkz-euclide} 
\usepackage{listings}
\usepackage{gvv}                                        
\def\inputGnumericTable{}                                 
\usepackage[latin1]{inputenc}                                
\usepackage{color}                                            
\usepackage{array}                                            
\usepackage{longtable}                                       
\usepackage{calc}                                             
\usepackage{multirow}                                         
\usepackage{hhline}                                           
\usepackage{ifthen}                                           
\usepackage{lscape}
\usepackage{circuitikz}
\tikzstyle{block} = [rectangle, draw, fill=blue!20, 
text width=4em, text centered, rounded corners, minimum height=3em]
\tikzstyle{sum} = [draw, fill=blue!10, circle, minimum size=1cm, node distance=1.5cm]
\tikzstyle{input} = [coordinate]
\tikzstyle{output} = [coordinate]



\begin{document}
	
	\bibliographystyle{IEEEtran}
	\vspace{3cm}
	
	\title{1.5.24}
	\author{EE25BTECH11036 - M Chanakya Srinivas}
	\maketitle
	% \newpage
	% \bigskip
	{\let\newpage\relax\maketitle}
	
	\renewcommand{\thefigure}{\theenumi}
	\renewcommand{\thetable}{\theenumi}
	\setlength{\intextsep}{10pt} % Space between text and floats
	
	
\renewcommand\theequation{\arabic{equation}}
\renewcommand\thefigure{\arabic{figure}}
\renewcommand{\thetable}{\theenumi}

\textbf{1.5.24}\quad
A line intersects the $Y$-axis and $X$-axis at the points
$P=(0,b)$ and $Q=(c,0)$ respectively. If $(2,-5)$ is the midpoint of
$\overline{PQ}$, then find the coordinates of $P$ and $Q$.

% Solution for 1.5.24 (matrix method: rank + midpoint)
\providecommand{\myvec}[1]{\begin{pmatrix}#1\end{pmatrix}}

\begin{align}
\vec P&=\myvec{0\\ b},
\end{align}
\begin{align}
    \vec Q=\myvec{c\\ 0},
\end{align} 

\begin{align}
\vec M=\myvec{2\\ -5}.
\end{align}

\textbf{(i) Rank/collinearity: } 
\begin{align}
\text{Since } \vec P,\vec Q,\vec M \text{ are collinear, } 
\operatorname{rank}\myvec{\vec P-\vec M & \vec Q-\vec M}^{\top}=1.
\end{align}
\begin{align}
\myvec{\vec P-\vec M & \vec Q-\vec M}^{\top}
&=\myvec{-2 & c-2\\[2pt] b+5 & 5}
\xrightarrow{\,R_2\leftarrow -2R_2-(b+5)R_1\,}
\myvec{-2 & c-2\\[2pt] 0 & -10-(b+5)(c-2)}.
\end{align}
For rank \(=1\), the last entry must be \(0\):
\begin{align}
\label{eq:theory}
-10-(b+5)(c-2)=0
\;\;\Longrightarrow\;\;
(b+5)(c-2)=-10.
\end{align}

\textbf{(ii) Midpoint: }
\begin{align}
\vec M=\frac{\vec P+\vec Q}{2}
\;\Longrightarrow\;
\myvec{2\\ -5}
=\frac{1}{2}\myvec{0\\ b}+\frac{1}{2}\myvec{c\\ 0}
=\frac{1}{2}\myvec{c\\ b}
\;\Longrightarrow\;
\myvec{c\\ b}=\myvec{4\\ -10}.
\end{align}
Thus \(c=4,\; b=-10\), and these satisfy \ref{eq:theory}.

\textbf{Answer: }
\[
\vec P=\myvec{0\\ -10},\qquad
\vec Q=\myvec{4\\ 0}.
\]
\begin{figure}[h!]
    \centering
    \includegraphics[width=0.6\columnwidth]{figs/Figure_1.png}
    \caption{Plot using Shared output}
\end{figure}

\begin{figure}[h!]
    \centering
    \includegraphics[width=0.6\columnwidth]{figs/fig.png}
    \caption{Plot using Python}
\end{figure}

\end{document}


