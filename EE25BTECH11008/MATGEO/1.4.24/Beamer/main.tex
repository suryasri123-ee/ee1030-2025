\documentclass{beamer}
\usepackage{listings}
\usepackage{color}
\usepackage{amsmath}

\title{Matrices in Geometry - 1.4.24}
\author{EE25BTECH11008-Anirudh M Abhilash}
\date{\today}

\begin{document}

%-------------------------------
\begin{frame}
\titlepage
\end{frame}

%-------------------------------
\begin{frame}{Problem Statement}
If $P(9a-2, -b)$ divides line segment joining $A(3a+1, -3)$ and $B(8a, 5)$ in the ratio 3:1, find a and b.

\end{frame}

%-------------------------------
\begin{frame}{Solution}
\textbf{Section Formula:}  
If $P$ divides $AB$ in $m:n$, then
\[
P = \left(\frac{mx_2 + nx_1}{m+n}, \; \frac{my_2 + ny_1}{m+n}\right).
\]
By applying the section formula:
\[
9a - 2 = \frac{27a+1}{4}, \qquad -b = 3.
\]
\[
\Rightarrow \; a = 1, \quad b = -3.
\]

We will now compute $(a,b)$ in Python and verify them using C.
\end{frame}

%-------------------------------
\begin{frame}[fragile]{Python Code (Computation + Plot)}
\begin{lstlisting}[language=Python, basicstyle=\ttfamily\scriptsize]

import numpy as np
import numpy.linalg as la
import matplotlib.pyplot as plt

# Solving eqns
coeff = np.array([[9-27/4, 0], [0, -1]])
rhs = np.array([2+1/4, 3])
soln = la.solve(coeff, rhs)
a = soln[0]
b = soln[1]

# Coordinates
A = (3*a + 1, -3)
B = (8*a, 5)
P = (9*a - 2, -b)

# Plotting points
plt.figure(figsize=(6,6))
plt.plot([A[0], B[0]], [A[1], B[1]], 'k--', label='Line AB')  
# Line AB
plt.scatter(*A, color='blue', s=100, label='A (3a+1, -3)')
plt.scatter(*B, color='green', s=100, label='B (8a, 5)')
plt.scatter(*P, color='red', s=100, label='P (9a-2, -b)')

\end{lstlisting}
\end{frame}

\begin{frame}[fragile]{Python Code (Cont..)}
\begin{lstlisting}[language=Python, basicstyle=\ttfamily\scriptsize]

# Annotating points
plt.text(A[0]+0.2, A[1], 'A', fontsize=12)
plt.text(B[0]+0.2, B[1], 'B', fontsize=12)
plt.text(P[0]+0.2, P[1], 'P', fontsize=12)



# Axis labels and grid
plt.xlabel('x')
plt.ylabel('y')
plt.title('Plot of Points A, B and P dividing AB in 3:1')
plt.grid(True)
plt.legend()
plt.axis('equal')
plt.show()
\end{lstlisting}
\end{frame}

%-------------------------------
\begin{frame}[fragile]{C Code for verification}
\begin{lstlisting}[language=C, basicstyle=\ttfamily\scriptsize]
#include<stdio.h>

int solve(float a, float b) {
    float A[2] = {3*a + 1, -3};
    float B[2] = {8*a, 5};
    float P[2] = {9*a - 2, -b};
    int ratio = 3;

    if (((A[0]+ratio*B[0])/(ratio+1) == P[0]) && ((A[1]+ratio*B[1])/(ratio+1) == P[1])) {
        return 1;
    }
    return 0;
}
\end{lstlisting}

This function is compiled as a shared library
and called from Python using \texttt{ctypes}.
\end{frame}

%-------------------------------
\begin{frame}[fragile]{Using the C code in Python}
\begin{lstlisting}[language=Python, basicstyle=\ttfamily\scriptsize]

import ctypes
import numpy as np
import numpy.linalg as la
import matplotlib.pyplot as plt

check = ctypes.CDLL("./verify.so")
check.solve.argtypes = [ctypes.c_float, ctypes.c_float]
check.solve.restype = ctypes.c_int

coeff = np.array([[9-27/4, 0], [0, -1]])
rhs = np.array([2+1/4, 3])
soln = la.solve(coeff, rhs)
a = soln[0]
b = soln[1]

correct = check.solve(a,b)
A = (3*a + 1, -3)
B = (8*a, 5)
P = (9*a - 2, -b)

\end{lstlisting}
\end{frame}

\begin{frame}[fragile]{Using the C code in Python (Cont..)}
\begin{lstlisting}[language=Python, basicstyle=\ttfamily\scriptsize]

if correct:
    # Plotting points
    plt.figure(figsize=(6,6))
    plt.plot([A[0], B[0]], [A[1], B[1]], 'k--', label='Line AB')  
    plt.scatter(*A, color='blue', s=100, label='A (3a+1, -3)')
    plt.scatter(*B, color='green', s=100, label='B (8a, 5)')
    plt.scatter(*P, color='red', s=100, label='P (9a-2, -b)')

    # Annotating points
    plt.text(A[0]+0.2, A[1], 'A', fontsize=12)
    plt.text(B[0]+0.2, B[1], 'B', fontsize=12)
    plt.text(P[0]+0.2, P[1], 'P', fontsize=12)

    # Axis labels and grid
    plt.xlabel('x')
    plt.ylabel('y')
    plt.title('Plot of Points A, B and P dividing AB in 3:1')
    plt.grid(True)
    plt.legend()
    plt.axis('equal')
    plt.show()
    
\end{lstlisting}
\end{frame}

%-------------------------------
\begin{frame}{Resulting Plot}
\begin{figure}
\centering
\includegraphics[width=0.75\columnwidth]{figs/plt.png}
\caption{Plot}
\label{fig:plot}
\end{figure}
\end{frame}

\end{document}