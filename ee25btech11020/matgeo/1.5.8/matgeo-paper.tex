\let\negmedspace\undefined
\let\negthickspace\undefined
\documentclass[journal,12pt,onecolumn]{IEEEtran}
\usepackage{cite}
\usepackage{amsmath,amssymb,amsfonts,amsthm}
\usepackage{algorithmic}
\usepackage{graphicx}
\graphicspath{{./figs/}}
\usepackage{textcomp}
\usepackage{xcolor}
\usepackage{txfonts}
\usepackage{listings}
\usepackage{enumitem}
\usepackage{mathtools}
\usepackage{gensymb}
\usepackage{comment}
\usepackage{caption}
\usepackage[breaklinks=true]{hyperref}
\usepackage{tkz-euclide} 
\usepackage{listings}
\usepackage{gvv}                                        
%\def\inputGnumericTable{}                                 
\usepackage[latin1]{inputenc}     
\usepackage{xparse}
\usepackage{color}                                            
\usepackage{array}                                            
\usepackage{longtable}                                       
\usepackage{calc}                                             
\usepackage{multirow}
\usepackage{multicol}
\usepackage{hhline}                                           
\usepackage{ifthen}                                           
\usepackage{lscape}
\usepackage{tabularx}
\usepackage{array}
\usepackage{float}
%\newtheorem{theorem}{Theorem}[section]
%\newtheorem{theorem}{Theorem}[section]
%\newtheorem{problem}{Problem}
%\newtheorem{proposition}{Proposition}[section]
%\newtheorem{lemma}{Lemma}[section]
%\newtheorem{corollary}[theorem]{Corollary}
%\newtheorem{example}{Example}[section]
%\newtheorem{definition}[problem]{Definition}

\begin{document}

%\textbf{\Large 1.2.1} \\
%\textbf{\large AI25BTECH11001 - Abhisek Mohapatra} \\
\title{1.5.8}
\author{EE25BTECH11020 - Darsh Pankaj Gajare}
% \maketitle
% \newpage
% \bigskip
%\begin{document}
{\let\newpage\relax\maketitle}
%\renewcommand{\thefigure}{\theenumi}
%\renewcommand{\thetable}{\theenumi}
Question:\\
Find the ratio in which $\vec{P}\brak{4, 5}$ divides the line segment joining $\vec{A}\brak{2, 3}$ and $\vec{B}\brak{7, 8}$.\\
\textbf{Solution:}
Using Section Formula,\\
\begin{align}
\vec{P}=\frac{k\vec{B}+\vec{A}}{k+1}
\end{align}
\begin{align}
\myvec{4\\5}=\frac{k\myvec{7\\8}+\myvec{2\\3}}{k+1}
\end{align}
\begin{align}
3k\myvec{1\\1}=2\myvec{1\\1}
\end{align}
\begin{align}
or, k=\frac{2}{3}
\end{align}
\begin{figure}[H]
	\centering
	\includegraphics[scale=0.5]{img}
	\caption*{}
	\label{img}
\end{figure}
\end{document}
