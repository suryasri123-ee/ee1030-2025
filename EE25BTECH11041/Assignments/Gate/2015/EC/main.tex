\documentclass[a4paper, 11pt]{article}
\usepackage{graphicx}
\usepackage{multicol}
\usepackage{tabularx}
\usepackage{enumitem}
\usepackage[a4paper, margin=1.8cm]{geometry}
\usepackage{listings}
\usepackage{amssymb}
\usepackage{gvv}
\usepackage{gvv-book}
\usepackage{amsmath}
\usepackage{setspace}
\usepackage{caption}
\usepackage{tfrupee}
\usepackage{float}

\graphicspath{./figs/}

\begin{document}
\begin{center}
    \huge{EC : ELECTRONICS AND COMMUNICATION ENGINEERING}\\
    \large{EE25BTECH11041 - Naman Kumar}
\end{center}

\begin{enumerate}
\section*{General Aptitude \brak{GA}}
    \item Choose the most appropriate word from the options given below to complete the following sentence.
    
    The principal presented the chief guest with a \underline{\hspace{2cm}} as token of appreciation.
    \begin{enumerate}
        \begin{multicols}{4}
            \item momento
            \item memento
            \item momentum
            \item moment
        \end{multicols}
    \end{enumerate}
    
    \hfill{\brak{\text{GATE EC 2015}}}

    \item Choose the appropriate word/phrase, out of the four options given below, to complete the following sentence:\\Frogs \underline{\hspace{2cm}}.
    \begin{enumerate}
        \begin{multicols}{4}
            \item croak
            \item roar
            \item hiss
            \item patter
        \end{multicols}
    \end{enumerate}
    
    \hfill{\brak{\text{GATE EC 2015}}}

    \item Choose the word most similar in meaning to the given word:\\Educe
    \begin{enumerate}
        \begin{multicols}{4}
            \item Exert
            \item Educate
            \item Extract
            \item Extend
        \end{multicols}
    \end{enumerate}
    
    \hfill{\brak{\text{GATE EC 2015}}}

    \item Operators $\Box$, $\Diamond$ and $\rightarrow$ are defined by: $a \Box b=\frac{a-b}{a+b}$; $a \Diamond b=\frac{a+b}{a-b}$; $a \rightarrow b=ab$. Find the value of $\brak{66\Box6}-\brak{66\Diamond6}$.
    \begin{enumerate}
        \begin{multicols}{4}
            \item -2
            \item -1
            \item 1
            \item 2
        \end{multicols}
    \end{enumerate}
    
    \hfill{\brak{\text{GATE EC 2015}}}

    \item If $\log_{x}\brak{5/7}=-1/3$, then the value of x is
    \begin{enumerate}
        \begin{multicols}{2}
            \item $343/125$
            \item $125/343$
            \item $-25/49$
            \item $-49/25$
        \end{multicols}
    \end{enumerate}
    
    \hfill{\brak{\text{GATE EC 2015}}}

    \item The following question presents a sentence, part of which is underlined. Beneath the sentence you find four ways of phrasing the underlined part. Following the requirements of the standard written English, select the answer that produces the most effective sentence.\\Tuberculosis, together with its effects, \underline{ranks one of the leading causes of death in India}.
    \begin{enumerate}
        \item ranks as one of the leading causes of death
        \item rank as one of the leading causes of death
        \item has the rank of one of the leading causes of death
        \item are one of the leading causes of death
    \end{enumerate}
    
    \hfill{\brak{\text{GATE EC 2015}}}

    \item Read the following paragraph and choose the correct statement.
    
    Climate change has reduced human security and threatened human well being. An ignored reality of human progress is that human security largely depends upon environmental security. But on the contrary, human progress seems contradictory to environmental security. To keep up both at the required level is a challenge to be addressed by one and all. One of the ways to curb the climate change may be suitable scientific innovations, while the other may be the Gandhian perspective on small scale progress with focus on sustainability.
    \begin{enumerate}
        \item Human progress and security are positively associated with environmental security.
        \item Human progress is contradictory to environmental security.
        \item Human security is contradictory to environmental security.
        \item Human progress depends upon environmental security.
    \end{enumerate}
    
    \hfill{\brak{\text{GATE EC 2015}}}

    \item Fill in the missing value
    \begin{figure}[H]
        \centering
        \includegraphics[width=0.5\columnwidth]{figs/q8.png}
        \caption*{}
        \label{fig:q8}
    \end{figure}
    
    \underline{\hspace{2cm}}
    
    \hfill{\brak{\text{GATE EC 2015}}}

    \item A cube of side 3 units is formed using a set of smaller cubes of side 1 unit. Find the proportion of the number of faces of the smaller cubes visible to those which are NOT visible.
    \begin{enumerate}
        \begin{multicols}{4}
            \item 1:4
            \item 1:3
            \item 1:2
            \item 2:3
        \end{multicols}
    \end{enumerate}
    
    \hfill{\brak{\text{GATE EC 2015}}}

    \item Humpty Dumpty sits on a wall every day while having lunch. The wall sometimes breaks. A person sitting on the wall falls if the wall breaks.\\Which one of the statements below is logically valid and can be inferred from the above sentences?
    \begin{enumerate}
        \item Humpty Dumpty always falls while having lunch
        \item Humpty Dumpty does not fall sometimes while having lunch
        \item Humpty Dumpty never falls during dinner
        \item When Humpty Dumpty does not sit on the wall, the wall does not break
    \end{enumerate}
    
    \hfill{\brak{\text{GATE EC 2015}}}
\section*{Electronics and Communication Engineering \brak{EC}}
    \item Consider a system of linear equations:
    \begin{align*}
        x-2y+3z &= -1, \\
        x-3y+4z &= 1, \quad \text{and} \\
        -2x+4y-6z &= k
    \end{align*}
    The value of k for which the system has infinitely many solutions is \underline{\hspace{2cm}}.
    
    \hfill{\brak{\text{GATE EC 2015}}}

    \item A function $f\brak{x}=1-x^{2}+x^{3}$ is defined in the closed interval [-1,1]. The value of x, in the open interval \brak{-1, 1} for which the mean value theorem is satisfied, is
    \begin{enumerate}
        \begin{multicols}{2}
            \item $-1/2$
            \item $-1/3$
            \item $1/3$
            \item $1/2$
        \end{multicols}
    \end{enumerate}
    
    \hfill{\brak{\text{GATE EC 2015}}}

    \item Suppose A and B are two independent events with probabilities $P\brak{A}\ne0$ and $P\brak{B}\ne0$. Let $\overline{A}$ and $\overline{B}$ be their complements. Which one of the following statements is FALSE?
    \begin{enumerate}
        \begin{multicols}{2}
            \item $P\brak{A\cap B}=P\brak{A}P\brak{B}$
            \item $P\brak{A|B}=P\brak{A}$
            \item $P\brak{A\cup B}=P\brak{A}+P\brak{B}$
            \item $P\brak{\overline{A}\cap\overline{B}}=P\brak{\overline{A}}P\brak{\overline{B}}$
        \end{multicols}
    \end{enumerate}
    
    \hfill{\brak{\text{GATE EC 2015}}}

    \item Let $z=x+iy$ be a complex variable. Consider that contour integration is performed along the unit circle in anticlockwise direction. Which one of the following statements is NOT TRUE?
    \begin{enumerate}
        \item The residue of $\frac{z}{z^{2}-1}$ at $z=1$ is $1/2$
        \item $\oint_{c}z^{2}dz=0$
        \item $\frac{1}{2\pi i}\oint_{C}\frac{1}{z}dz=1$
        \item $\overline{z}$ \brak{\text{complex conjugate of z}} is an analytical function
    \end{enumerate}
    
    \hfill{\brak{\text{GATE EC 2015}}}

    \item The value of p such that the vector $\myvec{1\\ 2\\ 3}$ is an eigenvector of the matrix $\myvec{4 & 1 & 2\\ p & 2 & 1\\ 14 & -4 & 10}$ is \underline{\hspace{2cm}}.

    \hfill{\brak{\text{GATE EC 2015}}}

    \item In the circuit shown, at resonance, the amplitude of the sinusoidal voltage \brak{in Volts} across the capacitor is \underline{\hspace{2cm}}.
    \begin{figure}[H]
        \centering
        \includegraphics[width=0.4\columnwidth]{figs/q16.png}
        \caption*{}
        \label{fig:q16}
    \end{figure}
    
    \hfill{\brak{\text{GATE EC 2015}}}

    \item In the network shown in the figure, all resistors are identical with $R=300~\ohm$. The resistance $R_{ab} \brak{\text{in } \ohm}$ of the network is \underline{\hspace{2cm}}.
    \begin{figure}[H]
        \centering
        \includegraphics[width=0.6\columnwidth]{figs/q17.png}
        \caption*{}
        \label{fig:q17}
    \end{figure}
    
    \hfill{\brak{\text{GATE EC 2015}}}

    \item In the given circuit, the values of $V_{1}$ and $V_{2}$ respectively are
    \begin{figure}[H]
        \centering
        \includegraphics[width=0.5\columnwidth]{figs/q18.png}
        \caption*{}
        \label{fig:q18}
    \end{figure}
    \begin{enumerate}
        \begin{multicols}{2}
            \item 5 V, 25 V
            \item 10 V, 30 V
            \item 15 V, 35 V
            \item 0 V, 20 V
        \end{multicols}
    \end{enumerate}
    
    \hfill{\brak{\text{GATE EC 2015}}}

    \item A region of negative differential resistance is observed in the current voltage characteristics of a silicon PN junction if
    \begin{enumerate}
        \item both the P-region and the N-region are heavily doped
        \item the N-region is heavily doped compared to the P-region
        \item the P-region is heavily doped compared to the N-region
        \item an intrinsic silicon region is inserted between the P-region and the N-region
    \end{enumerate}
    
    \hfill{\brak{\text{GATE EC 2015}}}

    \item A silicon sample is uniformly doped with donor type impurities with a concentration of $10^{16}/cm^{3}$. The electron and hole mobilities in the sample are $1200~cm^{2}/V-s$ and $400~cm^{2}/V-s$ respectively. Assume complete ionization of impurities. The charge of an electron is $1.6\times10^{-19}C$. The resistivity of the sample \brak{in $\ohm$-cm} is \underline{\hspace{2cm}}.
    
    \hfill{\brak{\text{GATE EC 2015}}}

    \item For the circuit with ideal diodes shown in the figure, the shape of the output $\brak{v_{out}}$ for the given sine wave input $\brak{v_{in}}$ will be
    \begin{figure}[H]
        \centering
        \includegraphics[width=0.5\columnwidth]{figs/q21.png}
        \caption*{}
        \label{fig:q21}
    \end{figure}
    \begin{enumerate}
        \item \includegraphics[width=0.4\columnwidth]{figs/q21A.png}
        \item \includegraphics[width=0.4\columnwidth]{figs/q21B.png}
        \item \includegraphics[width=0.4\columnwidth]{figs/q21C.png}
        \item \includegraphics[width=0.4\columnwidth]{figs/q21D.png}
    \end{enumerate}
    
    \hfill{\brak{\text{GATE EC 2015}}}

    \item In the circuit shown below, the Zener diode is ideal and the Zener voltage is 6 V. The output voltage $V_{0} \brak{\text{in volts}}$ is \underline{\hspace{2cm}}.
    \begin{figure}[H]
        \centering
        \includegraphics[width=0.4\columnwidth]{figs/q22.png}
        \caption*{}
        \label{fig:q22}
    \end{figure}
    
    \hfill{\brak{\text{GATE EC 2015}}}

    \item In the circuit shown, the switch SW is thrown from position A to position B at time $t=0$. The energy \brak{in $\mu$J} taken from the 3 V source to charge the 0.1 $\mu$F capacitor from 0 V to 3 V is
    \begin{figure}[H]
        \centering
        \includegraphics[width=0.5\columnwidth]{figs/q23.png}
        \caption*{}
        \label{fig:q23}
    \end{figure}
    \begin{enumerate}
        \begin{multicols}{4}
            \item 0.3
            \item 0.45
            \item 0.9
            \item 3
        \end{multicols}
    \end{enumerate}
    
    \hfill{\brak{\text{GATE EC 2015}}}

    \item In an 8085 microprocessor, the shift registers which store the result of an addition and the overflow bit are, respectively
    \begin{enumerate}
        \begin{multicols}{2}
            \item B and F
            \item A and F
            \item H and F
            \item A and C
        \end{multicols}
    \end{enumerate}
    
    \hfill{\brak{\text{GATE EC 2015}}}

    \item A 16 Kb \brak{=16,384 bit} memory array is designed as a square with an aspect ratio of one \brak{\text{number of rows is equal to the number of columns}}. The minimum number of address lines needed for the row decoder is \underline{\hspace{2cm}}.
    
    \hfill{\brak{\text{GATE EC 2015}}}

    \item Consider a four bit D to A converter. The analog value corresponding to digital signals of values 0000 and 0001 are 0 V and 0.0625 V respectively. The analog value \brak{in Volts} corresponding to the digital signal 1111 is \underline{\hspace{2cm}}.
    
    \hfill{\brak{\text{GATE EC 2015}}}

    \item The result of the convolution $x\brak{-t}*\delta\brak{-t-t_{0}}$ is
    \begin{enumerate}
        \begin{multicols}{4}
            \item $x\brak{t+t_{0}}$
            \item $x\brak{t-t_{0}}$
            \item $x\brak{-t+t_{0}}$
            \item $x\brak{-t-t_{0}}$
        \end{multicols}
    \end{enumerate}
    
    \hfill{\brak{\text{GATE EC 2015}}}

    \item The waveform of a periodic signal $x\brak{t}$ is shown in the figure.
    \begin{figure}[H]
        \centering
        \includegraphics[width=0.5\columnwidth]{figs/q28.png}
        \caption*{}
        \label{fig:q28}
    \end{figure}
    A signal $g\brak{t}$ is defined by $g\brak{t}=x\brak{\frac{t-1}{2}}$. The average power of $g\brak{t}$ is \underline{\hspace{2cm}}.
    
    \hfill{\brak{\text{GATE EC 2015}}}

    \item Negative feedback in a closed-loop control system DOES NOT
    \begin{enumerate}
        \item reduce the overall gain
        \item reduce bandwidth
        \item improve disturbance rejection
        \item reduce sensitivity to parameter variation
    \end{enumerate}
    
    \hfill{\brak{\text{GATE EC 2015}}}

    \item A unity negative feedback system has the open-loop transfer function $G\brak{s}=\frac{K}{s\brak{s+1}\brak{s+3}}$. The value of the gain K \brak{$>0$} at which the root locus crosses the imaginary axis is \underline{\hspace{2cm}}.
    
    \hfill{\brak{\text{GATE EC 2015}}}

    \item The polar plot of the transfer function $G\brak{s}=\frac{10\brak{s+1}}{s+10}$ for $0\le\omega<\infty$ will be in the
    \begin{enumerate}
        \begin{multicols}{2}
            \item first quadrant
            \item second quadrant
            \item third quadrant
            \item fourth quadrant
        \end{multicols}
    \end{enumerate}
    
    \hfill{\brak{\text{GATE EC 2015}}}

    \item A sinusoidal signal of 2 kHz frequency is applied to a delta modulator. The sampling rate and step-size $\Delta$ of the delta modulator are 20,000 samples per second and 0.1 V, respectively. To prevent slope overload, the maximum amplitude of the sinusoidal signal \brak{in Volts} is
    \begin{enumerate}
        \begin{multicols}{2}
            \item $\frac{1}{2\pi}$
            \item $\frac{1}{\pi}$
            \item $\frac{2}{\pi}$
            \item $\pi$
        \end{multicols}
    \end{enumerate}
    
    \hfill{\brak{\text{GATE EC 2015}}}

    \item Consider the signals $s\brak{t}=m\brak{t}\cos\brak{2\pi f_{c}t}+\hat{m}\brak{t}\sin\brak{2\pi f_{c}t}$ where $\hat{m}\brak{t}$ denotes the Hilbert transform of $m\brak{t}$ and the bandwidth of $m\brak{t}$ is very small compared to $f_{c}$. The signal $s\brak{t}$ is a
    \begin{enumerate}
        \item high-pass signal
        \item low-pass signal
        \item band-pass signal
        \item double sideband suppressed carrier signal
    \end{enumerate}
    
    \hfill{\brak{\text{GATE EC 2015}}}

    \item Consider a straight, infinitely long, current carrying conductor lying on the z-axis. Which one of the following plots \brak{in linear scale} qualitatively represents the dependence of $H_{\phi}$ on r, where $H_{\phi}$ is the magnitude of the azimuthal component of magnetic field outside the conductor and r is the radial distance from the conductor?
    \begin{enumerate}
        \item \begin{figure}[H]\centering\includegraphics[width=0.4\columnwidth]{figs/q34A.png}\end{figure}
        \item \begin{figure}[H]\centering\includegraphics[width=0.4\columnwidth]{figs/q34B.png}\end{figure}
        \item \begin{figure}[H]\centering\includegraphics[width=0.4\columnwidth]{figs/q34C.png}\end{figure}
        \item \begin{figure}[H]\centering\includegraphics[width=0.4\columnwidth]{figs/q34D.png}\end{figure}
    \end{enumerate}
    
    \hfill{\brak{\text{GATE EC 2015}}}

    \item The electric field component of a plane wave traveling in a lossless dielectric medium is given by $\vec{E}\brak{z,t}=\hat{a}_{y}2 \cos\brak{10^{8}t-\frac{z}{\sqrt{2}}}V/m$. The wavelength \brak{in m} for the wave is \underline{\hspace{2cm}}.
    
    \hfill{\brak{\text{GATE EC 2015}}}

    \item The solution of the differential equation $\frac{d^{2}y}{dt^{2}}+2\frac{dy}{dt}+y=0$ with $y\brak{0}=y'\brak{0}=1$ is
    \begin{enumerate}
        \begin{multicols}{2}
            \item $\brak{2-t}e^{t}$
            \item $\brak{1+2t}e^{-t}$
            \item $\brak{2+t}e^{-t}$
            \item $\brak{1-2t}e^{t}$
        \end{multicols}
    \end{enumerate}
    
    \hfill{\brak{\text{GATE EC 2015}}}

    \item A vector $\vec{P}$ is given by $\vec{P}=x^{3}y\vec{a}_{x}-x^{2}y^{2}\vec{a}_{y}-x^{2}yz\vec{a}_{z}$. Which one of the following statements is TRUE?
    \begin{enumerate}
        \item $\vec{P}$ is solenoidal, but not irrotational
        \item $\vec{P}$ is irrotational, but not solenoidal
        \item $\vec{P}$ is neither solenoidal nor irrotational
        \item $\vec{P}$ is both solenoidal and irrotational
    \end{enumerate}
    
    \hfill{\brak{\text{GATE EC 2015}}}

    \item Which one of the following graphs describes the function $f\brak{x}=e^{-x}\brak{x^{2}+x+1}$?
    \begin{enumerate}
        \item \begin{figure}[H]\centering\includegraphics[width=0.4\columnwidth]{figs/q38A.png}\end{figure}
        \item \begin{figure}[H]\centering\includegraphics[width=0.4\columnwidth]{figs/q38B.png}\end{figure}
        \item \begin{figure}[H]\centering\includegraphics[width=0.4\columnwidth]{figs/q38C.png}\end{figure}
        \item \begin{figure}[H]\centering\includegraphics[width=0.4\columnwidth]{figs/q38D.png}\end{figure}
    \end{enumerate}
    
    \hfill{\brak{\text{GATE EC 2015}}}

    \item The maximum area \brak{in square units} of a rectangle whose vertices lie on the ellipse $x^{2}+4y^{2}=1$ is \underline{\hspace{2cm}}.
    
    \hfill{\brak{\text{GATE EC 2015}}}

    \item The damping ratio of a series RLC circuit can be expressed as
    \begin{enumerate}
        \begin{multicols}{2}
            \item $\frac{R^{2}C}{2L}$
            \item $\frac{2L}{R^{2}C}$
            \item $\frac{R}{2}\sqrt{\frac{C}{L}}$
            \item $\frac{2}{R}\sqrt{\frac{L}{C}}$
        \end{multicols}
    \end{enumerate}
    
    \hfill{\brak{\text{GATE EC 2015}}}

    \item In the circuit shown, switch SW is closed at $t=0$. Assuming zero initial conditions, the value of $v_{c}\brak{t}$ \brak{in Volts} at $t=1$ sec is \underline{\hspace{2cm}}.
    \begin{figure}[H]
        \centering
        \includegraphics[width=0.5\columnwidth]{figs/q41.png}
        \caption*{}
        \label{fig:q41}
    \end{figure}
    
    \hfill{\brak{\text{GATE EC 2015}}}

    \item In the given circuit, the maximum power \brak{in Watts} that can be transferred to the load $R_{L}$ is \underline{\hspace{2cm}}.
    \begin{figure}[H]
        \centering
        \includegraphics[width=0.5\columnwidth]{figs/q42.png}
        \caption*{}
        \label{fig:q42}
    \end{figure}
    
    \hfill{\brak{\text{GATE EC 2015}}}

    \item The built-in potential of an abrupt p-n junction is 0.75 V. If its junction capacitance \brak{$C_{J}$} at a reverse bias \brak{$V_{R}$} of 1.25 V is 5 pF, the value of $C_{J}$ \brak{in pF} when $V_{R}=7.25 V$ is \underline{\hspace{2cm}}.
    
    \hfill{\brak{\text{GATE EC 2015}}}

    \item A MOSFET in saturation has a drain current of 1 mA for $V_{DS}=0.5 V$. If the channel length modulation coefficient is 0.05 $V^{-1}$, the output resistance \brak{in k$\ohm$} of the MOSFET is \underline{\hspace{2cm}}.
    
    \hfill{\brak{\text{GATE EC 2015}}}

    \item For a silicon diode with long P and N regions, the accepter and donor impurity concentrations are $1\times10^{17}cm^{-3}$ and $1\times10^{15}cm^{-3}$ respectively. The lifetimes of electrons in P region and holes in N region are both 100 µs. The electron and hole diffusion coefficients are 49 $cm^{2}/s$ and 36 $cm^{2}/s$, respectively. Assume kT/q = 26 mV, the intrinsic carrier concentration is $1\times10^{10}cm^{-3}$ and $q=1.6\times10^{-19}C$. When a forward voltage of 208 mV is applied across the diode, the hole current density \brak{in $nA/cm^{2}$} injected from P region to N region is \underline{\hspace{2cm}}.
    
    \hfill{\brak{\text{GATE EC 2015}}}

    \item The Boolean expression $F\brak{X,Y,Z}=\overline{X}Y\overline{Z}+X\overline{Y}\overline{Z}+X Y\overline{Z}+XYZ$ converted into the canonical product of sum \brak{POS} form is
    \begin{enumerate}
        \item $\brak{X+Y+Z}\brak{X+Y+\overline{Z}}\brak{X+\overline{Y}+\overline{Z}}\brak{\overline{X}+Y+\overline{Z}}$
        \item $\brak{X+\overline{Y}+Z}\brak{\overline{X}+Y+\overline{Z}}\brak{\overline{X}+\overline{Y}+Z}\brak{\overline{X}+\overline{Y}+\overline{Z}}$
        \item $\brak{X+Y+Z}\brak{\overline{X}+Y+\overline{Z}}\brak{X+\overline{Y}+Z}\brak{\overline{X}+\overline{Y}+\overline{Z}}$
        \item $\brak{X+\overline{Y}+\overline{Z}}\brak{\overline{X}+Y+Z}\brak{\overline{X}+\overline{Y}+Z}\brak{X+Y+Z}$
    \end{enumerate}
    
    \hfill{\brak{\text{GATE EC 2015}}}

    \item All the logic gates shown in the figure have a propagation delay of 20 ns. Let $A=C=0$ and $B=1$ until time $t=0$. At $t=0$, all the inputs flip \brak{i.e., $A=C=1$ and $B=0$} and remain in that state. For $t>0$, output $Z=1$ for a duration \brak{in ns} of \underline{\hspace{2cm}}.
    \begin{figure}[H]
        \centering
        \includegraphics[width=0.4\columnwidth]{figs/q47.png}
        \caption*{}
        \label{fig:q47}
    \end{figure}
    
    \hfill{\brak{\text{GATE EC 2015}}}

    \item A 3-input majority gate is defined by the logic function $M\brak{a,b,c}=ab+bc+ca$. Which one of the following gates is represented by the function $M\brak{\overline{M\brak{a,b,c}}, M\brak{a,b,\overline{c}},c}$?
    \begin{enumerate}
    \begin{multicols}{2}
        \item 3-input NAND gate
        \item 3-input XOR gate
        \item 3-input NOR gate
        \item 3-input XNOR gate
    \end{multicols}
    \end{enumerate}
    
    \hfill{\brak{\text{GATE EC 2015}}}

    \item For the NMOSFET in the circuit shown, the threshold voltage is $V_{th}$, where $V_{th}>0$. The source voltage $V_{SS}$ is varied from 0 to $V_{DD}$. Neglecting the channel length modulation, the drain current $I_{D}$ as a function of $V_{SS}$ is represented by
    \begin{figure}[H]
        \centering
        \includegraphics[width=0.3\columnwidth]{figs/q49.png}
        \caption*{}
        \label{fig:q49}
    \end{figure}
    \begin{enumerate}
        \item \begin{figure}[H]\centering\includegraphics[width=0.4\columnwidth]{figs/q49A.png}\end{figure}
        \item \begin{figure}[H]\centering\includegraphics[width=0.4\columnwidth]{figs/q49B.png}\end{figure}
        \item \begin{figure}[H]\centering\includegraphics[width=0.4\columnwidth]{figs/q49C.png}\end{figure}
        \item \begin{figure}[H]\centering\includegraphics[width=0.4\columnwidth]{figs/q49D.png}\end{figure}
    \end{enumerate}
    
    \hfill{\brak{\text{GATE EC 2015}}}

    \item In the circuit shown, assume that the opamp is ideal. The bridge output voltage $V_{0}$ \brak{in mV} for $\delta=0.05$ is \underline{\hspace{2cm}}.
    \begin{figure}[H]
        \centering
        \includegraphics[width=0.7\columnwidth]{figs/q50.png}
        \caption*{}
        \label{fig:q50}
    \end{figure}
    
    \hfill{\brak{\text{GATE EC 2015}}}

    \item The circuit shown in the figure has an ideal opamp. The oscillation frequency and the condition to sustain the oscillations, respectively, are
    \begin{figure}[H]
        \centering
        \includegraphics[width=0.5\columnwidth]{figs/q51.png}
        \caption*{}
        \label{fig:q51}
    \end{figure}
    \begin{enumerate}
        \item $\frac{1}{CR}$ and $R_{1}=R_{2}$
        \item $\frac{1}{CR}$ and $R_{1}=4R_{2}$
        \item $\frac{1}{2CR}$ and $R_{1}=R_{2}$
        \item $\frac{1}{2CR}$ and $R_{1}=4R_{2}$
    \end{enumerate}
    
    \hfill{\brak{\text{GATE EC 2015}}}

    \item In the circuit shown, $I_{1}=80$ mA and $I_{2}=4$ mA. Transistors $T_{1}$ and $T_{2}$ are identical. Assume that the thermal voltage $V_{T}$ is 26 mV at $27^{\circ}$C. At $50^{\circ}$C, the value of the voltage $V_{12}=V_{1}-V_{2}$ \brak{in mV} is \underline{\hspace{2cm}}.
    \begin{figure}[H]
        \centering
        \includegraphics[width=0.3\columnwidth]{figs/q52.png}
        \caption*{}
        \label{fig:q52}
    \end{figure}
    
    \hfill{\brak{\text{GATE EC 2015}}}

    \item Two sequences $[a,b,c]$ and $[A,B,C]$ are related as,
    \[ \myvec{A\\ B\\ C} = \myvec{1 & 1 & 1\\ 1 & W_{3}^{-1} & W_{3}^{-2}\\ 1 & W_{3}^{-2} & W_{3}^{-4}} \myvec{a\\ b\\ c} \]
    where $W_{3}=e^{j\frac{2\pi}{3}}$. If another sequence $[p,q,r]$ is derived as,
    \[ \myvec{p\\ q\\ r} = \myvec{1 & 1 & 1\\ 1 & W_{3}^{1} & W_{3}^{2}\\ 1 & W_{3}^{2} & W_{3}^{4}} \myvec{1 & 0 & 0\\ 0 & W_{3}^{2} & 0\\ 0 & 0 & W_{3}^{4}} \myvec{A/3\\ B/3\\ C/3} \]
    then the relationship between the sequences $[p,q,r]$ and $[a,b,c]$ is
    \begin{enumerate}
        \item $[p,q,r]=[b,a,c]$
        \item $[p,q,r]=[b,c,a]$
        \item $[p,q,r]=[c,a,b]$
        \item $[p,q,r]=[c,b,a]$
    \end{enumerate}
    
    \hfill{\brak{\text{GATE EC 2015}}}

    \item For the discrete-time system shown in the figure, the poles of the system transfer function are located at
    \begin{figure}[H]
        \centering
        \includegraphics[width=0.6\columnwidth]{figs/q54.png}
        \caption*{}
        \label{fig:q54}
    \end{figure}
    \begin{enumerate}
        \item 2, 3
        \item $\frac{1}{2}$, 3
        \item $\frac{1}{2}$, $\frac{1}{3}$
        \item 2, $\frac{1}{3}$
    \end{enumerate}
    
    \hfill{\brak{\text{GATE EC 2015}}}

    \item The pole-zero diagram of a causal and stable discrete-time system is shown in the figure. The zero at the origin has multiplicity 4. The impulse response of the system is $h[n]$. If $h[0]=1$, we can conclude
    \begin{figure}[H]
        \centering
        \includegraphics[width=0.4\columnwidth]{figs/q55.png}
        \caption*{}
        \label{fig:q55}
    \end{figure}
    \begin{enumerate}
        \item $h[n]$ is real for all n
        \item $h[n]$ is purely imaginary for all n
        \item $h[n]$ is real for only even n
        \item $h[n]$ is purely imaginary for only odd n
    \end{enumerate}
    
    \hfill{\brak{\text{GATE EC 2015}}}

    \item The open-loop transfer function of a plant in a unity feedback configuration is given as $G\brak{s}=\frac{K\brak{s+4}}{\brak{s+8}\brak{s^{2}-9}}$. The value of the gain $K\brak{>0}$ for which $-1+j2$ lies on the root locus is \underline{\hspace{2cm}}.
    
    \hfill{\brak{\text{GATE EC 2015}}}

    \item A lead compensator network includes a parallel combination of R and C in the feed-forward path. If the transfer function of the compensator is $G_{c}\brak{s}=\frac{s+2}{s+4}$, the value of RC is \underline{\hspace{2cm}}.
    
    \hfill{\brak{\text{GATE EC 2015}}}

    \item A plant transfer function is given as $G\brak{s}=\brak{K_{p}+\frac{K_{I}}{s}}\frac{1}{s\brak{s+2}}$. When the plant operates in a unity feedback configuration, the condition for the stability of the closed loop system is
    \begin{enumerate}
        \begin{multicols}{2}
            \item $K_{p}>\frac{K_{I}}{2}>0$
            \item $2K_{I}>K_{p}>0$
            \item $2K_{I}<K_{P}$
            \item $2K_{I}>K_{P}$
        \end{multicols}
    \end{enumerate}
    
    \hfill{\brak{\text{GATE EC 2015}}}

    \item The input X to the Binary Symmetric Channel \brak{BSC} shown in the figure is '1' with probability 0.8. The cross-over probability is $1/7$. If the received bit $Y=0$, the conditional probability that '1' was transmitted is \underline{\hspace{2cm}}.
    \begin{figure}[H]
        \centering
        \includegraphics[width=0.4\columnwidth]{figs/q59.png}
        \caption*{}
        \label{fig:q59}
    \end{figure}
    
    \hfill{\brak{\text{GATE EC 2015}}}

    \item The transmitted signal in a GSM system is of 200 kHz bandwidth and 8 users share a common bandwidth using TDMA. If at a given time 12 users are talking in a cell, the total bandwidth of the signal received by the base station of the cell will be at least \brak{in kHz} \underline{\hspace{2cm}}.
    
    \hfill{\brak{\text{GATE EC 2015}}}

    \item In the system shown in Figure \brak{a}, $m\brak{t}$ is a low-pass signal with bandwidth W Hz. The frequency response of the band-pass filter $H\brak{f}$ is shown in Figure \brak{b}. If it is desired that the output signal $z\brak{t}=10x\brak{t}$, the maximum value of W \brak{in Hz} should be strictly less than \underline{\hspace{2cm}}.
    \begin{figure}[H]
        \centering
        \includegraphics[width=0.8\columnwidth]{figs/q61.png}
        \caption*{}
        \label{fig:q61}
    \end{figure}
    
    \hfill{\brak{\text{GATE EC 2015}}}

    \item A source emits bit 0 with probability $\frac{1}{3}$ and bit 1 with probability $\frac{2}{3}$. The emitted bits are communicated to the receiver. The receiver decides for either 0 or 1 based on the received value R. It is given that the conditional density functions of R are as\\$f_{R|0}\brak{r}=\begin{cases}\frac{1}{4}, & -3\le r\le1,\\ 0, & otherwise,\end{cases}$ and $f_{R|1}\brak{r}=\begin{cases}\frac{1}{6}, & -1\le r\le5,\\ 0, & otherwise.\end{cases}$\\The minimum decision error probability is
    \begin{enumerate}
        \begin{multicols}{2}
            \item 0
            \item $1/12$
            \item $1/9$
            \item $1/6$
        \end{multicols}
    \end{enumerate}
    
    \hfill{\brak{\text{GATE EC 2015}}}

    \item The longitudinal component of the magnetic field inside an air-filled rectangular waveguide made of a perfect electric conductor is given by the following expression\\$H_{z}\brak{x,y,z,t}=0.1 \cos\brak{25\pi x}\cos\brak{30.3 \pi y}\cos\brak{12\pi\times10^{9}t-\beta z}\brak{A/m}$.\\The cross-sectional dimensions of the waveguide are given as $a=0.08$ m and $b=0.033$ m. The mode of propagation inside the waveguide is
    \begin{enumerate}
        \begin{multicols}{2}
            \item $TM_{12}$
            \item $TM_{21}$
            \item $TE_{21}$
            \item $TE_{12}$
        \end{multicols}
    \end{enumerate}
    
    \hfill{\brak{\text{GATE EC 2015}}}

    \item The electric field intensity of a plane wave traveling in free space is given by the following expression\\$E\brak{x,t}=a_{y}24\pi \cos\brak{\omega t-k_{0}x}\brak{V/m}$.\\In this field, consider a square area 10 cm x 10 cm on a plane $x+y=1$. The total time-averaged power \brak{\texin mW} passing through the square area is \underline{\hspace{2cm}}.
    
    \hfill{\brak{\text{GATE EC 2015}}}

    \item Consider a uniform plane wave with amplitude \brak{$E_{0}$} of 10 V/m and 1.1 GHz frequency travelling in air, and incident normally on a dielectric medium with complex relative permittivity \brak{$\epsilon_{r}$} and permeability \brak{$\mu_{r}$} as shown in the figure.
    \begin{figure}[H]
        \centering
        \includegraphics[width=0.7\columnwidth]{figs/q65.png}
        \caption*{}
        \label{fig:q65}
    \end{figure}
    The magnitude of the transmitted electric field component \brak{in V/m} after it has travelled a distance of 10 cm inside the dielectric region is \underline{\hspace{2cm}}.
    
    \hfill{\brak{\text{GATE EC 2015}}}

\end{enumerate}
\end{document}

