\documentclass[12pt]{article}
\usepackage{amsmath}
\usepackage{graphicx}
\usepackage{caption}
\usepackage{geometry}
\geometry{margin=1in}
\usepackage{gvv
\usepackage{gvv-book}

\title{Multiple Choice Questions}
\date{}

\begin{document}

\maketitle

\textbf{Q.1 -- Q.20 Carry one mark each}

\begin{enumerate}

   \item   The total number of isomers of Co(en)\textsubscript{2}Cl\textsubscript{2} (en = ethylenediamine) is\\
    (A) 4 \quad (B) 3 \quad (C) 6 \quad (D) 5                         
      \textbf{(GATE-EE 2025)}

    \vspace{0.5cm}

   \textbf{Q.2}  Metal-metal quadruple bonds are well-known for the metal\\
    (A) Ni \quad (B) Co \quad (C) Fe \quad (D) Re   
  \textbf{(GATE-EE 2025)}
    
    \vspace{0.5cm}

    \textbf{Q.3} The reaction of Al\textsubscript{4}C\textsubscript{3} with water leads to the formation of\\
    (A) methane \quad (B) propyne \quad (C) propene \quad (D) propane  
  \textbf{(GATE-EE 2025)}

    \vspace{0.5cm}

   \textbf{Q.4}  The correct statement about C\textsubscript{60} is\\
    (A) C\textsubscript{60} is soluble in benzene\\
    (B) C\textsubscript{60} does not react with \textit{tert}-butyllithium\\
    (C) C\textsubscript{60} is made up of 10 five-membered and 15 six-membered rings\\
    (D) Two adjacent five-membered rings share a common edge   \textbf{(GATE-EE 2025)}

    \vspace{0.5cm}

   \textbf{Q.5} The lattice parameters for a monoclinic crystal are\\
    (A) $a \neq b \neq c; \ \alpha = \gamma = 90^\circ$\\
    (B) $a = b \neq c; \ \alpha \neq \beta \neq \gamma$\\
    (C) $a \neq b \neq c; \ \alpha \neq \beta \neq \gamma$\\
    (D) $a = b = c; \ \alpha = \gamma = 90^\circ$   \textbf{(GATE-EE 2025)}


    \vspace{0.5cm}

    \textbf{Q.6} The magnetic moment of [Ru(H\textsubscript{2}O)\textsubscript{6}]\textsuperscript{2+} corresponds to the presence of\\
    (A) four unpaired electrons \quad
    (B) three unpaired electrons\\
    (C) two unpaired electrons \quad
    (D) zero unpaired electrons   \textbf{(GATE-EE 2025)}


    \vspace{0.5cm}

\textbf{Q.7} \quad The compound that is \textbf{NOT} aromatic is

\begin{center}
  \includegraphics[width=0.6\textwidth]{q7.png} 
\end{center}

\bigskip
   \textbf{(GATE-EE 2025)}


\textbf{Q.8} The order of stability for the following cyclic olefins is

\begin{center}
  \includegraphics[width=0.7\textwidth]{q8.png} 
\end{center}

\begin{itemize}
  \item[(A)] I \(<\) II \(<\) III \(<\) IV
  \item[(B)] I \(<\) III \(<\) IV \(<\) I
  \item[(C)] II \(<\) III \(<\) I \(<\) IV
  \item[(D)] IV \(<\) II \(<\) I \(<\) III
\end{itemize}   \textbf{(GATE-EE 2025)}


\textbf{Q.9} The most acidic species is

\begin{center}
  \includegraphics[width=0.7\textwidth]{q9.png} 
\end{center}   \textbf{(GATE-EE 2025)}


\vspace{0.5cm}

\textbf{Q.10} The major product of the following reaction is

\begin{center}
  \includegraphics[width=0.7\textwidth]{q10.png} 
\end{center}   \textbf{(GATE-EE 2025)}


\vspace{0.5cm} 

\textbf{Q.11} In the carbylamine reaction, R–X is converted to R–Y \textit{via} the intermediate Z.\\
R–X, R–Y and Z, respectively, are

    \item[(A)] R–NH\textsubscript{2}, R–NC, carbene
    \item[(B)] R–NH\textsubscript{2}, R–NC, nitrene
    \item[(C)] R–NC, R–NH\textsubscript{2}, carbene
    \item[(D)] R–OH, R–NC, nitrene    \textbf{(GATE-EE 2025)}


    \vspace{0.5cm}

   \textbf{Q.12} \quad The compound that is \textbf{NOT} oxidized by KMnO$_4$ is

\begin{center}
    \includegraphics[width=0.6\textwidth]{q12.png}
\end{center}   \textbf{(GATE-EE 2025)}


 \vspace{0.5cm}
 
   \textbf{Q.13}  Cyanogen bromide (CNBr) specifically hydrolyses the peptide bond formed by the C-side of\\
    (A) methionine \quad (B) glycine \quad (C) proline \quad (D) serine   \textbf{(GATE-EE 2025)}


    \vspace{0.5cm}

    \textbf{Q.14} The Hammett reaction constant $\rho$ is based on\\
    (A) the rates of alkaline hydrolysis of substituted ethyl benzoates\\
    (B) the dissociation constants of substituted acetic acids\\
    (C) the dissociation constants of substituted benzoic acids\\
    (D) the dissociation constants of substituted phenols   \textbf{(GATE-EE 2025)}


    \vspace{0.5cm}

    \textbf{Q.15} The lifetime of a molecule in an excited electronic state is $10^{-10}$ s. The uncertainty in the energy (eV) approximately is\\
    (A) $2 \times 10^5$ \quad (B) $3 \times 10^6$ \quad (C) 0 \quad (D) $10^{-14}$   \textbf{(GATE-EE 2025)}


    \vspace{0.5cm}

    \textbf{Q.16}For a one component system, the maximum number of phases that can coexist at equilibrium is\\
    (A) 3 \quad (B) 2 \quad (C) 1 \quad (D) 4   \textbf{(GATE-EE 2025)}


    \vspace{0.5cm}

    \textbf{Q.17} At $T = 300 \ \mathrm{K}$, the thermal energy ($k_B T$) in cm\textsuperscript{-1} is approximately\\
    (A) 20000 \quad (B) 8000 \quad (C) 5000 \quad (D) 200   \textbf{(GATE-EE 2025)}


    \vspace{0.5cm}

    \textbf{Q.18} For the reaction $2 X_3 \rightarrow 3 X_2$, the rate of formation of $X_2$ is

(A) $3 \left(-\frac{d[X_3]}{dt}\right)$ \hspace{1cm}
(B) $\frac{1}{2} \left(-\frac{d[X_3]}{dt}\right)$ \hspace{1cm}
(C) $\frac{1}{3} \left(-\frac{d[X_3]}{dt}\right)$ \hspace{1cm}
(D) $\frac{3}{2} \left(-\frac{d[X_3]}{dt}\right)$   \textbf{(GATE-EE 2025)}


\vspace{0.5cm}

\textbf{Q.19} The highest occupied molecular orbital of HF is

(A) bonding \hspace{1cm}
(B) antibonding \hspace{1cm}
(C) ionic \hspace{1cm}
(D) nonbonding   \textbf{(GATE-EE 2025)}


\vspace{0.5cm}

\textbf{Q.20} The residual entropy of the asymmetric molecule N$_2$O in its crystalline state is $5.8\ \mathrm{J\ K^{-1}\ mol^{-1}}$ at absolute zero. The number of orientations that can be adopted by N$_2$O in its crystalline state is

(A) 4 \hspace{1cm}
(B) 3 \hspace{1cm}
(C) 2 \hspace{1cm}
(D) 1   \textbf{(GATE-EE 2025)}


\vspace{0.5cm}

\textbf{Q.21 to Q.75 Carry two marks each}

\vspace{0.5cm}

\textbf{Q.21} The spectroscopic ground state symbol and the total number of electronic transitions of [Ti(H$_2$O)$_6$]$^{3+}$ are

(A) $^3T_{1g}$ and 2 \hspace{1cm}
(B) $^3A_{2g}$ and 3 \hspace{1cm}
(C) $^1T_{1g}$ and 3 \hspace{1cm}
(D) $^3A_{2g}$ and 2   \textbf{(GATE-EE 2025)}


\vspace{0.5cm}

\textbf{Q.22} The structures of the complexes [Cu(NH$_3$)$_4$](ClO$_4$)$_2$ and [Cu(NH$_3$)$_4$](ClO$_4$) in solution respectively are

(A) square planar and tetrahedral \hspace{1cm}
(B) octahedral and square pyramidal\\
(C) octahedral and trigonal bipyramidal \hspace{1cm}
(D) tetrahedral and square planar   \textbf{(GATE-EE 2025)}


\vspace{0.5cm}

\textbf{Q.23} In biological systems, the metal ions involved in electron transport are

(A) Na$^+$ and K$^+$ \hspace{1cm}
(B) Zn$^{2+}$ and Mg$^{2+}$ \hspace{1cm}
(C) Ca$^{2+}$ and Mg$^{2+}$ \hspace{1cm}
(D) Cu$^{2+}$ and Fe$^{3+}$   \textbf{(GATE-EE 2025)}


\vspace{0.5cm}

\textbf{Q.24} In a homogeneous catalytic reaction, 1.0 M of a substrate and 1.0 $\mu$M of a catalyst yields 1.0 mM of a product in 10 seconds. The turnover frequency (TOF) of the reaction (s$^{-1}$) is

(A) $10^2$ \hspace{1cm}
(B) $10^1$ \hspace{1cm}
(C) $10^{-3}$ \hspace{1cm}
(D) $10^3$   \textbf{(GATE-EE 2025)}


\textbf{Q.25} The expected magnetic moments of the first-row transition metal complexes and those of the lanthanide metal complexes are usually calculated using

(A) $\mu_{\text{so}}$ equation (s.o. = spin only) for both lanthanide and transition metal complexes\\
(B) $\mu_{\text{so}}$ equation for lanthanide metal complexes and $\mu$ equation for transition metal complexes\\
(C) $\mu_{\text{so}}$ equation for transition metal complexes and $\mu$ equation for lanthanide metal complexes\\
(D) $\mu_{\text{eff}}$ equation for transition metal complexes and $\mu_{\text{so}}$ equation for lanthanide metal complexes   \textbf{(GATE-EE 2025)}


\vspace{0.5cm}

\textbf{Q.26} \quad The Brønsted acidity of boron hydrides follows the order

\begin{enumerate}
    \item[(A)] $\mathrm{B_2H_6 > B_4H_{10} > B_5H_9 > B_{10}H_{14}}$
    \item[(B)] $\mathrm{B_2H_6 = B_4H_{10} > B_5H_9 = B_{10}H_{14}}$
    \item[(C)] $\mathrm{B_{10}H_{14} > B_5H_9 > B_4H_{10} > B_2H_6}$
    \item[(D)] $\mathrm{B_5H_9 > B_4H_{10} > B_2H_6 > B_{10}H_{14}}$
\end{enumerate} \textbf{(GATE-EE 2025)}

\vspace{0.5cm}

\textbf{Q.27} NaCl is crystallised by slow evaporation of its aqueous solution at room temperature. The correct statement is

(A) The crystals will be non-stoichiometric\\
(B) The crystals should have Frenkel defects\\
(C) The percentage of defects in the crystals will depend on the concentration of the solution and its rate of evaporation\\
(D) The nature of defects will depend upon the concentration of the solution and its rate of evaporation   \textbf{(GATE-EE 2025)}


\vspace{0.5cm}

\textbf{Q.28} CaTiO$_3$ has a perovskite crystal structure. The coordination number of titanium in CaTiO$_3$ is

(A) 9 \hspace{1cm}
(B) 6 \hspace{1cm}
(C) 3 \hspace{1cm}
(D) 12   \textbf{(GATE-EE 2025)}


\vspace{0.5cm}

\textbf{Q.29} If ClF$_5$ were to be stereochemically rigid, its $^{19}$F NMR spectrum (I for $^{19}$F = $\frac{1}{2}$) would be (assume that Cl is not NMR active)

(A) a doublet and a triplet\\
(B) a singlet\\
(C) a doublet and a singlet\\
(D) two singlets   \textbf{(GATE-EE 2025)}


\vspace{0.5cm}

\textbf{Q.30} The point group of NSF$_3$ is

(A) D$_{3d}$ \hspace{1cm}
(B) C$_{3h}$ \hspace{1cm}
(C) D$_{3h}$ \hspace{1cm}
(D) C$_{3v}$   \textbf{(GATE-EE 2025)}


\textbf{Q.31} When NiO is heated with a small amount of Li$_2$O in air at 1200$^\circ$C, a non-stoichiometric compound Li$_x$Ni$_{1-x}$O is formed. This compound is

(A) an n-type semiconductor containing only Ni$^{1+}$\\
(B) an n-type semiconductor containing Ni$^{1+}$ and Ni$^{2+}$\\
(C) a p-type semiconductor containing Ni$^{2+}$ and Ni$^{3+}$\\
(D) a p-type semiconductor containing only Ni$^{3+}$   \textbf{(GATE-EE 2025)}


\vspace{0.5cm}

\textbf{Q.32} White phosphorus, P$_4$, belongs to the

(A) \textit{closo} system \hspace{1cm}
(B) \textit{nido} system \hspace{1cm}
(C) \textit{arachno} system \hspace{1cm}
(D) \textit{hypho} system   \textbf{(GATE-EE 2025)}


\vspace{0.5cm}

\textbf{Q.33} Among the compounds Fe$_3$O$_4$, NiFe$_2$O$_4$ and Mn$_3$O$_4$

(A) NiFe$_2$O$_4$ and Mn$_3$O$_4$ are normal spinels\\
(B) Fe$_3$O$_4$ and Mn$_3$O$_4$ are normal spinels\\
(C) Fe$_3$O$_4$ and Mn$_3$O$_4$ are inverse spinels\\
(D) Fe$_3$O$_4$ and NiFe$_2$O$_4$ are inverse spinels   \textbf{(GATE-EE 2025)}


\vspace{0.5cm}

\textbf{Q.34} The number of M-M bonds in Ir$_4$(CO)$_{12}$ are

(A) four \hspace{1cm}
(B) six \hspace{1cm}
(C) eight \hspace{1cm}
(D) zero   \textbf{(GATE-EE 2025)}


\vspace{0.5cm}

\textbf{Q.35} Schrock carbenes are

(A) triplets and nucleophilic \hspace{1cm}
(B) triplets and electrophilic\\
(C) singlets and nucleophilic \hspace{1cm}
(D) singlets and electrophilic   \textbf{(GATE-EE 2025)}


\vspace{0.5cm}

\textbf{Q.36} The \textbf{INCORRECT} statement about linear dimethylpolysiloxane, [(CH$_3$)$_2$SiO]$_n$, is

(A) it is extremely hydrophilic\\
(B) it is prepared by a KOH catalysed ring-opening reaction of [Me$_2$SiO]$_4$\\
(C) it has a very low glass transition temperature\\
(D) it can be reinforced to give silicon elastomers   \textbf{(GATE-EE 2025)}


\vspace{0.5cm}

\textbf{Q.37} Match the entries \textbf{a–d} with their corresponding structures \textbf{p–s}.\\[1em]

\begin{center}
    \includegraphics[width=0.8\textwidth]{q37.png}
\end{center}
\begin{itemize}
    \item a - s,\quad b - r,\quad c - q,\quad d - p
    \item a - p,\quad b - s,\quad c - q,\quad d - r
    \item a - q,\quad b - p,\quad c - s,\quad d - r
    \item a - s,\quad b - r,\quad c - p,\quad d - q
\end{itemize}   \textbf{(GATE-EE 2025)}


\textbf{Q.38} The reaction between \textbf{X} and \textbf{Y} to give \textbf{Z} proceeds via

\begin{center}
    \includegraphics[width=0.8\textwidth]{q38.png}
\end{center}
\begin{enumerate}
    \item [(A)]$4\pi$-conrotatory opening of X followed by \textit{endo} Diels--Alder cycloaddition
    \item [(B)]$4\pi$-disrotatory opening of X followed by \textit{endo} Diels--Alder cycloaddition
    \item [(C)]$4\pi$-conrotatory opening of X followed by \textit{exo} Diels--Alder cycloaddition
    \item [(D)] $4\pi$-disrotatory opening of X followed by \textit{exo} Diels--Alder cycloaddition
\end{enumerate}   \textbf{(GATE-EE 2025)}


\textbf{Q.39} \quad The major products $P_1$ and $P_2$, respectively, in the following reaction sequence are

\begin{center}
    \includegraphics[width=0.8\textwidth]{q39.png}
\end{center}   \textbf{(GATE-EE 2025)}


\textbf{Q.40} \quad The products $Y$ and $Z$ are formed, respectively, from $X$ via

\begin{center}
    \includegraphics[width=0.8\textwidth]{q40.png}
\end{center}
\begin{enumerate}
    \item [(A)]$h\nu$, conrotatory opening and $\Delta$, disrotatory opening
    \item [(B)]$h\nu$, disrotatory opening and $\Delta$, conrotatory opening
    \item [(C)]$\Delta$, conrotatory opening and $h\nu$, disrotatory opening
    \item [(D)]$\Delta$, disrotatory opening and $h\nu$, conrotatory opening
\end{enumerate}   \textbf{(GATE-EE 2025)}



\textbf{Q.41} \textit{o}-Bromophenol is readily prepared from phenol using the following conditions:

\item[(A)] i) (CH$_3$CO)$_2$O; \quad ii) Br$_2$; \quad iii) HCl–H$_2$O, $\Delta$
\item[(B)] i) H$_2$SO$_4$, 100$^\circ$C; \quad ii) Br$_2$; \quad iii) H$_3$O$^+$, 100$^\circ$C
\item[(C)] N-Bromosuccinimide, dibenzoyl peroxide, CCl$_4$, $\Delta$
\item[(D)] Br$_2$/FeBr$_3$   \textbf{(GATE-EE 2025)}


\vspace{0.5cm}

\textbf{Q.42} The major product of the following reaction is

\begin{center}
\includegraphics[width=0.6\textwidth]{q42.png}
\end{center}   \textbf{(GATE-EE 2025)}


\textbf{Q.43} The photochemical reaction of 2-methylpropane with F$_2$ gives 2-fluoro-2-methylpropane and 1-fluoro-2-methylpropane in 14:86 ratio. The corresponding ratio of the bromo products in the above reaction using Br$_2$ is most likely to be:


\item[(A)] 14 : 86
\item[(B)] 50 : 50
\item[(C)] 1 : 9
\item[(D)] 99 : 1   \textbf{(GATE-EE 2025)}


\vspace{0.5cm}

\textbf{Q.44} The major product $P$ of the following reaction is

\begin{center}
\includegraphics[width=0.6\textwidth]{q44.png}
\end{center}   \textbf{(GATE-EE 2025)}


\textbf{Q.45} \quad The reagent \textbf{X} in the following reaction is

\begin{center}
\includegraphics[width=0.6\textwidth]{q45.png}
\end{center}   \textbf{(GATE-EE 2025)}


\vspace{0.5cm}

\textbf{Q.46} The major product of the following reactions is

\begin{center}
\includegraphics[width=0.6\textwidth]{q46.png}
\end{center}   \textbf{(GATE-EE 2025)}


\vspace{0.5cm}

\textbf{Q.47} The major product of the following reaction is

\begin{center}
\includegraphics[width=0.6\textwidth]{q47.png}
\end{center}   \textbf{(GATE-EE 2025)}


\vspace{0.5cm}

\textbf{Q.48} \quad In the following compound, the hydroxy group that is most readily methylated with CH$_2$N$_2$ is

\begin{center}
\includegraphics[width=0.45\textwidth]{q48.png}
\end{center}

\begin{tabbing}
\hspace{1cm} \= (A) p \hspace{2cm} \= (B) q \\
\> (C) r \> (D) s
\end{tabbing}   \textbf{(GATE-EE 2025)}


\textbf{Q.49} \quad The most appropriate sequence of reactions for carrying out the following transformation is

\begin{center}
\includegraphics[width=0.45\textwidth]{q49.png} 
\end{center}

\begin{tabbing}
\hspace{1cm} \= (A) i) O$_3$/H$_2$O$_2$; \quad ii) excess SOCl$_2$/pyridine; \quad iii) excess NH$_3$; \quad iv) LiAlH$_4$ \\
\> (B) i) O$_3$/Me$_2$S; \quad ii) excess SOCl$_2$/pyridine; \quad iii) LiAlH$_4$; \quad iv) excess NH$_3$ \\
\> (C) i) O$_3$/H$_2$O$_2$; \quad ii) excess SOCl$_2$/pyridine; \quad iii) LiAlH$_4$; \quad iv) excess NH$_3$ \\
\> (D) i) O$_3$/Me$_2$S; \quad ii) excess SOCl$_2$/pyridine; \quad iii) excess NH$_3$; \quad iv) LiAlH$_4$ 
\end{tabbing}   \textbf{(GATE-EE 2025)}


\textbf{Q.50} The number of optically active stereoisomers possible for 1,3-cyclohexanediol in its chair conformation is

\item[(A)] 4
\item[(B)] 3
\item[(C)] 2
\item[(D)] 1   \textbf{(GATE-EE 2025)}


\textbf{Q.51} The major product of the following reactions is

\begin{center}
\includegraphics[width=0.6\textwidth]{q51.png}
\end{center}   \textbf{(GATE-EE 2025)}


\vspace{0.5cm}

\textbf{Q.52} In the folloowing reaction,

\begin{center}
\includegraphics[width=0.6\textwidth]{q52.png}
\end{center}

\textbf{Q.} The absolute configurations of the chiral centres in X and Y are

\begin{tabbing}
\hspace{1cm} \= (A) \quad 2S, 3R \quad and \quad 2R, 3R \\
\> (B) \quad 2R, 3R \quad and \quad 2R, 3S \\
\> (C) \quad 2S, 3S \quad and \quad 2R, 3R \\
\> (D) \quad 2S, 3R \quad and \quad 2S, 3R
\end{tabbing}   \textbf{(GATE-EE 2025)}


\textbf{Q.53} \\
The IR stretching frequencies (cm$^{-1}$) for the compound X are as follows: 3300--3500 (s, br); 3000 (m); 2225 (s); 1680 (s). \\

\begin{center}
\includegraphics[width=0.6\textwidth]{q53.png}
\end{center}

The correct assignment of the absorption bands is:

\begin{tabbing}
\hspace{1cm} \= (A) \quad $\bar{\nu}_{\text{OH}} = 3300$--$3500$; $\bar{\nu}_{\text{CH}} = 3000$; $\bar{\nu}_{\text{CN}} = 2225$; $\bar{\nu}_{\text{CO}} = 1680$ \\
\> (B) \quad $\bar{\nu}_{\text{OH}} = 3000$; $\bar{\nu}_{\text{CH}} = 3300$--$3500$; $\bar{\nu}_{\text{CN}} = 2225$; $\bar{\nu}_{\text{CO}} = 1680$ \\
\> (C) \quad $\bar{\nu}_{\text{OH}} = 3300$--$3500$; $\bar{\nu}_{\text{CH}} = 3000$; $\bar{\nu}_{\text{CN}} = 1680$; $\bar{\nu}_{\text{CO}} = 2225$ \\
\> (D) \quad $\bar{\nu}_{\text{OH}} = 3000$; $\bar{\nu}_{\text{CH}} = 3300$--$3500$; $\bar{\nu}_{\text{CN}} = 1680$; $\bar{\nu}_{\text{CO}} = 2225$
\end{tabbing}   \textbf{(GATE-EE 2025)}


\textbf{Q.54} The T$_d$ point group has 24 elements and 5 classes. Given that it has two 3-dimensional irreducible representations, the number of one-dimensional irreducible representations is

\item[(A)] 1
\item[(B)] 6
\item[(C)] 2
\item[(D)] 3   \textbf{(GATE-EE 2025)}


\vspace{0.5cm}

\textbf{Q.55} The total number of ways in which two nonidentical spin $\frac{1}{2}$ particles can be oriented relative to a constant magnetic field is

\item[(A)] 1
\item[(B)] 2
\item[(C)] 3
\item[(D)] 4   \textbf{(GATE-EE 2025)}


\textbf{Q.56} Approximately one hydrogen atom per cubic meter is present in interstellar space. Assuming that the H-atom has a diameter of $10^{-10}$ m, the mean free path (m) approximately is

\begin{enumerate}
\item[(A)] $10^{10}$
\item[(B)] $10^{19}$
\item[(C)] $10^{24}$
\item[(D)] $10^{14}$
\end{enumerate}   \textbf{(GATE-EE 2025)}


\vspace{0.5cm}

\textbf{Q.57} The wavefunction of a diatomic molecule has the form $\psi = 0.89\, \varphi_{\text{covalent}} + 0.45\, \varphi_{\text{ionic}}$. The chance that both electrons of the bond will be found on the same atom in 100 inspections of the molecule approximately is

\begin{enumerate}
\item[(A)] 79
\item[(B)] 20
\item[(C)] 45
\item[(D)] 60
\end{enumerate}   \textbf{(GATE-EE 2025)}


\vspace{0.5cm}

\textbf{Q.58} For the reaction given below, the relaxation time is $10^{-4}$ s. Given that 10\% of A remains at equilibrium, the value of $k_1$ (s$^{-1}$) is

\begin{center}
\includegraphics[width=0.6\textwidth]{q58.png}
\end{center}

\begin{enumerate}
\item[(A)] $9 \times 10^5$
\item[(B)] $10^5$
\item[(C)] $10^6$
\item[(D)] $9 \times 10^6$
\end{enumerate}   \textbf{(GATE-EE 2025)}


\vspace{0.5cm}

\textbf{Q.59} The minimum number of electrons needed to form a chemical bond between two atoms is

\begin{enumerate}
\item[(A)] 1
\item[(B)] 2
\item[(C)] 3
\item[(D)] 4
\end{enumerate}   \textbf{(GATE-EE 2025)}


\vspace{0.5cm}

\textbf{Q.60} The ground state electronic energy (Hartree) of a helium atom, neglecting the inter-electron repulsion, is

\begin{enumerate}
\item[(A)] -1.0
\item[(B)] -0.5
\item[(C)] -2.0
\item[(D)] -4.0
\end{enumerate}   \textbf{(GATE-EE 2025)}


\vspace{0.5cm}

\textbf{Q.61} A particle is confined to a one-dimensional box of length 1 mm. If the length is changed by $10^{-9}$ m, the \% change in the ground state energy is

\begin{enumerate}
\item[(A)] $2 \times 10^4$
\item[(B)] $2 \times 10^7$
\item[(C)] $2 \times 10^2$
\item[(D)] 0
\end{enumerate}   \textbf{(GATE-EE 2025)}


\vspace{0.5cm}

\textbf{Q.62} A certain molecule can be treated as having only a doubly degenerate state lying at 360 cm$^{-1}$ above the nondegenerate ground state. The approximate temperature (K) at which 15\% of the molecules will be in the upper state is

\begin{enumerate}
\item[(A)] 500
\item[(B)] 150
\item[(C)] 200
\item[(D)] 300
\end{enumerate}   \textbf{(GATE-EE 2025)}


\vspace{0.5cm}

\textbf{Q.63} A box of volume $V$ contains one mole of an ideal gas. The probability that all $N$ particles will be found occupying one half of the volume leaving the other half empty is

\begin{enumerate}
\item[(A)] $1/2$
\item[(B)] $2/N$
\item[(C)] $(1/2)^N$
\item[(D)] $(1/2)^{6N}$
\end{enumerate}   \textbf{(GATE-EE 2025)}


\vspace{0.5cm}

\textbf{Q.64} According to the Debye-Hückel limiting law, the mean activity coefficient of $5 \times 10^{-4}~\text{mol kg}^{-1}$ aqueous solution of CaCl$_2$ at 25$^\circ$C is (the Debye-Hückel constant ‘A’ can be taken to be 0.509)

\begin{enumerate}
\item[(A)] 0.63
\item[(B)] 0.72
\item[(C)] 0.80
\item[(D)] 0.91
\end{enumerate}   \textbf{(GATE-EE 2025)}


\vspace{0.5cm}

\textbf{Q.65} The operation of the commutator $[x, d/dx]$ on a function $f(x)$ is equal to

\begin{enumerate}
\item[(A)] 0
\item[(B)] $f(x)$
\item[(C)] $-f(x)$
\item[(D)] $x \frac{df}{dx}$
\end{enumerate}   \textbf{(GATE-EE 2025)}


\vspace{0.5cm}

\textbf{Q.66} If a gas obeys the equation of state $P (V - nb) = nRT$, the ratio $(C_P - C_V)/(C_P - C_V)_{\text{ideal}}$ is

\begin{enumerate}
\item[(A)] $> 1$
\item[(B)] $< 1$
\item[(C)] 1
\item[(D)] $(1 - b)$
\end{enumerate}   \textbf{(GATE-EE 2025)}


\vspace{0.5cm}

\textbf{Q.67} Physisorbed particles undergo desorption at 27$^\circ$C with an activation energy of 16.628 kJ mol$^{-1}$. Assuming first-order process and a frequency factor of $10^{12}$ Hz, the average residence time (in seconds) of the particles on the surface is

\begin{enumerate}
\item[(A)] $8 \times 10^{-10}$
\item[(B)] $8 \times 10^{-11}$
\item[(C)] $2 \times 10^{-9}$
\item[(D)] $1 \times 10^{-12}$
\end{enumerate}   \textbf{(GATE-EE 2025)}


\vspace{0.5cm}

\textbf{Q.68} The rotational constants for CO in the ground and the first excited vibrational states are 1.9 and 1.6 cm$^{-1}$, respectively. The \% change in the internuclear distance due to vibrational excitation is

\begin{enumerate}
\item[(A)] 9
\item[(B)] 30
\item[(C)] 16
\item[(D)] 0
\end{enumerate}   \textbf{(GATE-EE 2025)}


\vspace{0.5cm}

\textbf{Q.69} \\
The mechanism of enzyme (E) catalysed reaction of a substrate (S) to yield product (P) is:

\begin{center}
\includegraphics[width=0.6\textwidth]{q69.png}
\end{center}

If a small amount of S is converted to P, the maximum rate for the reaction will be observed for:

\begin{tabbing}
\hspace{1cm} \= (A) \quad $(k_1 + k_2) \gg k_1 [S]_0$ \\
\> (B) \quad $(k_1 + k_2) \ll k_1 [S]_0$ \\
\> (C) \quad $(k_2 + k_{-1}) = (k_1 + k_1)$ \\
\> (D) \quad $k_2 \ll k_1$
\end{tabbing}   \textbf{(GATE-EE 2025)}


\textbf{Q.70} The lowest energy state of the $(1s)^2(2s)^1(3s)^1$ configuration of Be is

\begin{enumerate}
\item[(A)] $^1S_0$
\item[(B)] $^1D_2$
\item[(C)] $^3S_1$
\item[(D)] $^3P_1$
\end{enumerate}   \textbf{(GATE-EE 2025)}


\vspace{0.5cm}


\section*{Common Data Questions}

\textbf{Common Data for Questions 71, 72 and 73:} \\
An electron accelerated through a potential difference of $\varphi$ volts impinges on a nickel surface, whose (100) planes have a spacing $d = 351.8 \times 10^{-12}$ m (351.8 pm).

\begin{itemize}
    \item[Q.71] The de-Broglie wavelength of the electron is $\lambda/\text{pm} = (a/\varphi)^{1/2}$. The value of ‘a’ in volts is:
    \begin{enumerate}
        \item[(A)] $1.5 \times 10^{-18}$
        \item[(B)] $1.5 \times 10^6$
        \item[(C)] $6.63 \times 10^5$
        \item[(D)] $2.5 \times 10^{18}$
    \end{enumerate}   \textbf{(GATE-EE 2025)}


    \item[Q.72] The condition for observing diffraction from the nickel surface is:
    \begin{enumerate}
        \item[(A)] $\lambda \gg 2d$
        \item[(B)] $\lambda \leq 2d$
        \item[(C)] $\lambda \leq d$
        \item[(D)] $\lambda \geq d$
    \end{enumerate}   \textbf{(GATE-EE 2025)}


    \item[Q.73] The minimum value of $\varphi$ (V) for the electron to diffract from the (100) planes is:
    \begin{enumerate}
        \item[(A)] 3000
        \item[(B)] 300
        \item[(C)] 30
        \item[(D)] 3
    \end{enumerate}
\end{itemize}   \textbf{(GATE-EE 2025)}


\textbf{Common Data for Questions 74 and 75:} \\
An iron complex $[\text{FeL}_3]^{2+}$ (L = neutral monodentate ligand) catalyses the oxidation of (CH$_3$)$_2$S by perbenzoic acid.

\begin{itemize}
    \item[Q.74] The formation of the organic product in the above reaction can be monitored by:
    \begin{enumerate}
        \item[(A)] gas chromatography
        \item[(B)] cyclic voltammetry
        \item[(C)] electron spin resonance
        \item[(D)] fluorescence spectroscopy
    \end{enumerate}   \textbf{(GATE-EE 2025)}


    \item[Q.75] The oxidation state of the metal ion in the catalyst can be detected by:
    \begin{enumerate}
        \item[(A)] atomic absorption spectroscopy
        \item[(B)] M\"ossbauer spectroscopy
        \item[(C)] HPLC
        \item[(D)] gas chromatography
    \end{enumerate}
\end{itemize}   \textbf{(GATE-EE 2025)}


\textbf{Linked Answer Questions: Q.76 to Q.85 carry two marks each}

\textbf{Linked Answer Questions 76 and 77:}

In the reaction,

\begin{center}
\includegraphics[width=0.6\textwidth]{q76 1.png}
\end{center}

\textbf{Q.76} Compound $X$ is

\begin{center}
\includegraphics[width=0.6\textwidth]{q76 2.png}
\end{center}   \textbf{(GATE-EE 2025)}


\textbf{Q.77}Rh(PPh$_3$)$_3$Cl reacts very fast with a gaseous mixture of H$_2$ and C$_2$H$_4$ to immediately give $Z$.The structure of $Z$ is

\begin{center}
\includegraphics[width=0.6\textwidth]{q77.png}
\end{center}   \textbf{(GATE-EE 2025)}


\section*{Linked Answer Questions 78 and 79}

The reaction of PCl$_3$ with methanol in the presence of triethylamine affords compound X. EI mass spectrum of X shows a parent ion peak at $m/z = 124$. Microanalysis of X shows that it contains C, H, O and P. The $^1$H NMR spectrum of X shows a doublet at 4.0 ppm. The separation between the two lines of the doublet is approximately 15 Hz (J for $^1$H and $^{31}$P = $\tfrac{1}{2}$).

\begin{itemize}
    \item[Q.78] Compound X is:
    \begin{enumerate}
        \item[(A)] (CH$_3$O)$_2$P
        \item[(B)] (CH$_3$O)$_2$PO
        \item[(C)] (CH$_3$O)$_2$P(O)OH
        \item[(D)] (CH$_3$O)$_2$PH
    \end{enumerate}   \textbf{(GATE-EE 2025)}


    \item[Q.79] Upon heating, compound X is converted to Y, which has the same molecular formula as that of X. The $^1$H NMR spectrum of Y shows two doublets centered at 3.0 ppm (separation of two lines = 20 Hz) and 4.0 ppm (separation of two lines = 15 Hz) respectively.

    Compound Y is:
    \begin{enumerate}
        \item[(A)] (CH$_3$O)$_2$P(O)(OH)
        \item[(B)] (CH$_3$O)$_2$P
        \item[(C)] (CH$_3$O)(CH$_3$)P(O)
        \item[(D)] (CH$_3$O)(CH$_3$)P(OH)
    \end{enumerate}
\end{itemize}   \textbf{(GATE-EE 2025)}


\section*{Linked Answer Questions 80 and 81}

For butyrophenone (\texttt{PhCOCH\(_2\)CH\(_2\)CH\(_3\)}),

\textbf{Q.80} \quad The most probable fragmentation observed in the electron impact ionization (EI) mass spectrometry is

\begin{center}
\includegraphics[width=0.6\textwidth]{q80.png}
\end{center}   \textbf{(GATE-EE 2025)}


\textbf{Q.81} \quad Photoirradiation leads to the following set of products.

\begin{center}
\includegraphics[width=0.6\textwidth]{q81.png}
\end{center}   \textbf{(GATE-EE 2025)}


\textbf{Linked Answer Questions 82 and 83:}

In the following reaction,

\begin{center}
\includegraphics[width=0.6\textwidth]{q82 1.png}
\end{center}

\textbf{Q.82} the reactive intermediate $I$ and the product $P$ are

\begin{center}
\includegraphics[width=0.6\textwidth]{q82 2.png}
\end{center}   \textbf{(GATE-EE 2025)}


\textbf{Q.83} The product P shows ‘m’ and ‘n’ number of signals in $^1$H and $^{13}$C NMR spectra, respectively. The values of ‘m’ and ‘n’ are

\begin{choices}
\item[(A)]m = 3 and n = 2
\item[(B)] m = 2 and n = 3
\item[(C)]m = 2 and n = 2
\item[(D)] m = 4 and n = 3
\end{choices}   \textbf{(GATE-EE 2025)}


\textbf{Linked Answer Questions 84 and 85:}

The infrared spectrum of a diatomic molecule exhibits transitions at 2144, 4262 and 6354~cm$^{-1}$ 
corresponding to excitations from the ground state to the first, second, and third vibration states respectively.

\textbf{Q.84} \quad The fundamental transition (cm$^{-1}$) of the diatomic molecule is at

\begin{tabular}{ll}
(A) 2157 & (B) 2170 \\
(C) 2183 & (D) 2196 \\
\end{tabular}   \textbf{(GATE-EE 2025)}


\textbf{Q.85} \quad The anharmonicity constant (cm$^{-1}$) of the diatomic molecule is

\begin{tabular}{ll}
(A) 0.018 & (B) 0.012 \\
(C) 0.006 & (D) 0.003 \\
\end{tabular}   \textbf{(GATE-EE 2025)}


\begin{center}
\textbf{END OF THE QUESTION PAPER}
\end{center}

\end{numerate}
\end{document}

