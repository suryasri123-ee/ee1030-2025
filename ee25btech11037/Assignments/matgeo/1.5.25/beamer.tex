\documentclass{beamer}

\mode<presentation>
\usepackage{amsmath}
\usepackage{amssymb}
%\usepackage{advdate}
\usepackage{adjustbox}
%\usepackage{subcaption}
\usepackage{enumitem}
\usepackage{multicol}
\usepackage{mathtools}
\usepackage{listings}
\usepackage{url}
\usetheme{Boadilla}
\usecolortheme{lily}
\setbeamertemplate{footline}
{
  \leavevmode%
  \hbox{%
  \begin{beamercolorbox}[wd=\paperwidth,ht=2.25ex,dp=1ex,right]{author in head/foot}%
    \insertframenumber{} / \inserttotalframenumber\hspace*{2ex} 
  \end{beamercolorbox}}%
  \vskip0pt%
}
\setbeamertemplate{navigation symbols}{}
\providecommand{\nCr}[2]{\,^{#1}C_{#2}} % nCr
\providecommand{\nPr}[2]{\,^{#1}P_{#2}} % nPr
\providecommand{\mbf}{\mathbf}
\providecommand{\pr}[1]{\ensuremath{\Pr\left(#1\right)}}
\providecommand{\qfunc}[1]{\ensuremath{Q\left(#1\right)}}
\providecommand{\sbrak}[1]{\ensuremath{{}\left[#1\right]}}
\providecommand{\lsbrak}[1]{\ensuremath{{}\left[#1\right.}}
\providecommand{\rsbrak}[1]{\ensuremath{{}\left.#1\right]}}
\providecommand{\brak}[1]{\ensuremath{\left(#1\right)}}
\providecommand{\lbrak}[1]{\ensuremath{\left(#1\right.}}
\providecommand{\rbrak}[1]{\ensuremath{\left.#1\right)}}
\providecommand{\cbrak}[1]{\ensuremath{\left\{#1\right\}}}
\providecommand{\lcbrak}[1]{\ensuremath{\left\{#1\right.}}
\providecommand{\rcbrak}[1]{\ensuremath{\left.#1\right\}}}
\theoremstyle{remark}
\newtheorem{rem}{Remark}
\newcommand{\sgn}{\mathop{\mathrm{sgn}}}

\providecommand{\res}[1]{\Res\displaylimits_{#1}} 
\providecommand{\norm}[1]{\lVert#1\rVert}
\providecommand{\mtx}[1]{\mathbf{#1}}

\providecommand{\fourier}{\overset{\mathcal{F}}{ \rightleftharpoons}}
%\providecommand{\hilbert}{\overset{\mathcal{H}}{ \rightleftharpoons}}
\providecommand{\system}{\overset{\mathcal{H}}{ \longleftrightarrow}}
	%\newcommand{\solution}[2]{\textbf{Solution:}{#1}}
%\newcommand{\solution}{\noindent \textbf{Solution: }}
\providecommand{\dec}[2]{\ensuremath{\overset{#1}{\underset{#2}{\gtrless}}}}
\newcommand{\myvec}[1]{\ensuremath{\begin{pmatrix}#1\end{pmatrix}}}

\title{Matrices in Geometry - 1.5.25}
\author{EE25BTECH11037  Divyansh}
\date{Aug, 2025}

\begin{document}

\maketitle

\section*{Outline}
\begin{frame}
\tableofcontents
\end{frame}
\section{Problem}
\begin{frame}
\frametitle{Problem Statement}
In what ratio does the point $\vec{R}=\myvec{\frac{24}{11} \\ y}$ divide the line segment joining the points $\vec{P}=\myvec{2 \\ -2}$ and $\vec{Q}=\myvec{3 \\ 7}$? Also find the value of y.
;\end{frame}

\section{Solution}
\begin{frame}{Solution}
   $\vec{P}=\myvec{2\\-2}$, $\vec{Q}=\myvec{3\\7}$ and a point $\vec{R}=  \myvec{\frac{24}{11} \\ y}$ on $PQ$. \\
   Let $R$ divide $PQ$ internally in the ratio $k:1$.\\
   Therefore, they are defined to be collinear if,
   \begin{align*}
    \text{rank}\myvec{\vec{R}- \vec{P} & \vec{Q}-\vec{R}}=1\\
    \vec{R}-\vec{P}=\myvec{\dfrac{2}{11} \\ y+2}\\
    \vec{Q}-\vec{R}=\myvec{\dfrac{9}{11} \\ 7-y}\\
    \implies \text{rank}\myvec{\dfrac{2}{11} & \dfrac{9}{11} \\ y+2 & 7-y} = 1
\end{align*}
\end{frame}



\begin{frame}{Solution}
\begin{align*}
 \implies \Delta=0\\
    \dfrac{2}{11}\brak{7-y} - \dfrac{9}{11}\brak{y+2}=0\\
    14-2y-18-9y=0\\
    \implies y=\dfrac{-4}{11}    
\end{align*}

We know that $k$ is the ratio in which $\vec{R}$ divides $\vec{P}$ and $\vec{Q}$,
\begin{align*}
    k=\dfrac{\norm{\overline{PR}}}{\norm{\overline{RQ}}}\\
    \overline{PR}=\myvec{-2/11 \\ -18/11}
\end{align*}
\end{frame}

\begin{frame}{Solution}
   \begin{align*}
    \implies \norm{\overline{PR}}=\sqrt{4/121 + 324/121}=\sqrt{328/21}\\
    \implies \norm{\overline{PR}}=2\sqrt{82}/11 \\
    \overline{QR}=\myvec{9/11 \\ 81/11} \\ 
    \implies \norm{\overline{QR}}=\sqrt{81/121 + 6561/121}=\sqrt{6642/121}\\
    \implies \norm{\overline{QR}}=9\sqrt{82}/11\\
    \therefore k=\dfrac{\norm{\overline{PR}}}{\norm{\overline{RQ}}}=\dfrac{2}{9}
   \end{align*}


\end{frame}
\section{Final Answer}
\begin{frame}{Final Answer}
\begin{align*}
    \text{Hence, the final answer is }\fbox{k = \dfrac{2}{9}} \; \text{and} \; \fbox{y = \dfrac{-4}{11}}  
\end{align*}
\begin{figure}[H]
    \centering
    \includegraphics[width=0.7\columnwidth]{figs/1.png}
    \caption{Plot for 1.5.25}
    \label{fig:placeholder}
\end{figure}
\end{frame}
\end{document}
