\documentclass[journal,12pt,onecolumn]{IEEEtran}
\usepackage{cite}
 \usepackage{caption}
\usepackage{graphicx}
\usepackage{amsmath,amssymb,amsfonts,amsthm}
\usepackage{algorithmic}
\usepackage{graphicx}
\usepackage{textcomp}
\usepackage{xcolor}
\usepackage{txfonts}
\usepackage{listings}
\usepackage{enumitem}
\usepackage{mathtools}
\usepackage{gensymb}
\usepackage{comment}
\usepackage[breaklinks=true]{hyperref}
\usepackage{tkz-euclide} 
\usepackage{listings}
\usepackage{gvv}
\usepackage{gvv-book}
%\def\inputGnumericTable{}                                 
\usepackage[latin1]{inputenc} 
\usetikzlibrary{arrows.meta, positioning}
\usepackage{xparse}
\usepackage{color}                                            
\usepackage{array}                                            
\usepackage{longtable}                                       
\usepackage{calc}                                             
\usepackage{multirow}
\usepackage{multicol}
\usepackage{hhline}                                           
\usepackage{ifthen}                                           
\usepackage{lscape}
\usepackage{tabularx}
\usepackage{array}
\usepackage{float}

\usepackage{float}
%\newcommand{\define}{\stackrel{\triangle}{=}}
\theoremstyle{remark}
\usepackage{circuitikz}
\captionsetup{justification=centering}
\usepackage{tikz}

\title{Matrices in Geometry 1.5.25}
\author{EE25BTECH11037 - Divyansh}
\begin{document}
\vspace{3cm}
\maketitle
{\let\newpage\relax\maketitle}
\textbf{Question: }
In what ratio does the point \myvec{\frac{24}{11} \\ y} divide the line segment joining the points \textbf{P}=\myvec{2 \\ -2} and \textbf{Q}=\myvec{3 \\ 7}? Also find the value of y.\\

\textbf{Given: } 
$\vec{P}=\myvec{2\\-2}$, $\vec{Q}\myvec{3\\7}$ and a point $\vec{R} \myvec{\frac{24}{11} \\ y}$ on $PQ$.

Let $R$ divide $PQ$ internally in the ratio $k:1$.\\
Therefore, they are defined to be collinear if,
\begin{align*}
    \text{rank}\myvec{\vec{R}- \vec{P} & \vec{Q}-\vec{R}}=1\\
    \vec{R}-\vec{P}=\myvec{\dfrac{2}{11} \\ y+2}\\
    \vec{Q}-\vec{R}=\myvec{\dfrac{9}{11} \\ 7-y}\\
    \implies \text{rank}\myvec{\dfrac{2}{11} & \dfrac{9}{11} \\ y+2 & 7-y} = 1\\
    \implies \Delta=0\\
    \dfrac{2}{11}\brak{7-y} - \dfrac{9}{11}\brak{y+2}=0\\
    14-2y-18-9y=0\\
    \implies y=\dfrac{-4}{11}
\end{align*}

We know that $k$ is the ratio in which $\vec{R}$ divides $\vec{P}$ and $\vec{Q}$,
\begin{align*}
    k=\dfrac{\norm{\overline{PR}}}{\norm{\overline{RQ}}}\\
    \overline{PR}=\myvec{-2/11 \\ -18/11} \\ \implies \norm{\overline{PR}}=\sqrt{4/121 + 324/121}=\sqrt{328/21}
    \implies \norm{\overline{PR}}=2\sqrt{82}/11 \\
    \overline{QR}=\myvec{9/11 \\ 81/11} \\ 
    \implies \norm{\overline{QR}}=\sqrt{81/121 + 6561/121}=\sqrt{6642/121}
    \implies \norm{\overline{QR}}=9\sqrt{82}/11\\
    \therefore k=\dfrac{\norm{\overline{PR}}}{\norm{\overline{RQ}}}=\dfrac{2}{9}
\end{align*}

\begin{align*}
 \text{Hence, the final answer is }\fbox{k = \dfrac{2}{9}} \; \text{and} \; \fbox{y = \dfrac{-4}{11}}  
\end{align*}



\begin{figure}[H]
    \centering
    \includegraphics[width=0.8\columnwidth]{figs/1.png}
    \caption{Plot for 1.5.25}
    \label{fig:placeholder}
\end{figure}
\end{document}