\documentclass{beamer}
\let\vec\mathbf
\mode<presentation>
\usepackage{amsmath}
\usepackage{amssymb}
%\usepackage{advdate}
\usepackage{adjustbox}
%\usepackage{subcaption}
\usepackage{enumitem}
\usepackage{multicol}
\usepackage{mathtools}
\usepackage{listings}
\usepackage{url}
\usetheme{Boadilla}
\usecolortheme{lily}
\setbeamertemplate{footline}
{
  \leavevmode%
  \hbox{%
  \begin{beamercolorbox}[wd=\paperwidth,ht=2.25ex,dp=1ex,right]{author in head/foot}%
    \insertframenumber{} / \inserttotalframenumber\hspace*{2ex} 
  \end{beamercolorbox}}%
  \vskip0pt%
}
\setbeamertemplate{navigation symbols}{}
\providecommand{\nCr}[2]{\,^{#1}C_{#2}} % nCr
\providecommand{\nPr}[2]{\,^{#1}P_{#2}} % nPr
\providecommand{\mbf}{\mathbf}
\providecommand{\pr}[1]{\ensuremath{\Pr\left(#1\right)}}
\providecommand{\qfunc}[1]{\ensuremath{Q\left(#1\right)}}
\providecommand{\sbrak}[1]{\ensuremath{{}\left[#1\right]}}
\providecommand{\lsbrak}[1]{\ensuremath{{}\left[#1\right.}}
\providecommand{\rsbrak}[1]{\ensuremath{{}\left.#1\right]}}
\providecommand{\brak}[1]{\ensuremath{\left(#1\right)}}
\providecommand{\lbrak}[1]{\ensuremath{\left(#1\right.}}
\providecommand{\rbrak}[1]{\ensuremath{\left.#1\right)}}
\providecommand{\cbrak}[1]{\ensuremath{\left\{#1\right\}}}
\providecommand{\lcbrak}[1]{\ensuremath{\left\{#1\right.}}
\providecommand{\rcbrak}[1]{\ensuremath{\left.#1\right\}}}
\theoremstyle{remark}
\newtheorem{rem}{Remark}
\newcommand{\sgn}{\mathop{\mathrm{sgn}}}

\providecommand{\res}[1]{\Res\displaylimits_{#1}} 
\providecommand{\norm}[1]{\left\lVert#1\right\rVert}
\providecommand{\mtx}[1]{\mathbf{#1}}

\providecommand{\fourier}{\overset{\mathcal{F}}{ \rightleftharpoons}}
%\providecommand{\hilbert}{\overset{\mathcal{H}}{ \rightleftharpoons}}
\providecommand{\system}{\overset{\mathcal{H}}{ \longleftrightarrow}}
	%\newcommand{\solution}[2]{\textbf{Solution:}{#1}}
%\newcommand{\solution}{\noindent \textbf{Solution: }}align
\providecommand{\dec}[2]{\ensuremath{\overset{#1}{\underset{#2}{\gtrless}}}}
\newcommand{\myvec}[1]{\ensuremath{\begin{pmatrix}#1\end{pmatrix}}}

\title{Matrices in Geometry - 1.9.26}
\author{EE25BTECH11037  Divyansh}
\date{Aug, 2025}

\begin{document}

\maketitle


\section{Problem}
\begin{frame}
\frametitle{Problem Statement}
Find the value of k, if the point $\vec{P}\brak{2,4}$ is equidistant from point $\vec{A}\brak{5,k}$ and $\vec{B}\brak{k,7}$
\end{frame}

\section{Solution}
\begin{frame}{Solution}
   \textbf{Given: } 
$\vec{P}\myvec{2\\4}$, $\vec{A}\myvec{5\\k}$ and a point $\vec{B} \myvec{k \\ 7}$ such that $\vec{P}$ is equidistant from $\vec{A}$ and $\vec{B}$. 
\begin{align}
     \therefore \norm{\vec{P}-\vec{A}}=\norm{\vec{P}-\vec{B}}\\
    \text{On squaring both the sides, we get }\\
    \norm{\vec{P}-\vec{A}}^2=\norm{\vec{P}-\vec{A}}^2\\
    \myvec{\vec{P}-\vec{A}}^{\top}\myvec{\vec{P}-\vec{A}}=\myvec{\vec{P}-\vec{B}}^{\top}\myvec{\vec{P}-\vec{B}}
\end{align}
\end{frame}

\begin{frame}{Solution}
\begin{align}
    \vec{P}^{\top}\vec{P} - 2\vec{P}^{\top}\vec{A} + \vec{A}^{\top}\vec{A} =\vec{P}^{\top}\vec{P} - 2\vec{P}^{\top}\vec{B} + \vec{B}^{\top}\vec{B}\\
    \norm{\vec{A}}^2 - \norm{\vec{B}}^2=2\vec{P}^{\top}\myvec{\vec{A}-\vec{B}}\\
    \norm{\myvec{5\\k}} - \norm{\myvec{k\\7}}=2\myvec{2 & 4}\myvec{5-k \\ k-7}\\
    25+k^2 - 49 -k^2=2\brak{10-2k + 4k-28}\\
    -24=2\brak{2k-18} \implies -12=2k-18 \implies 2k=6 \implies k=3
\end{align}
\end{frame}



\section{Final Answer}
\begin{frame}{Final Answer}
\begin{align}
    \text{Hence, the final answer is }\fbox{k = 3} 
\end{align}
\begin{figure}
    \centering
    \includegraphics[width=0.6\columnwidth]{figs/1.png}
\end{figure}
\end{frame}
\end{document}