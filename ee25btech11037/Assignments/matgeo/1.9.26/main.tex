\documentclass[journal,12pt,onecolumn]{IEEEtran}
\usepackage{cite}
 \usepackage{caption}
\usepackage{graphicx}
\usepackage{amsmath,amssymb,amsfonts,amsthm}
\usepackage{algorithmic}
\usepackage{graphicx}
\usepackage{textcomp}
\usepackage{xcolor}
\usepackage{txfonts}
\usepackage{listings}
\usepackage{enumitem}
\usepackage{mathtools}
\usepackage{gensymb}
\usepackage{comment}
\usepackage[breaklinks=true]{hyperref}
\usepackage{tkz-euclide} 
\usepackage{listings}
\usepackage{gvv}
%\def\inputGnumericTable{}                                 
\usepackage[latin1]{inputenc} 
\usetikzlibrary{arrows.meta, positioning}
\usepackage{xparse}
\usepackage{color}                                            
\usepackage{array}                                            
\usepackage{longtable}                                       
\usepackage{calc}                                             
\usepackage{multirow}
\usepackage{multicol}
\usepackage{hhline}                                           
\usepackage{ifthen}                                           
\usepackage{lscape}
\usepackage{tabularx}
\usepackage{array}
\usepackage{float}

\usepackage{float}
%\newcommand{\define}{\stackrel{\triangle}{=}}
\theoremstyle{remark}
\usepackage{circuitikz}
\captionsetup{justification=centering}
\usepackage{tikz}

\title{Matrices in Geometry 1.9.26}
\author{EE25BTECH11037 - Divyansh}
\begin{document}
\vspace{3cm}
\maketitle
{\let\newpage\relax\maketitle}
\textbf{Question: }
Find the value of k, if the point $\vec{P}\brak{2,4}$ is equidistant from point $\vec{A}\brak{5,k}$ and $\vec{B}\brak{k,7}$

\textbf{Given: } 
$\vec{P}\myvec{2\\4}$, $\vec{A}\myvec{5\\k}$ and a point $\vec{B} \myvec{k \\ 7}$ such that $\vec{P}$ is equidistant from $\vec{A}$ and $\vec{B}$. 

\begin{align}
    \therefore \norm{\vec{A}-\vec{P}}=\norm{\vec{B}-\vec{P}}\\
    \text{On squaring both the sides, we get }\\
    \norm{\vec{A}-\vec{P}}^2=\norm{\vec{B}-\vec{P}}^2\\
    \myvec{\vec{A}-\vec{P}}^{\top}\myvec{\vec{A}-\vec{P}}=\myvec{\vec{B}-\vec{P}}^{\top}\myvec{\vec{B}-\vec{P}}\\
    \myvec{\vec{A}-\vec{P}}=\myvec{3 \\ k-4}\\
    \myvec{\vec{B}-\vec{P}}=\myvec{k-2 \\ 3}\\
    \myvec{\vec{A}-\vec{P}}^{\top}\myvec{\vec{A}-\vec{P}}=\myvec{3 & k-4}\myvec{3 \\ k-4}\\ =9+\brak{k-4}^2=9 + k^2 -8k + 16 =k^2 - 8k + 25\\
    \myvec{\vec{B}-\vec{P}}^{\top}\myvec{\vec{B}-\vec{P}}=\myvec{k-2 & 3}\myvec{k-2 \\ 3}\\= \brak{k-2}^2 + 9=k^2 - 4k + 4+ 9=k^2 -4k+13\\
    \text{From \brak{8} and \brak{10}, \ \ } k^2 - 8k + 25=k^2 -4k+13\\
    \implies -4k+ 8k =25 -13\implies 4k=12 \implies k=3
\end{align}


\begin{align}
 \text{Hence, the final answer is }\boxed{k = 3}   
\end{align}
\begin{figure}[H]
    \centering
    \includegraphics[width=1\columnwidth]{figs/1.png}
    \caption{Plot for 1.9.26}
    \label{fig:placeholder}
\end{figure}
\end{document}