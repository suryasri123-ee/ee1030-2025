\documentclass{beamer}
\usepackage[utf8]{inputenc}

\usetheme{Madrid}
\usecolortheme{default}
\usepackage{amsmath,amssymb,amsfonts,amsthm}
\usepackage{txfonts}
\usepackage{tkz-euclide}
\usepackage{listings}
\usepackage{adjustbox}
\usepackage{array}
\usepackage{tabularx}
\usepackage{gvv}
\usepackage{lmodern}
\usepackage{circuitikz}
\usepackage{tikz}
\usepackage{graphicx}

\setbeamertemplate{page number in head/foot}[totalframenumber]

\usepackage{tcolorbox}
\tcbuselibrary{minted,breakable,xparse,skins}



\definecolor{bg}{gray}{0.95}
\DeclareTCBListing{mintedbox}{O{}m!O{}}{%
  breakable=true,
  listing engine=minted,
  listing only,
  minted language=#2,
  minted style=default,
  minted options={%
    linenos,
    gobble=0,
    breaklines=true,
    breakafter=,,
    fontsize=\small,
    numbersep=8pt,
    #1},
  boxsep=0pt,
  left skip=0pt,
  right skip=0pt,
  left=25pt,
  right=0pt,
  top=3pt,
  bottom=3pt,
  arc=5pt,
  leftrule=0pt,
  rightrule=0pt,
  bottomrule=2pt,
  toprule=2pt,
  colback=bg,
  colframe=orange!70,
  enhanced,
  overlay={%
    \begin{tcbclipinterior}
    \fill[orange!20!white] (frame.south west) rectangle ([xshift=20pt]frame.north west);
    \end{tcbclipinterior}},
  #3,
}
\lstset{
    language=C,
    basicstyle=\ttfamily\small,
    keywordstyle=\color{blue},
    stringstyle=\color{orange},
    commentstyle=\color{green!60!black},
    numbers=left,
    numberstyle=\tiny\color{gray},
    breaklines=true,
    showstringspaces=false,
}
%------------------------------------------------------------
%This block of code defines the information to appear in the
%Title page
\title %optional
{1.5.36}
\date{}
%\subtitle{A short story}

\author % (optional)
{Sai Krishna Bakki - EE25BTECH11049}



\begin{document}


\frame{\titlepage}
\begin{frame}{Question}
The point P (x, 4) lies in the line segment that joins the points A (5, 8) and B (4, 10). Find the ratio in which point P divides the line segment AB. Also, find the value of x
\end{frame}


\begin{frame}{Theoretical Solution}
Let 
\begin{align*}
\vec{A}= \myvec{-5 \\ 8}, \vec{B}= \myvec{4 \\ -10}
\end{align*}
Let $\vec{P} = \lambda\vec{A} + \mu\vec{B}$ with $\lambda + \mu =1$. Using the y-coordinates:

\begin{align}
    \myvec{8 & -10 \\ 1 & 1} \myvec{\lambda \\ \mu} = \myvec{4 \\ 1}
\end{align}
Hence the internal division ratio
\begin{align}
    AP:PB = \mu:\lambda=2:7
\end{align}
and
\begin{align}
   x=\lambda\brak{-5}+\mu\brak{4}= -\frac{35}{9} + \frac{8}{9} = -3
\end{align}
So, $\vec{P}=\brak{-3,4}$

\end{frame}
\begin{frame}
\frametitle{Solution by Laplace Transform}
\begin{align}
    \frac{dy}{dx} = x+y-5
\end{align}
Applying the Laplace transform to both sides:

\begin{align}
    \mathcal{L}\left\{\frac{dy}{dx}\right\} = \mathcal{L}\left\{x + (y - 5)\right\}
\end{align}
\begin{align}
    sY(s) - y(0) = \frac{1}{s^2} + Y(s) - \frac{5}{s}
\end{align}
\begin{align}
    Y(s) = \frac{\frac{1}{s^2} - \frac{5}{s} + y(0)}{s - 1}
\end{align}
\begin{align}
    Y(s) = \frac{1}{s^2 (s - 1)} - \frac{1}{s (s - 1)} \cdot 5 + \frac{y(0)}{s - 1}
\end{align}
\end{frame}
\begin{frame}
\frametitle{Solution by Laplace Transform}
Inverse Laplace Transform of Each Term
\begin{align}
    \frac{1}{s^2 (s - 1)} = \frac{A}{s} + \frac{B}{s^2} + \frac{C}{s - 1}
\end{align}
\begin{align}
    1 = A\text{ } s\text{ } (s - 1) + B \text{ }(s - 1) + C \text{ }s^2
\end{align}
we get
\begin{align}
    \frac{1}{s^2 (s - 1)} = -\frac{1}{s} -\frac{1}{s^2} + \frac{1}{s - 1}
\end{align}
Taking the inverse Laplace transform
\begin{align}
    \mathcal{L}^{-1}\left\{ -\frac{1}{s} -\frac{1}{s^2}+ \frac{1}{s - 1} \right\} = -1 - x + e^x
\end{align}
\end{frame}
\begin{frame}
\frametitle{Solution by Laplace Transform}
Inverse Laplace of \( \frac{5}{s (s - 1)} \)
\begin{align}
    \frac{5}{s (s - 1)} = \frac{A}{s} + \frac{B}{s - 1}
\end{align}
\begin{align}
    5 = A (s - 1) + B s
\end{align}
Solving gives \( A = 5 \) and \( B = -5 \)
\begin{align}
    \frac{5}{s (s - 1)} = \frac{5}{s} - \frac{5}{s - 1}
\end{align}
Taking the inverse Laplace transform
\begin{align}
    \mathcal{L}^{-1}\left\{ \frac{5}{s} - \frac{5}{s - 1} \right\} = 5 - 5 e^x
\end{align}

\end{frame}
\begin{frame}{Solution by Laplace Transform}
Inverse Laplace of \( \frac{y(0)}{s - 1} \)
\begin{align}
    \mathcal{L}^{-1}\left\{ \frac{y(0)}{s - 1} \right\} = y(0) e^x
\end{align}
combining the results from all parts, we have the solution for $y(x)$
The general solution too this differential equation is 
\begin{align}
    y(x) = 4 - x + (y(0) - 4) e^x 
\end{align}
\begin{align}
     y(x) = 4 - x + c  e^x
    \label{0.26}
\end{align}
To find c, put x=0 and y=2 in \ref{0.26}
\begin{align}
    c = -2  
\end{align}
The curve is 
\begin{align}
   x+y = 4 -2e^x
\end{align}
\end{frame}
\begin{frame}{Verification}
   Now lets verify the solution computationally from the definition of $\frac{dy}{dx}$
\begin{align}
    y_{n+1}= y_{n} + \frac{dy}{dx} \cdot h
    \label{0.29}
\end{align}
From the differential equation given,
\begin{align}
    \frac{dy}{dx} = x+y-5
    \label{0.30}
\end{align}
Substituting \ref{0.30} in \ref{0.29}
\begin{align}
    y_{n+1} = y_n + \brak{x_n + y_n -5} \cdot h 
\end{align}
\end{frame}
\begin{frame}[fragile]
    \frametitle{C Code - Eulers Method }

    \begin{lstlisting}
#include <stdio.h>
#include <stdio.h>
#include <math.h>
#include <stdlib.h>
// Function to calculate dy/dx for the differential equation
float dy_dx(float x, float y) {
    return x + y - 5;  // Differential equation dy/dx = x + y - 5
}
// Function to calculate points using Euler's method
void points(float x_0, float y_0, float x_end, float h, float *x_points, float *y_points, int steps) {
    float x_n = x_0;
    float y_n = y_0;

    for (int i = 0; i < steps; i++) {
        x_points[i] = x_n;  // Store current x value
        y_points[i] = y_n;  // Store current y value
}
    \end{lstlisting}
\end{frame}
\begin{frame}[fragile]
    \frametitle{C Code - Eulers Method}

    \begin{lstlisting}
        // Calculate the next y using Euler's method
        y_n = y_n + h * dy_dx(x_n, y_n);
        x_n = x_n + h;  // Move to the next x value
    }
    // Main function
int main() {
    float x_0 = 0.0;    // Initial condition for x
    float y_0 = 2.0;    // Initial condition for y
    float x_end = 1.0;  // Final value of x
    float step_size = 0.001; // Step size for Euler's method
    int steps = (int)((x_end - x_0) / step_size) + 1;
    // Allocate memory for arrays to store points
    float *x_points = (float *)malloc(steps * sizeof(float));
    float *y_points = (float *)malloc(steps * sizeof(float));
    if (x_points == NULL || y_points == NULL) {
        printf("Memory allocation failed.\n");
        return 1;
    }
}
\end{lstlisting}

\end{frame}

\begin{frame}[fragile]
    \frametitle{C Code - Eulers Method}
    \begin{lstlisting}
// Call the points function
    points(x_0, y_0, x_end, step_size, x_points, y_points, steps);

    // Print the calculated points (optional, for debugging purposes)
    printf("x\t\ty\n");
    for (int i = 0; i < steps; i++) {
        printf("%f\t%f\n", x_points[i], y_points[i]);
    }

    // Free allocated memory
    free(x_points);
    free(y_points);

    return 0;
}
    \end{lstlisting}
\end{frame}

\begin{frame}[fragile]
    \frametitle{Python Code}
    \begin{lstlisting}
import ctypes
import numpy as np
import matplotlib.pyplot as plt

# Load the shared library
lib = ctypes.CDLL("./c.so")

# Define the function signature for points
lib.points.argtypes = [
    ctypes.c_float,  # x_0
    ctypes.c_float,  # y_0
    ctypes.c_float,  # x_end
    ctypes.c_float,  # h
    np.ctypeslib.ndpointer(dtype=np.float32, ndim=1),  # x_points
    np.ctypeslib.ndpointer(dtype=np.float32, ndim=1),  # y_points
    ctypes.c_int     # stepsclass 12 differential equations
]

    \end{lstlisting}
\end{frame}

\begin{frame}[fragile]
    \frametitle{Python Code}
    \begin{lstlisting}
# Parameters for simulation
x_0 = 0.0  # Initial condition for x
y_0 = 2.0  # Initial condition for y
x_end = 1.0  # Final value of x
step_size = 0.001  # Reduced step size for higher accuracy
steps = int((x_end - x_0) / step_size) + 1

# Create numpy arrays to hold the points
x_points = np.zeros(steps, dtype=np.float32)
y_points = np.zeros(steps, dtype=np.float32)

# Call the points function from the C shared library
lib.points(x_0, y_0, x_end, step_size, x_points, y_points, steps)

# Define the theoretical solution with C = -2
def theoretical_solution(x):
    return (-x + 4 - 2* np.exp(x))  # C = -2
    \end{lstlisting}
\end{frame}

\begin{frame}[fragile]
    \frametitle{Python Code}

    \begin{lstlisting}
# Generate theoretical values for y
x_theory = np.linspace(x_0, x_end, 1000)
y_theory = theoretical_solution(x_theory)

# Plot the results
plt.figure(figsize=(10, 6))

# Plot Euler's method results
plt.plot(x_points, y_points, 'ro-', markersize=2, linewidth=4, label="sim")

# Plot the theoretical solution
plt.plot(x_theory, y_theory, 'b-', linewidth=2, label="theory")

    \end{lstlisting}
\end{frame}

\begin{frame}[fragile]
    \frametitle{Python Code}

    \begin{lstlisting}
  # Add labels, title, grid, and legend
plt.xlabel("x") 1
plt.ylabel("y")
plt.grid(True, linestyle="--")
plt.legend()

# Display the plot
plt.show()
    \end{lstlisting}
\end{frame}

\begin{frame}{Plot}
    \begin{center}
        \includegraphics[width=0.6\columnwidth]{figs/Figure_1.png}
    \end{center}
\end{frame}




\end{document}
