\let\negmedspace\undefined
\let\negthickspace\undefined
\documentclass[journal]{IEEEtran}
\usepackage[a5paper, margin=10mm, onecolumn]{geometry}
\usepackage{lmodern} % Ensure lmodern is loaded for pdflatex
\usepackage{tfrupee} % Include tfrupee package

\setlength{\headheight}{1cm} % Set the height of the header box
\setlength{\headsep}{0mm}     % Set the distance between the header box and the top of the text

\usepackage{gvv-book}
\usepackage{gvv}
\usepackage{cite}
\usepackage{amsmath,amssymb,amsfonts,amsthm}
\usepackage{algorithmic}
\usepackage{graphicx}
\graphicspath{{./figs/}}
\usepackage{textcomp}
\usepackage{xcolor}
\usepackage{txfonts}
\usepackage{listings}
\usepackage{enumitem}
\usepackage{mathtools}
\usepackage{gensymb}
\usepackage{comment}
\usepackage[breaklinks=true]{hyperref}
\usepackage{tkz-euclide} 
\usepackage{listings}
\usepackage{gvv}                                        
\def\inputGnumericTable{}                                 
\usepackage[latin1]{inputenc}                                
\usepackage{color}                                            
\usepackage{array}                                            
\usepackage{longtable}                                       
\usepackage{calc}                                             
\usepackage{multirow}                                         
\usepackage{hhline}                                           
\usepackage{ifthen}                                           
\usepackage{lscape}
\usepackage{circuitikz}
\tikzstyle{block} = [rectangle, draw, fill=blue!20, 
text width=4em, text centered, rounded corners, minimum height=3em]
\tikzstyle{sum} = [draw, fill=blue!10, circle, minimum size=1cm, node distance=1.5cm]
\tikzstyle{input} = [coordinate]
\tikzstyle{output} = [coordinate]
\begin{document}
\bibliographystyle{IEEEtran}
	\vspace{3cm}
	
	\title{1.5.37}
	\author{EE25BTECH11050-Hema Havil}
	\maketitle
	% \newpage
	% \bigskip
	{\let\newpage\relax\maketitle}
	
	\renewcommand{\thefigure}{\theenumi}
	\renewcommand{\thetable}{\theenumi}
	\setlength{\intextsep}{12pt} % Space between text and floats
	
	\numberwithin{equation}{enumi}
	\numberwithin{figure}{enumi}
	\renewcommand{\thetable}{\theenumi}
	
	\textbf{Question}:\\
    
         The center of a circle whose end points of diameter are (-6,3) and (6,4) is \underline{\hspace{2cm}}\\
         
         \solution \\
         Let the given end points of the diameter of the circle be A and B, then\\
         \begin{align}
             \vec{A} = \myvec{-6 \\ 3}, \vec{B} = \myvec{6 \\ 4}
         \end{align}
         The midpoint of the two points is the center of the circle,\\
         let the center be C, then
         \begin{align}
             \vec{C}=\frac{1}{2} \brak{\vec{A}+\vec{B}}
         \end{align}
         by substituting A and B
         \begin{align}
             \vec{C}=\frac{1}{2}\brak{\myvec{-6 \\ 3} + \myvec{6 \\ 4}}
         \end{align}
         \begin{align}
             \vec{C}=\frac{1}{2}\myvec{-6+6 \\ 3+4}
         \end{align}
         \begin{align}
             \vec{C}=\frac{1}{2}\myvec{0 \\ 7}
         \end{align}
         \begin{align}
             \vec{C}=\myvec{0 \\ 3.5}
         \end{align}
         Therefore, the center of the circle is \\
         \begin{align*}
             \vec{C}=\myvec{0 \\ 3.5}
         \end{align*}
         \begin{figure}
             \centering
             \includegraphics[width=1\columnwidth]{problem1.png}
             \caption{Plot for the center of the cicle }
             \label{1.5.37}
         \end{figure}
         
\end{document}