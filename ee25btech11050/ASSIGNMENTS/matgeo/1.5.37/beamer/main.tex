\documentclass{beamer}
\usepackage[utf8]{inputenc}

\usetheme{Madrid}
\usecolortheme{default}
\usepackage{amsmath,amssymb,amsfonts,amsthm}
\usepackage{txfonts}
\usepackage{tkz-euclide}
\usepackage{listings}
\usepackage{adjustbox}
\usepackage{array}
\usepackage{tabularx}
\usepackage{gvv}
\usepackage{lmodern}
\usepackage{circuitikz}
\usepackage{tikz}
\usepackage{graphicx}

\setbeamertemplate{page number in head/foot}[totalframenumber]

\usepackage{tcolorbox}
\tcbuselibrary{minted,breakable,xparse,skins}



\definecolor{bg}{gray}{0.95}
\DeclareTCBListing{mintedbox}{O{}m!O{}}{%
	breakable=true,
	listing engine=minted,
	listing only,
	minted language=#2,
	minted style=default,
	minted options={%
		linenos,
		gobble=0,
		breaklines=true,
		breakafter=,,
		fontsize=\small,
		numbersep=8pt,
		#1},
	boxsep=0pt,
	left skip=0pt,
	right skip=0pt,
	left=25pt,
	right=0pt,
	top=3pt,
	bottom=3pt,
	arc=5pt,
	leftrule=0pt,
	rightrule=0pt,
	bottomrule=2pt,
	toprule=2pt,
	colback=bg,
	colframe=orange!70,
	enhanced,
	overlay={%
		\begin{tcbclipinterior}
			\fill[orange!20!white] (frame.south west) rectangle ([xshift=20pt]frame.north west);
	\end{tcbclipinterior}},
	#3,
}
\lstset{
	language=C,
	basicstyle=\ttfamily\small,
	keywordstyle=\color{blue},
	stringstyle=\color{orange},
	commentstyle=\color{green!60!black},
	numbers=left,
	numberstyle=\tiny\color{gray},
	breaklines=true,
	showstringspaces=false,
}
%------------------------------------------------------------
%This block of code defines the information to appear in the
%Title page
\title %optional
{1.5.37}
%\subtitle{A short story}

\author % (optional)
{Hema Havil - EE25BTECH11050}



\begin{document}
	
	\frame{\titlepage}
	\begin{frame}{Question}
		The centre of a circle whose end points of a diameter are (−6, 3) and (6, 4) is \underline{\hspace{3cm}}
	\end{frame}

	
\begin{frame}{Theoretical Solution}
	Let the given end points of the diameter of the circle be A and B, then\\
         \begin{align}
             \vec{A} = \myvec{-6 \\ 3}, \vec{B} = \myvec{6 \\ 4}
         \end{align}
         The midpoint of the two points is the center of the circle,\\
         let the center be C, then
         \begin{align}
             \vec{C}=\frac{1}{2} \brak{\vec{A}+\vec{B}}
         \end{align}
         by substituting A and B
         \begin{align}
             \vec{C}=\frac{1}{2}\brak{\myvec{-6 \\ 3} + \myvec{6 \\ 4}}
         \end{align}
         \begin{align}
             \vec{C}=\frac{1}{2}\myvec{-6+6 \\ 3+4}
         \end{align}
\end{frame}
\begin{frame}{Theoretical Solution}
\begin{align}
             \vec{C}=\frac{1}{2}\myvec{0 \\ 7}
         \end{align}
         \begin{align}
             \vec{C}=\myvec{0 \\ 3.5}
         \end{align}
         Therefore the center of the circle is \\
         \begin{align*}
             \vec{C}=\myvec{0 , 3.5}
         \end{align*}
	\end{frame}
	
	\begin{frame}[fragile]
	\frametitle{C Code- equidistant check function }
	
	\begin{lstlisting}
// circle.c
#include <math.h>

#ifdef _WIN32
#define API __declspec(dllexport)
#else
#define API
#endif

// Given endpoints (x1,y1), (x2,y2), returns center (cx,cy) and radius r
API void compute_circle(double x1, double y1,
                        double x2, double y2,
                        double *cx, double *cy, double *r) {
    *cx = 0.5 * (x1 + x2);
    *cy = 0.5 * (y1 + y2);
    double dx = x2 - x1, dy = y2 - y1;
    *r = 0.5 * sqrt(dx*dx + dy*dy);
}
	\end{lstlisting}
\end{frame}

\begin{frame}[fragile]
	\frametitle{Python Code using shared output}
	\begin{lstlisting}
		import ctypes, os, numpy as np, matplotlib.pyplot as plt

# load the shared library (adjust name for macOS: libcircle.dylib, Windows: circle.dll)
lib = ctypes.CDLL(os.path.abspath("./libcircle.so"))

lib.compute_circle.argtypes = [ctypes.c_double, ctypes.c_double,
                               ctypes.c_double, ctypes.c_double,
                               ctypes.POINTER(ctypes.c_double),
                               ctypes.POINTER(ctypes.c_double),
                               ctypes.POINTER(ctypes.c_double)]

def compute_circle(x1, y1, x2, y2):
    cx = ctypes.c_double()
    cy = ctypes.c_double()
    r  = ctypes.c_double()
	\end{lstlisting}
\end{frame}
\begin{frame}[fragile]
	\frametitle{Python Code using shared output}
	\begin{lstlisting}	
     lib.compute_circle(x1, y1, x2, y2, ctypes.byref(cx), ctypes.byref(cy), ctypes.byref(r))
     return cx.value, cy.value, r.value

# given endpoints
x1, y1 = -6.0, 3.0
x2, y2 =  6.0, 4.0
cx, cy, r = compute_circle(x1, y1, x2, y2)
print("Center:", (cx, cy), "Radius:", r)

# make a circle for plotting
t = np.linspace(0, 2*np.pi, 400)
xc = cx + r*np.cos(t)
yc = cy + r*np.sin(t)

	\end{lstlisting}
\end{frame}
\begin{frame}[fragile]
	\frametitle{Python Code using shared output}
	\begin{lstlisting}
fig, ax = plt.subplots()
ax.plot(xc, yc, label="Circle")
ax.plot([x1, x2], [y1, y2], 'o-', label="Diameter endpoints")
ax.plot(cx, cy, 'o', label="Center")

ax.set_aspect('equal', adjustable='box')
ax.grid(True, linestyle="--", alpha=0.5)
ax.legend()
ax.set_title(f"Circle with diameter [({x1},{y1}) ↔ ({x2},{y2})]\nCenter=({cx:.2f},{cy:.2f}), r={r:.4f}")
plt.show()
	\end{lstlisting}
\end{frame}
\begin{frame}{Plot by python using shared output from c}
	\begin{center}
	\begin{figure}[H]
		\centering
		\includegraphics[width = 0.6\columnwidth]{figs/fig1.png}
		\caption{Plot of the center and ends of the diameter}
		\label{fig1}
	\end{figure}
	\end{center}
\end{frame}
\begin{frame}[fragile]
     \frametitle{Python code for the plot}
\begin{lstlisting}
    import numpy as np
    import matplotlib.pyplot as plt

# Endpoints of diameter
x1, y1 = -6, 3
x2, y2 =  6, 4

# Compute center (midpoint)
cx = 0.5 * (x1 + x2)
cy = 0.5 * (y1 + y2)

# Compute radius
dx, dy = x2 - x1, y2 - y1
r = 0.5 * np.sqrt(dx**2 + dy**2)
\end{lstlisting}
\end{frame}
\begin{frame}[fragile]
   \frametitle{Python code for the plot}
    \begin{lstlisting}
print("Center:", (cx, cy))
print("Radius:", r)

# Parametric circle
theta = np.linspace(0, 2*np.pi, 400)
xc = cx + r*np.cos(theta)
yc = cy + r*np.sin(theta)

# Plot
fig, ax = plt.subplots()

# Circle (blue)
ax.plot(xc, yc, color="blue", label="Circle")

# Diameter endpoints + line (green)
ax.plot([x1, x2], [y1, y2], 'o-', color="green", label="Diameter")
 \end{lstlisting}
\end{frame}
 \begin{frame}[fragile]
       \frametitle{python code for plot}
       \begin{lstlisting}
       # Center (red point)
ax.plot(cx, cy, 'ro', label="Center")

# Formatting
ax.set_aspect('equal', adjustable='box')
ax.grid(True, linestyle="--", alpha=0.5)
ax.legend()
ax.set_title(f"Circle with diameter endpoints ({x1},{y1}) and ({x2},{y2})")

plt.show()
    \end{lstlisting}
 \end{frame}
 \begin{figure}
     \centering
     \includegraphics[width=0.5\linewidth]{figs/fig2.png}
     \caption{Plot for the center of the circle}
     \label{fig2}
 \end{figure}
\end{document}