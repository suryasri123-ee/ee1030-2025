\documentclass[journal]{IEEEtran}
\usepackage[a5paper, margin=10mm, onecolumn]{geometry}
\usepackage[cmex10]{amsmath}
\usepackage{amssymb,amsfonts,amsthm}
\usepackage{gvv-book}
\usepackage{gvv}
\usepackage{hyperref}

\begin{document}

\title{1.5.33}
\author{Puni Aditya - EE25BTECH11046}
\maketitle

\textbf{Question:}\\
Find the ratio in which the Y-axis divides the line segment joining the points A\brak{5, -6} and B\brak{-1, -4}. Also find the coordinates of the point of intersection.

\textbf{Solution:}\\
Let the given points be A and B
\begin{align*} \vec{A} = \myvec{5 \\ -6}, \vec{B} = \myvec{-1 \\ -4} \end{align*}
Let the Y-axis divide the line segment $\vec{AB}$ at point $\vec{P}$ in the ratio $k:1$.

The coordinates of any point $\vec{P}$ on the line segment $\vec{AB}$ can be found using the section formula
\begin{align}
    \vec{P} \equiv \myvec{x \\ y} = \frac{k\vec{B} + \vec{A}}{k+1}
\end{align}
Here, substituting the values, we get
\begin{align}
    \vec{P} \equiv \myvec{x \\ y} = \frac{k\myvec{-1 \\ -4}+1\myvec{5 \\ -6}}{k+1} \\
    \vec{P} = \myvec{\frac{k\brak{-1} + 1\brak{5}}{k+1} \\ \frac{k\brak{-4} + 1\brak{-6}}{k+1}} \\
    \vec{P} = \myvec{\frac{-k+5}{k+1} \\ \frac{-4k-6}{k+1}}
\end{align}

Since the point $\vec{P}$ lies on the Y-axis only, its x component must be 0.
\begin{align}
    \frac{-k+5}{k+1} = 0 \\
    -k + 5 = 0 \implies k = 5
\end{align}
Thus, the ratio in which the Y-axis divides the line segment $\vec{AB}$ is \textbf{5:1}.

Now, we find the coordinates of the point of intersection, $\vec{P}$, by substituting $k=5$ into the equations for x and y components.
The x component is 0. For the y component,
\begin{align}
    y = \frac{-4\brak{5} - 6}{5+1} = \frac{-20 - 6}{6} = \frac{-26}{6} = -\frac{13}{3}
\end{align}
$\therefore$ The coordinates of the point of intersection are 
\begin{align*}
\vec{P} = \myvec{0 \\ -\frac{13}{3}}
\end{align*}

\begin{figure}
    \centering
    \includegraphics[width=0.7\columnwidth]{figs/plot_c.jpg}
    \caption*{Plot of Intersection of AB by Y-axis}
    \label{fig:fig}
\end{figure}

\end{document}
