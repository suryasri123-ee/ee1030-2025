\documentclass[a4paper,10pt]{article}
\usepackage[top=2cm, left=2.5cm, right=2.5cm, bottom=2cm]{geometry}
\usepackage[cmex10]{amsmath}
\usepackage{graphicx}
\usepackage{gvv-book}
\usepackage{gvv}
\usepackage{textcomp}
\usepackage{multicol}
\usepackage{amssymb}

\begin{document}

\title{GATE 2025 - AR : Architecture and Planning Exam}
\author{Your Name}
\date{\today}
\maketitle

Duration: Three Hours \hfill Maximum Marks:100

\section*{General Aptitude (GA)}
\subsection*{Q.1 - Q.5 Carry ONE mark Each}
\begin{enumerate}
    \item Fish: Shoal :: Lion:
    \begin{multicols}{4}
    \begin{enumerate}
        \item Pride
        \item School
        \item Forest
        \item Series
    \end{enumerate}
    \end{multicols}

    \item Identify the grammatically correct sentence:
    \begin{enumerate}
        \item It is I who am responsible for this fiasco.
        \item It is myself who is responsible for this fiasco.
        \item It is I who is responsible for this fiasco.
        \item It is I who are responsible for this fiasco.
    \end{enumerate}

    \item Two cars, P and Q, start from a point X in India at 10 AM. Car P travels North with a speed of $25~km/h$ and car Q travels East with a speed of $30~km/h.$ Car P travels continuously but car Q stops for some time after travelling for one hour. If both the cars are at the same distance from X at 11:30 AM, for how long (in minutes) did car Q stop?
    \begin{multicols}{4}
    \begin{enumerate}
        \item 15
        \item 20
        \item 25
        \item 30
    \end{enumerate}
    \end{multicols}

    \item In a company, 40\% of employees are males. 75\% of the male employees earn more than Rs. 35,000 per year and 60\% of the employees in the company earn more than Rs. 35,000 per year. What percentage of female employees earn Rs. 35,000 per year or less?
    \begin{multicols}{4}
    \begin{enumerate}
        \item 50\%
        \item 25\%
        \item 75\%
        \item 100\%
    \end{enumerate}
    \end{multicols}

    \item Find the missing number in the series: 4, 10, 22, 46, ?
    \begin{multicols}{4}
    \begin{enumerate}
        \item 94
        \item 92
        \item 90
        \item 88
    \end{enumerate}
    \end{multicols}
\end{enumerate}

\subsection*{Q.6 - Q.10 Carry TWO marks Each}
\begin{enumerate}
    \setcounter{enumi}{5}
    \item The ratio of the number of boys and girls who participated in an examination is 4:3. The total percentage of candidates who passed the examination is 80 and the percentage of girls who passed is 90. The percentage of boys who passed is:
    \begin{multicols}{4}
    \begin{enumerate}
        \item 72.50\%
        \item 55.57\%
        \item 80.50\%
        \item 90.00\%
    \end{enumerate}
    \end{multicols}

    \item A container contains 40 litres of milk. From this container, 4 litres of milk was taken out and replaced by water. This process was repeated further two times. How much milk is now contained by the container?
    \begin{multicols}{4}
    \begin{enumerate}
        \item 29.16 litres
        \item 26.34 litres
        \item 27.36 litres
        \item 28 litres
    \end{enumerate}
    \end{multicols}
    
    \item A and B can do a piece of work in 30 days, while B and C can do the same work in 24 days and C and A in 20 days. They all work together for 10 days, then B and C leave. How many days more will A take to finish the work?
    \begin{multicols}{4}
    \begin{enumerate}
        \item 18 days
        \item 24 days
        \item 30 days
        \item 36 days
    \end{enumerate}
    \end{multicols}
    
    \item A vendor bought toffees at 6 for a rupee. How many for a rupee must he sell to gain 20\%?
    \begin{multicols}{4}
    \begin{enumerate}
        \item 5
        \item 3
        \item 4
        \item 6
    \end{enumerate}
    \end{multicols}
    
    \item A boat running upstream takes 8 hours 48 minutes to cover a certain distance, while it takes 4 hours to cover the same distance running downstream. What is the ratio between the speed of the boat and the speed of the water current respectively?
    \begin{multicols}{4}
    \begin{enumerate}
        \item 8:3
        \item 2:1
        \item 3:2
        \item Cannot be determined
    \end{enumerate}
    \end{multicols}
\end{enumerate}

\section*{Architecture and Planning (AR)}
\subsection*{Q.11 - Q.35 Carry ONE mark Each}
\begin{enumerate}
    \setcounter{enumi}{10}
    \item Which of the following is NOT a primary pollutant?
    \begin{multicols}{4}
    \begin{enumerate}
        \item Carbon Monoxide (CO)
        \item Sulphur Dioxide (SO$_2$)
        \item Ozone (O$_3$)
        \item Particulate Matter (PM$_{2.5}$)
    \end{enumerate}
    \end{multicols}
    
    \item The 'Golden Rectangle' has its sides in the proportion of:
    \begin{multicols}{4}
    \begin{enumerate}
        \item 1:1.618
        \item 1:1.414
        \item 1:1.732
        \item 1:2.0
    \end{enumerate}
    \end{multicols}
    
    \item The architect of the Indian Institute of Management, Ahmedabad is:
    \begin{multicols}{4}
    \begin{enumerate}
        \item Charles Correa
        \item B.V. Doshi
        \item Louis Kahn
        \item Le Corbusier
    \end{enumerate}
    \end{multicols}
    
    \item A building has a floor area of 250 sq.m. The total area of openings (doors and windows) is 50 sq.m. The Floor Area Ratio (FAR) is 1.5. What is the total built-up area of the building?
    \begin{multicols}{4}
    \begin{enumerate}
        \item 250 sq.m
        \item 375 sq.m
        \item 400 sq.m
        \item 500 sq.m
    \end{enumerate}
    \end{multicols}

    \item The term 'Zeitgeist' in architecture refers to:
    \begin{enumerate}
        \item The spirit of the age
        \item The structural system
        \item The functional aspects
        \item The aesthetic quality
    \end{enumerate}

    \item In the context of urban planning, 'Gentrification' means:
    \begin{enumerate}
        \item The process of renovating and improving a house or district so that it conforms to middle-class taste.
        \item The development of rural areas.
        \item The creation of new satellite towns.
        \item The preservation of historical monuments.
    \end{enumerate}
    
    \item The Taj Mahal is an example of which style of architecture?
    \begin{enumerate}
        \item Mughal
        \item Gothic
        \item Baroque
        \item Neoclassical
    \end{enumerate}
    
    \item A 'cantilever' is a beam supported at:
    \begin{enumerate}
        \item Both ends
        \item One end only
        \item Its center
        \item No ends
    \end{enumerate}
    
    \item The "Garden City" concept is attributed to:
    \begin{enumerate}
        \item Ebenezer Howard
        \item Le Corbusier
        \item Frank Lloyd Wright
        \item Patrick Geddes
    \end{enumerate}
    
    \item Which of the following materials has the highest embodied energy?
    \begin{enumerate}
        \item Wood
        \item Steel
        \item Concrete
        \item Aluminum
    \end{enumerate}
    
\documentclass[a4paper,10pt]{article}
\usepackage[top=2cm, left=2.5cm, right=2.5cm, bottom=2cm]{geometry}
\usepackage[cmex10]{amsmath}
\usepackage{graphicx}
\usepackage{gvv-book}
\usepackage{gvv}
\usepackage{textcomp}
\usepackage{multicol}
\usepackage{amssymb}

\begin{document}

\title{GATE 2025 - AR : Architecture and Planning Exam (Questions 21-40)}
\author{Your Name}
\date{\today}
\maketitle

\section*{Architecture and Planning (AR)}
\subsection*{Q.11 - Q.35 Carry ONE mark Each (Continued)}
\begin{enumerate}
    \setcounter{enumi}{20}
    
    \item A 'Setback' in building regulations refers to:
    \begin{enumerate}
        \item The minimum distance a building must be from the property line
        \item The height limit of a building
        \item The total floor area allowed
        \item The type of materials to be used
    \end{enumerate}

    \item The 'Kyoto Protocol' is an international treaty that commits state parties to reduce:
    \begin{enumerate}
        \item Greenhouse gas emissions
        \item Ozone depletion
        \item Deforestation
        \item Water pollution
    \end{enumerate}

    \item Which of the following is a 'cool' color in the context of color theory?
    \begin{multicols}{4}
    \begin{enumerate}
        \item Red
        \item Orange
        \item Yellow
        \item Blue
    \end{enumerate}
    \end{multicols}

    \item A 'Topiary' is the art of:
    \begin{enumerate}
        \item Arranging flowers
        \item Clipping shrubs or trees into ornamental shapes
        \item Designing water gardens
        \item Creating miniature landscapes
    \end{enumerate}
    
    \item The 'flying buttress' is a characteristic feature of which architectural style?
    \begin{multicols}{4}
    \begin{enumerate}
        \item Romanesque
        \item Gothic
        \item Renaissance
        \item Baroque
    \end{enumerate}
    \end{multicols}
    
    \item 'Ergonomics' is the study of:
    \begin{enumerate}
        \item The history of architecture
        \item People in their working environment
        \item The structural behavior of materials
        \item The flow of water in pipes
    \end{enumerate}
    
    \item In a site plan, a 'contour' line connects points of equal:
    \begin{multicols}{4}
    \begin{enumerate}
        \item Temperature
        \item Pressure
        \item Elevation
        \item Rainfall
    \end{enumerate}
    \end{multicols}
    
    \item The term 'curtilage' refers to:
    \begin{enumerate}
        \item The decorative molding around a doorway
        \item The land immediately surrounding a house or dwelling
        \item A type of roof truss
        \item The main entrance of a public building
    \end{enumerate}
    
    \item The architect of the 'Baháʼí House of Worship' (Lotus Temple) in Delhi is:
    \begin{multicols}{4}
    \begin{enumerate}
        \item Fariborz Sahba
        \item Charles Correa
        \item B.V. Doshi
        \item Raj Rewal
    \end{enumerate}
    \end{multicols}

    \item The 'Universal Transverse Mercator' (UTM) is a:
    \begin{enumerate}
        \item Building material
        \item Construction technique
        \item Geographic coordinate system
        \item Type of landscape design
    \end{enumerate}

    \item In HVAC systems, 'CFM' stands for:
    \begin{multicols}{4}
    \begin{enumerate}
        \item Cubic Feet per Minute
        \item Centigrade Fahrenheit Measurement
        \item Critical Flow Mass
        \item Circular Fan Motor
    \end{enumerate}
    \end{multicols}

    \item The 'slump test' is used to measure the:
    \begin{enumerate}
        \item Compressive strength of concrete
        \item Workability of fresh concrete
        \item Tensile strength of steel
        \item Hardness of stone
    \end{enumerate}

    \item The 'Charbagh' is a Persian-style garden layout where the quadrilateral garden is divided into four smaller parts by walkways or flowing water. A famous example of this is at:
    \begin{enumerate}
        \item The Taj Mahal
        \item The Palace of Versailles
        \item The Villa d'Este
        \item The Central Park, New York
    \end{enumerate}

    \item The 'Acropolis of Athens' is home to which famous building?
    \begin{multicols}{4}
    \begin{enumerate}
        \item The Colosseum
        \item The Pantheon
        \item The Parthenon
        \item St. Peter's Basilica
    \end{enumerate}
    \end{multicols}

    \item The 'Rule of thirds' is a principle of:
    \begin{multicols}{4}
    \begin{enumerate}
        \item Structural design
        \item Composition
        \item Acoustic design
        \item Urban planning
    \end{enumerate}
    \end{multicols}

\end{enumerate}

\subsection*{Q.36 - Q.65 Carry TWO marks Each}
\begin{enumerate}
    \setcounter{enumi}{35}
    \item Match the urban planning theories in Group I with their proponents in Group II.
    
    \begin{tabular}{ll}
    \textbf{Group I} & \textbf{Group II} \\
    P. Concentric Zone Model & 1. Homer Hoyt \\
    Q. Sector Model & 2. Chauncy Harris \& Edward Ullman \\
    R. Multiple Nuclei Model & 3. Ernest Burgess \\
    S. Garden City & 4. Ebenezer Howard \\
    \end{tabular}

    \begin{enumerate}
        \item P-3, Q-1, R-2, S-4
        \item P-1, Q-3, R-4, S-2
        \item P-3, Q-2, R-1, S-4
        \item P-4, Q-1, R-3, S-2
    \end{enumerate}
    
    \item A room of size 4m x 5m has a ceiling height of 3m. The total area of walls and ceiling to be painted is \rule{2cm}{0.4pt} sq.m.
    
    \item If the scale of a map is 1:1000, a distance of 5 cm on the map represents an actual distance of \rule{2cm}{0.4pt} m.
    
    \item A cantilever beam of length L is subjected to a point load W at its free end. The maximum bending moment is:
    \begin{multicols}{4}
    \begin{enumerate}
        \item WL
        \item WL/2
        \item WL/4
        \item WL/8
    \end{enumerate}
    \end{multicols}
    
    \item In the given figure, identify the type of roof.
    % Image for question 40 should be here
    % \includegraphics[width=0.4\columnwidth]{path/to/roof_image.jpg}
    \begin{enumerate}
        \item Gable roof
        \item Hip roof
        \item Gambrel roof
        \item Mansard roof
    \end{enumerate}

\end{enumerate}

\end{document}
\documentclass[a4paper,10pt]{article}
\usepackage[top=2cm, left=2.5cm, right=2.5cm, bottom=2cm]{geometry}
\usepackage[cmex10]{amsmath}
\usepackage{graphicx}
\usepackage{gvv-book}
\usepackage{gvv}
\usepackage{textcomp}
\usepackage{multicol}
\usepackage{amssymb}

\begin{document}

\title{GATE 2025 - AR : Architecture and Planning Exam (Questions 41-60)}
\author{Your Name}
\date{\today}
\maketitle

\section*{Architecture and Planning (AR)}
\subsection*{Q.36 - Q.65 Carry TWO marks Each (Continued)}
\begin{enumerate}
    \setcounter{enumi}{40}
    
    \item A project has an initial investment of Rs. 1,00,000 and generates a net cash flow of Rs. 30,000 per year for 5 years. The payback period for the project is \rule{2cm}{0.4pt} years. (rounded off to two decimal places)

    \item In landscape architecture, 'xeriscaping' refers to:
    \begin{enumerate}
        \item Landscaping with exotic and rare plants
        \item Landscaping in a way that reduces or eliminates the need for supplemental water from irrigation
        \item Creating large, open green spaces
        \item Designing gardens with a formal, symmetrical layout
    \end{enumerate}

    \item A sound source has a sound pressure level of 80 dB. If two such identical sound sources are combined, the resulting sound pressure level will be \rule{2cm}{0.4pt} dB.

    \item Match the following building materials in Group I with their primary constituent in Group II.
    
    \begin{tabular}{ll}
    \textbf{Group I} & \textbf{Group II} \\
    P. Glass & 1. Calcium Silicate \\
    Q. Cement & 2. Iron Oxide \\
    R. Steel & 3. Silicon Dioxide \\
    S. Terracotta & 4. Iron Carbide \\
    \end{tabular}

    \begin{enumerate}
        \item P-3, Q-1, R-4, S-2
        \item P-3, Q-1, R-2, S-4
        \item P-1, Q-3, R-4, S-2
        \item P-2, Q-4, R-1, S-3
    \end{enumerate}

    \item The 'stack effect' in a building is a phenomenon related to:
    \begin{enumerate}
        \item Structural stability
        \item Natural ventilation
        \item Fire safety
        \item Acoustic performance
    \end{enumerate}

    \item A 230 mm thick brick wall has a thermal conductivity of 0.8 W/mK. The thermal resistance of the wall in m$^2$K/W is \rule{2cm}{0.4pt}. (rounded off to two decimal places)

    \item Which of the following planning concepts is LEAST associated with sustainable urban development?
    \begin{enumerate}
        \item Compact City
        \item Transit-Oriented Development (TOD)
        \item Urban Sprawl
        \item Mixed-Use Development
    \end{enumerate}

    \item The critical path in a project network is the sequence of activities that:
    \begin{enumerate}
        \item has the maximum total float.
        \item determines the total project duration.
        \item involves the least number of activities.
        \item has the lowest cost.
    \end{enumerate}
    
    \item A rectangular room 6m x 4m has a reverberation time of 1.5 seconds. If the volume of the room is 72 m$^3$, the total sound absorption in the room is \rule{2cm}{0.4pt} sabins.

    \item Identify the building shown in the figure below, designed by Zaha Hadid.
    % Image for question 50 should be here
    % \includegraphics[width=0.5\columnwidth]{path/to/hadid_building.jpg}
    \begin{enumerate}
        \item Guangzhou Opera House
        \item Heydar Aliyev Center
        \item Vitra Fire Station
        \item London Aquatics Centre
    \end{enumerate}

    \item In a housing project, 30\% of the area is for roads, 15\% for open spaces, 5\% for amenities, and the rest is the net residential area. If the total site area is 10 hectares, the net residential area is \rule{2cm}{0.4pt} hectares.
    
    \item Which of the following are NOT part of the classical orders of architecture? (MSQ - Multiple Select Question)
    \begin{enumerate}
        \item Doric
        \item Ionic
        \item Corinthian
        \item Byzantine
        \item Gothic
    \end{enumerate}

    \item A city's population was 1,00,000 in the year 2010 and 1,20,000 in 2020. Using the arithmetic increase method, the projected population for the year 2030 will be \rule{2cm}{0.4pt}.

    \item The primary function of a 'French drain' is to:
    \begin{enumerate}
        \item Store rainwater for irrigation
        \item Collect and redirect surface and groundwater away from an area
        \item Treat domestic wastewater
        \item Serve as a decorative water channel in a garden
    end{enumerate}
    
    \item A building has a total connected load of 50 kW. If the demand factor is 0.6 and the diversity factor is 1.2, the maximum demand on the transformer is \rule{2cm}{0.4pt} kW.

    \item The 'Geodesic dome' was popularized by which architect/engineer?
    \begin{enumerate}
        \item Frei Otto
        \item Pier Luigi Nervi
        \item Buckminster Fuller
        \item Santiago Calatrava
    \end{enumerate}

    \item A plot of land has an area of 2000 sq.m. The permissible Floor Area Ratio (FAR) is 1.8 and the maximum permissible ground coverage is 40\%. What is the maximum number of floors that can be constructed if the building covers the maximum permissible ground area?
    \begin{multicols}{4}
    \begin{enumerate}
        \item 3.5
        \item 4
        \item 4.5
        \item 5
    \end{enumerate}
    \end{multicols}

    \item In GIS, 'raster' data represents:
    \begin{enumerate}
        \item Features as points, lines, and polygons
        \item Data in a grid of cells or pixels
        \item Tabular data linked to spatial features
        \item 3D models of the terrain
    \end{enumerate}
    
    \item The R-value of a building material is a measure of its:
    \begin{enumerate}
        \item Thermal conductivity
        \item Thermal resistance
        \item Heat capacity
        \item Solar reflectivity
    \end{enumerate}

    \item A developer proposes to build a commercial complex with a total built-up area of 50,000 sq.m. As per the parking norms, 1 ECS (Equivalent Car Space) is required for every 100 sq.m of built-up area. The total number of car parking spaces required is \rule{2cm}{0.4pt}.

\end{enumerate}

\end{document}
\documentclass[a4paper,10pt]{article}
\usepackage[top=2cm, left=2.5cm, right=2.5cm, bottom=2cm]{geometry}
\usepackage[cmex10]{amsmath}
\usepackage{graphicx}
\usepackage{gvv-book}
\usepackage{gvv}
\usepackage{textcomp}
\usepackage{multicol}
\usepackage{amssymb}
\usepackage{longtable}

\begin{document}

\title{GATE 2025 - AR : Architecture and Planning Exam (Questions 61-81)}
\author{Your Name}
\date{\today}
\maketitle

\section*{Architecture and Planning (AR)}
\subsection*{Q.36 - Q.65 Carry TWO marks Each (Continued)}
\begin{enumerate}
    \setcounter{enumi}{60}
    
    \item Which of the following are NOT principles of Universal Design? (MSQ)
    \begin{enumerate}
        \item Equitable Use
        \item Flexibility in Use
        \item High Physical Effort
        \item Perceptible Information
        \item Tolerance for Error
    \end{enumerate}

    \item A contour map has a contour interval of 2m. If the horizontal equivalent between two adjacent contours is 50m, the slope of the ground in percent is \rule{2cm}{0.4pt}\%.

    \item In the context of building acoustics, 'flanking transmission' refers to:
    \begin{enumerate}
        \item Sound passing directly through a partition.
        \item Sound that travels around the partition or through adjoining structures.
        \item The reflection of sound off a hard surface.
        \item The absorption of sound by a porous material.
    \end{enumerate}

    \item A construction project has the following activities, durations, and dependencies. What is the minimum project completion time?
    
    \begin{longtable}{|c|c|c|}
    \hline
    \textbf{Activity} & \textbf{Duration (days)} & \textbf{Predecessor(s)} \\
    \hline
    A & 5 & - \\
    \hline
    B & 7 & A \\
    \hline
    C & 6 & A \\
    \hline
    D & 4 & B, C \\
    \hline
    \end{longtable}

    \begin{multicols}{4}
    \begin{enumerate}
        \item 18 days
        \item 19 days
        \item 20 days
        \item 22 days
    \end{enumerate}
    \end{multicols}

    \item The 'Urban Heat Island' effect is characterized by:
    \begin{enumerate}
        \item Urban areas being significantly warmer than their surrounding rural areas.
        \item Rural areas being significantly warmer than surrounding urban areas.
        \item Uniform temperatures across urban and rural areas.
        \item Higher wind speeds in urban areas compared to rural areas.
    \end{enumerate}

\end{enumerate}

\subsection*{Part B1: Architecture (For Architecture Candidates Only)}
\subsubsection*{Q.66 - Q.75 Carry TWO marks Each}
\begin{enumerate}
    \setcounter{enumi}{65}
    \item A reinforced concrete beam of 250mm width and 450mm effective depth is reinforced with 3 bars of 16mm diameter. Using M20 concrete and Fe415 steel, the moment of resistance of the beam is \rule{2cm}{0.4pt} kNm. (Use Limit State Method)

    \item Match the famous domes in Group I with their locations in Group II.
    
    \begin{tabular}{ll}
    \textbf{Group I} & \textbf{Group II} \\
    P. Florence Cathedral Dome & 1. Istanbul, Turkey \\
    Q. St. Peter's Basilica Dome & 2. Rome, Italy \\
    R. Hagia Sophia Dome & 3. Florence, Italy \\
    S. Gol Gumbaz & 4. Bijapur, India \\
    \end{tabular}

    \begin{enumerate}
        \item P-3, Q-2, R-1, S-4
        \item P-2, Q-3, R-1, S-4
        \item P-3, Q-2, R-4, S-1
        \item P-4, Q-1, R-2, S-3
    \end{enumerate}
    
    \item In a framed structure, what is the primary function of a 'shear wall'?
    \begin{enumerate}
        \item To resist vertical loads from the slabs and beams.
        \item To provide insulation against heat and sound.
        \item To resist lateral forces such as wind and earthquakes.
        \item To serve as a non-structural partition between rooms.
    \end{enumerate}

    \item A building requires a cooling load of 10 Tons of Refrigeration (TR). The equivalent cooling capacity in kW is \rule{2cm}{0.4pt} kW. (1 TR = 3.517 kW)

    \item Which of the following are key features of 'Deconstructivism' in architecture? (MSQ)
    \begin{enumerate}
        \item Absence of harmony, continuity, or symmetry.
        \item Fragmentation and controlled chaos.
        \item Emphasis on pure geometric forms (cubes, spheres, cylinders).
        \item Use of traditional building materials and techniques.
        \item Non-linear design processes.
    \end{enumerate}

    \item A column of size 400mm x 400mm is subjected to an axial load of 1200 kN. The bearing capacity of the soil is 200 kN/m$^2$. The required area of an isolated square footing is \rule{2cm}{0.4pt} m$^2$. (neglecting the self-weight of the footing)

    \item The 'Villa Savoye' by Le Corbusier is a prime example of:
    \begin{enumerate}
        \item Postmodern Architecture
        \item The International Style
        \item Art Deco
        \item Brutalism
    \end{enumerate}

    \item A simply supported steel I-beam spans 6m. It supports a total uniformly distributed load of 40 kN/m. The required section modulus for the beam, if the permissible bending stress in steel is 165 MPa, is \rule{2cm}{0.4pt} x 10$^3$ mm$^3$.
    
    \item What is the purpose of a 'Damp-Proof Course' (DPC) in a building?
    \begin{enumerate}
        \item To prevent the rise of moisture from the ground into the building.
        \item To improve the thermal insulation of the walls.
        \item To increase the structural strength of the walls.
        \item To provide a decorative finish to the exterior of the building.
    \end{enumerate}

    \item The 'CCTV' in a building's security system stands for:
    \begin{enumerate}
        \item Central Circuit Television
        \item Closed-Circuit Television
        \item Computer Controlled Television
        \item Coded Circuit Television
    \end{enumerate}
\end{enumerate}

\subsection*{Part B2: Planning (For Planning Candidates Only)}
\subsubsection*{Q.76 - Q.85 Carry TWO marks Each}
\begin{enumerate}
    \setcounter{enumi}{75}
    \item A city has a total area of 100 sq. km. and a population of 2 million. The gross population density of the city is \rule{2cm}{0.4pt} persons per hectare.

    \item Match the planning models in Group I with their primary focus in Group II.
    
    \begin{tabular}{ll}
    \textbf{Group I} & \textbf{Group II} \\
    P. Advocacy Planning & 1. Public Participation \& Social Equity \\
    Q. Rational Planning & 2. Incremental Change \& Adaptation \\
    R. Collaborative Planning & 3. Scientific Method \& Objectivity \\
    S. Incremental Planning & 4. Consensus Building \& Stakeholder Engagement \\
    \end{tabular}

    \begin{enumerate}
        \item P-1, Q-3, R-4, S-2
        \item P-3, Q-1, R-2, S-4
        \item P-1, Q-4, R-3, S-2
        \item P-4, Q-2, R-1, S-3
    \end{enumerate}

    \item The 'Transfer of Development Rights' (TDR) is a planning tool that:
    \begin{enumerate}
        \item Allows for the transfer of land ownership from public to private entities.
        \item Separates the development potential of a parcel of land from the land itself and allows it to be sold.
        \item Facilitates the compulsory acquisition of land for public projects.
        \item Provides tax incentives for developing in specific zones.
    \end{enumerate}

    \item A transportation survey reveals that the average trip length in a city is 8 km and the modal split is 40\% for public transport and 60\% for private vehicles. If the total number of trips made per day is 5,00,000, the total vehicle kilometers traveled (VKT) by private vehicles per day is \rule{2cm}{0.4pt} million km.

    \item Which of the following are components of a 'Livable City'? (MSQ)
    \begin{enumerate}
        \item High levels of traffic congestion.
        \item Access to quality public open spaces.
        \item Affordable and diverse housing options.
        \item Safe and secure neighborhoods.
        \item Poor air and water quality.
    \end{enumerate}

    \item A four-arm un-signalized intersection is shown in the figure. The conflicting traffic movements that need to be managed are \rule{2cm}{0.4pt}. (in integer)
    % Image for question 81 should be here
    % \includegraphics[width=0.5\columnwidth]{path/to/intersection.jpg}

\end{enumerate}

\end{document}