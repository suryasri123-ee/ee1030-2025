\documentclass{beamer}

\usepackage{amsmath}     % For math equations
\usepackage{amssymb}     % For advanced math symbols
\usepackage{amsfonts}    % For math fonts
\usepackage{graphicx}    % For including images
\usepackage{array}
\usepackage{multirow}
\usepackage{float}
\usepackage{caption}
\usepackage{pgfplots}
\pgfplotsset{compat=1.18}

% Custom vector command
\newcommand{\myvec}[1]{\begin{bmatrix}#1\end{bmatrix}}

\title{AI25BTECH11034 - SUJAL CHAUHAN}
\subtitle{Problem 1.4.4}
\author{}
\date{}

\begin{document}

% Title slide
\frame{\titlepage}

% Problem Statement
\begin{frame}{Problem Statement}
Find the coordinate of the point which divides the line segment joining the point 
\[
\vec{P}(4,3) \quad \text{and} \quad \vec{Q}(8,5)
\]
in the ratio \(3:1\) internally.
\end{frame}

% Input Data
\begin{frame}{Input Data}
\centering
\begin{tabular}{|c|c|}
    \hline
    \textbf{Input variable} & \textbf{Value} \\ 
    \hline
    $\vec{P}$ & $\myvec{4 \\ 3}$ \\
    \hline 
    $\vec{Q}$ & $\myvec{8 \\ 5}$ \\
    \hline
    $\vec{PR}:\vec{RQ}$ & $3:1$ \\
    \hline
\end{tabular}
\end{frame}

% Solution
\begin{frame}{Solution}
Let the position vectors be
\[
\vec{P} = \myvec{4 \\ 3}, \qquad
\vec{Q} = \myvec{8 \\ 5}.
\] 
\hfill{(1)}

If $\vec{R}$ is the position vector of $R$, then
\begin{align}
\vec{R} = \frac{3\vec{Q} + \vec{P}}{3+1} \tag{2}
\end{align}
\end{frame}

% Calculation
\begin{frame}{Calculation}
So,
\begin{align}
\vec{R} &= \frac{3\myvec{8 \\ 5} + \myvec{4 \\ 3}}{4} \tag{3}
\end{align}

Therefore, the required point is
\[
\boxed{\vec{R} = \myvec{7 \\ \tfrac{9}{2}}} \tag{4}
\]

which indeed satisfies
\begin{align}
\vec{R} - \vec{P} = 3(\vec{Q} - \vec{P}). \tag{5}
\end{align}
\end{frame}

% Figure
\begin{frame}{Figure}
\centering
\includegraphics[width=0.75\textwidth]{figures/plot.png}
\\
{\small Visualization of point $R$ dividing $PQ$ in the ratio $3:1$}
\end{frame}

\end{document}

