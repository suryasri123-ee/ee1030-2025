\documentclass[12pt]{article}
\usepackage{hyperref}
\usepackage{listings}
\usepackage[margin=1in]{geometry}
\usepackage{enumitem}
\usepackage{multicol}
\usepackage{array}
\usepackage{titlesec}
\usepackage{helvet}
\renewcommand{\familydefault}{\sfdefault}
\usepackage{amsmath}     % For math equations
\usepackage{amssymb}     % For advanced math symbols
\usepackage{amsfonts} % For math fonts
\usepackage{gvv}
\usepackage{esint}
\usepackage[utf8]{inputenc}
\usepackage{graphicx}
\usepackage{pgfplots}
\pgfplotsset{compat=1.18}
\titleformat{\section}{\bfseries\large}{\thesection.}{1em}{}
\setlength{\parindent}{0pt}
\setlength{\parskip}{6pt}
\usepackage{multirow}
\usepackage{float}
\usepackage{caption}





\begin{document}
\begin{center}
    \textbf{\Large AI25BTECH11034-SUJAL CHAUHAN}
\end{center}
\textbf{Problem 1.4.4 .}  
Find the coordinate of the point which divides the line segment joining the point $\vec{P}(4,3) $and $\vec{Q}(8,5)$ in the ratio 3:1 internally.

\textbf{Solution.}  

\begin{table}[H]
\centering
\begin{tabular}[12pt]{ |c| c|}
    \hline
    \textbf{Input variable} & \textbf{Value}\\ 
    \hline
    $\vec{P}$ & \myvec{4 \\3 } \\
    \hline 
    $\vec{Q}$ & \myvec{8 \\ 5}\\
    \hline
    $\vec{PR}:\vec{RP}$ & $3:1$\\
    \hline
    \end{tabular}
    \caption{
    \label{}
    }
 \end{table}
Let the position vectors be

$$
\vec{P} = \myvec{4\\ 3}, \qquad
\vec{Q} = \myvec{8 \\ 5}.
$$
\hfill{(0)} \\

If \(\vec{R}\) is the position vector of \(R\), then
\begin{align}
\vec{R}=\frac{3\vec{Q}+\vec{R}}{3+1}
\end{align}
So,

    


$$    \vec{R}
= \frac{3 \myvec{8 \\ 5} + \myvec{ 4 \\3}}{4}$$



Therefore, the required point is
$$
\boxed{\vec{R}=\myvec{7 \\ \frac{9}{2}}}
$$
which indeed satisfies \begin{align}
    \vec{R} - \vec{P} = 3(\vec{Q} - \vec{P})
\end{align}

\begin{figure}[H]
    \centering
    \includegraphics[width=1\columnwidth]{figures/plot.png}j
    \caption{}
    \label{fig:placeholder}
\end{figure}
\end{document}