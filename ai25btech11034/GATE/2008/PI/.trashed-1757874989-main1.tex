\let\negmedspace\undefined
\let\negthickspace\undefined
\documentclass[journal,12pt,onecolumn]{IEEEtran}
\usepackage{cite}
\usepackage{amsmath,amssymb,amsfonts,amsthm}
\usepackage{algorithmic}
\usepackage{graphicx}
\usepackage{textcomp}
\usepackage{xcolor}
\usepackage{txfonts}
\usepackage{listings}
\usepackage{enumitem}
\usepackage{mathtools}
\usepackage{gensymb}
\usepackage{comment}
\usepackage[breaklinks=true]{hyperref}
\usepackage{tkz-euclide} 
\usepackage{gvv}                                        
%\def\inputGnumericTable{}                                 
\usepackage[latin1]{inputenc}     
\usepackage{xparse}
\usepackage{color}                                            
\usepackage{array}                                            
\usepackage{longtable}                                       
\usepackage{calc}                                             
\usepackage{multirow}
\usepackage{multicol}
\usepackage{hhline}                                           
\usepackage{ifthen}                                           
\usepackage{lscape}
\usepackage{tabularx}
\usepackage{float}
\newtheorem{theorem}{Theorem}[section]
\newtheorem{problem}{Problem}
\newtheorem{proposition}{Proposition}[section]
\newtheorem{lemma}{Lemma}[section]
\newtheorem{corollary}[theorem]{Corollary}
\newtheorem{example}{Example}[section]
\newtheorem{definition}[problem]{Definition}
\newcommand{\BEQA}{\begin{eqnarray}}
\newcommand{\EEQA}{\end{eqnarray}}
\newcommand{\define}{\stackrel{\triangle}{=}}
\theoremstyle{remark}
\newtheorem{rem}{Remark}
% Marks the beginning of the document
\begin{document}
\title{PI : PRODUCTION AND INDUSTRIAL ENGINEERING}
\author{AI25BTECH11034 - Sujal Chauhan}
\maketitle
\renewcommand{\thefigure}{\theenumi}
\renewcommand{\thetable}{\theenumi}
\textbf{Q.1-Q.20 carry one marks each}
\vspace{1cm}

\begin{enumerate}

    

    \item[\textnormal{Q.1}]  The value of the integral $\int_{-\frac{\pi}{2}}^{\frac{\pi}{2}}xcos(x)dx$ is 
    \hfill{(PI 2008)}
    \begin{multicols}{4}
    \begin{enumerate}[label=(\Alph*)]
        \item $0$
        \item $\pi-2$
        \item $\pi$
        \item $\pi+2$
    \end{enumerate}
\end{multicols}
\vspace{1cm}
\item[\textnormal{Q.2}]  The value of the expression $$\frac{-5+i10}{3+i4}$$ is 
    \hfill{(PI 2008)}
    \begin{multicols}{4}
    \begin{enumerate}[label=(\Alph*)]
        \item $1-i2$
        \item $1+i2$
        \item $2-i$
        \item $2+i$
    \end{enumerate}
\end{multicols}
\vspace{1cm}
\item[\textnormal{Q.3}] The value of the expression $$\lim_{x\to0}\left[\frac{\sin(x)}{e^xx}\right]$$  
    \hfill{(PI 2008)}
    \begin{multicols}{4}
    \begin{enumerate}[label=(\Alph*)]
        \item $0$
        \item $\frac{1}{2}$
        \item $1$
        \item $\frac{1}{1+e}$
    \end{enumerate}
\end{multicols}
\vspace{1cm}
\item[\textnormal{Q.4}]  In inventory cost structure, set up cost is a part of  replenishment cost when it
    \hfill{(PI 2008)}
    \begin{itemize}[label={}]
        \item (A) has taken place externally
        \item (B) is dependent on supply conditions
        \item (C) is independent of supply conditions
        \item (D) has taken place internally
    \end{itemize}
    \vspace{1cm}
    \item[\textnormal{Q.5}] Acceptable Quantity Level(AQL) is associated with
    \hfill{(PI 2008)}

\begin{itemize}[label={}]
        \item (A) Producer's risk
        \item (B) Consumer's risk
        \item (C) Lot tolerance percent defective
        \item (D) Average outgoing quality limit
    \end{itemize}
\vspace{1cm}
\item[\textnormal{Q.6}] The REL chart is used for 
    \hfill{(PI 2008)}

\begin{itemize}[label={}]
        \item (A) designing the layout of plants
        \item (B) estimating the valuation of stock
        \item (C) analysing the movement of an item in a store
        \item (D) maintaining the issue and reciept record
    \end{itemize}
    \vspace{1cm}
    \item[\textnormal{Q.7}]  If $\Vec{r}$ is the position vector of any point on a closed surface $S$ that encloses the volume $V$, then    
    $\int_{S}\left(\vec{r}.d\vec{S}\right)$ 
    \hfill{(PI 2008)}
    \begin{multicols}{4}
    \begin{enumerate}[label=(\Alph*)]
        \item $\frac{1}{2}V$
        \item $V$
        \item $2V$
        \item $3V$
    \end{enumerate}
\end{multicols}
\vspace{1cm}
\item[\textnormal{Q.8}]  Laplace transform of $8t^3$ is     \hfill{(PI 2008)}
    \begin{multicols}{4}
    \begin{enumerate}[label=(\Alph*)]
        \item $\frac{8}{s^4}$
        \item $\frac{16}{s^4}$
        \item $\frac{24}{s^4}$
        \item $\frac{48}{s^4}$
    \end{enumerate}
\end{multicols}
\vspace{1cm}
\item[\textnormal{Q.9}]  For a random variable $x (-\infty<x<\infty)$ following normal distribution, the mean is $\mu=100$. If the probability is $P=\alpha$ for $x\geq110$, then the probability of $x$ lying between 90 and 110,i.e.,$P(90\leq x \leq 110 )$ will be equal to
    \hfill{(PI 2008)}
    \begin{multicols}{4}
    \begin{enumerate}[label=(\Alph*)]
        \item $1-2\alpha$
        \item $1-\alpha$
        \item $1-\frac{\alpha}{2}$
        \item $2\alpha$
    \end{enumerate}
\end{multicols}
\vspace{1cm}
\item[\textnormal{Q.10}] Consider a steady,reversible flow process in a system with one inlet stream and one outlet stream. Potential and kinetic energy effects are negligibly small. Given : $\nu=$specific volume and $p=$pressure of the system. The net work done by system per unit mass flow rate is 
    \hfill{(PI 2008)}
    \begin{multicols}{4}
    \begin{enumerate}[label=(\Alph*)]
        \item $\int pdv$
        \item $-\int pdv$
        \item $\int vdp$
        \item $-\int vdp$
    \end{enumerate}
\end{multicols}
\vspace{1cm}
\item[\textnormal{Q.11}]  A refrigerator,operating in a room at a temprature of $29.5^{\circ}C$, maintains the refrigerated space at $2^{\circ}C$. The maximum possible COP of the refrigerator is 
    \hfill{(PI 2008)}
    \begin{multicols}{4}
    \begin{enumerate}[label=(\Alph*)]
        \item $1.0$
        \item $7.0$
        \item $10.0$
        \item $11.0$
    \end{enumerate}
\end{multicols}
\vspace{1cm}
\item[\textnormal{Q.12}]  Self locking condition for a pair of square thread screw and nut having coeffficent of friction$\mu=$, lead of thread$=L$ and pitch diameter of thread $=d$, is given by
    \hfill{(PI 2008)}
    \begin{multicols}{4}
    \begin{enumerate}[label=(\Alph*)]
        \item $d>\frac{L}{\pi\mu}$
        \item $d>\pi\mu L$
        \item $d>\mu L$
        \item $\mu>Ld$
    \end{enumerate}
\end{multicols}
\vspace{1cm}
\item[\textnormal{Q.13}]  The state of stress at a point in a body under plane state of stress condition is given by\[ \begin{bmatrix}
60 && 0 \\
0 && 20
\end{bmatrix}\]
    \hfill{(PI 2008)}
    \begin{multicols}{4}
    \begin{enumerate}[label=(\Alph*)]
        \item $0$
        \item $20$
        \item $30$
        \item $40$
    \end{enumerate}
\end{multicols}
\vspace{1cm}
\item[\textnormal{Q.14}]  Which one of the following is a heat treatment process for surface hardening?
    \hfill{(PI 2008)}
    \begin{multicols}{4}
    \begin{enumerate}[label=(\Alph*)]
        \item Normalizing
        \item Annneling
        \item Carburising
        \item Tempering
    \end{enumerate}
\end{multicols}
\vspace{1cm}
\item[\textnormal{Q.15}]  Which pair among the following solid waste welding process uses heat from a extrenal source?\\
P-Diffusion welding;Q-Friction welding;R-Ultrasonic welding;S-Forge welding
    \hfill{(PI 2008)}
    \begin{multicols}{4}
    \begin{enumerate}[label=(\Alph*)]
        \item P and R
        \item R and S
        \item Q and S
        \item P and S
    \end{enumerate}
\end{multicols}
\vspace{1cm}
\item[\textnormal{Q.16}]  In hollow cylindrical parts,made by centrifugal casting, the density of the part is 
    \hfill{(PI 2008)}
   \begin{itemize}[label={}]
        \item (A) maximum at outer region
        \item (B) maximum at inner region
        \item (C) maximum at the mid-point between outer and inner surfaces
        \item (D) uniform throught
    \end{itemize}
\vspace{1cm}

\item[\textnormal{Q.17}] Brittle material are machined with tools having zero or negative rake angle because it  
    \hfill{(PI 2008)}

\begin{itemize}[label={}]
        \item (A) results in lower cutting force  
        \item (B) improve surface finish
        \item (C) provides adequate strenght to cutting tool
        \item (D) results in more accurate dimensions 
    \end{itemize}
\vspace{1cm}

\item[\textnormal{Q.18}] When 0.8\% carbon eutectoid steel is slowly cooled from $750^{\circ}$ to room temorature,
    \hfill{(PI 2008)}

\begin{itemize}[label={}]
        \item (A) austenite transforms to pearlite
        \item (B) pearlite transforms to austenite
        \item (C) austenite transforms to martensite  
        \item (D) pearlite transforms to martensite
    \end{itemize}
\vspace{1cm}
\item[\textnormal{Q.19}] Which one of the following is a unary operation performed in relational data model?
    \hfill{(PI 2008)}

\begin{itemize}[label={}]
        \item (A)Cartesian product            \item (B)Set union 
        \item (C)Set diffrence                \item (D)Selection 
    \end{itemize}
\vspace{1cm}
\item[\textnormal{Q.20}] The process of tracing through the MRP records and all levels in the product structure to identify how changes in the records of one component will effect the records of other components is known as
    \hfill{(PI 2008)}\\
    \begin{itemize}[label={}]
        \item (A) product exlosion
        \item (B) lead time offsetting
        \item (C) updating
        \item (D) pegging
    \end{itemize}
\vspace{1cm}
\textbf{Q.21-Q.75 carry two marks each}
\vspace{1cm}
\item[\textnormal{Q.21}] The eigenvector pair of the matrix
\[\begin{pmatrix} 
3 & 4 \\
4 & -3
\end{pmatrix}\] is
    \hfill{(PI 2008)}\\
    \[\begin{matrix}
   { (A) \begin{Bmatrix} 2 \\ 1
    \end{Bmatrix}  and \begin{Bmatrix} 1 \\-2
    \end{Bmatrix}} & {(B) \begin{Bmatrix} 2 \\ 1
    \end{Bmatrix} and \begin{Bmatrix} 1 \\-2
    \end{Bmatrix}}\\
    { (C) \begin{Bmatrix} 2 \\ 1
    \end{Bmatrix}  and \begin{Bmatrix} 1 \\-2
    \end{Bmatrix}} & {(D) \begin{Bmatrix} 2 \\ 1
    \end{Bmatrix} and \begin{Bmatrix} 1 \\-2
    \end{Bmatrix}}
    
    \end{matrix}\]
    \vspace{1cm}
    \item[\textnormal{Q.22}]  If the interval of integration is divided into two equal intervals of width 1.0, the value of the definite integral $\int_1^3 log_e x dx$, using Simpson's one-third rule, will be
    \hfill{(PI 2008)}
    \begin{multicols}{4}
    \begin{enumerate}[label=(\Alph*)]
        \item $0.50$
        \item $0.80$
        \item $1.00$
        \item $1.29$
    \end{enumerate}
\end{multicols}
\item[\textnormal{Q.23}]  In a game,two players X and Y toss a coin alternately. Whosoever gets a 'head' first, wins the game and game is terminated. Assuming  that player X starts the game, the probability of player X winning is 
    \hfill{(PI 2008)}
    \begin{multicols}{4}
    \begin{enumerate}[label=(\Alph*)]
        \item $\frac{1}{3}$
        \item $\frac{1}{2}$
        \item $\frac{2}{3}$
        \item $\frac{3}{4}$
    \end{enumerate}
\end{multicols}
\item[\textnormal{Q.24}] Laplace transform of $\sinh (t)$ is  
    \hfill{(PI 2008)}
    \begin{multicols}{4}
    \begin{enumerate}[label=(\Alph*)]
        \item $\frac{1}{s^2 -1}$
        \item $\frac{1}{1-s^2}$
        \item $\frac{s}{s^2-1}$
        \item $\frac{s}{1-s^2}$
    \end{enumerate}
\end{multicols}
\item[\textnormal{Q.25}] A resorvoir contains an estimated 30,00,000 barrel of oil. The initial cost of the reservoir is Rs. 1,50,00,000. If 2,00,000 barrrels of oil are produced from the resorvoir during a particular year, how much will be the deplition charge (cost depletion) for that year?  
    \hfill{(PI 2008)}
    \begin{multicols}{4}
    \begin{enumerate}[label=(\Alph*)]
        \item $Rs.10,00,000$
        \item $Rs.15,00,000$
        \item $Rs.20,00,000$
        \item $Rs.25,00,000$
    \end{enumerate}
\end{multicols}
\item[\textnormal{Q.26}]  Customer arrives at a service counter nammed by a single person according to a Poisson distribution with a mean arrival rate of 30per hour. The time required to serve a customer follow and exponential distribuation with a mean of 100 seconds. The average waiting time (in hour) of a customer in the system will be
    \hfill{(PI 2008)}
    \begin{multicols}{4}
    \begin{enumerate}[label=(\Alph*)]
        \item $0.138$
        \item $0.166$
        \item $0.276$
        \item $0.332$
    \end{enumerate}
\end{multicols}
\item[\textnormal{Q.27}] Consider the following linear programming problem (LPP)  \\ \\
Maximize $z=5x_1+3x_2$\\
Subject to the following constraints
$x_1-x_2\leq2$\\
$x_1+x_2\geq 3$\\
$x_1,x_2\geq 0$
    \hfill{(PI 2008)}
    \begin{itemize}[label={}]
        \item (A) no solution
        \item (B) unique solution
        \item (C) two solution
        \item (D) unbounded solution
    \end{itemize}
\vspace{1cm}
 \item[\textnormal{Q.28}]  A machine costing Rs.2 lakh (salvage value of the machine at end of 4 years $= 0$) is to be depreciated over 4 years using the double declining balance depreciation method. The amount of depresiation changes in $3^{rd}$ year is 
    \hfill{(PI 2008)}
    \begin{multicols}{4}
    \begin{enumerate}[label=(\Alph*)]
        \item Rs. $1.00$ lakh
        \item Rs. $0.50$ lakh
        \item Rs. $0.25$ lakh
        \item Rs. $0.125$ lakh
    \end{enumerate}
\end{multicols}
\vspace{1cm}
 \item[\textnormal{Q.29}]  During a survey of customers in a store,20 samples of size 200 customers were taken. The number of dissatisfied customers was found to be 180. The upper and lower control limits for the control chart of disstisfied customers will be 
    \hfill{(PI 2008)}\\
\[ \begin{matrix}
       { (A) 18.345,0.205} & { (A) 17.345,0.605} \\
       { (A) 17.345,0.805} & { (A) 16.345,0.705}
    \end{matrix}  \]
   
\vspace{1cm}

 \item[\textnormal{Q.30}]  An assembly has 10 components in series. Each component has an exponential time-to-failure distribution with a constant failure rare of 0.02 per 3000 hours of operation. Assuming that the failed component of the assembly is replaced immediately with another component that has the same failure rate, the relibility of the assmebly for 2000 hours of operation and the mean time-to-failure(MTTF) is 
    \hfill{(PI 2008)}
    \[
    \begin{matrix}
        {(A) 0.875,10,000 hours } & {(B)0.675,20,000 hours}\\
        {(C)0.975,40,000 hours}  & {(D)0.875,15,000 hours}
    \end{matrix}
    \]
    \vspace{1cm}
    \item[\textnormal{Q.31}] Match the following: 
    \hfill{(PI 2008)}\\
    \vspace{1cm}
    \begin{tabular}{p{4cm} p{6cm}}
\textbf{Group 1} & \textbf{Group 2} \\
    P--SLP                  & 1--Intellectual property system \\
    Q--Margin of Safety     & 2--Assembly line balancing\\
    R--LOB                  & 3--Facility design\\
    S--TRIPS                & Break even analysis\\
    \end{tabular}
    \vspace{1cm}
     \item[\textnormal{Q.32}]  A man has deposited Rs. 1,000 per year for three year in bank that paid him 5\% intrest compounded annually. At the end three years,he had Rs 3,153 in his account. How much more would he have earned if the bannk had paid him 5\% intrest compounded continuously?
    \hfill{(PI 2008)}
    \begin{multicols}{4}
    \begin{enumerate}[label=(\Alph*)]
        \item $Rs.300$
        \item $Rs30$
        \item $Rs3$
        \item $Rs0.30$
    \end{enumerate}
\end{multicols}
\vspace{1cm}

    \item[\textnormal{Q.33}]  Two pipes of uniform section but differnt diameter carry  water at the same volumetric flow rate. Water properties are the same in the two pipes. The Reynolds number,based on the pipe diameter.
    \hfill{(PI 2008)}
    \[
    \begin{matrix}
        {(A) \text{is the same in both pipes}  } & {(B)\text{is larger in the narrower pipe}}\\
        {(C)\text{is smaller in the narrower pipe}}  & {(D)\text{depends on the pipe material}}
    \end{matrix}
    \]
    \vspace{1cm}
     \item[\textnormal{Q.34}]  A single cylinder compression ignition engine, operating on the air-standard diesel cycle,has a mean effective pressure of 1.0MPa and a compression ratio of 21. The engine has a clearence volume of $5\times10^{-5}m^3$. The heat added at constant pressure is 2.0KJ. The thermal effciency of the engine is 
    \hfill{(PI 2008)}
    \begin{multicols}{4}
    \begin{enumerate}[label=(\Alph*)]
        \item $10\%$
        \item $35\%$
        \item $50\%$
        \item $70\%$
    \end{enumerate}
\end{multicols}
\vspace{1cm}
 \item[\textnormal{Q.35}]  An industrial gas $(C_p=1 KJ/kgK)$enters a parallel-flow heat exchanger at $250^{\circ}C$ with a flow rate of $2Kg/s$. The heat a water stream. Thenwater stream $(C_p=4KJ/kgK)$ enters the heat exchanger at  $50^{\circ}$ with a flow rate of $1kg/s$. The heat exchanger has an effectiveness of 0.75. The gas stream exit temprature will be 

    \hfill{(PI 2008)}
    \begin{multicols}{4}
    \begin{enumerate}[label=(\Alph*)]
        \item  $75^{\circ}$
        \item  $100^{\circ}$

        \item  $125^{\circ}$

        \item  $150^{\circ}$

    \end{enumerate}
\end{multicols}
\vspace{1cm}
 \item[\textnormal{Q.36}] Oil is being pumped through a straight pipe. The pipe lenght, diameter and volumetric flow rate are all doubled in a new arrangment. The pipe friction factor, however, remains constant. The ratio of pipe frictional loses in the new arrangment to that in the original configuration would be   
    \hfill{(PI 2008)}
    \begin{multicols}{4}
    \begin{enumerate}[label=(\Alph*)]
        \item $\frac{1}{4}$
        \item $\frac{1}{2}$
        \item $2$
        \item $4$
    \end{enumerate}
\end{multicols}
\vspace{1cm}
 \item[\textnormal{Q.37}] Air flows steadily at low speed throught a horizontal nozzle, which dischrages the air into the atmosphere. The area at the nozzle inlet and outlet are $0.1m^2$ and $0.2m^2$ respectively. If the air density remains constant at $1.0kg/m^3$, the gauge pressure(in kPa) required at the nozzle inlet to produce an outlet speed of 50 m/s would be  
    \hfill{(PI 2008)}
    \begin{multicols}{4}
    \begin{enumerate}[label=(\Alph*)]
        \item $0.6$
        \item $1.2$
        \item $100.2$
        \item $101.2$
    \end{enumerate}
\end{multicols}
\vspace{1cm}
 \item[\textnormal{Q.38}] Heat is being transferred convextively from a cylindrical nuclear reactor fuel rod of 50 mm diameter to water at $75^{\circ}C$, Under steady state condition, the rate of heat genration within the fuel element is $5\times10^7W/m^3$ and the convection heat transfer coefficient is $1kW/m^2K$. \\
 The outer surface surface temprature of the fuel element would be
    \hfill{(PI 2008)}
    \begin{multicols}{4}
    \begin{enumerate}[label=(\Alph*)]
        \item $700^{\circ}C$
        \item $625^{\circ}C$
        \item $550^{\circ}C$
        \item $400^{\circ}C$
    \end{enumerate}
\end{multicols}
\vspace{1cm}
 \item[\textnormal{Q.39}] In an assembly , the dimension of a component should be between 20mm and 30 mm. Twenty five components were taken at random during the manufacturing of the components. The mean value of the dimension and the standard deviation of the 25 components were 26mm and 2mm respectively. The process capability index $C_{pk}$of the concerned manufacturing process 
 would be 
    \hfill{(PI 2008)}
    \begin{multicols}{4}
    \begin{enumerate}[label=(\Alph*)]
        \item $0.33$
        \item $0.67$
        \item $0.83$
        \item $1.00$
    \end{enumerate}
\end{multicols}
\vspace{1cm}
 \item[\textnormal{Q.40}] A three-component welded cylindrical assembly is shown below. The mean lenght of the three components and their respective tolerance (both in mm) are given int the table below. 
    \hfill{(PI 2008)}\\
    
    
        \includegraphics[width=0.5\linewidth]{figures/gate-pi-2008-40.png}
        \label{fig:placeholder}
        
\begin{tabularx}{\linewidth}{|X|X|X|}
\hline
Component & Mean Length (mm) & Tolerance (mm) \\
\hline 
P & $X_1=18$   & $\pm1.2$   \\
\hline
Q & $X_2=23$  & $\pm1.0$ \\
\hline
R & $X_3=24$ & $\pm1.5$\\
\hline
\end{tabularx}
    \begin{multicols}{4}
    \begin{enumerate}[label=(\Alph*)]
        \item $65\pm2.16$
        \item $65\pm1.16$
        \item $65\pm6.16$
        \item $65\pm0.16$
    \end{enumerate}
\end{multicols}
\vspace{1cm}


















    




    
    \end{enumerate}
\end{document}
