\documentclass{beamer}
\usepackage[utf8]{inputenc}

\usetheme{Madrid}
\usecolortheme{default}
\usepackage{amsmath,amssymb,amsfonts,amsthm}
\usepackage{txfonts}
\usepackage{tkz-euclide}
\usepackage{listings}
\usepackage{adjustbox}
\usepackage{array}
\usepackage{tabularx}
\usepackage{gvv}
\usepackage{lmodern}
\usepackage{circuitikz}
\usepackage{tikz}
\usepackage{graphicx}

\setbeamertemplate{page number in head/foot}[totalframenumber]

\usepackage{tcolorbox}
\tcbuselibrary{minted,breakable,xparse,skins}



\definecolor{bg}{gray}{0.95}
\DeclareTCBListing{mintedbox}{O{}m!O{}}{%
  breakable=true,
  listing engine=minted,
  listing only,
  minted language=#2,
  minted style=default,
  minted options={%
    linenos,
    gobble=0,
    breaklines=true,
    breakafter=,,
    fontsize=\small,
    numbersep=8pt,
    #1},
  boxsep=0pt,
  left skip=0pt,
  right skip=0pt,
  left=25pt,
  right=0pt,
  top=3pt,
  bottom=3pt,
  arc=5pt,
  leftrule=0pt,
  rightrule=0pt,
  bottomrule=2pt,
  toprule=2pt,
  colback=bg,
  colframe=orange!70,
  enhanced,
  overlay={%
    \begin{tcbclipinterior}
    \fill[orange!20!white] (frame.south west) rectangle ([xshift=20pt]frame.north west);
    \end{tcbclipinterior}},
  #3,
}
\lstset{
    language=C,
    basicstyle=\ttfamily\small,
    keywordstyle=\color{blue},
    stringstyle=\color{orange},
    commentstyle=\color{green!60!black},
    numbers=left,
    numberstyle=\tiny\color{gray},
    breaklines=true,
    showstringspaces=false,
}
%------------------------------------------------------------
%This block of code defines the information to appear in the
%Title page
\title %optional
{1.3.5}
\date{August 22,2025}
%\subtitle{A short story}

\author % (optional)
{Rathlavath Jeevan-AI25BTECH11026}



\begin{document}


\frame{\titlepage}
\begin{frame}{Question}
If $(3,3), (6,y), (x,7)$ and $(5,6)$ are the vertices of a parallelogram taken in order, 
find the values of $x$ and $y$.
\end{frame}



\begin{frame}{Theoretical Solution}
\textbf{Solution:} \\
In a parallelogram, the diagonals bisect each other. Therefore, the midpoint of diagonal 
joining $(3,3)$ and $(x,7)$ is equal to the midpoint of diagonal joining $(6,y)$ and $(5,6)$.
begin{align}
($\frac{3+x}{2}, \frac{3+7}{2}$) 
= ($\frac{6+5}{2}, \frac{y+6}{2}$)
end{align}

$
\left(\frac{3+x}{2}, 5\right) = \left(\frac{11}{2}, \frac{y+6}{2}\right)
$

Equating the coordinates, we get:
$
\frac{3+x}{2} = \frac{11}{2} \quad \Rightarrow \quad x=8
$
$
5 = \frac{y+6}{2} \quad \Rightarrow \quad y=4
$

\bigskip
\textbf{Final Answer:} \quad $x=8, \; y=4$
\end{frame}




\begin{frame}[fragile]
    \frametitle{C Code }

    \begin{lstlisting}
#include <stdio.h>

int main() {
    int x, y;

    // Using midpoint property of diagonals of parallelogram
    x = 11 - 3;   // From (x+3)/2 = 11/2
    y = 10 - 6;   // From (y+6)/2 = 5

    printf("The values are: x = %d, y = %d\n", x, y);

    return 0;
}

    \end{lstlisting}
\end{frame}



\begin{frame}[fragile]
    \frametitle{Python Code}
    \begin{lstlisting}
# Using midpoint property of diagonals of parallelogram

# From (x+3)/2 = 11/2  => x = 11 - 3
x = 11 - 3  

# From (y+6)/2 = 5  => y = 10 - 6
y = 10 - 6  

print(f"The values are: x = {x}, y = {y}")


    \end{lstlisting}
\end{frame}

\begin{frame}[fragile]
    \frametitle{Python Code Using C Functions}
    \begin{lstlisting}
    
# Function to calculate x using diagonal midpoint property
def find_x():
    # (3 + x)/2 = (6 + 5)/2
    return 11 - 3   # x = 8

# Function to calculate y using diagonal midpoint property
def find_y():
    # (3 + 7)/2 = (y + 6)/2
    return 10 - 6   # y = 4

# Main function (like in C)
def main():
    x = find_x()
    y = find_y()
    print(f"The values are: x = {x}, y = {y}")

# Run main
if __name__ == "__main__":
    main()
     \end{lstlisting}

\end{frame}

    

\begin{frame}{Plot}
    \centering
    \includegraphics[width=\columnwidth, height=0.8\textheight, keepaspectratio]{Figs/IMG-20250829-WA0000.jpg}     
\end{frame}




\end{document}