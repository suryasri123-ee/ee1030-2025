\documentclass{beamer}
\usepackage[utf8]{inputenc}

\usetheme{Madrid}
\usecolortheme{default}
\usepackage{amsmath,amssymb,amsfonts,amsthm}
\usepackage{txfonts}
\usepackage{tkz-euclide}
\usepackage{listings}
\usepackage{adjustbox}
\usepackage{array}
\usepackage{tabularx}
\usepackage{gvv}
\usepackage{lmodern}
\usepackage{circuitikz}
\usepackage{tikz}
\usepackage{graphicx}

\setbeamertemplate{page number in head/foot}[totalframenumber]

\usepackage{tcolorbox}
\tcbuselibrary{minted,breakable,xparse,skins}



\definecolor{bg}{gray}{0.95}
\DeclareTCBListing{mintedbox}{O{}m!O{}}{%
  breakable=true,
  listing engine=minted,
  listing only,
  minted language=#2,
  minted style=default,
  minted options={%
    linenos,
    gobble=0,
    breaklines=true,
    breakafter=,,
    fontsize=\small,
    numbersep=8pt,
    #1},
  boxsep=0pt,
  left skip=0pt,
  right skip=0pt,
  left=25pt,
  right=0pt,
  top=3pt,
  bottom=3pt,
  arc=5pt,
  leftrule=0pt,
  rightrule=0pt,
  bottomrule=2pt,
  toprule=2pt,
  colback=bg,
  colframe=orange!70,
  enhanced,
  overlay={%
    \begin{tcbclipinterior}
    \fill[orange!20!white] (frame.south west) rectangle ([xshift=20pt]frame.north west);
    \end{tcbclipinterior}},
  #3,
}
\lstset{
    language=C,
    basicstyle=\ttfamily\small,
    keywordstyle=\color{blue},
    stringstyle=\color{orange},
    commentstyle=\color{green!60!black},
    numbers=left,
    numberstyle=\tiny\color{gray},
    breaklines=true,
    showstringspaces=false,
}
%------------------------------------------------------------
%This block of code defines the information to appear in the
%Title page
\title %optional
{1.3.5}
\date{August 22,2025}
%\subtitle{A short story}

\author % (optional)
{Rathlavath Jeevan-AI25BTECH11026}



\begin{document}


\frame{\titlepage}
\begin{frame}{Question}
If $(3,3), (6,y), (x,7)$ and $(5,6)$ are the vertices of a parallelogram taken in order, 
find the values of $x$ and $y$.
\end{frame}



\begin{frame}{Theoretical Solution}
\textbf{Solution:} \\
In a parallelogram, the diagonals bisect each other. Therefore, the midpoint of diagonal 
joining $(3,3)$ and $(x,7)$ is equal to the midpoint of diagonal joining $(6,y)$ and $(5,6)$.
\begin{align}
    \vec{A}=\begin{myvec}{3\\3}\end{myvec}\;
    \vec{B}=\begin{myvec}{6\\y}\end{myvec}\;
    \vec{C}=\begin{myvec}{x\\7}\end{myvec}\;         \vec{D}=\begin{myvec}{5\\6}\end{myvec}\
\end{align}
condition for the given points to form a parallelogram.\\
\begin{align}
    \vec{B}-\vec{A}=\vec{C}-\vec{D}
\end{align}


\begin{align}
    \vec{B}-\vec{A}=\begin{myvec}
        {3\\y-3}
    \end{myvec} \;
    \vec{C}-\vec{D}=\begin{myvec}
        {x-5\\1}\
    \end{myvec}
\end{align}
$\therefore$ $x=8, \; y=4$  
\end{frame}




\begin{frame}[fragile]
    \frametitle{C Code }

    \begin{lstlisting}
#include <stdio.h>

int main() {
    int x, y;

    // Using midpoint property of diagonals of parallelogram
    x = 11 - 3;   // From (x+3)/2 = 11/2
    y = 10 - 6;   // From (y+6)/2 = 5

    printf("The values are: x = %d, y = %d\n", x, y);

    return 0;
}

    \end{lstlisting}
\end{frame}



\begin{frame}[fragile]
    \frametitle{Python Code}
    \begin{lstlisting}
import matplotlib.pyplot as plt
from mpl_toolkits.mplot3d import Axes3D

# Given and solved coordinates
A = (3, 3, 0)
B = (6, 4, 0)   # y = 4
C = (8, 7, 0)   # x = 8
D = (5, 6, 0)

# Vertices in order, and close the parallelogram by repeating the first point
vertices = [A, B, C, D, A]

# Unpack coordinates
xs, ys, zs = zip(*vertices)

# Plotting
fig = plt.figure()
ax = fig.add_subplot(111, projection='3d')




    \end{lstlisting}
\end{frame}

\begin{frame}[fragile]
    \frametitle{Python Code }
    \begin{lstlisting}
    
# Plot the edges
ax.plot(xs, ys, zs, label='Parallelogram', color='blue')

# Plot the points
ax.scatter(xs, ys, zs, color='red', s=50)

# Annotate each point
labels = ['A(3,3)', 'B(6,4)', 'C(8,7)', 'D(5,6)', 'A']
for i, (x, y, z) in enumerate(vertices):
    ax.text(x, y, z + 0.1, labels[i], fontsize=10)

# Setting labels
ax.set_xlabel('X')
ax.set_ylabel('Y')
ax.set_zlabel('Z')
ax.set_title('Parallelogram in 3D (Z=0)')
     \end{lstlisting}

\end{frame}
\begin{frame}[fragile]
    \frametitle{Python Code }
    \begin{lstlisting}
    # Set the view angle for better 3D effect
ax.view_init(elev=20, azim=30)

# Save as PNG
plt.savefig('parallelogram_3d.png', dpi=300)
plt.show()
   \end{lstlisting}

\end{frame}
\begin{frame}{Plot}
    \centering
    \includegraphics[width=\columnwidth, height=0.8\textheight, keepaspectratio]{Figs/Img.jpg}     
\end{frame}




\end{document}