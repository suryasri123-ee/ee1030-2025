\let\negmedspace\undefined
\let\negthickspace\undefined
\documentclass[journal,12pt,onecolumn]{IEEEtran}
\usepackage{cite}
\usepackage{amsmath,amssymb,amsfonts,amsthm}
\usepackage{algorithmic}
\usepackage{graphicx}
\graphicspath{{./figs/}}
\usepackage{textcomp}
\usepackage{xcolor}
\usepackage{txfonts}
\usepackage{listings}
\usepackage{enumitem}
\usepackage{mathtools}
\usepackage{gensymb}
\usepackage{comment}
\usepackage{caption}
\usepackage[breaklinks=true]{hyperref}
\usepackage{tkz-euclide} 
\usepackage{listings}
\usepackage{gvv}                                        
%\def\inputGnumericTable{}                                 
\usepackage[latin1]{inputenc}     
\usepackage{xparse}
\usepackage{color}                                            
\usepackage{array}                                            
\usepackage{longtable}                                       
\usepackage{calc}                                             
\usepackage{multirow}
\usepackage{multicol}
\usepackage{hhline}                                           
\usepackage{ifthen}                                           
\usepackage{lscape}
\usepackage{tabularx}
\usepackage{array}
\usepackage{float}
%\newtheorem{theorem}{Theorem}[section]
%\newtheorem{theorem}{Theorem}[section]
%\newtheorem{problem}{Problem}
%\newtheorem{proposition}{Proposition}[section]
%\newtheorem{lemma}{Lemma}[section]
%\newtheorem{corollary}[theorem]{Corollary}
%\newtheorem{example}{Example}[section]
%\newtheorem{definition}[problem]{Definition}

\begin{document}

%\textbf{\Large 1.3.6} \\
%\textbf{\large AI25BTECH11027 - NAGA BHUVANA} \\
\title{1.3.6}
\author{AI25BTECH11027 - NAGA BHUVANA}
% \maketitle
% \newpage
% \bigskip
%\begin{document}
{\let\newpage\relax\maketitle}
%\renewcommand{\thefigure}{\theenumi}
%\renewcommand{\thetable}{\theenumi}

		\textbf{Question}:

		\noindent Show that the points $\textbf{A}\brak{6,2}$,$\textbf{B}\brak{2,1}$,$\textbf{C}\brak{1,5}$ and $\textbf{D}\brak{5,6}$ are vertices of a square.\\
		\textbf{Solution:}\\
      Given that


		\begin{align}
			\vec{A} = \myvec{6\\2} ,\vec{B}=\myvec{2\\1} ,\vec{C}=\myvec{1\\5} , \vec{D}=\myvec{5\\6}
		\end{align}
		\begin{align}
		    \vec{B}-\vec{A}=\myvec{2-6\\1-2}=\myvec{-4\\-1}
		\end{align}
        \begin{align}
            \vec{C}-\vec{D}=\myvec{1-5\\5-6}=\myvec{-4\\-1}
        \end{align}
        \begin{align}
            \vec{B}-\vec{A}=\vec{C}-\vec{D}
        \end{align}
        By the above property we can say that \textbf{ABCD} is a parallelogram.\\
        Consider the sides\\
        \begin{align}
            \vec{A-D}=\myvec{6-5\\2-6}=\myvec{1\\-4}
        \end{align}
        \begin{align}
            \vec{(B-A)}^T=\myvec{-4 & -1}
        \end{align}
        \begin{align}
            \|\vec{B-A}\|=\sqrt{17}\\
            \|\vec{A-D}\|=\sqrt{17}\\
        \end{align}
      Consider the angle $\theta$ between the sides $\vec{B-A}$ and $\vec{A-D}$ of the parallelogram\\
      \begin{align}
      \cos{\theta}=\frac{\myvec{B-A}^T\myvec{A-D}}{\|\vec{B-A}\|\|\vec{A-D}\|}
      \end{align}
      \begin{align}
          \cos{\theta}=\frac{\myvec{-4 & -1}\myvec{1\\-4}}{\sqrt{17}\sqrt{17}}\\
          \cos{\theta}=\frac{(-4)(1)+(-1)(-4)}{17}\\
      \end{align}
      \begin{align}
          \cos{\theta}=0\\
      \end{align}
      \begin{center}
      $\theta=90^\circ$
      \end{center}
      \textbf{Property:}\\
      A parallelogram with one angle $90^\circ$ is a rectangle\\
      Hence the parallelogram is a rectangle\\
      \begin{align}
      \vec{A-C}=\myvec{5\\-3}\\
      \implies \vec{(A-C)}^T=\myvec{5 & -3}
      \end{align}
      \begin{align}
          \vec{B-D}=\myvec{-3\\-5}
      \end{align}
      Let the angle between the diagonals of the rectangle be $\alpha$\\
      Now Consider the inner product of the diagonals of rectangle $\vec{A-C}$ and $\vec{B-D}$ \\
      \begin{align}
      \cos{\alpha}=\frac{\myvec{A-C}^T\myvec{B-D}}{\|\vec{A-C}\|\|\vec{B-D}\|}=\frac{\myvec{5&-3}\myvec{-3\\-5}}{\sqrt{34}\sqrt{34}}
      \end{align}
      \begin{align}
          \cos{\alpha}=0\\
          \cos{\alpha}=90^\circ
      \end{align}
      \textbf{Property:}\\
      Rectangle with diagonals at right angle is a square\\
      Hence given points forms a square\\
      \begin{figure}[H]
          \centering
          \includegraphics[width=0.7\linewidth]{figs/fig1.png}
	      \caption{}
	      \label{fig}
      \end{figure}
\end{document}
