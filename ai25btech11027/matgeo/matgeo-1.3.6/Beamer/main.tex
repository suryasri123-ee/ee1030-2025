\documentclass{beamer}
\mode<presentation>
\usepackage{amsmath}
\usepackage{amssymb}
%\usepackage{advdate}
\usepackage{graphicx}
\graphicspath{{./figs/}}
\usepackage{adjustbox}
\usepackage{subcaption}
\usepackage{enumitem}
\usepackage{multicol}
\usepackage{mathtools}
\usepackage{listings}
\usepackage{url}
\def\UrlBreaks{\do\/\do-}
\usetheme{Boadilla}
\usecolortheme{lily}
\setbeamertemplate{footline}
{
  \leavevmode%
  \hbox{%
  \begin{beamercolorbox}[wd=\paperwidth,ht=2.25ex,dp=1ex,right]{author in head/foot}%
    \insertframenumber{} / \inserttotalframenumber\hspace*{2ex} 
  \end{beamercolorbox}}%
  \vskip0pt%
}
\setbeamertemplate{navigation symbols}{}

\providecommand{\nCr}[2]{\,^{#1}C_{#2}} % nCr
\providecommand{\nPr}[2]{\,^{#1}P_{#2}} % nPr
\providecommand{\mbf}{\mathbf}
\providecommand{\pr}[1]{\ensuremath{\Pr\left(#1\right)}}
\providecommand{\qfunc}[1]{\ensuremath{Q\left(#1\right)}}
\providecommand{\sbrak}[1]{\ensuremath{{}\left[#1\right]}}
\providecommand{\lsbrak}[1]{\ensuremath{{}\left[#1\right.}}
\providecommand{\rsbrak}[1]{\ensuremath{{}\left.#1\right]}}
\providecommand{\brak}[1]{\ensuremath{\left(#1\right)}}
\providecommand{\lbrak}[1]{\ensuremath{\left(#1\right.}}
\providecommand{\rbrak}[1]{\ensuremath{\left.#1\right)}}
\providecommand{\cbrak}[1]{\ensuremath{\left\{#1\right\}}}
\providecommand{\lcbrak}[1]{\ensuremath{\left\{#1\right.}}
\providecommand{\rcbrak}[1]{\ensuremath{\left.#1\right\}}}
\theoremstyle{remark}
\newtheorem{rem}{Remark}
\newcommand{\sgn}{\mathop{\mathrm{sgn}}}
\providecommand{\abs}[1]{\left\vert#1\right\vert}
\providecommand{\res}[1]{\Res\displaylimits_{#1}} 
\providecommand{\norm}[1]{\lVert#1\rVert}
\providecommand{\mtx}[1]{\mathbf{#1}}
\providecommand{\mean}[1]{E\left[ #1 \right]}
\providecommand{\fourier}{\overset{\mathcal{F}}{ \rightleftharpoons}}
%\providecommand{\hilbert}{\overset{\mathcal{H}}{ \rightleftharpoons}}
\providecommand{\system}[1]{\overset{\mathcal{#1}}{ \longleftrightarrow}}
%\providecommand{\system}{\overset{\mathcal{H}}{ \longleftrightarrow}}
	%\newcommand{\solution}[2]{\textbf{Solution:}{#1}}
%\newcommand{\solution}{\noindent \textbf{Solution: }}
\providecommand{\dec}[2]{\ensuremath{\overset{#1}{\underset{#2}{\gtrless}}}}
\newcommand{\myvec}[1]{\ensuremath{\begin{pmatrix}#1\end{pmatrix}}}
\let\vec\mathbf

\lstset{
%language=C,
frame=single, 
breaklines=true,
columns=fullflexible
}

\numberwithin{equation}{section}
\title{1.3.6}
\author{AI25BTECH11027 - NAGA BHUVANA}
% \maketitle
% \newpage
% \bigskip
\begin{document}
{\let\newpage\relax\maketitle}
\renewcommand{\thefigure}{\theenumi}
\renewcommand{\thetable}{\theenumi}
		\textbf{Question}:

		\noindent Show that the points $\vec{A}\brak{6,2}$,$\vec{B}\brak{2,1}$,$\vec{C}\brak{1,5}$ and $\vec{D}\brak{5,6}$ are vertices of a square.\\
		\textbf{Solution:}\\
       
      Given that


		\begin{align}
			\vec{A} = \myvec{6\\2} ,\vec{B}=\myvec{2\\1} ,\vec{C}=\myvec{1\\5} , \vec{D}=\myvec{5\\6}
		\end{align}
		\begin{align}
		    \vec{B}-\vec{A}=\myvec{2-6\\1-2}=\myvec{-4\\-1}
		\end{align}
        \begin{align}
            \vec{C}-\vec{D}=\myvec{1-5\\5-6}=\myvec{-4\\-1}
        \end{align}
        \begin{align}
            \vec{B}-\vec{A}=\vec{C}-\vec{D}
        \end{align}
        By the above property we can say that \textbf{ABCD} is a parallelogram.\\
      Consider the inner product of the vectors $\vec{(B-A)}$ and $\vec{(C-B)}$ \\
      \begin{align}
	      \implies \vec{(B-A)}^T \vec{(C-B)}=(-4)(-1)+(-1)(4)=0
      \end{align}
      Hence the angle at the vertex B is $90^\circ$ \\
      \textbf{Property:}\\
      A parallelogram with one angle $90^\circ$ is a rectangle\\
      Hence the parallelogram is a rectangle\\
      Now consider the inner product of the diagonals of the rectangle $\vec{(C-A)}$ and $\vec{(D-B)}$\\
	\begin{align}
		\implies \vec{(C-A)}^T \vec{(D-B)} =(-5)(3)+(3)(5)=0
	\end{align}
	Hence the angle between the diagonals of the rectangle is $90^\circ$\\
      \textbf{Property:}\\
      Rectangle with diagonals at right angle is a square\\
      Hence given points forms a square\\
      % Graphical representation
      \begin{frame}
      \frametitle{Graphical Representation}
          \centering
          \includegraphics[width=0.6\linewidth]{figs/fig1.png}
      \end{frame}
\end{document}
     
