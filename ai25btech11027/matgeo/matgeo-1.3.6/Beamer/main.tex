\documentclass{beamer}
\mode<presentation>
\usepackage{amsmath}
\usepackage{amssymb}
%\usepackage{advdate}
\usepackage{graphicx}
\graphicspath{{./figs/}}
\usepackage{adjustbox}
\usepackage{subcaption}
\usepackage{enumitem}
\usepackage{multicol}
\usepackage{mathtools}
\usepackage{listings}
\usepackage{url}
\def\UrlBreaks{\do\/\do-}
\usetheme{Boadilla}
\usecolortheme{lily}
\setbeamertemplate{footline}
{
  \leavevmode%
  \hbox{%
  \begin{beamercolorbox}[wd=\paperwidth,ht=2.25ex,dp=1ex,right]{author in head/foot}%
    \insertframenumber{} / \inserttotalframenumber\hspace*{2ex} 
  \end{beamercolorbox}}%
  \vskip0pt%
}
\setbeamertemplate{navigation symbols}{}

\providecommand{\nCr}[2]{\,^{#1}C_{#2}} % nCr
\providecommand{\nPr}[2]{\,^{#1}P_{#2}} % nPr
\providecommand{\mbf}{\mathbf}
\providecommand{\pr}[1]{\ensuremath{\Pr\left(#1\right)}}
\providecommand{\qfunc}[1]{\ensuremath{Q\left(#1\right)}}
\providecommand{\sbrak}[1]{\ensuremath{{}\left[#1\right]}}
\providecommand{\lsbrak}[1]{\ensuremath{{}\left[#1\right.}}
\providecommand{\rsbrak}[1]{\ensuremath{{}\left.#1\right]}}
\providecommand{\brak}[1]{\ensuremath{\left(#1\right)}}
\providecommand{\lbrak}[1]{\ensuremath{\left(#1\right.}}
\providecommand{\rbrak}[1]{\ensuremath{\left.#1\right)}}
\providecommand{\cbrak}[1]{\ensuremath{\left\{#1\right\}}}
\providecommand{\lcbrak}[1]{\ensuremath{\left\{#1\right.}}
\providecommand{\rcbrak}[1]{\ensuremath{\left.#1\right\}}}
\theoremstyle{remark}
\newtheorem{rem}{Remark}
\newcommand{\sgn}{\mathop{\mathrm{sgn}}}
\providecommand{\abs}[1]{\left\vert#1\right\vert}
\providecommand{\res}[1]{\Res\displaylimits_{#1}} 
\providecommand{\norm}[1]{\lVert#1\rVert}
\providecommand{\mtx}[1]{\mathbf{#1}}
\providecommand{\mean}[1]{E\left[ #1 \right]}
\providecommand{\fourier}{\overset{\mathcal{F}}{ \rightleftharpoons}}
%\providecommand{\hilbert}{\overset{\mathcal{H}}{ \rightleftharpoons}}
\providecommand{\system}[1]{\overset{\mathcal{#1}}{ \longleftrightarrow}}
%\providecommand{\system}{\overset{\mathcal{H}}{ \longleftrightarrow}}
	%\newcommand{\solution}[2]{\textbf{Solution:}{#1}}
%\newcommand{\solution}{\noindent \textbf{Solution: }}
\providecommand{\dec}[2]{\ensuremath{\overset{#1}{\underset{#2}{\gtrless}}}}
\newcommand{\myvec}[1]{\ensuremath{\begin{pmatrix}#1\end{pmatrix}}}
\let\vec\mathbf

\lstset{
%language=C,
frame=single, 
breaklines=true,
columns=fullflexible
}

\numberwithin{equation}{section}
\title{1.3.6}
\author{AI25BTECH11027 - NAGA BHUVANA}
% \maketitle
% \newpage
% \bigskip
\begin{document}
{\let\newpage\relax\maketitle}
\renewcommand{\thefigure}{\theenumi}
\renewcommand{\thetable}{\theenumi}
		\textbf{Question}:

		\noindent Show that the points $\textbf{A}\brak{6,2}$,$\textbf{B}\brak{2,1}$,$\textbf{C}\brak{1,5}$ and $\textbf{D}\brak{5,6}$ are vertices of a square.\\
		\textbf{Solution:}\\
       
      Given that


		\begin{align}
			\vec{A} = \myvec{6\\2} ,\vec{B}=\myvec{2\\1} ,\vec{C}=\myvec{1\\5} , \vec{D}=\myvec{5\\6}
		\end{align}
		\begin{align}
		    \vec{B}-\vec{A}=\myvec{2-6\\1-2}=\myvec{-4\\-1}
		\end{align}
        \begin{align}
            \vec{C}-\vec{D}=\myvec{1-5\\5-6}=\myvec{-4\\-1}
        \end{align}
        \begin{align}
            \vec{B}-\vec{A}=\vec{C}-\vec{D}
        \end{align}
        By the above property we can say that \textbf{ABCD} is a parallelogram.\\
        Consider the sides\\
        \begin{align}
            \vec{A-D}=\myvec{6-5\\2-6}=\myvec{1\\-4}
        \end{align}
        \begin{align}
            \vec{(B-A)}^T=\myvec{-4 & -1}
        \end{align}
        \begin{align}
            \|\vec{B-A}\|=\sqrt{17}\\
            \|\vec{A-D}\|=\sqrt{17}\\
        \end{align}
      Consider the angle $\theta$ between the sides $\vec{B-A}$ and $\vec{A-D}$ of the parallelogram\\
      \begin{align}
      \cos{\theta}=\frac{\myvec{B-A}^T\myvec{A-D}}{\|\vec{B-A}\|\|\vec{A-D}\|}
      \end{align}
      \begin{align}
          \cos{\theta}=\frac{\myvec{-4 & -1}\myvec{1\\-4}}{\sqrt{17}\sqrt{17}}\\
          \cos{\theta}=\frac{(-4)(1)+(-1)(-4)}{17}\\
      \end{align}
      \begin{align}
          \cos{\theta}=0\\
     \implies  \theta=90^\circ
      \end{align}
      \textbf{Property:}\\
      A parallelogram with one angle $90^\circ$ is a rectangle\\
      Hence the parallelogram is a rectangle\\
      \begin{align}
      \vec{A-C}=\myvec{5\\-3}\\
      \implies \vec{(A-C)}^T=\myvec{5 & -3}
      \end{align}
      \begin{align}
          \vec{B-D}=\myvec{-3\\-5}
      \end{align}
      Let the angle between the diagonals of the rectangle be $\alpha$\\
      Now Consider the inner product of the diagonals of rectangle $\vec{A-C}$ and $\vec{B-D}$ \\
      \begin{align}
      \cos{\alpha}=\frac{\myvec{A-C}^T\myvec{B-D}}{\|\vec{A-C}\|\|\vec{B-D}\|}=\frac{\myvec{5&-3}\myvec{-3\\-5}}{\sqrt{34}\sqrt{34}}
      \end{align}
      \begin{align}
          \cos{\alpha}=0\\
       \implies \alpha=90^\circ
      \end{align}
      \textbf{Property:}\\
      Rectangle with diagonals at right angle is a square\\
      Hence given points forms a square\\
      % Graphical representation
      \begin{frame}
      \frametitle{Graphical Representation}
          \centering
          \includegraphics[width=0.6\linewidth]{figs/fig1.png}
      \end{frame}
\end{document}
     
