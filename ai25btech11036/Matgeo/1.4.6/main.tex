\let\negmedspace\undefined
\let\negthickspace\undefined
\documentclass[journal]{IEEEtran}
\usepackage[a5paper, margin=10mm, onecolumn]{geometry}
%\usepackage{lmodern} % Ensure lmodern is loaded for pdflatex
\usepackage{tfrupee} % Include tfrupee package

\setlength{\headheight}{1cm} % Set the height of the header box
\setlength{\headsep}{0mm}     % Set the distance between the header box and the top of the text

\usepackage{gvv-book}
\usepackage{gvv}
\usepackage{cite}
\usepackage{amsmath,amssymb,amsfonts,amsthm}
\usepackage{algorithmic}
\usepackage{graphicx}
\usepackage{textcomp}
\usepackage{xcolor}
\usepackage{txfonts}
\usepackage{listings}
\usepackage{enumitem}
\usepackage{mathtools}
\usepackage{gensymb}
\usepackage{comment}
\usepackage[breaklinks=true]{hyperref}
\usepackage{tkz-euclide} 
\usepackage{listings}
% \usepackage{gvv}                                        
\def\inputGnumericTable{}                                 
\usepackage[latin1]{inputenc}                                
\usepackage{color}                                            
\usepackage{array}                                            
\usepackage{longtable}                                       
\usepackage{calc}                                             
\usepackage{multirow}                                         
\usepackage{hhline}                                           
\usepackage{ifthen}                                           
\usepackage{lscape}
\usepackage{circuitikz}
\tikzstyle{block} = [rectangle, draw, fill=blue!20, 
    text width=4em, text centered, rounded corners, minimum height=3em]
\tikzstyle{sum} = [draw, fill=blue!10, circle, minimum size=1cm, node distance=1.5cm]
\tikzstyle{input} = [coordinate]
\tikzstyle{output} = [coordinate]


\begin{document}

\bibliographystyle{IEEEtran}
\vspace{3cm}

\title{1.4.6}
\author{AI25BTECH11033--SNEHAMRUDULA}
 \maketitle
% \newpage
% \bigskip
{\let\newpage\relax\maketitle}

\renewcommand{\thefigure}{\theenumi}
\renewcommand{\thetable}{\theenumi}
\setlength{\intextsep}{10pt} % Space between text and floats


\numberwithin{equation}{enumi}
\numberwithin{figure}{enumi}
\renewcommand{\thetable}{\theenumi}


\title{Solution: Collinearity Problem}
\author{AI25BTECH11036-- SNEHAMRUDULA}
\date{}

\begin{document}
\maketitle

\item If the point $P(2,1)$ lies on the line segment joining points $A(4,2)$ and $B(8,4)$, then
\begin{enumerate}
    \item $AP = \tfrac{1}{4}AB$
    \item $AP = PE$
    \item $PB = \tfrac{1}{3}AB$
    \item $AP = \tfrac{3}{5}AB$
\end{enumerate}
\textbf{Solution:}\\
 \[
    \vec A=\myvec{4\\2},\quad 
    \vec B=\myvec{8\\4},\quad 
    \vec P=\myvec{2\\1}
    \]
    Direction vector:
    \[
    \vec m=\vec B-\vec A=\myvec{8\\4}-\myvec{4\\2}=\myvec{4\\2}
    \]

    Parameter:
    \[
    k=\frac{(\vec P-\vec A)^\top(\vec B-\vec A)}{\|\vec B-\vec A\|^2}
    =\frac{(\myvec{2\\1}-\myvec{4\\2})^\top\myvec{4\\2}}{4^2+2^2}
    =\frac{\myvec{-2\\-1}^\top\myvec{4\\2}}{20}
    =\frac{-8-2}{20}=-\frac{1}{2}.
    \]

    Hence,
    \[
    \vec P=\vec A+k(\vec B-\vec A)=\vec A-\frac{1}{2}(\vec B-\vec A)
    \]

    Since \(k<0\), point \(P\) lies on the line \(AB\) but \emph{outside} the segment (beyond \(A\)).

    \item \textbf{Lengths:}
    \[
    AB=\|\vec B-\vec A\|=\sqrt{4^2+2^2}=2\sqrt{5}, \qquad
    AP=\|\vec P-\vec A\|=\|\myvec{-2\\-1}\|=\sqrt{5}.
    \]

    Therefore,
    \[
    AP=\tfrac{1}{2}AB.
    \]

    \item \textbf{Conclusion:}  
    The correct relation is
    \[
    AP=\frac{1}{2}AB
    \]
    and \(P\) does not lie on the segment \(AB\).  
    Hence, the given statement is \textbf{false}.
\end{enumerate}


\begin{figure}[ht!]
    \centering
    \includegraphics[width=0.8\textwidth]{figs/fig.1.4.6.jpeg}
    \caption{}
    \label{fig:1.2.27.jpg}
\end{figure}

\end{document}
