\let\negmedspace\undefined
\let\negthickspace\undefined
\documentclass[journal]{IEEEtran}
\usepackage[a5paper, margin=10mm, onecolumn]{geometry}
%\usepackage{lmodern} % Ensure lmodern is loaded for pdflatex
\usepackage{tfrupee} % Include tfrupee package

\setlength{\headheight}{1cm} % Set the height of the header box
\setlength{\headsep}{0mm}     % Set the distance between the header box and the top of the text

\usepackage{gvv-book}
\usepackage{gvv}
\usepackage{cite}
\usepackage{amsmath,amssymb,amsfonts,amsthm}
\usepackage{algorithmic}
\usepackage{graphicx}
\usepackage{textcomp}
\usepackage{xcolor}
\usepackage{txfonts}
\usepackage{listings}
\usepackage{enumitem}
\usepackage{mathtools}
\usepackage{gensymb}
\usepackage{comment}
\usepackage[breaklinks=true]{hyperref}
\usepackage{tkz-euclide} 
\usepackage{listings}
% \usepackage{gvv}                                        
\def\inputGnumericTable{}                                 
\usepackage[latin1]{inputenc}                                
\usepackage{color}                                            
\usepackage{array}                                            
\usepackage{longtable}                                       
\usepackage{calc}                                             
\usepackage{multirow}                                         
\usepackage{hhline}                                           
\usepackage{ifthen}                                           
\usepackage{lscape}
\usepackage{circuitikz}
\tikzstyle{block} = [rectangle, draw, fill=blue!20, 
    text width=4em, text centered, rounded corners, minimum height=3em]
\tikzstyle{sum} = [draw, fill=blue!10, circle, minimum size=1cm, node distance=1.5cm]
\tikzstyle{input} = [coordinate]
\tikzstyle{output} = [coordinate]


\begin{document}

\bibliographystyle{IEEEtran}
\vspace{3cm}

\title{1.8.12}
\author{AI25BTECH11033--SNEHAMRUDULA}
 \maketitle
% \newpage
% \bigskip
{\let\newpage\relax\maketitle}

\renewcommand{\thefigure}{\theenumi}
\renewcommand{\thetable}{\theenumi}
\setlength{\intextsep}{10pt} % Space between text and floats


\numberwithin{equation}{enumi}
\numberwithin{figure}{enumi}
\renewcommand{\thetable}{\theenumi}


\title{Solution: Collinearity Problem}
\author{AI25BTECH11036-- SNEHAMRUDULA}
\date{}

\begin{document}
\maketitle
\textbf{1.8.12:}The perimeter of a triangle with vertices $(0, 4)$, $(0, 0)$ and $(3, 0)$ is \\\\.



\begin{enumerate}[a)]
    \item \textbf{Given:}  
    Vertices of the triangle are  
    \[
    A = \myvec{0\\4},\quad 
    B = \myvec{0\\0},\quad 
    C = \myvec{3\\0}
    \]

    \item \textbf{Lengths of sides:}
    \[
    AB = \|A-B\| = \left\|\myvec{0\\4}-\myvec{0\\0}\right\|
    = \left\|\myvec{0\\4}\right\| = 4
    \]

    \[
    BC = \|B-C\| = \left\|\myvec{0\\0}-\myvec{3\\0}\right\|
    = \left\|\myvec{-3\\0}\right\| = 3
    \]

    \[
    CA = \|C-A\| = \left\|\myvec{3\\0}-\myvec{0\\4}\right\|
    = \left\|\myvec{3\\-4}\right\|
    = \sqrt{3^2+(-4)^2} = 5
    \]

    \item \textbf{Perimeter:}
    \[
    P = AB+BC+CA = 4+3+5=12
    \]

    \item \textbf{Conclusion:}  
    The perimeter of the triangle is
    \[
    \boxed{12}
    \]
\end{enumerate}
\begin{figure}[ht!]
    \centering
    \includegraphics[width=0.5\textwidth]{1.8.12.}
    \caption{}
    \label{1.8.12.}
\end{figure}

\end{document}

