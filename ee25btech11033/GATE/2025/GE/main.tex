
\documentclass[journal,12pt,onecolumn]{IEEEtran}
\usepackage{cite}
\usepackage{amsmath,amssymb,amsfonts,amsthm}
\usepackage{algorithmic}
\usepackage{graphicx}
\usepackage{textcomp}
\usepackage{xcolor}
\usepackage{txfonts}
\usepackage{listings}
\usepackage{enumitem}
\usepackage{mathtools}
\usepackage{gensymb}
\usepackage{comment}
\usepackage[breaklinks=true]{hyperref}
\usepackage{tkz-euclide} 
\usepackage{listings}
\usepackage{gvv}                                        
                                 
\usepackage[latin1]{inputenc}     
\usepackage{xparse}
\usepackage{color}                     

\usepackage{array}                                            
\usepackage{longtable}                                        
\usepackage{calc}                
                             
\usepackage{multirow}
\usepackage{multicol}
\usepackage{hhline}                                           
\usepackage{ifthen}                            
               
\usepackage{lscape}
\usepackage{tabularx}
\usepackage{float}
\usepackage{circuitikz}
\usepackage{tikz}

\newtheorem{theorem}{Theorem}[section]
\newtheorem{problem}{Problem}
\newtheorem{proposition}{Proposition}[section]
\newtheorem{lemma}{Lemma}[section]
\newtheorem{corollary}[theorem]{Corollary}
\newtheorem{example}{Example}[section]
\newtheorem{definition}[problem]{Definition}
\newcommand{\BEQA}{\begin{eqnarray}}
\newcommand{\EEQA}{\end{eqnarray}}
%\newcommand{\define}{\stackrel{\triangle}{=}}
\theoremstyle{remark}
%\newtheorem{rem}{Remark}

\begin{document}
\title{GATE $2024$ Geomatics Engineering (GE)}
\author{EE$25$BTECH$11033$- Kavin B}
\maketitle
\renewcommand{\thefigure}{\theenumi}
\renewcommand{\thetable}{\theenumi}

\underline{\textbf{General Aptitude (GA)}}
\\


\textbf{Q.$1$ $-$ Q.$5$ Carry ONE mark Each}
\vspace{0.5cm}
\begin{enumerate}
\item If `$\rightarrow$' denotes increasing order of intensity, then the meaning of the words [smile $\rightarrow$ giggle $\rightarrow$ laugh] is analogous to [disapprove $\rightarrow$ \rule{2.5cm}{1pt}$\rightarrow$ chide].
Which one of the given options is appropriate to fill the blank?
\begin{enumerate}
\begin{multicols}{4}
\item reprove
\item praise
\item reprise
\item grieve
\end{multicols}
\end{enumerate}
\hfill $\brak{\text{GATE GE 2025}}$
\bigskip
\item Find the odd one out in the set : $\cbrak{19,37,21,17, 23, 29, 31, 11}$
\begin{enumerate}
\begin{multicols}{4}
\item $21$
\item $29$
\item $37$
\item $23$
\end{multicols}
\end{enumerate}
\hfill $\brak{\text{GATE GE 2025}}$
\bigskip
\item In the following series, identify the number that needs to be changed to form the Fibonacci series.\\
\\
$1$, $1$, $2$, $3$, $6$, $8$, $13$, $21$,...
\begin{enumerate}
\begin{multicols}{4}
\item $8$
\item $21$
\item $6$
\item $13$
\end{multicols}
\end{enumerate}
\hfill $\brak{\text{GATE GE 2025}}$
\bigskip

\item The real variables $x$,$y$,$z$, and the real constants p,q,r satisfy \\
\\
\centerline{\large $\frac{x}{pq-r^{2}}=\frac{y}{qr-p^{2}}=\frac{z}{rp-q^{2}}$}
\\ \\Given that the denominators are non-zero, the value of $px+qy+rz$ is
\begin{enumerate}
\begin{multicols}{4}
\item $0$
\item $1$
\item pqr
\item $p^{2}+q^{2}+r^{2}$
\end{multicols}
\end{enumerate}
\hfill $\brak{\text{GATE GE 2025}}$
\bigskip
\item Take two long dice (rectangular parallelepiped), each having four rectangular faces
labelled as $2$, $3$, 
$5$, and $7$. If thrown, the long dice cannot land on the square faces and has $1/4$ probability of landing on any of the four rectangular faces.
The label on the top face of the dice is the score of the throw.\\
\\If thrown together, what is the probability of getting the sum of the two long dice scores greater than $11$?
\begin{enumerate}
\begin{multicols}{4}
\item $3/8$
\item $1/8$
\item $1/16$
\item $3/16$
\end{multicols}
\end{enumerate}
\end{enumerate}
\hfill $\brak{\text{GATE GE 2025}}$
\bigskip
\\
\textbf{Q.$6$ $-$ Q.$10$ Carry TWO markS Each}\\
\begin{enumerate}
\setcounter{enumi}{5}
\item In the given text, the blanks are numbered $\brak{i}-\brak{iv}$.
Select the best match for all the blanks. \\
\\Prof. P \underline{\hspace{1cm}} (i) \underline{\hspace{1cm}} merely a man who narrated funny stories.
\underline{\hspace{1cm}} (ii) \underline{\hspace{1cm}} in his blackest moments he was capable of self-deprecating humor. \\
\\Prof.
Q \underline{\hspace{1cm}} (iii) \underline{\hspace{1cm}} a man who hardly narrated funny stories.
\underline{\hspace{1cm}} (iv) \underline{\hspace{1cm}} in his blackest moments was he able to find humor.
\begin{enumerate}
\item $\brak{i}$ was \ \ \ $\brak{ii}$ Only \ \ \   $\brak{iii}$ wasn't \ \ \ $\brak{iv}$ Even
\item $\brak{i}$ wasn't\ \ \  $\brak{ii}$ Even\ \ \  $\brak{iii}$ was\ \ \  $\brak{iv}$ Only
\item $\brak{i}$ was \ \ \ $\brak{ii}$ Even \ \ \ $\brak{iii}$ wasn't\ \ \  $\brak{iv}$ Only
\item $\brak{i}$ wasn't\ \ \  $\brak{ii}$ Only \ \ \ $\brak{iii}$ was \ \ \ $\brak{iv}$ Even
\end{enumerate}
\hfill $\brak{\text{GATE GE 2025}}$
\bigskip
\item How many combinations of non-null sets A, B, C are possible from the subsets of $\cbrak{2,3,5}$ satisfying the conditions: $\brak{i}$ A is a subset of B, and 
$\brak{ii}$ B is a subset of C?
\begin{enumerate}
\begin{multicols}{4}
\item $28$
\item $27$
\item $18$
\item $19$
\end{multicols}
\end{enumerate}
\hfill $\brak{\text{GATE GE 2025}}$
\bigskip
\item The bar chart gives the batting averages of VK and RS for $11$ calendar years from $2012$ to $2022$. Considering that $2015$ and $2019$ are world cup years, which one of the following options is true?\\
\begin{figure}[H]
    \centering
    \includegraphics[width=0.8\columnwidth]{figs/fig1.png}
    \caption{\centering{Figure}}
    \label{fig:fig.1}
\end{figure}
\textit{}
\begin{enumerate}
\item RS has a higher yearly batting average than that of VK in every world cup year.
\item VK has a higher yearly batting average than that of RS in every world cup year.
\item VK's yearly batting average is consistently higher than that of RS between the two world cup years.
\item RS's yearly batting average is consistently higher than that of VK in the last three years.
\end{enumerate}
\hfill $\brak{\text{GATE GE 2025}}$
\bigskip
\item A planar rectangular paper has two V-shaped pieces attached as shown below .

\begin{figure}[H]
    \centering
    \includegraphics[width=0.3\columnwidth]{figs/fig2.png}
    \caption{\centering{Figure}}
    \label{figs:fig2}
\end{figure}
\textit{} 
This piece of paper is folded to make the following closed three-dimensional object.

\begin{figure}[H]
    \centering
    \includegraphics[width=0.2\columnwidth]{figs/fig3.png}
    \caption{\centering{Figure}}
    \label{figs:fig3}
\end{figure}
\textit{} \\
The number of folds required to form the above object is
\begin{enumerate}
\begin{multicols}{4}
\item $9$
\item $7$
\item $11$
\item $8$
\end{multicols}
\end{enumerate}
\hfill $\brak{\text{GATE GE 2025}}$
\bigskip
\item Four equilateral triangles are used to form a regular closed three-dimensional object by joining along the edges.
The angle between any two faces is
\begin{enumerate}
\begin{multicols}{4}
\item $30\degree$
\item $60\degree$
\item $45\degree$
\item $90\degree$
\end{multicols}
\end{enumerate}
\end{enumerate}
\hfill $\brak{\text{GATE GE 2025}}$
\bigskip
\\
\textbf{PART A: Common FOR ALL CANDIDATES}\\
\\
\textbf{Q.$11$ $-$ Q.$27$ Carry ONE mark Each}\\

\begin{enumerate}
\setcounter{enumi}{10}
\item Which of the following options best describes the "uncertainty" in a measurement?
\begin{enumerate}
\item It includes both random and gross errors
\item It includes only systematic errors
\item It includes both systematic and gross errors
\item It includes both random and systematic errors
\end{enumerate}
\hfill $\brak{\text{GATE GE 2025}}$
\bigskip

\item A distance was measured as $200 m \pm 0.1 m$.
The relative precision of this measurement is
\begin{enumerate}
\begin{multicols}{4}
\item $1$:$20$
\item $1$:$200$
\item $1$:$2000$
\item $1$:$20000$
\end{multicols}
\end{enumerate}
\hfill $\brak{\text{GATE GE 2025}}$
\bigskip
\item Which of the following options describes the CORRECT relationship for a Gaussian distributed random error?
\begin{enumerate}
\item Probable error $<$ Average error $<$ Standard error $<$ $90$\% error
\item Standard error $<$ Average error $<$ Probable error $<$ $90$\% error
\item Average error $<$ Probable error $<$ $90$\% error $<$ Standard error
\item Probable error $<$ $90$\% error $<$ Average error $<$ Standard error
\end{enumerate}
\hfill $\brak{\text{GATE GE 2025}}$
\bigskip
\item The Chi-square distribution is used for comparing the
\begin{enumerate}
\item population variance with the sample variance for a given degree of freedom
\item population mean with the sample mean for a given degree of freedom
\item population median with the sample median for a given degree of freedom
\item population mean and standard deviation with the sample mean 
and standard deviation for a given degree of freedom
\end{enumerate}
\hfill $\brak{\text{GATE GE 2025}}$
\bigskip
\item Water bodies appear in dark tone in Near Infrared (NIR) image, because water \rule{2cm}{0.5mm} most of the NIR radiations incident on it.
\begin{enumerate}
\begin{multicols}{4}
\item absorbs
\item emits
\item reflects
\item scatters
\end{multicols}
\end{enumerate}
\hfill $\brak{\text{GATE GE 2025}}$
\bigskip
\item The approximate altitude $\brak{above\ earth\ suface}$ of polar sun-synchronous orbits of ISRO's remote sensing satellites is
\begin{enumerate}
\begin{multicols}{4}
\item $<$ $90$ km
\item $90$ km to $200$ km
\item $200$ km to $400$ km
\item $>$ $400$ km
\end{multicols}
\end{enumerate}
\hfill $\brak{\text{GATE GE 2025}}$
\bigskip
\item Hyperspectral sensor consists of
\begin{enumerate}
\item large number of wide and discrete bands
\item small number of wide and contiguous bands
\item large number of narrow and contiguous bands
\item small number of narrow and discrete bands
\end{enumerate}
\hfill $\brak{\text{GATE GE 2025}}$
\bigskip
\item Part of the solar radiation incident on the water surface gets refracted as per
\begin{enumerate}
\begin{multicols}{4}
\item Rayleigh's law
\item Snell's law
\item Moore's law
\item Newton's law
\end{multicols}
\end{enumerate}
\hfill $\brak{\text{GATE GE 2025}}$
\bigskip
\item Which 
of the following mathematical principles is applied for finding a geographic position on Earth's surface using GPS?
\begin{enumerate}
\begin{multicols}{4}
\item Triangulation
\item Analytical traversing
\item Trilateration
\item Analytical leveling
\end{multicols}
\end{enumerate}
\hfill $\brak{\text{GATE GE 2025}}$
\bigskip
\item Which of the following is NOT a segment of GPS to determine position and time?
\begin{enumerate}
\begin{multicols}{4}
\item Space segment
\item Control segment
\item Launch segment
\item User segment
\end{multicols}
\end{enumerate}
\hfill $\brak{\text{GATE GE 2025}}$
\bigskip
\item Dilution of Precision $\brak{DOP}$ in GPS based survey is primarily used to assess the quality of
\begin{enumerate}
\item satellite's altitude
\item satellite's geometry
\item satellite's atomic clocks
\item satellite's velocity
\end{enumerate}
\hfill $\brak{\text{GATE GE 2025}}$
\bigskip
\item How many NAVSTAR GPS satellites in standard constellation are operational and provide uninterrupted service?
\begin{enumerate}
\begin{multicols}{4}
\item $4$
\item $12$
\item $24$
\item $36$
\end{multicols}
\end{enumerate}
\hfill $\brak{\text{GATE GE 2025}}$
\bigskip
\item Identify the type of digitizing error in the following figure.
\\
\begin{figure}[H]
    \centering
    \includegraphics[width=0.25\columnwidth]{figs/fig4.png}
    \caption{\centering{Figure}}
    \label{figs:fig4}
\end{figure}

\begin{enumerate}
\begin{multicols}{4}
\item Dangling arc
\item Overshoot
\item Undershoot
\item Missing label
\end{multicols}
\end{enumerate}
\hfill $\brak{\text{GATE GE 2025}}$
\bigskip
\item Which of the following is NOT a derivative of digital elevation model $\brak{DEM}$?
\begin{enumerate}
\begin{multicols}{4}
\item Slope
\item Aspect
\item Contour
\item Emissivity
\end{multicols}
\end{enumerate}
\hfill $\brak{\text{GATE GE 2025}}$
\bigskip
\item Which of the following is a core vector GIS operation?
\begin{enumerate}
\begin{multicols}{2}
\item Overlaying
\item Contrast stretching
\item Histogram equalization
\item Band ratioing
\end{multicols}
\end{enumerate}
\hfill $\brak{\text{GATE GE 2025}}$
\bigskip
\item The wavelength at which maximum energy is radiated or emitted from the forest fire at temperature of $700$\degree C is \rule{2cm}{0.5mm} $\mu$m $\brak{rounded\ of\ to\ one\ decimal\ place}$.
\\
\\.
\hfill $\brak{\text{GATE GE 2025}}$
\bigskip

\item The standard error of a unit weight for a set of angle observations is $10''$.
The minimum number of observations required to reduce the standard error of the mean for this set of observations to $2''$ is \rule{2cm}{0.5mm} $\brak{in\ integer}$.\\
\end{enumerate}
\hfill $\brak{\text{GATE GE 2025}}$
\bigskip
\\
\textbf{Q.$28$ $-$ Q.$46$ Carry TWO marks Each}\\
\begin{enumerate}
\setcounter{enumi}{27}
\item An angle is observed independently twice, and the values are as follows:\\
$60\degree30'10'' \pm 10''$\\
$60\degree30'20'' \pm 20''$\\
The most probable value $\brak{MPV}$ of the angle is
\begin{enumerate}
\begin{multicols}{4}
\item $60\degree30'12''$
\item $60\degree30'15''$
\item $60\degree30'18''$
\item $60\degree30'14''$
\end{multicols}
\end{enumerate}
\hfill $\brak{\text{GATE GE 2025}}$
\bigskip
\item In the figure, d1, d2, d3 are three independently measured distances for estimating the unknown distances $x$ and $y$.The correlation coefficient between the unknown estimates approximately equals to\\
\\

\begin{figure}[H]
    \centering
    \includegraphics[width=0.5\columnwidth]{figs/fig5.png}
    \caption{\centering{Figure}}
    \label{figs:fig5}
\end{figure}\\

\begin{align*}
d_1 &= 100 \text{ m} \pm 1 \text{ cm} \\
d_2 &= 150 \text{ m} \pm 2 \text{ cm} \\
d_3 &= 175 \text{ m} \pm 3 \text{ cm}
\end{align*}

\begin{enumerate}
\begin{multicols}{4}
\item + $0.325$
\item - $0.496$
\item + $0.755$
\item - $0.592$
\end{multicols}
\end{enumerate}
\hfill $\brak{\text{GATE GE 2025}}$
\bigskip
\item Independent angles AOB, BOC and AOC were observed as shown in figure.
The standard error of all observations is same. The adjusted values of these angles using the least squares adjustment are\\
AOB = $30\degree00'20''$\\
BOC = $30\degree00'05''$\\
AOC = $60\degree00'10''$\\

\begin{figure}[H]
    \centering
    \includegraphics[width=0.4\columnwidth]{figs/fig6.png}
    \caption{\centering{Figure}}
    \label{figs:fig6}
\end{figure}
\begin{enumerate}
\item AOB = $30\degree00'15''$, BOC = $30\degree00'00''$, AOC = $60\degree00'15''$
\item AOB = $30\degree00'10''$, BOC = $30\degree00'05''$, AOC = $60\degree00'15''$
\item AOB = $30\degree00'05''$, BOC = $30\degree00'10''$, AOC = $60\degree00'15''$
\item AOB = $30\degree00'10''$, BOC = $30\degree00'10''$, AOC = $60\degree00'20''$
\end{enumerate}
\hfill $\brak{\text{GATE GE 2025}}$
\bigskip
\item To reduce the slope distance (S) to an equivalent horizontal distance (H) as shown in the figure given below, the following independent observations were taken.\\ 
\\ $S = 
29.95$ m $\pm$ $0.01$ m; $\theta = 4\degree30'$ $\pm$ $10'$ \\
\\The required precision of computed horizontal distance is $\pm$ $0.005$ m.
Assume a "balanced accuracy" where the contribution to precision of the horizontal distance comes equally from the slope distance and angle measurements.
The minimum number of angle observations to achieve the desired precision is \\
\\(Given $1$ radian = $206265$ seconds)\\
\begin{figure}[H]
    \centering
    \includegraphics[width=0.4\columnwidth]{figs/fig7.png}
    \caption{\centering{Figure}}
    \label{figs:fig7}
\end{figure}
\begin{enumerate}
\begin{multicols}{4}
\item $1$
\item $2$
\item $3$
\item $4$
\end{multicols}
\end{enumerate}
\hfill $\brak{\text{GATE GE 2025}}$
\bigskip
\item Find the best match between remote sensing sensors (Column A) with their characteristics (Column B)
\begin{table}[h]
    \centering
    \begin{table}[H]
     \centering
     \begin{tabular}{c|c}
       Column A    & Column B  \\
         1. Simple cubic   & P.0.74\\
          2. Hexagonal close-packed & Q.0.68\\
          3. Body-centered cubic  & R.0.52\\
          4. Face-centered cubic \\
     \end{tabular}
     \caption{Table-1}
     \label{tab:tables/table1.tex}
 \end{table}
\end{table}
\begin{enumerate}
\begin{multicols}{4}
\item P-$1$, Q-$5$, R-$2$, S-$3$
\item P-$3$, Q-$2$ , R-$4$, S-$1$
\item P-$2$, Q-$3$, R-$1$, S-$5$
\item P-$1$, Q-$3$, R-$4$, S-$5$
\end{multicols}
\end{enumerate}
\hfill $\brak{\text{GATE GE 2025}}$
\bigskip
\item Find the best match between Column A and Column B
\begin{table}[h]
    \centering
    \begin{center}
\begin{tabular}{|p{1cm}|p{5.2cm}|p{1cm}|p{4cm}|}
\hline
\multicolumn{2}{|c|}{Measuring feature} & \multicolumn{2}{c|}{Measuring instrument} \\
\hline
P & Flatness error of a surface plate & 1 & Auto collimator \\
\hline
Q & Profile of a cam & 2 & Tool maker's microscope \\
\hline
R & Alignment error of a machine tool slide way & 3 & Dividing head and dial gauge \\
\hline
S & Pitch and angle errors of screw thread & 4 & Optical interferometer \\
\hline
\end{tabular}
\end{center}
\end{table}
\begin{enumerate}
\begin{multicols}{4}
\item P-$5$, Q-$4$, R-$3$, S-$1$
\item P-$5$, Q-$4$, R-$2$, S-$3$
\item 
P-$3$, Q-$1$, R-$2$, S-$4$
\item P-$2$, Q-$3$, R-$4$, S-$1$
\end{multicols}
\end{enumerate}
\hfill $\brak{\text{GATE GE 2025}}$
\bigskip
\item Which of the following factors is/are responsible for ionospheric delay in GNSS observations?
\begin{enumerate}
\item Total electron count in the ionosphere
\item Carrier signal frequency
\item Size of GPS receivers
\item Size and accuracy of atomic clocks
\end{enumerate}
\hfill $\brak{\text{GATE GE 2025}}$
\bigskip
\item Which of the following statements is/are CORRECT in the context of GPS data collection methods?
\begin{enumerate}
\item CORS $\brak{Continuously\ Operating\ Reference\ Station}$ can be used as a reference $\brak{base}$ GPS receiver
\item Reference $\brak{base}$ receiver should record the observations for longer period as compared to remote $\brak{rover}$ GPS receiver for applying corrections
\item Remote $\brak{rover}$ GPS receiver must always be placed on a known location for applying the corrections of reference $\brak{base}$ GPS receiver
\item Reference $\brak{base}$ and remote $\brak{rover}$ GPS receivers must be placed on top of each other for applying corrections
\end{enumerate}
\hfill $\brak{\text{GATE GE 2025}}$
\bigskip
\item Which of the following errors is/are corrected in Differential GPS $\brak{DGPS}$ ?
\begin{enumerate}
\item Tropospheric delays
\item Orbital errors
\item Ionospheric delays
\item Ambiguity in atomic clocks
\end{enumerate}
\hfill $\brak{\text{GATE GE 2025}}$
\bigskip
\item Which of the following statements is/are CORRECT?
\begin{enumerate}
\item Network analysis can be done with vector data.
\item Linear features are clearly identified as discrete features in vector database.
\item Satellite images are in vector format.
\item Digital elevation model is in raster format.
\end{enumerate}
\hfill $\brak{\text{GATE GE 2025}}$
\bigskip
\item In GIS, buffer is a zone with a specified width surrounding a spatial feature.
Which of the following statements regarding buffer is/are CORRECT?
\begin{enumerate}
\item For a point feature, buffer is an ellipse with minor and major axes as buffer distances
\item For a line feature, buffer is a band with a specified distance created around the line conforming to the line's curve
\item Buffer zones are polylines
\item For a polygon feature, buffer is a belt of a specified distance from the edge of the polygon and conforming to its shape
\end{enumerate}
\hfill $\brak{\text{GATE GE 2025}}$
\bigskip
\item Which of the following statements about the Triangulated Irregular Network $\brak{TIN}$ model is/are INCORRECT?
\begin{enumerate}
\item TIN contains irregularly spaced sampled points.
\item Triangulation is performed to form network of triangles.
\item In the TIN model, the edges represent features such as peaks and depression.
\item In the TIN model, the vertices represent features such as peaks and depression.
\end{enumerate}
\hfill $\brak{\text{GATE GE 2025}}$
\bigskip
\item Which of the following statements is/are INCORRECT in the context of GIS?
\begin{enumerate}
\item CLIP erases a part of one of the input layers.
\item SPLIT overlays polygons and keeps all areas in both layers.
\item INTERSECT overlays polygons and keeps only the common portions of both layers.
\item UNION overlays polygons and keeps all areas in both layers.
\end{enumerate}
\hfill $\brak{\text{GATE GE 2025}}$
\bigskip
\item Which of the following is/are method(s) used for compact storage of raster GIS data?
\begin{enumerate}
\item Chain code
\item Run-length code
\item Quadtree
\item Decision-tree
\end{enumerate}
\hfill $\brak{\text{GATE GE 2025}}$
\bigskip
\item Which of the following statements is/are CORRECT?
\begin{enumerate}
\item CARTOSAT-$1$ satellite can acquire across-track stereoscopic pairs of images of a geographical region on the same day.
\item CARTOSAT-$1$ satellite can acquire across-track stereoscopic pairs of images of a geographical region on successive days.
\item CARTOSAT-$1$ satellite can acquire along-track stereoscopic pairs of images of a geographical region on the same day.
\item CARTOSAT-$1$ satellite can acquire along-track stereoscopic pairs of images of a geographical region on successive days.
\end{enumerate}
\hfill $\brak{\text{GATE GE 2025}}$
\bigskip
\item Which of the following statements is/are CORRECT for satellite image interpretation?
\begin{enumerate}
\item SWIR band is sensitive to moisture in soil and vegetation
\item Blue band is not useful to discriminate between water and snow
\item NIR band is useful to discriminate between land and water
\item Green band is useful to discriminate between cloud and snow
\end{enumerate}
\hfill $\brak{\text{GATE GE 2025}}$
\bigskip
\item Which of the following CANNOT be used as visual interpretation key(s) for satellite images?
\begin{enumerate}
\item Texture
\item Projection
\item Pattern
\item Association
\end{enumerate}
\hfill $\brak{\text{GATE GE 2025}}$
\bigskip
\item Which of the following parts of the electromagnetic spectrum is/are used in satellite remote sensing for earth observation?
\begin{enumerate}
\item Visible wavelengths
\item Thermal Infrared wavelengths
\item Radio wavelengths
\item Gamma wavelengths
\end{enumerate}
\hfill $\brak{\text{GATE GE 2025}}$
\bigskip
\item Using the following data, the spatial resolution of a push-broom sensor is \rule{2cm}{0.5mm} m $\brak{in\ integer}$.
\\
\underline{\textbf{Data:}} \\
Orbital altitude (above earth surface) = $1000$ km \\
Number of spectral bands = $5$ \\
Number of detectors/CCDs (charged coupled devices) in a row = $4000$ \\
Ground swath = $20$ km
\end{enumerate}
\hfill $\brak{\text{GATE GE 2025}}$
\bigskip
\\
\textbf{PART B1: FOR Surveying and Mapping CANDIDATES ONLY}\\

\textbf{Q.$47$ $-$ Q.$54$ Carry ONE mark Each}\\
\begin{enumerate}
\setcounter{enumi}{46}
\item If the plotting accuracy of a map is $0.25$ mm and the scale of the same map is $1$:$100000$, what will be the minimum ground distance that can be plotted on the map?
\begin{enumerate}
\begin{multicols}{4}
\item $2.5$ m
\item $25$ m
\item $250$ m
\item $2500$ m
\end{multicols}
\end{enumerate}
\hfill $\brak{\text{GATE GE 2025}}$
\bigskip
\item The Survey of India toposheet number $43\frac{D}{6}$ covers ground area of
\begin{enumerate}
\begin{multicols}{4}
\item $1$\degree by $1$\degree
\item $25$' by $25$'
\item $15$' by $15$'
\item $7.5$' by $7.5$'
\end{multicols}
\end{enumerate}
\hfill $\brak{\text{GATE GE 2025}}$
\bigskip
\item Universal Transverse Mercator $\brak{UTM}$ is a
\begin{enumerate}
\item conical projection
\item azimuthal projection
\item polyconic projection
\item cylindrical projection
\end{enumerate}
\hfill $\brak{\text{GATE GE 2025}}$
\bigskip
\item Change Point $\brak{CP}$ in levelling refers to a location where
\begin{enumerate}
\item only backsight reading is taken
\item both backsight and foresight readings are taken
\item survey work ends
\item staff reading is taken on a benchmark
\end{enumerate}
\hfill $\brak{\text{GATE GE 2025}}$
\bigskip
\item At a fixed instrument location in levelling, if the backsight reading at a point 
P is more than the foresight reading at a point Q, then
\begin{enumerate}
\item point P has lower elevation than point Q
\item point P has higher elevation than point Q
\item the elevation difference between P and Q depends on height of the instrument
\item the elevation difference between P and Q depends on benchmark elevation
\end{enumerate}
\hfill $\brak{\text{GATE GE 2025}}$
\bigskip
\item "Transit the telescope" of a theodolite involves
\begin{enumerate}
\item rotating the theodolite about its vertical axis
\item rotating the telescope about its trunnion axis
\item rotating the telescope about its line of collimation
\item rotating the theodolite by $90$\degree in horizontal plane
\end{enumerate}
\hfill $\brak{\text{GATE GE 2025}}$
\bigskip
\item Scale of a vertical aerial photograph of 
an undulating terrain is
\begin{enumerate}
\item directly proportional to the height of terrain
\item inversely proportional to the focal length of camera lens
\item directly proportional to the flying height of aircraft
\item uniform throughout the photograph
\end{enumerate}
\hfill $\brak{\text{GATE GE 2025}}$
\bigskip
\item Isocentre of a tilted photograph is
\begin{enumerate}
\item intersection of the optical axis of the aerial camera with the plane of the photograph
\item the point of aerial photograph where a plumb line dropped from exposure station pierces the photograph
\item angle of tilt of the photograph
\item the point on the photograph where the bisector of the angle of tilt meets the photograph
\end{enumerate}
\end{enumerate}
\hfill $\brak{\text{GATE GE 2025}}$
\bigskip
\\
\textbf{Q.$55$ $-$ Q.$65$ Carry TWO marks 
Each}\\
\begin{enumerate}
\setcounter{enumi}{54}
\item The magnetic bearing of a line in the year $1990$ was found to be N $40\degree30'$ W and magnetic declination was $3\degree30'$ E. If the present magnetic declination is $2\degree10'$ W, the magnetic bearing now (in reduced bearing system) would be
\begin{enumerate}
\begin{multicols}{4}
\item S $30\degree50'$ W
\item N $30\degree50'$ W
\item S $34\degree50'$ W
\item N $34\degree50'$ W
\end{multicols}
\end{enumerate}
\hfill $\brak{\text{GATE GE 2025}}$
\bigskip
\item Map $\brak{A}$ represents all the roads, street lights, trees and buildings of a campus of $5$ km$^2$.
Another map $\brak{B}$ represents the forest and agricultural area of a district of $10000$ km$^2$.
Considering the physical size of both the maps $\brak{A}$ \& $\brak{B}$ same, which of the following statements is/are CORRECT?
\begin{enumerate}
\item Map $\brak{A}$ is at relatively large scale
\item Map $\brak{B}$ is at relatively large scale
\item Both maps are at same scale
\item Both maps are not at same scale
\end{enumerate}
\hfill $\brak{\text{GATE GE 2025}}$
\bigskip
\item Which of the following statements is/are CORRECT?
\begin{enumerate}
\item Triangulation is preferred in plain areas, whereas trilateration is preferred in hilly areas
\item Triangulation is preferred in hilly areas, whereas trilateration is preferred in plain areas
\item In triangulation, the angles are measured with greater accuracy, while in trilateration, sides are measured with greater accuracy
\item In trilateration, the angles are measured with greater accuracy, while in triangulation, sides of triangles are measured with greater accuracy
\end{enumerate}
\hfill $\brak{\text{GATE GE 2025}}$
\bigskip
\item Which of the following statements is/are CORRECT?
\begin{enumerate}
\item Bowditch rule in traverse adjustment is particularly useful, where angular and linear measurements are equally precise
\item Transit rule in traverse adjustment is particularly useful, where angular measurements are more precise than linear measurements
\item In Bowditch rule, the traverse adjustment is done using arithmetic sum of latitudes or departures of the traverse
\item In Transit rule, the traverse adjustment is done using perimeter of the traverse
\end{enumerate}
\hfill $\brak{\text{GATE GE 2025}}$
\bigskip
\item Consider a point A on the surface of Earth, its elevation with respect to EGM$2008$ (geoid) is $95.5$ m.
The geoidal undulation at point A is $4.5$ m. The orthometric height of point A is \rule{2cm}{0.5mm} m $\brak{rounded\ off\ to\ one\ decimal\ place}$.\\.
\hfill $\brak{\text{GATE GE 2025}}$
\bigskip
\item If the longitudinal overlap in aerial photographs is kept as $65$\%, the common overlap $\brak{superlap}$ between three successive photographs is \rule{2cm}{0.5mm} \% $\brak{in\ integer}$.\\.
\hfill $\brak{\text{GATE GE 2025}}$
\bigskip
\item The Representative Fraction $\brak{RF}$ of the graphical scale given below is $1$/X, where X is \rule{2cm}{0.5mm} $\brak{in\ integer}$.
\begin{figure}[H]
    \centering
    \includegraphics[width=0.7\columnwidth]{figs/fig8.png}
    \caption{\centering{Figure}}
    \label{figs:fig8}
\end{figure}\\.
\hfill $\brak{\text{GATE GE 2025}}$
\bigskip

\item The combined correction for curvature of Earth and refraction in levelling for a distance of $6$ km would be \rule{2cm}{0.5mm} m $\brak{rounded\ off\ to\ two\ decimal\ places}$.\\
\\Assume the radius of earth is $6370$ km.\\.
\hfill $\brak{\text{GATE GE 2025}}$
\bigskip
\item In tangential method of tacheometry, two vanes in a staff were fixed at a distance of $1.0$ m with the bottom vane fixed at $1.0$ m.
The levelling staff was held vertical at a point P and the vertical angles of the vanes observed were $5\degree30'$ and $3\degree15'$, respectively.
The vertical distance between the instrument axis and the bottom vane would be \rule{2cm}{0.5mm} m $\brak{rounded\ off\ to\ two\ decimal\ places}$\\.
\hfill $\brak{\text{GATE GE 2025}}$
\bigskip
\item A line measures $15$ cm on an aerial photograph, while it measures $5$ cm on a map at $1$:$24000$ scale.
The photograph was taken using a camera lens of $20$ cm focal length.
Average elevation of terrain is $240$ m above mean sea level.
The flying height of the aircraft above mean sea level is \rule{2cm}{0.5mm} m $\brak{in\ integer}$\\.
\hfill $\brak{\text{GATE GE 2025}}$
\bigskip
\item A high tower appeared on an aerial photograph taken at $1000$ m above mean sea level with a camera lens of $15$ cm focal length.
The radial distances of the top and bottom images of the tower from principal point of photograph are $92.6$ mm and $78.3$ mm, respectively.
If the average elevation of terrain is $300$ m above mean sea level, then the height of the tower above ground is \rule{2cm}{0.5mm} m $\brak{rounded\ off\ to\ the\ nearest\ integer}$.
\end{enumerate}
\hfill $\brak{\text{GATE GE 2025}}$
\bigskip
\\
\textbf{PART B2: FOR Image Processing and Analysis CANDIDATES ONLY}\\
\\

\textbf{Q.$66$ $-$ Q.$73$ Carry ONE mark Each}
\bigskip
\begin{enumerate}
\setcounter{enumi}{65}
\item A four-band multispectral image of size $64 \times 64$ pixels has $560$ header bytes.
The per pixel depth of the image is $2$ bytes.
The total number of bytes required to store this image on the disk in the Band Interleaved by Line $\brak{BIL}$ format will be
\begin{enumerate}
\begin{multicols}{4}
\item $33328$
\item $32338$
\item $33823$
\item $33283$
\end{multicols}
\end{enumerate}
\hfill $\brak{\text{GATE GE 2025}}$
\bigskip
\item A one-dimensional normalized kernel $\frac{1}{4}\myvec{1 & 2 & 1}$ is convolved with an image to produce an intermediate result.
The intermediate image of this operation is again convolved with the same kernel to produce a final result.
The equivalent kernel to achieve the same final result in one step from the original image is given as
\begin{enumerate}
\item $\frac{1}{16}\myvec{1 & 4 & 6 & 4 & 1}$
\item $\frac{1}{16}\myvec{1 & 2 & 1 & 2 & 1}$
\item $\frac{1}{8}\myvec{1 & 2 & 4 & 2 & 1}$
\item $\frac{1}{10}\myvec{1 & 2 & 4 & 2 & 1}$
\end{enumerate}
\hfill $\brak{\text{GATE GE 2025}}$
\bigskip
\item The histogram equalization applied to a digital image generally DOES NOT yield a truly uniform histogram of the transformed image due to
\begin{enumerate}
\item discrete nature of pixel values
\item poor contrast of the original image
\item low frequency image information
\item 
presence of edges
\end{enumerate}
\hfill $\brak{\text{GATE GE 2025}}$
\bigskip
\item Which type of contrast stretching is represented by the following figure?\\
\begin{figure}[H]
    \centering
    \includegraphics[width=0.4\columnwidth]{figs/fig9.png}
    \caption{\centering{Figure}}
    \label{figs:fig9}
\end{figure}

\begin{enumerate}
\begin{multicols}{2}
\item Linear contrast stretch
\item Multiple linear stretch
\item Logarithmic stretch
\item Gaussian stretch
\end{multicols}
\end{enumerate}
\hfill $\brak{\text{GATE GE 2025}}$
\bigskip
\item Contrast enhancement is a type of \rule{2cm}{0.5mm} enhancement.
\begin{enumerate}
\begin{multicols}{4}
\item spectral
\item spatial
\item radiometric
\item temporal
\end{multicols}
\end{enumerate}
\hfill $\brak{\text{GATE GE 2025}}$
\bigskip
\item \rule{2cm}{0.5mm} is a raster image resampling technique that DOES NOT alter any of the output cell values from the input raster dataset.
\begin{enumerate}
\begin{multicols}{4}
\item Nearest neighbor
\item Cubic convolution
\item Bilinear
\item Kriging
\end{multicols}
\end{enumerate}
\hfill $\brak{\text{GATE GE 2025}}$
\bigskip
\item De-stripping in radiometric correction is used to correct a type of
\begin{enumerate}
\item sensor defect
\item atmospheric effect
\item path radiance
\item geometric error
\end{enumerate}
\hfill $\brak{\text{GATE GE 2025}}$
\bigskip
\item The figure given below shows the Fourier spectrum obtained by applying filter on a remote sensing image in frequency domain.
Zone A represents the location of \rule{2cm}{0.5mm} components. \\
\\
\begin{figure}[H]
    \centering
    \includegraphics[width=0.2\columnwidth]{figs/fig10.png}
    \caption{\centering{Figure}}
    \label{figs:fig10}
\end{figure}

\begin{enumerate}
\item low frequency
\item mid frequency
\item mid to high frequency
\item high frequency
\end{enumerate}

\end{enumerate}
\hfill $\brak{\text{GATE GE 2025}}$
\bigskip
\\
\textbf{Q.$74$ $-$ Q.$84$ Carry TWO marks each}
\\
\begin{enumerate}
\setcounter{enumi}{73}
\item For the following covariance matrix $\brak{\sum}$ of a multispectral image, which of the statements is/are INCORRECT?
\[
\Sigma =
\\
\begin{array}{r @{\hspace{0.5em}} l}
    & 
\begin{array}{ccc}
        \text{band-$1$} & \quad \text{band-$2$} & \quad \text{band-$3$}
    \end{array} \\
   
\begin{array}{r}
        \text{band-$1$} \\ \text{band-$2$} \\ \text{band-$3$}
    \end{array}
    & 
\myvec{
        34.14 & 46.71 & 40.68 \\
        46.71 & 68.83 & 69.59 \\
        40.68 & 69.59 & 248.40}
\end{array}
\]

\begin{enumerate}
\item band-$1$ and band-$2$ have maximum correlation
\item band-$2$ and band-$3$ are least correlated
\item band-$3$ conveys the maximum information content
\item band-$1$ conveys the minimum information content
\end{enumerate}
\hfill $\brak{\text{GATE GE 2025}}$
\bigskip
\item Which of the following statistical measures CANNOT be computed from the multispectral image histograms?
\begin{enumerate}
\item Mean, skewness, kurtosis
\item Covariance matrix
\item Co-occurrence matrix
\item Correlation matrix
\end{enumerate}
\hfill $\brak{\text{GATE GE 2025}}$
\bigskip
\item Which of the following statements about Principal Component Analysis $\brak{PCA}$ is/are CORRECT?
\begin{enumerate}
\item A two-dimensional data set can have up to four principal components.
\item The first principal component accounts for the majority of conceivable data variation.
\item The second principal component attempts to encapsulate the mode of the data.
\item The transformed principal components are linear combinations of the original variables and are orthogonal.
\end{enumerate}
\hfill $\brak{\text{GATE GE 2025}}$
\bigskip
\item In the context of satellite image classification, which of the following statements is/are CORRECT?
\begin{enumerate}
\item Both ANN and Fuzzy C-means clustering are parametric classifiers
\item Both ANN and Fuzzy C-means clustering are non-parametric classifiers
\item ANN can be both supervised and unsupervised classification method
\item Fuzzy C-means clustering is a supervised classification method
\end{enumerate}
\hfill $\brak{\text{GATE GE 2025}}$
\bigskip
\item Which of the following filters can be used to suppress the low frequency component of a raster image?
\begin{center}

\begin{tabular}{cc}


\begin{tabular}{c}
  \begin{tabular}{|c|c|c|}
  \hline
  $1$ & $1$ & $1$ \\
  \hline
  $1$ & $1$ & $1$ \\
  \hline
  $1$ & $1$ & $1$ \\
  \hline
  \end{tabular} \\
  (i)
\end{tabular} &


\begin{tabular}{c}
  \begin{tabular}{|c|c|c|}
  \hline
  $-1$ & $-1$ & $-1$ \\
  \hline
  $-1$ & $9$ & $-1$ \\
  \hline
  $-1$ & $-1$ & $-1$ \\
  \hline
  \end{tabular} \\
  (ii)
\end{tabular} \\

\\[5ex] 
\begin{tabular}{c}
  \begin{tabular}{|c|c|c|c|c|}
  \hline
  $1$ & $1$ & $1$ & $1$ & $1$ \\
  \hline
  $1$ & $1$ & $1$ & 
$1$ & $1$ \\
  \hline
  $1$ & $1$ & $1$ & $1$ & $1$ \\
  \hline
  $1$ & $1$ & $1$ & $1$ & $1$ \\
  \hline
  $1$ & $1$ & $1$ & $1$ & $1$ \\
  \hline
  \end{tabular} \\
  (iii)
\end{tabular} &


\begin{tabular}{c}
  \begin{tabular}{|c|c|c|c|c|}
  \hline
  $-1$ & $-1$ & $-1$ & $-1$ & $-1$ \\
  \hline
  $-1$ & $-1$ & $-1$ & $-1$ & $-1$ \\
  \hline
  $-1$ & $-1$ & $25$ & $-1$ & $-1$ \\
  \hline
  $-1$ & $-1$ & $-1$ 
& $-1$ & $-1$ \\
  \hline
  $-1$ & $-1$ & $-1$ & $-1$ & $-1$ \\
  \hline
  \end{tabular} \\
  (iv)
\end{tabular}

\end{tabular}
\end{center}
\begin{enumerate}
\begin{multicols}{4}
\item $\brak{i}$
\item $\brak{ii}$
\item $\brak{iii}$
\item $\brak{iv}$
\end{multicols}
\end{enumerate}
\hfill $\brak{\text{GATE GE 2025}}$
\bigskip
\item Which of the following statements about image ratio is/are CORRECT?
\begin{enumerate}
\item It cannot be used to suppress the effects of topography
\item It cannot be used to suppress the effects of differential sun-illumination
\item It helps in suppressing the effects of differential sun-illumination
\item It helps in suppressing the effects of topography
\end{enumerate}
\hfill $\brak{\text{GATE GE 2025}}$
\bigskip
\item Which of the following statistical classification algorithms is/are represented by the figure given below?
\\
\begin{figure}[H]
    \centering
    \includegraphics[width=0.3\columnwidth]{figs/fig11.png}
    \caption{\centering{Figure}}
    \label{figs:fig11}
\end{figure}

\begin{enumerate}
\item Minimum distance to mean classification
\item Parallelepiped classification
\item Maximum likelihood classification
\item k-means clustering
\end{enumerate}
\hfill $\brak{\text{GATE GE 2025}}$
\bigskip
\item Using the given $3 \times 3$ pixel kernel and original image and applying the concept of convolution, the value of central pixel of the output image is \rule{2cm}{0.5mm} $\brak{in\ integer}$.
\\
\begin{figure}[H]
    \centering
    \begin{tabular}{|c|c|c|}
        \hline
        $1/9$ & $1/9$ & $1/9$ \\
        \hline
        $1/9$ & $1/9$ & $1/9$ \\
        \hline
        $1/9$ & $1/9$ & $1/9$ \\
        \hline
    \end{tabular}
    \hspace{2cm} % Horizontal space between tables
    \begin{tabular}{|c|c|c|}
    
    \hline
        $67$ & $67$ & $72$ \\
        \hline
        $70$ & $68$ & $71$ \\
        \hline
        $72$ & $71$ & $72$ \\
        \hline
    \end{tabular}
    \hspace{2cm} % Horizontal space between tables
    \begin{tabular}{|c|c|c|}
        \hline
        
& & \\
        \hline
        & \huge{?} & \\
        \hline
        & & \\
        \hline
    \end{tabular}
    
    \vspace{0.5cm} % Vertical space between tables and labels
    
    \begin{tabular}{c c c}
        \hspace{2cm}
        \textbf{KERNEL} & \hspace{1cm} \textbf{ORIGINAL IMAGE} & \hspace{1cm} \textbf{OUTPUT 
IMAGE}
    \end{tabular}
\end{figure}\\.
\hfill $\brak{\text{GATE GE 2025}}$
\bigskip
\item A four-band multispectral image with pixel size of $50$ m $\times$ $50$ m covers a ground area of $20$ km $\times$ $20$ km.
If the radiometric resolution of the satellite data is $8$ bits, then the uncompressed satellite image contains \rule{2cm}{0.5mm} kilobytes $\brak{kB}$ of data $\brak{in\ integer}$\\.
\hfill $\brak{\text{GATE GE 2025}}$
\bigskip
\item In spatial interpolation using coordinate transformations for image-to-map rectification, the minimum number of ground control points $\brak{GCPs}$ required to perform a third-order transformation is \rule{2cm}{0.5mm} $\brak{in\ integer}$\\.
\hfill $\brak{\text{GATE GE 2025}}$
\bigskip
\item In an image with $6$-bit quantization level, the pixel values of a scene are between $25$ and $55$. A linear contrast stretch is applied to the image covering the full dynamic range.
A pixel value $40$ in the original image will be mapped to \rule{2cm}{0.5mm} $\brak{rounded\ off\ to\ nearest\ integer}$ in the stretched image.
\end{enumerate}
\hfill $\brak{\text{GATE GE 2025}}$
\bigskip

\end{document}


