\documentclass{beamer}
\usepackage[utf8]{inputenc}

\usetheme{Madrid}
\usecolortheme{default}
\usepackage{amsmath,amssymb,amsfonts,amsthm}
\usepackage{txfonts}
\usepackage{tkz-euclide}
\usepackage{listings}
\usepackage{adjustbox}
\usepackage{array}
\usepackage{tabularx}
\usepackage{gvv}
\usepackage{lmodern}
\usepackage{circuitikz}
\usepackage{tikz}
\usepackage{graphicx}

\setbeamertemplate{page number in head/foot}[totalframenumber]

\usepackage{tcolorbox}
\tcbuselibrary{minted,breakable,xparse,skins}



\definecolor{bg}{gray}{0.95}
\DeclareTCBListing{mintedbox}{O{}m!O{}}{%
  breakable=true,
  listing engine=minted,
  listing only,
  minted language=#2,
  minted style=default,
  minted options={%
    linenos,
    gobble=0,
    breaklines=true,
    breakafter=,,
    fontsize=\small,
    numbersep=8pt,
    #1},
  boxsep=0pt,
  left skip=0pt,
  right skip=0pt,
  left=25pt,
  right=0pt,
  top=3pt,
  bottom=3pt,
  arc=5pt,
  leftrule=0pt,
  rightrule=0pt,
  bottomrule=2pt,
  toprule=2pt,
  colback=bg,
  colframe=orange!70,
  enhanced,
  overlay={%
    \begin{tcbclipinterior}
    \fill[orange!20!white] (frame.south west) rectangle ([xshift=20pt]frame.north west);
    \end{tcbclipinterior}},
  #3,
}
\lstset{
    language=C,
    basicstyle=\ttfamily\small,
    keywordstyle=\color{blue},
    stringstyle=\color{orange},
    commentstyle=\color{green!60!black},
    numbers=left,
    numberstyle=\tiny\color{gray},
    breaklines=true,
    showstringspaces=false,
}
\begin{document}

\title 
{2.8.16}
\date{August 27,2025}


\author 
{Kavin B-EE25BTECH11033}






\frame{\titlepage}
\begin{frame}{Question}
Prove that the lines $x=py+q , z=ry+s \text{ and } x=p^{\prime}y+q^{\prime}, z=r^{\prime}y+s^{\prime} $ are perpendicular if $pp^{\prime}+rr^{\prime}+1=0$..
\end{frame}



\begin{frame}{Theoretical Solution}

Let line $L_1$ be the intersection of the planes,
\begin{center}
    $x-py-q=0\ ,\ z-ry-s=0$
\end{center}
Let line $L_2$ be the intersection of the planes,
\begin{center}
    $x-p^{\prime}y-q^{\prime}=0\ ,\ z-r^{\prime}y-s^{\prime}=0$
\end{center}
\bigskip

\end{frame}

\begin{frame}{Theoretical Solution}
Let $\vec{n_1}$, $\vec{n_2}$ be the normals for the planes $x-py-q=0\ \text{and}\ z-ry-s=0$ respectively.

\begin{align}
    direction\ vector\ of\ \vec{n_1}\ = \myvec{1\\-p\\0}
\end{align}
\begin{align}
    direction\ vector\ of\ \vec{n_2}\ = \myvec{0\\-r\\1}
\end{align}
Let $\vec{n_3}$, $\vec{n_4}$ be the normals for the planes $x-p^{\prime}y-q^{\prime}=0\ \text{and}\ z-r^{\prime}y-s^{\prime}=0$ respectively.

\begin{align}
    direction\ vector\ of\ \vec{n_3}\ = \myvec{1\\-p^{\prime}\\0}
\end{align}
\end{frame}


\begin{frame}{Theoretical Solution}
To find the direction vectors, the equations of the lines can be expressed as,
\begin{align}
	\vec{x} = \myvec{x_1 \\ y_1 \\ z_1} + \kappa \myvec{a \\ b \\ c}
\end{align}
where $\myvec{a\\b\\c}$ is the direction vector.\\
\end{frame}

\begin{frame}{Theoretical Solution}

The equations for the line $L_1$ are:
\begin{align}
    x &= py + q \\
    z &= ry + s
\end{align}
We can rearrange these to isolate $y$:
\begin{align}
    x - q = py \implies y = \frac{x-q}{p} \\
    z - s = ry \implies y = \frac{z-s}{r}
\end{align}
By equating these expressions for $y$, and including the term for $y$ itself, the line equation of $L_1$ to be:
\begin{align}
	\vec{x} = \myvec{q \\ 0 \\ s} + \kappa_1 \myvec{p \\ 1 \\ r}
\end{align}
\end{frame}
\begin{frame}{Theoretical Solution}
From this, the direction vectors of the line $L_1$ are $\myvec{p\\1\\r}$.\\
The equations for the line $L_2$ are:
\begin{align}
    x &= p^{\prime}y + q^{\prime} \\
    z &= r^{\prime}y + s^{\prime}
\end{align}
Similarly, we rearrange to isolate $y$:
\begin{align}
    x - q^{\prime} = p^{\prime}y \implies y = \frac{x-q^{\prime}}{p^{\prime}} \\
    z - s^{\prime} = r^{\prime}y \implies y = \frac{z-s^{\prime}}{r^{\prime}}
\end{align}

\end{frame}

\begin{frame}{Theoretical Solution}
By equating these expressions for $y$, and including the term for $y$ itself, the line equation of $L_2$ to be:
\begin{align}
	\vec{x} = \myvec{q^{\prime}\\ 0 \\ s^{\prime}} + \kappa_2 \myvec{p^{\prime} \\ 1 \\ r^{\prime}}
\end{align}\
From this, the direction vectors of the line $L_2$ are $\myvec{p^{\prime}\\1\\r^{\prime}}$.
\end{frame}

\begin{frame}{Theoretical Solution}
If the lines are perpendicular, then their dot product of direction vectors must be zero.
\begin{align}
    \implies \brak{direction\ vector\ of\ L_1}^\top\brak{direction\ vector\ of\ L_2} = 0
\end{align}
\begin{align}
\implies \myvec{p & 1 & r} \myvec{p^{\prime}\\1\\r^{\prime}} = 0
\end{align}
\begin{align}
\implies pp^{\prime}+rr^{\prime}+1=0
\end{align}\\
\bigskip
$\therefore$ The lines $x=py+q , z=ry+s \text{ and } x=p^{\prime}y+q^{\prime}, z=r^{\prime}y+s^{\prime} $ are perpendicular if $pp^{\prime}+rr^{\prime}+1=0$\\
\\
Hence proved.
\end{frame}


\begin{frame}[fragile]
    \frametitle{C Code - A function to check whether they are perpendicular}

    \begin{lstlisting}

#include <stdio.h>

int is_perpendicular(double p, double r, double p_prime, double r_prime) {
    if ((p * p_prime) + (r * r_prime) + 1 == 0) {
        return 1; // True, the lines are perpendicular
    }
    return 0; // False, the lines are not perpendicular
}
    \end{lstlisting}
\end{frame}

\begin{frame}[fragile]
    \frametitle{Python Code}
    \begin{lstlisting}
import ctypes
import os

# Load the shared library
lib = ctypes.CDLL('./code.so')
# Define the argument types for the C function
lib.is_perpendicular.argtypes = [ctypes.c_double, ctypes.c_double, ctypes.c_double, ctypes.c_double]
# Define the return type for the C function
lib.is_perpendicular.restype = ctypes.c_int

def check_perpendicular(p, r, p_prime, r_prime):
    result = lib.is_perpendicular(p, r, p_prime, r_prime)
    return bool(result)

    \end{lstlisting}
\end{frame}


\end{document}