\let\negmedspace\undefined
\let\negthickspace\undefined
\documentclass[journal]{IEEEtran}
\usepackage[a5paper, margin=10mm, onecolumn]{geometry}
%\usepackage{lmodern} % Ensure lmodern is loaded for pdflatex
\usepackage{tfrupee} % Include tfrupee package

\setlength{\headheight}{1cm} % Set the height of the header box
\setlength{\headsep}{0mm} % Set the distance between the header box and the top of the text

\usepackage{gvv-book}
\usepackage{gvv}
\usepackage{cite}
\usepackage{amsmath,amssymb,amsfonts,amsthm}
\usepackage{algorithmic}
\usepackage{graphicx}
\usepackage{textcomp}
\usepackage{xcolor}
\usepackage{txfonts}
\usepackage{listings}
\usepackage{enumitem}
\usepackage{mathtools}
\usepackage{gensymb}
\usepackage{comment}
\usepackage[breaklinks=true]{hyperref}
\usepackage{tkz-euclide} 
\usepackage{listings}
% \usepackage{gvv}                                        
\def\inputGnumericTable{}                                 
\usepackage[latin1]{inputenc}                                
\usepackage{color}                                            
\usepackage{array}                                            
\usepackage{longtable}                                       
\usepackage{calc}                                             
\usepackage{multirow}                                         
\usepackage{hhline}                                           
\usepackage{ifthen}                                           
\usepackage{lscape}
\begin{document}

\bibliographystyle{IEEEtran}
\vspace{3cm}

\title{2.8.16}
\author{EE25BTECH11033 - Kavin}
% \maketitle
% \newpage
% \bigskip
{\let\newpage\relax\maketitle}

\renewcommand{\thefigure}{\theenumi}
\renewcommand{\thetable}{\theenumi}
\setlength{\intextsep}{10pt} % Space between text and floats
\textbf{Question}:\\
Prove that the lines $x=py+q , z=ry+s \text{ and } x=p^{\prime}y+q^{\prime}, z=r^{\prime}y+s^{\prime} $ are perpendicular if $pp^{\prime}+rr^{\prime}+1=0$.\\
\bigskip

\textbf{Solution}:\\
Let line $L_1$ be the intersection of the planes,
\begin{center}
    $x-py-q=0\ ,\ z-ry-s=0$
\end{center}
Let line $L_2$ be the intersection of the planes,
\begin{center}
    $x-p^{\prime}y-q^{\prime}=0\ ,\ z-r^{\prime}y-s^{\prime}=0$
\end{center}
\bigskip
To find the direction vectors, the equations of the lines can be expressed as,
\begin{align}
	\vec{x} = \myvec{x_1 \\ y_1 \\ z_1} + \kappa \myvec{a \\ b \\ c}
\end{align}
where $\myvec{a\\b\\c}$ is the direction vector.\\
\bigskip

The equations for the line $L_1$ are:
\begin{align}
    x &= py + q \\
    z &= ry + s
\end{align}
We can rearrange these to isolate $y$:
\begin{align}
    x - q = py \implies y = \frac{x-q}{p} \\
    z - s = ry \implies y = \frac{z-s}{r}
\end{align}
By equating these expressions for $y$, and including the term for $y$ itself, the line equation of $L_1$ to be:
\begin{align}
	\vec{x} = \myvec{q \\ 0 \\ s} + \kappa_1 \myvec{p \\ 1 \\ r}
\end{align}\


From this, the direction vectors of the line $L_1$ are $\myvec{p\\1\\r}$.

The equations for the line $L_2$ are:
\begin{align}
    x &= p^{\prime}y + q^{\prime} \\
    z &= r^{\prime}y + s^{\prime}
\end{align}
Similarly, we rearrange to isolate $y$:
\begin{align}
    x - q^{\prime} = p^{\prime}y \implies y = \frac{x-q^{\prime}}{p^{\prime}} \\
    z - s^{\prime} = r^{\prime}y \implies y = \frac{z-s^{\prime}}{r^{\prime}}
\end{align}
By equating these expressions for $y$, and including the term for $y$ itself, the line equation of $L_2$ to be:
\begin{align}
	\vec{x} = \myvec{q^{\prime}\\ 0 \\ s^{\prime}} + \kappa_2 \myvec{p^{\prime} \\ 1 \\ r^{\prime}}
\end{align}\
From this, the direction vectors of the line $L_2$ are $\myvec{p^{\prime}\\1\\r^{\prime}}$.\\
\bigskip

If the lines are perpendicular, then their dot product of direction vectors must be zero.
\begin{align}
    \implies \brak{direction\ vector\ of\ L_1}^\top\brak{direction\ vector\ of\ L_2} = 0
\end{align}
\begin{align}
\implies \myvec{p & 1 & r} \myvec{p^{\prime}\\1\\r^{\prime}} = 0
\end{align}
\begin{align}
\implies pp^{\prime}+rr^{\prime}+1=0
\end{align}\\
\bigskip
$\therefore$ The lines $x=py+q , z=ry+s \text{ and } x=p^{\prime}y+q^{\prime}, z=r^{\prime}y+s^{\prime} $ are perpendicular if $pp^{\prime}+rr^{\prime}+1=0$\\
\\
Hence proved.

\end{document}