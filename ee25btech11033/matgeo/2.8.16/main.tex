\let\negmedspace\undefined
\let\negthickspace\undefined
\documentclass[journal]{IEEEtran}
\usepackage[a5paper, margin=10mm, onecolumn]{geometry}
%\usepackage{lmodern} % Ensure lmodern is loaded for pdflatex
\usepackage{tfrupee} % Include tfrupee package

\setlength{\headheight}{1cm} % Set the height of the header box
\setlength{\headsep}{0mm} % Set the distance between the header box and the top of the text

\usepackage{gvv-book}
\usepackage{gvv}
\usepackage{cite}
\usepackage{amsmath,amssymb,amsfonts,amsthm}
\usepackage{algorithmic}
\usepackage{graphicx}
\usepackage{textcomp}
\usepackage{xcolor}
\usepackage{txfonts}
\usepackage{listings}
\usepackage{enumitem}
\usepackage{mathtools}
\usepackage{gensymb}
\usepackage{comment}
\usepackage[breaklinks=true]{hyperref}
\usepackage{tkz-euclide} 
\usepackage{listings}
% \usepackage{gvv}                                        
\def\inputGnumericTable{}                                 
\usepackage[latin1]{inputenc}                                
\usepackage{color}                                            
\usepackage{array}                                            
\usepackage{longtable}                                       
\usepackage{calc}                                             
\usepackage{multirow}                                         
\usepackage{hhline}                                           
\usepackage{ifthen}                                           
\usepackage{lscape}
\begin{document}

\bibliographystyle{IEEEtran}
\vspace{3cm}

\title{2.8.16}
\author{EE25BTECH11033 - Kavin}
% \maketitle
% \newpage
% \bigskip
{\let\newpage\relax\maketitle}

\renewcommand{\thefigure}{\theenumi}
\renewcommand{\thetable}{\theenumi}
\setlength{\intextsep}{10pt} % Space between text and floats
\textbf{Question}:\\
Prove that the lines $x=py+q , z=ry+s \text{ and } x=p^{\prime}y+q^{\prime}, z=r^{\prime}y+s^{\prime} $ are perpendicular if $pp^{\prime}+rr^{\prime}+1=0$.\\
\bigskip

\textbf{Solution}:\\
Let line $L_1$ be the intersection of the planes,
\begin{center}
    $x-py-q=0\ ,\ z-ry-s=0$
\end{center}
Let line $L_2$ be the intersection of the planes,
\begin{center}
    $x-p^{\prime}y-q^{\prime}=0\ ,\ z-r^{\prime}y-s^{\prime}=0$
\end{center}
\bigskip
The direction vectors of the lines $L_1$ and $L_2$ are given by the cross product of the direction vectors of the normals of the intersecting planes.\\ 
Let $\vec{n_1}$, $\vec{n_2}$ be the normals for the planes $x-py-q=0\ \text{and}\ z-ry-s=0$ respectively.

\begin{align}
    direction\ vector\ of\ \vec{n_1}\ = \myvec{1\\-p\\0}
\end{align}
\begin{align}
    direction\ vector\ of\ \vec{n_2}\ = \myvec{0\\-r\\1}
\end{align}
Let $\vec{n_3}$, $\vec{n_4}$ be the normals for the planes $x-p^{\prime}y-q^{\prime}=0\ \text{and}\ z-r^{\prime}y-s^{\prime}=0$ respectively.

\begin{align}
    direction\ vector\ of\ \vec{n_3}\ = \myvec{1\\-p^{\prime}\\0}
\end{align}
\begin{align}
    direction\ vector\ of\ \vec{n_4}\ = \myvec{0\\-r^{\prime}\\1}
\end{align}
\newpage
\begin{align}
    \therefore direction\ vector\ of\ L_1\ = \vec{n_1} \times \vec{n_2} 
\end{align}
\begin{align}
    direction\ vector\ of\ L_2\ = \vec{n_3} \times \vec{n_4} 
\end{align}
\bigskip

The {\em cross product} or {\em vector product} of $\vec{n_1}, \vec{n_2}$ is defined as
\begin{align}
  \label{eq:cross3d}
	\vec{n_1} \times \vec{n_2} 
	 = \myvec{ \mydet{(\vec{n_1})_{23} & (\vec{n_2})_{23}} \\[1ex] \mydet{(\vec{n_1})_{31} & (\vec{n_2})_{31}} \\[1ex] \mydet{(\vec{n_1})_{12}  & (\vec{n_2})_{12}}}
\end{align}\\
\begin{align}
	\mydet{(\vec{n_1})_{23}&(\vec{n_2})_{23}}=\mydet{-p & -r \\ 0 & 1}=-p\\\
	\mydet{(\vec{n_1})_{31}&(\vec{n_2})_{31}}=\mydet{0 & 1 \\ 1 & 0}=-1\\
	\mydet{(\vec{n_1})_{12}&(\vec{n_2})_{12}}=\mydet{1 & 0 \\ -p & -r}=-r,
	\\
	\vec{n_1}\times\vec{n_2}
	 =\myvec{ \mydet{(\vec{n_1})_{23} & (\vec{n_2})_{23}} \\[1ex] \mydet{(\vec{n_1})_{31} & (\vec{n_2})_{31}} \\[1ex] \mydet{(\vec{n_1})_{12}  & (\vec{n_2})_{12}}}
=\myvec{-p\\-1\\-r}
\end{align}
\begin{align}
    \implies direction\ vector\ of\ L_1\ = \myvec{-p\\-1\\-r}
\end{align}\\
\bigskip

The {\em cross product} or {\em vector product} of $\vec{n_3}, \vec{n_4}$ is defined as
\begin{align}
  \label{eq:cross3d}
	\vec{n_3} \times \vec{n_4} 
	 = \myvec{ \mydet{(\vec{n_3})_{23} & (\vec{n_4})_{23}} \\[1ex] \mydet{(\vec{n_3})_{31} & (\vec{n_4})_{31}} \\[1ex] \mydet{(\vec{n_3})_{12}  & (\vec{n_4})_{12}}}
\end{align}\\
\begin{align}
	\mydet{(\vec{n_3})_{23}&(\vec{n_4})_{23}}=\mydet{-p^{\prime} & -r^{\prime} \\ 0 & 1}=-p^{\prime}\\\
	\mydet{(\vec{n_3})_{31}&(\vec{n_4})_{31}}=\mydet{0 & 1 \\ 1 & 0}=-1\\
	\mydet{(\vec{n_3})_{12}&(\vec{n_4})_{12}}=\mydet{1 & 0 \\ -p^{\prime} & -r^{\prime}}=-r^{\prime},
	\\
	\vec{n_3}\times\vec{n_4}
	 =\myvec{ \mydet{(\vec{n_3})_{23} & (\vec{n_4})_{23}} \\[1ex] \mydet{(\vec{n_3})_{31} & (\vec{n_4})_{31}} \\[1ex] \mydet{(\vec{n_3})_{12}  & (\vec{n_4})_{12}}}
=\myvec{-p^{\prime}\\-1\\-r^{\prime}}
\end{align}
\begin{align}
    \implies direction\ vector\ of\ L_2\ = \myvec{-p^{\prime}\\-1\\-r^{\prime}}
\end{align}\\
\bigskip

If the lines are perpendicular, then their dot product of direction vectors must be zero.
\begin{align}
    \implies \brak{direction\ vector\ of\ L_1}^\top\brak{direction\ vector\ of\ L_2} = 0
\end{align}
\begin{align}
\implies \myvec{-p & -1 & -r} \myvec{-p^{\prime}\\-1\\-r^{\prime}} = 0
\end{align}
\begin{align}
\implies pp^{\prime}+rr^{\prime}+1=0
\end{align}\\
\bigskip

$\therefore$the lines $x=py+q , z=ry+s \text{ and } x=p^{\prime}y+q^{\prime}, z=r^{\prime}y+s^{\prime} $ are perpendicular if $pp^{\prime}+rr^{\prime}+1=0$
\end{document}