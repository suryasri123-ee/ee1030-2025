\let\negmedspace\undefined
\let\negthickspace\undefined
\documentclass[journal]{IEEEtran}
\usepackage[a5paper, margin=10mm, onecolumn]{geometry}
%\usepackage{lmodern} % Ensure lmodern is loaded for pdflatex
\usepackage{tfrupee} % Include tfrupee package

\setlength{\headheight}{1cm} % Set the height of the header box
\setlength{\headsep}{0mm}     % Set the distance between the header box and the top of the text

\usepackage{gvv-book}
\usepackage{gvv}
\usepackage{cite}
\usepackage{amsmath,amssymb,amsfonts,amsthm}
\usepackage{algorithmic}
\usepackage{graphicx}
\usepackage{textcomp}
\usepackage{xcolor}
\usepackage{txfonts}
\usepackage{listings}
\usepackage{enumitem}
\usepackage{mathtools}
\usepackage{gensymb}
\usepackage{comment}
\usepackage[breaklinks=true]{hyperref}
\usepackage{tkz-euclide} 
\usepackage{listings}
% \usepackage{gvv}                                        
\def\inputGnumericTable{}                                 
\usepackage[latin1]{inputenc}                                
\usepackage{color}                                            
\usepackage{array}                                            
\usepackage{longtable}                                       
\usepackage{calc}                                             
\usepackage{multirow}                                         
\usepackage{hhline}                                           
\usepackage{ifthen}                                           
\usepackage{lscape}
\begin{document}

\bibliographystyle{IEEEtran}
\vspace{3cm}

\title{1.5.21}
\author{EE25BTECH11033 - Kavin}
% \maketitle
% \newpage
% \bigskip
{\let\newpage\relax\maketitle}

\renewcommand{\thefigure}{\theenumi}
\renewcommand{\thetable}{\theenumi}
\setlength{\intextsep}{10pt} % Space between text and floats
\textbf{Question}:\\
Find the ratio in which $\vec{P}(4,m)$ divides the line segment joining the points $\vec{A}(2,3)$ and $\vec{B}(6,-3)$. Hence, find $m$.\\
\bigskip


\textbf{Solution}:\\
Let the vector $\vec{P}$ be 
\begin{align}
    \vec{P}=\begin{myvec}{4\\m}\end{myvec} \;, 
\end{align}
Given the points,
\begin{align}
    \vec{A}=\begin{myvec}{2\\3}\end{myvec}
    \vec{B}=\begin{myvec}{6\\-3}\end{myvec}
\end{align}
We can use the section formula to find the ratio first and then we can compute the value of $m$.\\
\\
Section formula for a vector $\vec{P}$ which divides the line formed by vectors $\vec{A}$ and $\vec{B}$ in the ratio k:1 is given by
\begin{align}
    \vec{P}=\frac{k\vec{B}+\vec{A}}{k+1}
\end{align}
Using section formula,\\
\begin{align}
         \myvec{4\\m} &=\frac{{\myvec{2\\3}+k\myvec{6\\-3}}}{1+k}\\
    \implies\myvec{4\\m} + k\myvec{4\\m} &=\myvec{2\\3}+k\myvec{6\\-3}\\ 
	 \implies k\myvec{2 \\ -3-m} &= \myvec{2 \\ m-3}
	 \\
	 \text{or, } k &= \frac{1}{1}.\\
     \implies m &= 0.
\end{align}
\bigskip
\\
See Fig. 0 ,
\begin{figure}[H]
\begin{center}
\includegraphics[width=0.6\columnwidth]{figs/fig.png}
\end{center}
\caption{}
\label{fig:Fig1}
\end{figure}





\end{document}
