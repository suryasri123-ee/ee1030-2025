\documentclass{beamer}
\usepackage[utf8]{inputenc}

\usetheme{Madrid}
\usecolortheme{default}
\usepackage{amsmath,amssymb,amsfonts,amsthm}
\usepackage{txfonts}
\usepackage{tkz-euclide}
\usepackage{listings}
\usepackage{adjustbox}
\usepackage{array}
\usepackage{tabularx}
\usepackage{gvv}
\usepackage{lmodern}
\usepackage{circuitikz}
\usepackage{tikz}
\usepackage{graphicx}

\setbeamertemplate{page number in head/foot}[totalframenumber]

\usepackage{tcolorbox}
\tcbuselibrary{minted,breakable,xparse,skins}



\definecolor{bg}{gray}{0.95}
\DeclareTCBListing{mintedbox}{O{}m!O{}}{%
  breakable=true,
  listing engine=minted,
  listing only,
  minted language=#2,
  minted style=default,
  minted options={%
    linenos,
    gobble=0,
    breaklines=true,
    breakafter=,,
    fontsize=\small,
    numbersep=8pt,
    #1},
  boxsep=0pt,
  left skip=0pt,
  right skip=0pt,
  left=25pt,
  right=0pt,
  top=3pt,
  bottom=3pt,
  arc=5pt,
  leftrule=0pt,
  rightrule=0pt,
  bottomrule=2pt,
  toprule=2pt,
  colback=bg,
  colframe=orange!70,
  enhanced,
  overlay={%
    \begin{tcbclipinterior}
    \fill[orange!20!white] (frame.south west) rectangle ([xshift=20pt]frame.north west);
    \end{tcbclipinterior}},
  #3,
}
\lstset{
    language=C,
    basicstyle=\ttfamily\small,
    keywordstyle=\color{blue},
    stringstyle=\color{orange},
    commentstyle=\color{green!60!black},
    numbers=left,
    numberstyle=\tiny\color{gray},
    breaklines=true,
    showstringspaces=false,
}
\begin{document}

\title 
{2.10.62}
\date{August 28,2025}


\author 
{Kavin B-EE25BTECH11033}






\frame{\titlepage}
\begin{frame}{Question}
Find all values of $\lambda$ such that $x,y,z\neq\brak{0,0,0}$ and $$\brak{\hat{i}+\hat{j}+3\hat{k}}x+\brak{3\hat{i}-3\hat{j}+\hat{k}}y+\brak{-4\hat{i}+5\hat{j}}z=\lambda\brak{x\hat{i}+y\hat{j}+z\hat{k}}$$ where $\hat{i}, \hat{j}, \hat{k}$ are unit vectors along the coordinate axes.
\end{frame}



\begin{frame}{Theoretical Solution}

The given vector equation is:
\begin{align}
(\hat{i} + \hat{j} + 3\hat{k})x + (3\hat{i} - 3\hat{j} + \hat{k})y + (-4\hat{i} + 5\hat{j})z = \lambda(x\hat{i} + y\hat{j} + z\hat{k})
\end{align}\\

First, we group the terms on the left-hand side by the unit vectors $\hat{i}$, $\hat{j}$, and $\hat{k}$:
\begin{align}
    (x + 3y - 4z)\hat{i} + (x - 3y + 5z)\hat{j} + (3x + y)\hat{k} = \lambda x\hat{i} + \lambda y\hat{j} + \lambda z\hat{k}
\end{align}\\
which can be expressed as,
\begin{align}
x\myvec{1\\1\\3} + y\myvec{3\\-3\\1} + z\myvec{-4\\5\\0} = \lambda\myvec{x\\y\\z}
\end{align}

\end{frame}

\begin{frame}{Theoretical Solution}
\begin{align}
\implies\begin{myvec}{1 & 3 & -4 \\ 1 & -3 & 5 \\ 3 & 1 & 0}\end{myvec}\begin{myvec}{x \\ y \\ z}\end{myvec} = \lambda\begin{myvec}{x \\ y \\ z}\end{myvec}
\end{align}
\begin{align}
\implies A\vec{v} = \lambda\vec{v}
\end{align}

\end{frame}


\begin{frame}{Theoretical Solution}
This is a homogeneous system of linear equations. It can be expressed in matrix form as $(A - \lambda I)\vec{v} = 0$, where:
\begin{align}
A = \begin{myvec}{1 & 3 & -4 \\ 1 & -3 & 5 \\ 3 & 1 & 0}\end{myvec}, \quad I = \begin{myvec}{1 & 0 & 0 \\ 0 & 1 & 0 \\ 0 & 0 & 1}\end{myvec}, \quad \vec{v} = \begin{myvec}{x \\ y \\ z}\end{myvec}
\end{align}

The problem states that $(x, y, z) \neq (0, 0, 0)$, which means we are looking for a \textbf{non-trivial solution} for the vector $\vec{v}$. This is a \textbf{eigenvalue problem}. The values of $\lambda$ for which non-trivial solutions exist are the eigenvalues of the matrix $A$.\\
\end{frame}

\begin{frame}{Theoretical Solution}

A non-trivial solution exists if and only if the determinant of the coefficient matrix is zero. This gives us the characteristic equation:
\begin{align}
\mydet{A - \lambda I} = 0
\end{align}
\begin{align}
\mydet{
1-\lambda & 3 & -4 \\
1 & -3-\lambda & 5 \\
3 & 1 & -\lambda
}= 0
\end{align}

Now, we calculate the determinant by expanding along the first row:
\begin{align}
(1-\lambda) \mydet{-3-\lambda & 5 \\ 1 & -\lambda} - 3 \mydet{ 1 & 5 \\ 3 & -\lambda } + (-4) \mydet{ 1 & -3-\lambda \\ 3 & 1 } &=0 \\
(1-\lambda)((-3-\lambda)(-\lambda) - 5) - 3(-\lambda - 15) - 4(1 - 3(-3-\lambda)) &= 0\\
(1-\lambda)(\lambda^2 + 3\lambda - 5) + 3(\lambda + 15) - 4(10 + 3\lambda) &= 0\\
(\lambda^2 + 3\lambda - 5 - \lambda^3 - 3\lambda^2 + 5\lambda) + (3\lambda + 45) - (40 + 12\lambda) &= 0\\
-\lambda^3 - 2\lambda^2 + 8\lambda - 5 + 3\lambda + 45 - 40 - 12\lambda &= 0
\end{align}
\end{frame}
\begin{frame}{Theoretical Solution}
Combine like terms to get the characteristic polynomial:
\begin{align}
-\lambda^3 - 2\lambda^2 - \lambda &= 0 \\
\lambda^3 + 2\lambda^2 + \lambda &= 0
\end{align}

Factoring out $\lambda$:
\begin{align}
\lambda(\lambda^2 + 2\lambda + 1) = 0
\end{align}
The quadratic term is a perfect square:
\begin{align}
\lambda(\lambda + 1)^2 = 0
\end{align}

The solutions for $\lambda$ are:
\begin{align}
\lambda = 0 \quad \text{or} \quad \lambda = -1
\end{align}

Thus, the required values of $\lambda$ are 0 and -1.

\end{frame}

\end{document}