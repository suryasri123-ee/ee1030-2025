
\let\negmedspace\undefined
\let\negthickspace\undefined
\documentclass[journal,12pt,onecolumn]{IEEEtran}
\usepackage{cite}
\usepackage{amsmath,amssymb,amsfonts,amsthm}
\usepackage{algorithmic}
\usepackage{graphicx}
\graphicspath{{./figs/}}
\usepackage{textcomp}
\usepackage{xcolor}
\usepackage{txfonts}
\usepackage{listings}
\usepackage{enumitem}
\usepackage{mathtools}
\usepackage{gensymb}
\usepackage{comment}
\usepackage{caption}
\usepackage[breaklinks=true]{hyperref}
\usepackage{tkz-euclide} 
\usepackage{listings}
\usepackage{gvv}                                        
                                
\usepackage[latin1]{inputenc}     
\usepackage{xparse}
\usepackage{color}                                            
\usepackage{array}                                            
\usepackage{longtable}                                       
\usepackage{calc}                                             
\usepackage{multirow}
\usepackage{multicol}
\usepackage{hhline}                                           
\usepackage{ifthen}                                           
\usepackage{lscape}
\usepackage{tabularx}
\usepackage{array}
\usepackage{float}
\newtheorem{theorem}{Theorem}[section]
\newtheorem{problem}{Problem}
\newtheorem{proposition}{Proposition}[section]
\newtheorem{lemma}{Lemma}[section]
\newtheorem{corollary}[theorem]{Corollary}
\newtheorem{example}{Example}[section]
\newtheorem{definition}[problem]{Definition}
\newcommand{\BEQA}{\begin{eqnarray}}
\newcommand{\EEQA}{\end{eqnarray}}
\newcommand{\define}{\stackrel{\triangle}{=}}
\theoremstyle{remark}
\newtheorem{rem}{Remark}

\begin{document}
\begin{center}
\LARGE \textbf{Assignment 1: GATE 2018 MA}\\[2pt] 
\large EE25BTECH11061-- Vankudoth Sainadh
\end{center}

\begin{enumerate}
    \item ``The dress \underline{\hspace{2cm}} her so well that they all immediately \underline{\hspace{2cm}} her on her appearance." 
    
    \hfill{\brak{\text{GATE MA 2018}}}
    
    The word that best fill blanks in the above sentence are
    \begin{enumerate}
        \begin{multicols}{2}
            \item complemented, complemented
            \item complimented, complemented
            \item complimented, complimented
            \item complemented, complimented
        \end{multicols}
    \end{enumerate}

    \item ``The judge's standing in the legal community, though shaken by false allegations of wrongdoing, remained \underline{\hspace{2cm}}."
    
    \hfill{\brak{\text{GATE MA 2018}}} 
   
    The word that best fill blanks in the above sentence are
    \begin{enumerate}
        \begin{multicols}{2}
            \item undiminished
            \item damaged
            \item illegal
            \item uncertain
        \end{multicols}
    \end{enumerate}

\item Find the missing group of letters in the following series: BC, FGH, LMNO, \underline{\hspace{2cm}} \\ \makebox[\linewidth][r]{\brak{\text{GATE MA 2018}}}
\begin{enumerate}
    \begin{multicols}{4}
        \item UVWXY
        \item TUVWX
        \item STUVW
        \item RSTUV
    \end{multicols}
\end{enumerate}

    \item The perimeters of a circle, a square and an equilateral triangle are equal. Which one of the following statements is true?
    
    \hfill{\brak{\text{GATE MA 2018}}}
    \begin{enumerate}
        \item The circle has the largest area.
        \item The square has the largest area.
        \item The equilateral triangle has the largest area.
        \item All the three shapes have the same area.
    \end{enumerate}

    \item The value of the expression $\frac{1}{1+\log_u vw} + \frac{1}{1+\log_v wu} + \frac{1}{1+\log_w uv}$ is \underline{\hspace{2cm}}.
    
    \hfill{\brak{\text{GATE MA 2018}}}
    \begin{enumerate}
        \begin{multicols}{4}
            \item $-1$
            \item $0$
            \item $1$
            \item $3$
        \end{multicols}
    \end{enumerate}

   \item Forty students watched films A, B and C over a week. Each student watched either only one film or all three. Thirteen students watched film A, sixteen students watched film B and nineteen students watched film C. How many students watched all three films? \nolinebreak
   
   \hfill\mbox{\brak{\text{GATE MA 2018}}}
\begin{enumerate}
    \begin{multicols}{4}
        \item $0$
        \item $2$
        \item $4$
        \item $8$
    \end{multicols}
\end{enumerate}
\end{enumerate}

\begin{enumerate}[start=7]
    \item A wire would enclose an area of $1936 \, \text{m}^2$, if it is bent into a square. The wire is cut into two pieces. The longer piece is thrice as long as the shorter piece. The long and the short pieces are bent into a square and a circle, respectively. Which of the following choices is closest to the sum of the areas enclosed by the two pieces in square meters? \nolinebreak
    
    \hfill\mbox{\brak{\text{GATE MA 2018}}}
    \begin{enumerate}
        \begin{multicols}{4}
            \item $1096$
            \item $1111$
            \item $1243$
            \item $2486$
        \end{multicols}
    \end{enumerate}
\newpage
    \item A contract is to be completed in $52$ days and $125$ identical robots were employed, each operational for $7$ hours a day. After $39$ days, five-seventh of the work was completed. How many additional robots would be required to complete the work on time, if each robot is now operational for $8$ hours a day? \nolinebreak
    
    \hfill\mbox{\brak{\text{GATE MA 2018}}}
    \begin{enumerate}
        \begin{multicols}{4}
            \item $50$
            \item $89$
            \item $146$
            \item $175$
        \end{multicols}
    \end{enumerate}

    \item A house has a number which needs to be identified. The following three statements are given that can help in identifying the house number. \nolinebreak
    
    \hfill\mbox{\brak{\text{GATE MA 2018}}}

    i. If the house number is a multiple of $3$, then it is a number from $50$ to $59$. \\
    ii. If the house number is NOT a multiple of $4$, then it is a number from $60$ to $69$. \\
    iii. If the house number is NOT a multiple of $6$, then it is a number from $70$ to $79$.

    \begin{enumerate}
        \begin{multicols}{4}
            \item $54$
            \item $65$
            \item $66$
            \item $76$
        \end{multicols}
    \end{enumerate}
\end{enumerate}
\begin{enumerate}[start=10]
    \item An unbiased coin is tossed six times in a row and four different such trials are conducted. One trial implies six tosses of the coin. If H stands for head and T stands for tail, the following are the observations from the four trials: \newline (1) HTHTHT (2) TTHHHT (3) HTTHHT (4) HHHT\_\_ \_\_. \vspace{0.3cm}\newline Which statement describing the last two coin tosses of the fourth trial has the highest probability of being correct? \nolinebreak
    
    \hfill\mbox{\brak{\text{GATE MA 2018}}}
    \begin{enumerate}
                \item Two T will occur.
            \item One H and one T will occur.
            \item Two H will occur.
            \item One H will be followed by one T.
        
    \end{enumerate}

\newpage




\item The principal value of $\brak{-1}^{\brak{-2i/\pi}}$ is

\hfill{\brak{\text{GATE MA 2018}}}

\begin{enumerate}
\begin{multicols}{4}
\item $e^{2}$
\item $e^{2i}$
\item $e^{-2i}$
\item $e^{-2}$
\end{multicols}
\end{enumerate}

\item Let $f\colon \mathbb{C}\to \mathbb{C}$ be an entire function with $f\brak{0}=1$, $f\brak{1}=2$ and $f'\brak{0}=0$. If there exists $M>0$ such that $\abs{f''\brak{z}}\le M$ for all $z\in \mathbb{C}$, then $f\brak{2}=$ 

\hfill{\brak{\text{GATE MA 2018}}}

\begin{enumerate}
\begin{multicols}{4}
\item $2$
\item $5$
\item $2+5i$
\item $5+2i$
\end{multicols}
\end{enumerate}

\item In the Laurent series expansion of $f\brak{z}=\dfrac{1}{z\brak{z-1}}$ valid for $\abs{z-1}>1$, the coefficient of $\dfrac{1}{z-1}$ is

\hfill{\brak{\text{GATE MA 2018}}}

\begin{enumerate}
\begin{multicols}{4}
\item $-2$
\item $-1$
\item $0$
\item $1$
\end{multicols}
\end{enumerate}

\item Let $X$ and $Y$ be metric spaces, and let $f\colon X\to Y$ be a continuous map. For any subset $S$ of $X$, which one of the following statements is true? \hfill{\brak{\text{GATE MA 2018}}}

\begin{enumerate}

\item \text{If $S$ is open, then $f\brak{S}$ is open}
\item \text{If $S$ is connected, then $f\brak{S}$ is connected}
\item \text{If $S$ is closed, then $f\brak{S}$ is closed}
\item \text{If $S$ is bounded, then $f\brak{S}$ is bounded}

\end{enumerate}

\item The general solution of the differential equation $x y'= y+\sqrt{x^{2}+y^{2}}$ for $x>0$ is given by \newline \brak{\text{with an arbitrary positive constant $k$ }}

\hfill{\brak{\text{GATE MA 2018}}}

\begin{enumerate}

\item $k y^{2}= x+\sqrt{x^{2}+y^{2}}$
\item $k x^{2}= x+\sqrt{x^{2}+y^{2}}$
\item $k x^{2}= y+\sqrt{x^{2}+y^{2}}$
\item $k y^{2}= y+\sqrt{x^{2}+y^{2}}$
\end{enumerate}

\item Let $p_n\brak{x}$ be the polynomial solution of the differential equation
\begin{align*}
\frac{d}{dx}\brak{\brak{1- x^2}y'} + n\brak{n+ 1}y = 0
\end{align*}
with $p_n\brak{1} = 1$ for $n = 1, 2, 3, \ldots$. If
\begin{align*}
\frac{d}{dx}\brak{p_{n+2}\brak{x}- p_n\brak{x}} = \alpha_n\,p_{n+1}\brak{x},
\end{align*}
then $\alpha_n$ is

\hfill{\brak{\text{GATE MA 2018}}}
\begin{enumerate}
\begin{multicols}{4}
\item $2n$
\item $2n+ 1$
\item $2n+ 2$
\item $2n+ 3$
\end{multicols}
\end{enumerate}

\item In the permutation group $S_{6}$, the number of elements of order $8$ is

\hfill{\brak{\text{GATE MA 2018}}}
\begin{enumerate}
\begin{multicols}{4}
\item $0$
\item $1$
\item $2$
\item $4$
\end{multicols}
\end{enumerate}
\newpage
\item Let $R$ be a commutative ring with $1$ \brak{\text{unity}} which is not a field. Let $I \subset R$ be a proper ideal
such that every element of $R$ not in $I$ is invertible in $R$. Then the number of maximal ideals of
$R$ is

\hfill{\brak{\text{GATE MA 2018}}}
\begin{enumerate}
\begin{multicols}{4}
\item $1$
\item $2$
\item $3$
\item \text{infinite}
\end{multicols}
\end{enumerate}

\item Let $f\colon \mathbb{R} \to \mathbb{R}$ be a twice continuously differentiable function. The order of convergence of
the secant method for finding root of the equation $f\brak{x} = 0$ is

\hfill{\brak{\text{GATE MA 2018}}}
\begin{enumerate}
\begin{multicols}{4}
\item $\dfrac{1+\sqrt{5}}{2}$
\item $\dfrac{2}{1+\sqrt{5}}$
\item $\dfrac{1+\sqrt{5}}{3}$
\item $\dfrac{3}{1+\sqrt{5}}$
\end{multicols}
\end{enumerate}

\item The Cauchy problem $u\,u_x + y\,u_y = x$ with $u\brak{x, 1} = 2x$, when solved using its characteristic
equations with an independent variable $t$, is found to admit of a solution in the form
\begin{align*}
x &= \frac{3}{2}\,s e^{t} - \frac{1}{2}\,s e^{-t},\quad
y = e^{t},\quad
u = f\brak{s, t}.
\end{align*}
Then $f\brak{s, t}$ is

\hfill{\brak{\text{GATE MA 2018}}}
\begin{enumerate}
\begin{multicols}{2}
\item $\dfrac{3}{2}\,s e^{t} + \dfrac{1}{2}\,s e^{-t}$
\item $\dfrac{1}{2}\,s e^{t} + \dfrac{3}{2}\,s e^{-t}$
\item $\dfrac{1}{2}\,s e^{t} - \dfrac{3}{2}\,s e^{-t}$
\item $\dfrac{3}{2}\,s e^{t} - \dfrac{1}{2}\,s e^{-t}$
\end{multicols}
\end{enumerate}

\item An urn contains four balls, each ball having equal probability of being white or black. Three
black balls are added to the urn. The probability that five balls in the urn are black is

\hfill{\brak{\text{GATE MA 2018}}}
\begin{enumerate}
\begin{multicols}{4}
\item $\dfrac{2}{7}$
\item $\dfrac{3}{8}$
\item $\dfrac{1}{2}$
\item $\dfrac{5}{7}$
\end{multicols}
\end{enumerate}

\item For a linear programming problem, which one of the following statements is FALSE? \\ \makebox[\linewidth][r]{\brak{\text{GATE MA 2018}}}
\begin{enumerate}
\item If a constraint is an equality, then the corresponding dual variable is unrestricted in sign
\item Both primal and its dual can be infeasible
\item If primal is unbounded, then its dual is infeasible
\item Even if both primal and dual are feasible, the optimal values of the primal and the dual can differ
\end{enumerate}

\item Let $A =
\myvec{ a & 2f & 0 \\[2pt] 2f & b & 3f \\[2pt] 0 & 3f & c }$, where $a, b, c, f$ are real numbers and $f \ne 0$. The geometric multiplicity of the largest eigenvalue of $A$ equals \underline{\hspace{2cm}}.

\hfill{\brak{\text{GATE MA 2018}}}

\item Consider the subspaces
\begin{align*}
W_1 &= \{\brak{x_1, x_2, x_3} \in \mathbb{R}^3 \colon x_1 = x_2 + 2x_3\},\\
W_2 &= \{\brak{x_1, x_2, x_3} \in \mathbb{R}^3 \colon x_1 = 3x_2 + 2x_3\}
\end{align*}
of $\mathbb{R}^3$. Then the dimension of $W_1 + W_2$ equals \underline{\hspace{2cm}}. 

\hfill{\brak{\text{GATE MA 2018}}}
\newpage
\item Let $V$ be the real vector space of all polynomials of degree less than or equal to $2$ with real
coefficients. Let $T \colon V \to V$ be the linear transformation given by
\begin{align*}
T\brak{p} = 2p+ p',\quad \text{for } p \in V,
\end{align*}
where $p'$ is the derivative of $p$. Then the number of nonzero entries in the Jordan canonical
form of a matrix of $T$ equals \underline{\hspace{2cm}}. 
\hfill{\brak{\text{GATE MA 2018}}}

\item Let $I = [2, 3)$, $J$ be the set of all rational numbers in the interval $[4, 6]$, $K$ be the Cantor
\brak{\text{ternary}} set, and let $L = \{7 + x \colon x \in K\}$. Then the Lebesgue measure of the set $I \cup J \cup L$
equals \underline{\hspace{2cm}}. 

\hfill{\brak{\text{GATE MA 2018}}}

\item Let $u\brak{x, y, z} = x^2 - 2y + 4z^2$ for $\brak{x, y, z} \in \mathbb{R}^3$. Then the directional derivative of $u$ in the
direction $\dfrac{3}{5}\,\hat{\imath}- \dfrac{4}{5}\,\hat{k}$ at the point $\brak{5, 1, 0}$ is \underline{\hspace{2cm}}.

\hfill{\brak{\text{GATE MA 2018}}}

\item If the Laplace transform of $y\brak{t}$ is given by $Y\brak{s} = \mathcal{L}\brak{y\brak{t}} =
\dfrac{5}{2\brak{s- 1}} - \dfrac{2}{s- 2} + \dfrac{1}{2\brak{s- 3}}$,
then $y\brak{0} + y'\brak{0} = $ \underline{\hspace{2cm}}.

\hfill{\brak{\text{GATE MA 2018}}}

\item The number of regular singular points of the differential equation
\begin{align*}
\bigl[(x-1)^2 \sin x\bigr]\,y'' \;+\; \bigl[\cos x\,\sin(x-1)\bigr]\,y' \;+\; (x-1)\,y \;=\; 0.
\end{align*}

in the interval $\brak{0, \dfrac{\pi}{2}}$ is equal to \underline{\hspace{2cm}}.

\hfill{\brak{\text{GATE MA 2018}}}

\item Let $F$ be a field with $49$ elements and let $K$ be a subfield of $F$ with $7$ elements. Then the
dimension of $F$ as a vector space over $K$ is \underline{\hspace{2cm}}. 

\hfill{\brak{\text{GATE MA 2018}}}

\item Let $C\brak{[0, 1]}$ be the real vector space of all continuous real valued functions on $[0, 1]$, and let
$T$ be the linear operator on $C\brak{[0, 1]}$ given by
\begin{align*}
\brak{Tf}\brak{x} = \int_{0}^{1} \sin\brak{x+ y}\, f\brak{y}\, dy,\quad x \in [0, 1].
\end{align*}
Then the dimension of the range space of $T$ equals \underline{\hspace{2cm}}. 

\hfill{\brak{\text{GATE MA 2018}}}

\item Let $a \in \brak{-1, 1}$ be such that the quadrature rule $\int_{-1}^{1} f\brak{x}\, dx \approx f\brak{-a} + f\brak{a}$
is exact for all polynomials of degree less than or equal to $3$. Then $3a^2 = $ \underline{\hspace{2cm}}.\\ \makebox[\linewidth][r]{\brak{\text{GATE MA 2018}}}

\item Let $X$ and $Y$ have joint probability density function given by
\begin{align*}
f_{X,Y}\brak{x, y} =
\begin{cases}
2, & 0 \le x \le 1- y,\ 0 \le y \le 1,\\
0, & \text{otherwise.}
\end{cases}
\end{align*}
If $f_Y$ denotes the marginal probability density function of $Y$, then $f_Y\brak{1/2} = $ \underline{\hspace{2cm}}. \\ \makebox[\linewidth][r]{\brak{\text{GATE MA 2018}}}

\item Let the cumulative distribution function of the random variable $X$ be given by
\begin{align*}
F_X\brak{x} =
\begin{cases}
0, & x < 0,\\
x, & 0 \le x < 1/2,\\
\brak{1 + x}/2, & 1/2 \le x < 1,\\
1, & x \ge 1.
\end{cases}
\end{align*}
Then $P\brak{X = 1/2} = $ \underline{\hspace{2cm}}.

\hfill{\brak{\text{GATE MA 2018}}}

\item Let $\{X_j\}$ be a sequence of independent Bernoulli random variables with $P\brak{X_j = 1} = 1/4$
and let $Y_n = \dfrac{1}{n}\,\sum_{j=1}^{n} X_j^2$. Then $Y_n$ converges, in probability, to \underline{\hspace{2cm}}. 
\hfill{\brak{\text{GATE MA 2018}}}


\item Let $\Gamma$ be the circle given by $z = 4e^{i\theta}$, where $\theta$ varies from $0$ to $2\pi$. Then
\begin{align*}
\oint_{\Gamma} \frac{e^{z}}{z^2 - 2z}\, dz =
\end{align*}

\hfill{\brak{\text{GATE MA 2018}}}
\begin{enumerate}
\begin{multicols}{2}
\item $2\pi i\brak{e^{2} - 1}$
\item $\pi i\brak{1 - e^{2}}$
\item $\pi i\brak{e^{2} - 1}$
\item $2\pi i\brak{1 - e^{2}}$
\end{multicols}
\end{enumerate}

\item The image of the half plane $\Re\brak{z} + \Im\brak{z} > 0$ under the map $w = \dfrac{z - 1}{z + i}$ is given by 

\hfill{\brak{\text{GATE MA 2018}}}
\begin{enumerate}
\begin{multicols}{2}
\item $Re\brak{w} > 0$
\item $Im\brak{w} > 0$
\item $\abs{w} > 1$
\item $\abs{w} < 1$
\end{multicols}
\end{enumerate}

\item Let $D \subset \mathbb{R}^2$ denote the closed disc with center at the origin and radius $2$. Then
\begin{align*}
\iint_{D} e^{-\brak{x^2 + y^2}}\, dx\, dy =
\end{align*}

\hfill{\brak{\text{GATE MA 2018}}}
\begin{enumerate}
\begin{multicols}{2}
\item $\pi\brak{1 - e^{-4}}$
\item $\dfrac{\pi}{2}\brak{1 - e^{-4}}$
\item $\pi\brak{1 - e^{-2}}$
\item $\dfrac{\pi}{2}\brak{1 - e^{-2}}$
\end{multicols}
\end{enumerate}

\item Consider the polynomial $p\brak{X} = X^{4} + 4$ in the ring $\mathbb{Q}[X]$ of polynomials in the variable $X$
with coefficients in the field $\mathbb{Q}$ of rational numbers. Then \hfill{\brak{\text{GATE MA 2018}}}\
\begin{enumerate}
\item \text{the set of zeros of $p\brak{X}$ in $\mathbb{C}$ forms a group under multiplication}
\item \text{$p\brak{X}$ is reducible in the ring $\mathbb{Q}[X]$}
\item \text{the splitting field of $p\brak{X}$ has degree $3$ over $\mathbb{Q}$}
\item \text{the splitting field of $p\brak{X}$ has degree $4$ over $\mathbb{Q}$}
\end{enumerate}

\item Which one of the following statements is true? 

\hfill{\brak{\text{GATE MA 2018}}}
\begin{enumerate}
\item \text{Every group of order $12$ has a non-trivial proper normal subgroup}
\item \text{Some group of order $12$ does not have a non-trivial proper normal subgroup}
\item \text{Every group of order $12$ has a subgroup of order $6$}
\item \text{Every group of order $12$ has an element of order $12$}
\end{enumerate}

\item For an odd prime $p$, consider the ring $\mathbb{Z}\brak{\sqrt{-p}} = \{a + b\sqrt{-p} \colon a, b \in \mathbb{Z}\} \subseteq \mathbb{C}$. Then the
element $2$ in $\mathbb{Z}\brak{\sqrt{-p}}$ is 

\hfill{\brak{\text{GATE MA 2018}}}
\begin{enumerate}
\begin{multicols}{2}
\item \text{a unit}
\item \text{a square}
\item \text{a prime}
\item \text{irreducible}
\end{multicols}
\end{enumerate}
\newpage
\item Consider the following two statements:

\hfill{\brak{\text{GATE MA 2018}}}
\begin{align*}
\text{P}\colon &\ \myvec{0 & 5\\ 0 & 7}\ \text{has infinitely many LU factorizations, where $L$ is lower triangular}\\
&\ \text{with each diagonal entry $1$ and $U$ is upper triangular.}\\[4pt]
\text{Q}\colon &\ \myvec{0 & 0\\ 2 & 5}\ \text{has no LU factorization, where $L$ is lower triangular with each diag-}\\
&\ \text{onal entry $1$ and $U$ is upper triangular.}
\end{align*}
Then which one of the following options is correct?
\begin{enumerate}
\begin{multicols}{2}
\item \text{P is TRUE and Q is FALSE}
\item \text{Both P and Q are TRUE}
\item \text{P is FALSE and Q is TRUE}
\item \text{Both P and Q are FALSE}
\end{multicols}
\end{enumerate}

\item If the characteristic curves of the partial differential equation $x\,u_{xx} + 2x^{2}u_{xy} = u_{x} - 1$ are
$\mu\brak{x, y} = c_1$ and $\nu\brak{x, y} = c_2$, where $c_1$ and $c_2$ are constants, then 

\hfill{\brak{\text{GATE MA 2018}}}
\begin{enumerate}
\begin{multicols}{2}
\item $\mu\brak{x, y} = x^{2} - y,\ \nu\brak{x, y} = y$
\item $\mu\brak{x, y} = x^{2} + y,\ \nu\brak{x, y} = y$
\item $\mu\brak{x, y} = x^{2} + y,\ \nu\brak{x, y} = x^{2}$
\item $\mu\brak{x, y} = x^{2} - y,\ \nu\brak{x, y} = x^{2}$
\end{multicols}
\end{enumerate}

\item Let $f \colon X \to Y$ be a continuous map from a Hausdorff topological space $X$ to a metric space
$Y$. Consider the following two statements:

\hfill{\brak{\text{GATE MA 2018}}}
\begin{align*}
\text{P}\colon &\ f \text{ is a closed map and the inverse image } f^{-1}\brak{y} = \{x \in X \colon f\brak{x} = y\} \text{ is compact for each } y \in Y.\\
\text{Q}\colon &\ \text{For every compact subset } K \subset Y, \text{ the inverse image } f^{-1}\brak{K} \text{ is a compact subset of } X.
\end{align*}
Which one of the following is true?
\begin{enumerate}

\item \text{Q implies P but P does NOT imply Q}
\item \text{P implies Q but Q does NOT imply P}
\item \text{P and Q are equivalent}
\item \text{neither P implies Q nor Q implies P}

\end{enumerate}

\item Let $X$ denote $\mathbb{R}^2$ endowed with the usual topology. Let $Y$ denote $\mathbb{R}$ endowed with the co-finite
topology. If $Z$ is the product topological space $Y \times Y$, then 

\hfill{\brak{\text{GATE MA 2018}}}
\begin{enumerate}
\item \text{the topology of $X$ is the same as the topology of $Z$}
\item \text{the topology of $X$ is strictly coarser \brak{\text{weaker}} than that of $Z$}
\item \text{the topology of $Z$ is strictly coarser \brak{\text{weaker}} than that of $X$}
\item \text{the topology of $X$ cannot be compared with that of $Z$}
\end{enumerate}

\item Consider $\mathbb{R}^n$ with the usual topology for $n = 1, 2, 3$. Each of the following options gives
topological spaces $X$ and $Y$ with respective induced topologies. In which option is $X$ homeomorphic to $Y$?

\hfill{\brak{\text{GATE MA 2018}}}
\begin{enumerate}
\item $X = \{\brak{x, y, z} \in \mathbb{R}^3 \colon x^{2} + y^{2} = 1\},\ Y = \{\brak{x, y, z} \in \mathbb{R}^3 \colon z = 0,\ x^{2} + y^{2} \ne 0\}$
\item $X = \{\brak{x, y} \in \mathbb{R}^{2} \colon y = \sin\brak{1/x},\ 0 < x \le 1\}\cup\{\brak{x, y} \in \mathbb{R}^{2} \colon x = 0,\ -1 \le y \le 1\},\ Y = [0, 1] \subset \mathbb{R}$
\item $X = \{\brak{x, y} \in \mathbb{R}^{2} \colon y = x\sin\brak{1/x},\ 0 < x \le 1\},\ Y = [0, 1] \subset \mathbb{R}$
\item $X = \{\brak{x, y, z} \in \mathbb{R}^3 \colon x^{2} + y^{2} = 1\},\ Y = \{\brak{x, y, z} \in \mathbb{R}^3 \colon x^{2} + y^{2} = z^{2} \ne 0\}$
\end{enumerate}
\newpage
\item Let $\{X_i\}$ be a sequence of independent $\text{Poisson}\brak{\lambda}$ variables and let $W_n = \dfrac{1}{n}\sum_{i=1}^{n} X_i$. Then
the limiting distribution of $\sqrt{n}\brak{W_n - \lambda}$ is the normal distribution with zero mean and variance
given by

\hfill{\brak{\text{GATE MA 2018}}}
\begin{enumerate}
\begin{multicols}{4}
\item $1$
\item $\sqrt{\lambda}$
\item $\lambda$
\item $\lambda^{2}$
\end{multicols}
\end{enumerate}

\item Let $X_1, X_2, \ldots, X_n$ be independent and identically distributed random variables with probability
density function given by
\begin{align*}
f_X\brak{x;\theta} =
\begin{cases}
\theta e^{-\theta\brak{x- 1}}, & x \ge 1,\\
0, & \text{otherwise.}
\end{cases}
\end{align*}
Also, let $\bar{X} = \dfrac{1}{n}\sum_{i=1}^{n} X_i$. Then the maximum likelihood estimator of $\theta$ is

\hfill(GATE MA 2018)
\begin{enumerate}
\begin{multicols}{4}
\item $1/\bar{X}$
\item $\brak{1/\bar{X}} - 1$
\item $1/\brak{\bar{X} - 1}$
\item $\bar{X}$
\end{multicols}
\end{enumerate}

\item Consider the Linear Programming Problem \brak{\text{LPP}}:\\
Maximize $\alpha x_1 + x_2$\\
Subject to $2x_1 + x_2 \le 6$, $-x_1 + x_2 \le 1$, $x_1 + x_2 \le 4$, $x_1 \ge 0$, $x_2 \ge 0$,\\
where $\alpha$ is a constant. If $\brak{3, 0}$ is the only optimal solution, then

\hfill{\brak{\text{GATE MA 2018}}}
\begin{enumerate}
\begin{multicols}{2}
\item $\alpha < -2$
\item $-2 < \alpha < 1$
\item $1 < \alpha < 2$
\item $\alpha > 2$
\end{multicols}
\end{enumerate}

\item Let $M_2\brak{\mathbb{R}}$ be the vector space of all $2 \times 2$ real matrices over the field $\mathbb{R}$. Define the linear
transformation $S \colon M_2\brak{\mathbb{R}} \to M_2\brak{\mathbb{R}}$ by $S\brak{X} = 2X + X^{T}$, where $X^{T}$ denotes the transpose
of the matrix $X$. Then the trace of $S$ equals \underline{\hspace{2cm}}. 

\hfill{\brak{\text{GATE MA 2018}}}

\item Consider $\mathbb{R}^3$ with the usual inner product. If $d$ is the distance from $\brak{1, 1, 1}$ to the subspace
$\text{span}\{(1, 1, 0), (0, 1, 1)\}$ of $\mathbb{R}^3$, then $3d^{2} = $ \underline{\hspace{2cm}}.

\hfill{\brak{\text{GATE MA 2018}}}

\item Consider the matrix $A = I_{9} - 2u u^{T}$ with $u = \dfrac{1}{3}\,[1, 1, 1, 1, 1, 1, 1, 1, 1]$, where $I_{9}$ is the $9 \times 9$
identity matrix and $u^{T}$ is the transpose of $u$. If $\lambda$ and $\mu$ are two distinct eigenvalues of $A$, then
$\abs{\lambda - \mu} = $ \underline{\hspace{2cm}}. \hfill{\brak{\text{GATE MA 2018}}}

\item Let $f\brak{z} = z^{3} e^{z^{2}}$ for $z \in \mathbb{C}$ and let $\Gamma$ be the circle $z = e^{i\theta}$, where $\theta$ varies from $0$ to $4\pi$. Then
\begin{align*}
\frac{1}{2\pi i}\oint_{\Gamma} \frac{f'\brak{z}}{f\brak{z}}\, dz = \underline{\hspace{2cm}}.
\end{align*}


\hfill{\brak{\text{GATE MA 2018}}}

\item Let $S$ be the surface of the solid $V = \{(x, y, z) \colon 0 \le x \le 1,\ 0 \le y \le 2,\ 0 \le z \le 3\}$. Let $\hat{n}$ denote the unit
outward normal to $S$ and let $\vec{F}\brak{x, y, z} = x\,\hat{\imath} + y\,\hat{\jmath} + z\,\hat{k}$, $(x, y, z) \in V$. Then the surface integral
$\iint_{S} \vec{F} \cdot \hat{n}\, dS$ equals \underline{\hspace{2cm}}.

\hfill{\brak{\text{GATE MA 2018}}}
\newpage
\item Let $A$ be a $3\times 3$ matrix with real entries. If three solutions of the linear system of differential
equations $\dot{x}\brak{t} = A x\brak{t}$ are given by
\begin{align*}
\myvec{ e^{t} - e^{2t} \\ -e^{t} + e^{2t} \\ e^{t} + e^{2t} },\quad
\myvec{ -e^{2t} - e^{-t} \\ e^{2t} - e^{-t} \\ e^{2t} + e^{-t} },\quad
\myvec{ e^{-t} + 2e^{t} \\ e^{-t} - 2e^{t} \\ -e^{-t} + 2e^{t} },
\end{align*}
then the sum of the diagonal entries of $A$ is equal to \underline{\hspace{2cm}}. 

\hfill{\brak{\text{GATE MA 2018}}}

\item If $y_1\brak{x} = e^{-x^{2}}$ is a solution of the differential equation $x y'' + \alpha y' + \beta x^{3} y = 0$ for some real numbers
$\alpha$ and $\beta$, then $\alpha\beta = $ \underline{\hspace{2cm}}.

\hfill{\brak{\text{GATE MA 2018}}}

\item Let $L^{2}\brak{[0, 1]}$ be the Hilbert space of all real valued square integrable functions on $[0, 1]$ with
the usual inner product. Let $\phi$ be the linear functional on $L^{2}\brak{[0, 1]}$ defined by
\begin{align*}
\phi\brak{f} = \int_{1/4}^{3/4} {3}{\sqrt{2}}\, f\, d\mu,
\end{align*}
where $\mu$ denotes the Lebesgue measure on $[0, 1]$. Then $\|\phi\| = $ \underline{\hspace{2cm}}.

\hfill(GATE MA 2018)

\item Let $U$ be an orthonormal set in a Hilbert space $H$ and let $x \in H$ be such that $\|x\| = 2$.
Consider the set $E = \{\, u \in U \colon \abs{\langle x, u\rangle} \ge 1/4 \,\}$. Then the maximum possible number of elements in $E$ is \underline{\hspace{2cm}}.

\hfill{\brak{\text{GATE MA 2018}}}

\item If $p\brak{x} = 2 - \brak{x+ 1} + x\brak{x+ 1} - \beta x\brak{x+ 1}\brak{x- \alpha}$ interpolates the points $\brak{x, y}$ in the table\\
$\begin{array}{c|c|c|c|c}
x & -1 & 0 & 1 & 2\\ \hline
y & 2 & 1 & 2 & -7
\end{array}$\\
then $\alpha + \beta = $ \underline{\hspace{2cm}}.

\hfill{\brak{\text{GATE MA 2018}}}

\item If $\sin\brak{\pi x} = a_0 + \sum_{n=1}^{\infty} a_n \cos\brak{n\pi x}$ for $0 < x < 1$, then $\brak{a_0 + a_1}\pi = $ \underline{\hspace{2cm}}. \\ \makebox[\linewidth][r]{\brak{\text{GATE MA 2018}}}

\item For $n = 1, 2, \ldots$, let $f_n\brak{x} = \dfrac{2^{n} x^{n- 1}}{1 + x}$, $x \in [0, 1]$. Then $\displaystyle \lim_{n \to \infty} \int_{0}^{1} f_n\brak{x}\, dx = $ \underline{\hspace{2cm}}. \\ \makebox[\linewidth][r]{\brak{\text{GATE MA 2018}}}

\item Let $X_1, X_2, X_3, X_4$ be independent exponential random variables with mean $1$, $1/2$, $1/3$, $1/4$,
respectively. Then $Y = \min\brak{X_1, X_2, X_3, X_4}$ has exponential distribution with mean equal to \underline{\hspace{2cm}}. 

\hfill{\brak{\text{GATE MA 2018}}}

\item Let $X$ be the number of heads in $4$ tosses of a fair coin by Person $1$ and let $Y$ be the number of
heads in $4$ tosses of a fair coin by Person $2$. Assume that all the tosses are independent. Then
the value of $P\brak{X = Y}$ correct up to three decimal places is \underline{\hspace{2cm}}. \\ \makebox[\linewidth][r]{\brak{\text{GATE MA 2018}}}

\item Let $X_1$ and $X_2$ be independent geometric random variables with the same probability
mass function given by $P\brak{X = k} = p\brak{1 - p}^{k- 1}$, $k = 1, 2, \ldots$. Then the value of
$P\brak{X_1 = 2 \mid X_1 + X_2 = 4}$ correct up to three decimal places is \underline{\hspace{2cm}}.

\hfill(GATE MA 2018)
\newpage
\item A certain commodity is produced by the manufacturing plants $P_1$ and $P_2$ whose capacities are
$6$ and $5$ units, respectively. The commodity is shipped to markets $M_1$, $M_2$, $M_3$ and $M_4$ whose
requirements are $1$, $2$, $3$ and $5$ units, respectively. The transportation cost per unit from plant
$P_i$ to market $M_j$ is as follows:\\[2pt]
\begin{minipage}{\linewidth}
\centering
\begin{tabular}{c|c|c|c|c|c}
& $M_1$ & $M_2$ & $M_3$ & $M_4$  \\ \hline
$P_1$ & $1$ & $3$ & $5$ & $8$ & $6$\\ \hline
$P_2$ & $2$ & $5$ & $6$ & $7$ & $5$\\ \hline
 & $1$ & $2$ & $3$ & $5$ 
\end{tabular}
\end{minipage}\\[4pt]
Then the optimal cost of transportation is \underline{\hspace{2cm}}. 

\hfill{\brak{\text{GATE MA 2018}}}

\end{enumerate}


\end{document}
```

[1] https://ppl-ai-file-upload.s3.amazonaws.com/web/direct-files/attachments/91584593/2df4586e-bc4b-4c32-9800-b23ffdfc7591/ma-2018-p1.pdf

