
\documentclass{beamer}
\mode<presentation>
\usepackage{amsmath}
\usepackage{amssymb}
%\usepackage{advdate}
\usepackage{graphicx}
\graphicspath{{./figs/}}
\usepackage{adjustbox}
\usepackage{subcaption}
\usepackage{enumitem}
\usepackage{multicol}
\usepackage{mathtools}
\usepackage{listings}
\usepackage{url}
\def\UrlBreaks{\do\/\do-}
\usetheme{Boadilla}
\usecolortheme{lily}
\setbeamertemplate{footline}
{
  \leavevmode%
  \hbox{%
  \begin{beamercolorbox}[wd=\paperwidth,ht=2.25ex,dp=1ex,right]{author in head/foot}%
    \insertframenumber{} / \inserttotalframenumber\hspace*{2ex} 
  \end{beamercolorbox}}%
  \vskip0pt%
}
\setbeamertemplate{navigation symbols}{}

\providecommand{\nCr}[2]{\,^{#1}C_{#2}} % nCr
\providecommand{\nPr}[2]{\,^{#1}P_{#2}} % nPr
\providecommand{\mbf}{\mathbf}
\providecommand{\pr}[1]{\ensuremath{\Pr\left(#1\right)}}
\providecommand{\qfunc}[1]{\ensuremath{Q\left(#1\right)}}
\providecommand{\sbrak}[1]{\ensuremath{{}\left[#1\right]}}
\providecommand{\lsbrak}[1]{\ensuremath{{}\left[#1\right.}}
\providecommand{\rsbrak}[1]{\ensuremath{{}\left.#1\right]}}
\providecommand{\brak}[1]{\ensuremath{\left(#1\right)}}
\providecommand{\lbrak}[1]{\ensuremath{\left(#1\right.}}
\providecommand{\rbrak}[1]{\ensuremath{\left.#1\right)}}
\providecommand{\cbrak}[1]{\ensuremath{\left\{#1\right\}}}
\providecommand{\lcbrak}[1]{\ensuremath{\left\{#1\right.}}
\providecommand{\rcbrak}[1]{\ensuremath{\left.#1\right\}}}
\theoremstyle{remark}
\newtheorem{rem}{Remark}
\newcommand{\sgn}{\mathop{\mathrm{sgn}}}
\providecommand{\abs}[1]{\left\vert#1\right\vert}
\providecommand{\res}[1]{\Res\displaylimits_{#1}} 
\providecommand{\norm}[1]{\lVert#1\rVert}
\providecommand{\mtx}[1]{\mathbf{#1}}
\providecommand{\mean}[1]{E\left[ #1 \right]}
\providecommand{\fourier}{\overset{\mathcal{F}}{ \rightleftharpoons}}
%\providecommand{\hilbert}{\overset{\mathcal{H}}{ \rightleftharpoons}}
\providecommand{\system}[1]{\overset{\mathcal{#1}}{ \longleftrightarrow}}
%\providecommand{\system}{\overset{\mathcal{H}}{ \longleftrightarrow}}
	%\newcommand{\solution}[2]{\textbf{Solution:}{#1}}
%\newcommand{\solution}{\noindent \textbf{Solution: }}
\providecommand{\dec}[2]{\ensuremath{\overset{#1}{\underset{#2}{\gtrless}}}}
\newcommand{\myvec}[1]{\ensuremath{\begin{pmatrix}#1\end{pmatrix}}}
\let\vec\mathbf

\lstset{
%language=C,
frame=single, 
breaklines=true,
columns=fullflexible
}

\numberwithin{equation}{section}
\title{1.4.3}
\author{AI25BTECH11033 - Spoorthi N}
% \maketitle
% \newpage
% \bigskip
\begin{document}
{\let\newpage\relax\maketitle}
\renewcommand{\thefigure}{\theenumi}
\renewcommand{\thetable}{\theenumi}

\textbf{Question}:\\
\noindent Find the ratio in which the point $\Vec{P}\brak{\frac{3}{4},\frac{5}{12}} $ divides the line segment joining the points $\vec{A}\brak{\frac{1}{2},\frac{3}{2}}$ and $\vec{B}\brak{2,-5}$

\textbf{Solution}:

Let us solve the given equation theoretically and then verify the solution computationally \\
According to the question, \\
Now\\
\begin{align}
\vec{P}=\myvec{\frac{3}{4}\\ \frac{5}{12}},\vec{A}=\myvec{\frac{1}{2} \\ \frac{3}{2}},\vec{B}=\myvec{2\\-5}
\end{align}
Let $\vec{P}$ divide $\vec{A}$ and $\vec{B}$ in $k:1$ \\
We know that 
\begin{align}
	k= \frac{\vec{(A-P)}^T\vec{(P-B)}}{\|\vec{P-B}\|^2} 
   \end{align}
   \begin{align}
   k=\frac{\myvec{\frac{-1}{4} & \frac{13}{12}}\myvec{\frac{-5}{4} \\ \frac{65}{12}}}{(\frac{-5}{4})^2+(\frac{65}{12})^2} 
   \end{align}
   \begin{align}
       K=\frac{(\frac{-1}{4})(\frac{-5}{4})+(\frac{13}{12})(\frac{65}{12})}{\frac{25}{16}+\frac{4225}{144}}
 \end{align}
 \begin{align}
     K=\frac{(\frac{5}{16})+(\frac{845}{144})}{(\frac{4225+225}{144})}
 \end{align}
\begin{align}
    K=1/5
\end{align}
% Graphical representation
\begin{frame}
\frametitle{Graphical Representation}
    \centering
    \includegraphics[width=0.7\linewidth]{figs/fig1.png}
\end{frame}
\end{document}

