\let\negmedspace\undefined
\let\negthickspace\undefined
\documentclass[journal,12pt,onecolumn]{IEEEtran}
\usepackage{cite}
\usepackage{amsmath,amssymb,amsfonts,amsthm}
\usepackage{algorithmic}
\usepackage{graphicx}
\graphicspath{{./figs/}}
\usepackage{textcomp}
\usepackage{xcolor}
\usepackage{txfonts}
\usepackage{listings}
\usepackage{enumitem}
\usepackage{mathtools}
\usepackage{gensymb}
\usepackage{comment}
\usepackage{caption}
\usepackage[breaklinks=true]{hyperref}
\usepackage{tkz-euclide} 
\usepackage{listings}
\usepackage{gvv}                                        
%\def\inputGnumericTable{}                                 
\usepackage[latin1]{inputenc}     
\usepackage{xparse}
\usepackage{color}                                            
\usepackage{array}                                            
\usepackage{longtable}                                       
\usepackage{calc}                                             
\usepackage{multirow}
\usepackage{multicol}
\usepackage{hhline}                                           
\usepackage{ifthen}                                           
\usepackage{lscape}
\usepackage{tabularx}
\usepackage{array}
\usepackage{float}
%\newtheorem{theorem}{Theorem}[section]
%\newtheorem{theorem}{Theorem}[section]
%\newtheorem{problem}{Problem}
%\newtheorem{proposition}{Proposition}[section]
%\newtheorem{lemma}{Lemma}[section]
%\newtheorem{corollary}[theorem]{Corollary}
%\newtheorem{example}{Example}[section]
%\newtheorem{definition}[problem]{Definition}

\begin{document}

%\textbf{\Large 1.2.1} \\
%\textbf{\large AI25BTECH11001 - Abhisek Mohapatra} \\
\title{1.4.3}
\author{AI25BTECH11033- Spoorthi N}
% \maketitle
% \newpage
% \bigskip
%\begin{document}
{\let\newpage\relax\maketitle}
%\renewcommand{\thefigure}{\theenumi}
%\renewcommand{\thetable}{\theenumi}

\textbf{Question}:\\
\noindent Find the ratio in which the point $\Vec{P}\brak{\frac{3}{4},\frac{5}{12}} $ divides the line segment joining the points $\vec{A}\brak{\frac{1}{2},\frac{3}{2}}$ and $\vec{B}\brak{2,-5}$

\textbf{Solution}:

Let us solve the given equation theoretically and then verify the solution computationally \\
According to the question, \\
Now\\
\begin{align}
\vec{P}=\myvec{\frac{3}{4}\\ \frac{5}{12}},\vec{A}=\myvec{\frac{1}{2} \\ \frac{3}{2}},\vec{B}=\myvec{2\\-5}
\end{align}
Let $\vec{P}$ divide $\vec{A}$ and $\vec{B}$ in $k:1$ \\
We know that 
\begin{align}
	k= \frac{\vec{(A-P)}^T\vec{(P-B)}}{\|\vec{P-B}\|^2} 
   \end{align}
   \begin{align}
   k=\frac{\myvec{\frac{-1}{4} & \frac{13}{12}}\myvec{\frac{-5}{4} \\ \frac{65}{12}}}{(\frac{-5}{4})^2+(\frac{65}{12})^2} 
   \end{align}
   \begin{align}
       K=\frac{(\frac{-1}{4})(\frac{-5}{4})+(\frac{13}{12})(\frac{65}{12})}{\frac{25}{16}+\frac{4225}{144}}
 \end{align}
 \begin{align}
     K=\frac{(\frac{5}{16})+(\frac{845}{144})}{(\frac{4225+225}{144})}
 \end{align}
\begin{align}
    K=1/5
\end{align}
\begin{figure}[h!]
    \centering
    \includegraphics[width=0.8\linewidth]{figs/fig1.png}
    \caption{}
    \label{fig}
\end{figure}
\end{document}
