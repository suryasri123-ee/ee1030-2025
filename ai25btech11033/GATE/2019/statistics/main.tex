%iffalse
\let\negmedspace\undefined
\let\negthickspace\undefined
\documentclass[journal,12pt,onecolumn]{IEEEtran}
\usepackage{cite}
\usepackage{amsmath,amssymb,amsfonts,amsthm}
\usepackage{algorithmic}
\usepackage{graphicx}
\usepackage{textcomp}
\usepackage{xcolor}
\usepackage{txfonts}
\usepackage{listings}
\usepackage{enumitem}
\usepackage{mathtools}
\usepackage{gensymb}
\usepackage{comment}
\usepackage[breaklinks=true]{hyperref}
\usepackage{tkz-euclide} 
\usepackage{gvv}                                        
%\def\inputGnumericTable{}                                 
\usepackage[latin1]{inputenc}     
\usepackage{xparse}
\usepackage{color}                                            
\usepackage{array}                                            
\usepackage{longtable}                                       
\usepackage{calc}                                             
\usepackage{multirow}
\usepackage{multicol}
\usepackage{hhline}                                           
\usepackage{ifthen}                                           
\usepackage{lscape}
\usepackage{tabularx}
\usepackage{array}
\usepackage{float}
\newtheorem{theorem}{Theorem}[section]
\newtheorem{problem}{Problem}
\newtheorem{proposition}{Proposition}[section]
\newtheorem{lemma}{Lemma}[section]
\newtheorem{corollary}[theorem]{Corollary}
\newtheorem{example}{Example}[section]
\newtheorem{definition}[problem]{Definition}
\newcommand{\BEQA}{\begin{eqnarray}}
\newcommand{\EEQA}{\end{eqnarray}}
%\newcommand{\define}{\stackrel{\triangle}{=}}
\theoremstyle{remark}
%\newtheorem{rem}{Remark}
% Marks the beginning of the document
\begin{document}
\title{STATISTICS}
\maketitle
\renewcommand{\thefigure}{\theenumi}
\renewcommand{\thetable}{\theenumi}
\begin {center}
\large \textbf{2020}\\
\large \textbf{STATISTICS}\\
\end{center}

\begin{center}
    \LARGE \textbf{GATE 2019 General Aptitude (GA) Set-8}
\end{center}
\textbf{Q.1 -- Q.5 carry one mark each.}
\begin{enumerate}
 \item The fishermen, \underline{\hspace{1cm}} the flood victims owed their lives, were rewarded by the government. \hfill \textbf{(GATE EE 2025)}\\

\begin{enumerate} 
    \item whom
    \item to which
    \item to whom
    \item that
\end{enumerate} 

\item Some students were not involved in the strike.

If the above statement is true, which of the following conclusions is/are logically necessary?
\hfill \textbf{(GATE EE 2025)}\\
\begin{enumerate}
    \item Some who were involved in the strike were students.
    \item No student was involved in the strike.
    \item At least one student was involved in the strike.
    \item Some who were not involved in the strike were students.
\end{enumerate}

\begin{enumerate} 
    \item 1 and 2
    \item 3
    \item 4
    \item 2 and 3
\end{enumerate}

\item The radius as well as the height of a circular cone increases by 10\%. The percentage increase in its volume is \underline{\hspace{1cm}}.\hfill \textbf{(GATE EE 2025)}\\

\begin{enumerate} 
    \item 17.1
    \item 21.0
    \item 33.1
    \item 72.8
\end{enumerate}

\item Five numbers 10, 7, 5, 4, and 2 are to be arranged in a sequence from left to right following the directions given below:\hfill \textbf{(GATE EE 2025)}\\

\begin{enumerate}
    \item No two odd or even numbers are next to each other.
    \item The second number from the left is exactly half of the left-most number.
    \item The middle number is exactly twice the right-most number.
\end{enumerate}

Which is the second number from the right?

\begin{enumerate} 
    \item 2
    \item 4
    \item 7
    \item 10
\end{enumerate}

\item Until Iran came along, India had never been \underline{\hspace{1cm}} in kabaddi.
\hfill \textbf{(GATE EE 2025)}\\
\begin{enumerate} 
    \item defeated
    \item defeating
    \item defeat
    \item defeatist
\end{enumerate}

\textbf {Q.6 -- Q.10 carry two marks each.}

\item Since the last one year, after a 125 basis point reduction in repo rate by the Reserve Bank of India, banking institutions have been making a demand to reduce interest rates on small saving schemes. Finally, the government announced yesterday a reduction in interest rates on small saving schemes to bring them on par with fixed deposit interest rates.  
\hfill \textbf{(GATE EE 2025)}\\
Which one of the following statements can be inferred from the given passage?

\begin{enumerate} 
    \item Whenever the Reserve Bank of India reduces the repo rate, the interest rates on small saving schemes are also reduced.
    \item Interest rates on small saving schemes are always maintained on par with fixed deposit interest rates.
    \item The government sometimes takes into consideration the demands of banking institutions before reducing the interest rates on small saving schemes.
    \item A reduction in interest rates on small saving schemes follow only after a reduction in repo rate by the Reserve Bank of India.
\end{enumerate}

\item In a country of 1400 million population, 70\% own mobile phones. Among the mobile phone owners, only 294 million access the Internet. Among these Internet users, only half buy goods from e-commerce portals. What is the percentage of these buyers in the country?\hfill \textbf{(GATE EE 2025)}\\

\begin{enumerate} 
    \item 10.50
    \item 14.70
    \item 15.00
    \item 50.00
\end{enumerate}

\item The nomenclature of Hindustani music has changed over the centuries. Since the medieval period \textit{dhrupad} styles were identified as \textit{baanis}. Terms like \textit{gayaki} and \textit{baaj} were used to refer to vocal and instrumental styles, respectively. With the institutionalization of music education the term \textit{gharana} became acceptable. \textit{Gharana} originally referred to hereditary musicians from a particular lineage, including disciples and grand disciples. \hfill \textbf{(GATE EE 2025)}\\

Which one of the following pairings is \textbf{NOT} correct?

\begin{enumerate} 
    \item \textit{dhrupad}, \textit{baani}
    \item \textit{gayaki}, vocal
    \item \textit{baaj}, institution
    \item \textit{gharana}, lineage
\end{enumerate}


\item Two trains started at 7AM from the same point. The first train travelled north at a speed of 80km/h and the second train travelled south at a speed of 100 km/h. The time at which they were 540 km apart is \_\_\_\_\_ AM.\hfill \textbf{(GATE EE 2025)}\\

\begin{enumerate} 
    \item 9
    \item 10
    \item 11
    \item 11.30
\end{enumerate}

\item I read somewhere that in ancient times the prestige of a kingdom depended upon the number of taxes that it was able to levy on its people. It was very much like the prestige of a head-hunter in his own community.''\hfill \textbf{(GATE EE 2025)}\\


\begin{enumerate} 
    \item the prestige of the kingdom
    \item the prestige of the heads
    \item the number of taxes he could levy
    \item the number of heads he could gather
\end{enumerate}
\begin{center}
    \Large \textbf{Q.1 -- Q.25 carry one mark each.}
\end{center}

\item $\lim_{n \to \infty} \sum_{k=1}^n \frac{n}{n^2 + k^2}$ \hfill \textbf{(GATE EE 2025)}\\
is equal to:
\begin{enumerate} 
    \item $\dfrac{\pi}{3}$
    \item $\dfrac{5}{6}$
    \item $\dfrac{3}{4}$
    \item $\dfrac{\pi}{4}$
\end{enumerate}


\item Let $\vec{F} = (x - y + z)(\hat{i} + \hat{j})$ be a vector field on  ${R}^3$. The line integral $\displaystyle \int_C \vec{F} \cdot d\vec{r}$, where $C$ is the triangle with vertices $(0,0,0)$, $(5,0,0)$, and $(5,5,0)$ traversed in that order, is:
\hfill \textbf{(GATE EE 2025)}\\
\begin{enumerate} 
    \item $-25$
    \item $25$
    \item $50$
    \item $5$
\end{enumerate}


\item Let $\{1,2,3,4\}$ represent the outcomes of a random experiment, and $P(\{1\}) = P(\{2\}) = P(\{3\}) = P(\{4\}) = 1/4$. Suppose that $A_1 = \{1,2\}$, $A_2 = \{2,3\}$, $A_3 = \{3,4\}$, and $A_4 = \{1,2,3\}$. Which of the following statements is true?
\hfill \textbf{(GATE EE 2025)}\\

\begin{enumerate} 
    \item $A_1$ and $A_2$ are not independent.
    \item $A_3$ and $A_4$ are independent.
    \item $A_1$ and $A_4$ are not independent.
    \item $A_2$ and $A_4$ are independent.
\end{enumerate}

\item A fair die is rolled two times independently. Given that the outcome on the first roll is $1$, the expected value of the sum of the two outcomes is: \hfill \textbf{(GATE EE 2025)}\\
\begin{enumerate} 
    \item $4$
    \item $4.5$
    \item $3$
    \item $5.5$
\end{enumerate}


\item The dimension of the vector space of $7 \times 7$ real symmetric matrices with trace zero and the sum of the off-diagonal elements zero is: \hfill \textbf{(GATE EE 2025)}\\

\begin{enumerate} 
    \item $47$
    \item $28$
    \item $27$
    \item $26$
\end{enumerate}

\item Let $A$ be a $6 \times 6$ complex matrix with $A^3 \neq 0$ and $A^4 = 0$. Then the number of Jordan blocks of $A$ is:  \hfill \textbf{(GATE EE 2025)}\\
\begin{enumerate} 
    \item $1$ or $6$
    \item $2$ or $3$
    \item $4$
    \item $5$
\end{enumerate}

\item Let $X_1, \ldots, X_n$ be a random sample from a uniform distribution defined over $(0,\theta)$, where $\theta > 0$ and $n \geq 2$. Let $X_{(1)} = \min\{X_1,\ldots,X_n\}$ and $X_{(n)} = \max\{X_1,\ldots,X_n\}$. Then the covariance between $X_{(n)}$ and $\dfrac{X_{(1)}}{X_{(n)}}$ is:  \hfill \textbf{(GATE EE 2025)}\\
\begin{enumerate} 
    \item $0$
    \item $n(n+1)\theta$
    \item $n\theta$
    \item $n^2(n+1)\theta$
\end{enumerate}

\setlength{\parskip}{6pt}
\setlength{\parindent}{0pt}


\item Let $\{X_n\}_{n \geq 1}$ be a sequence of independent and identically distributed normal random variables with mean $4$ and variance $1$. Then  \hfill \textbf{(GATE EE 2025)}\\
\[
\lim_{n \to \infty} P\left( \frac{1}{n} \sum_{i=1}^n X_i > 4.0006 \right) 
\]
is equal to \dots

\item Let $(X_1, X_2)$ be a random vector following bivariate normal distribution with mean vector $(0,0)$, $\mathrm{Var}(X_1)=\mathrm{Var}(X_2)=1$ and correlation coefficient $\rho$, where $|\rho| < 1$. Then \hfill \textbf{(GATE EE 2025)}\\
\[
P(X_1 + X_2 > 0) 
\]
is equal to \dots

\item Let $X_1,\dots,X_n$ be a random sample from normal distribution with mean $\mu$ and variance $1$. Let $\Phi$ be the cumulative distribution function of the standard normal distribution. Given $\Phi(1.96) = 0.975$, the minimum sample size required such that the length of the $95\%$ confidence interval for $\mu$ does NOT exceed $2$ is \dots  \hfill \textbf{(GATE EE 2025)}\\

\item  $X$ be a random variable with probability density function \hfill \textbf{(GATE EE 2025)}\\
\[
f(x;\theta) = \theta e^{-\theta x},
\]
where $x \ge 0$ and $\theta > 0$. To test $H_0: \theta=1$ against $H_1: \theta>1$, the following test is used: \\

\textbf{Reject} $H_0$ if and only if $X > \log_e 20$. \\

Then the size of the test is \dots

\item Let $\{X_n\}_{n \geq 0}$ be a discrete-time Markov chain on the state space $\{1,2,3\}$ with one-step transition probability matrix: \hfill \textbf{(GATE EE 2025)}\\
\[
P = 
\begin{pmatrix}
0.4 & 0.3 & 0.3 \\
0.5 & 0.2 & 0.3 \\
0.2 & 0.4 & 0.4
\end{pmatrix}
\]
and initial distribution $P(X_0=1)=0.5$, $P(X_0=2)=0.2$, $P(X_0=3)=0.3$. \\

Then 
\[
P(X_1=2, X_2=3, X_3=1)
\]
(rounded off to three decimal places) is equal to \dots

\item Let $f$ be a continuous and positive real-valued function on $[0,1]$. Then  \hfill \textbf{(GATE EE 2025)}\\
\[
\int_0^{\pi} f(\sin x)\cos x \, dx 
\]
is equal to \dots

\item A random sample of size $100$ is classified into $10$ class intervals covering all the data points. To test whether the data comes from a normal population with unknown mean and unknown variance, the chi-squared goodness of fit test is used. The degrees of freedom of the test statistic is equal to \dots  \hfill \textbf{(GATE EE 2025)}\\

\item For $i = 1,2,3,4$, let  \hfill \textbf{(GATE EE 2025)}\\
\[
Y_i = \alpha + \beta x_i + \varepsilon_i
\] 
where $x_i$'s are fixed covariates and $\varepsilon_i$'s are uncorrelated random variables with mean $0$ and variance $3$. Here, $\alpha$ and $\beta$ are unknown parameters. Given the following observations,

\[
\begin{array}{|c|c|c|c|c|}
\hline
Y_i & 2 & 2.5 & -0.5 & 1 \\
\hline
x_i & 3 & 2 & -4 & -1 \\
\hline
\end{array}
\]

the variance of the least squares estimator of $\beta$ is equal to \ldots

\item Let $a_n = \dfrac{(-1)^{n+1}}{n!}, \; n \geq 0$, and $b_n = \sum_{k=0}^n a_k, \; n \geq 0$. Then, for $|x| < 1$, the series \hfill \textbf{(GATE EE 2025)}\\
\[
\sum_{n=0}^\infty b_n x^n 
\]
converges to
\begin{enumerate}
    \item $\dfrac{e^{-x}}{1+x}$
    \item $\dfrac{e^{-x}}{1-x^2}$
    \item $\dfrac{-e^{-x}}{1-x}$
    \item $-(1+x)e^{-x}$
\end{enumerate}
\item Let $\{X_k\}_{k \geq 1}$ be a sequence of independent and identically distributed Bernoulli random variables with success probability $p \in (0,1)$. Then, as $n \to \infty$, 
\hfill \textbf{(GATE EE 2025)}\\
\[
\dfrac{1}{n} \sum_{k=1}^n (X_k)^k
\]
converges almost surely to
\begin{enumerate}
    \item $p$
    \item $\dfrac{1}{1-p}$
    \item $\dfrac{1-p}{p}$
    \item $1$
\end{enumerate}
\item  Let $X$ and $Y$ be two independent random variables with $\chi_m^2$ and $\chi_n^2$ distributions, respectively, where $m$ and $n$ are positive integers. Then which of the following statements is true? \hfill \textbf{(GATE EE 2025)}\\

\begin{enumerate}
    \item For $m < n$, $P(X > a) \geq P(Y > a)$ for all $a \in R$.
    \item For $m > n$, $P(X > a) \geq P(Y > a)$ for all $a \in R$.
    \item For $m < n$, $P(X > a) \leq P(Y > a)$ for all $a \in R$.
    \item None of the above.
\end{enumerate}

\item The matrix \hfill \textbf{(GATE EE 2025)}\\
\[
\begin{bmatrix}
0 & 2 & y \\
0 & 0 & 1 \\
x & 0 & 1 
\end{bmatrix}
\]
is diagonalizable when $(x,y,z)$ equals
\begin{enumerate}
    \item[(A)] $(0,0,1)$
    \item[(B)] $(1,1,0)$
    \item[(C)] $(\sqrt{2},\sqrt{2},2)$
    \item[(D)] $(\sqrt{2},\sqrt{2},\sqrt{2})$
\end{enumerate}


\item Suppose that $P_1$ and $P_2$ are two populations with equal prior probabilities having bivariate normal distributions with mean vectors $(2,3)$ and $(1,1)$, respectively. The variance covariance matrix of both the distributions is the identity matrix. Let $z_1 = (2.5,2)$ and $z_2 = (2,1.5)$ be two new observations. According to Fisher's linear discriminant rule, \hfill \textbf{(GATE EE 2025)}\\

\begin{enumerate}
    \item $z_1$ is assigned to $P_1$, and $z_2$ is assigned to $P_2$.
    \item $z_1$ is assigned to $P_2$, and $z_2$ is assigned to $P_1$.
    \item $z_1$ is assigned to $P_1$, and $z_2$ is assigned to $P_1$.
    \item $z_1$ is assigned to $P_2$, and $z_2$ is assigned to $P_2$.
\end{enumerate}

\item Let $X_1,\dots,X_n$ be a random sample from a population having probability density function \hfill \textbf{(GATE EE 2025)}\\
\[
f_X(x;\theta) = \frac{2x}{\theta^2}, \quad 0 < x < \theta.
\]
Then the method of moments estimator of $\theta$ is
\begin{enumerate}
    \item $\dfrac{3 \sum_{i=1}^n X_i}{2n}$
    \item $\dfrac{3 \sqrt{\sum_{i=1}^n X_i^2}}{2n}$
    \item $\dfrac{\sum_{i=1}^n X_i}{n}$
    \item $\dfrac{3 \sum_{i=1}^n X_i (X_i-1)}{2n}$
\end{enumerate}

\item Let $X$ be a normal random variable having mean $\theta$ and variance $1$, where $1 \leq \theta \leq 10$. Then $X$ is \hfill \textbf{(GATE EE 2025)}\\
\begin{enumerate}
    \item sufficient but not complete.
    \item the maximum likelihood estimator of $\theta$.
    \item the uniformly minimum variance unbiased estimator of $\theta$.
    \item complete and ancillary.
\end{enumerate}

\item  Let $\{X_n\}_{n \geq 1}$ be a sequence of independent and identically distributed random variables with mean $\theta$ and variance $\theta$, where $\theta > 0$. Then $\dfrac{\sum_{i=1}^n X_i}{\sum_{i=1}^n X_i^2}$ is a consistent estimator of  \hfill \textbf{(GATE EE 2025)}\\
\begin{enumerate}
    \item $\dfrac{1}{1+\theta}$
    \item $\dfrac{1+\theta}{\theta}$
    \item $\dfrac{1}{\theta}$
    \item $\dfrac{\theta}{1+\theta}$
\end{enumerate}

\item Let $X_1, \ldots, X_{10}$ be a random sample from a population with probability density function \hfill \textbf{(GATE EE 2025)}\\
\[
f(x; \theta) = \frac{e^{-|x-\theta|}}{2}, \quad -\infty < x < \infty, \; -\infty < \theta < \infty.
\]
Then the maximum likelihood estimator of $\theta$
\begin{enumerate}
    \item does not exist.
    \item is not unique.
    \item is the sample mean.
    \item is the smallest observation.
\end{enumerate}

\item Consider the model $Y_i = \beta + \epsilon_i$, where $\epsilon_i$'s are independent normal random variables with zero mean and known variance $\sigma_i^2 > 0$, for $i=1, \ldots, n$. Then the best linear unbiased estimator of the unknown parameter $\beta$ is  \hfill \textbf{(GATE EE 2025)}\\
\begin{enumerate}
    \item $\dfrac{\sum_{i=1}^n (Y_i / \sigma_i^2)}{\sum_{i=1}^n (1/\sigma_i^2)}$
    \item $\dfrac{\sum_{i=1}^n Y_i}{n}$
    \item $\dfrac{\sum_{i=1}^n (Y_i / \sigma_i)}{n}$
    \item $\dfrac{\sum_{i=1}^n (Y_i / \sigma_i)}{\sum_{i=1}^n (1/\sigma_i)}$
\end{enumerate}

\item Let $(X, Y)$ be a bivariate random vector with probability density function \hfill \textbf{(GATE EE 2025)}\\
\[
f_{X,Y}(x,y) = 
\begin{cases}
e^{-y}, & 0 < x < y, \\
0, & \text{otherwise}.
\end{cases}
\]
Then the regression of $Y$ on $X$ is given by
\begin{enumerate}
    \item[(A)] $X+1$
    \item[(B)] $\dfrac{X}{2}$
    \item[(C)] $\dfrac{Y}{2}$
    \item[(D)] $Y+1$
\end{enumerate}
\item Consider a discrete time Markov chain on the state space $\{1,2\}$ with one-step transition probability matrix \hfill \textbf{(GATE EE 2025)}\\
\[
P = \begin{bmatrix}
0.2 & 0.8 \\
0.3 & 0.7
\end{bmatrix}.
\]
Then $\lim_{n \to \infty} P^n$ is
\[
\text{(A)} \; \begin{bmatrix}
\frac{3}{11} & \frac{8}{11} \\
\frac{3}{11} & \frac{8}{11}
\end{bmatrix} 
\quad
\text{(B)} \; \begin{bmatrix}
1 & 0 \\
0 & 1
\end{bmatrix}
\quad
\text{(C)} \; \begin{bmatrix}
0 & 1 \\
1 & 0
\end{bmatrix}
\quad
\text{(D)} \; \begin{bmatrix}
\frac{8}{11} & \frac{3}{11} \\
\frac{8}{11} & \frac{3}{11}
\end{bmatrix}.
\]

\item Let $(X_1, X_2)$ be a random vector with variance-covariance matrix  \hfill \textbf{(GATE EE 2025)}\\
\[
\begin{bmatrix}
4 & 0 \\
0 & 2
\end{bmatrix}.
\]
The two principal components are
\[
\text{(A)} \; X_1 \text{ and } X_2 
\quad
\text{(B)} \; -X_1 \text{ and } X_2
\quad
\text{(C)} \; X_1 \text{ and } -X_2
\quad
\text{(D)} \; X_1+X_2 \text{ and } X_2.
\]

\item Consider the objects $\{1,2,3,4\}$ with the distance matrix  \hfill \textbf{(GATE EE 2025)}\\
\[
\begin{bmatrix}
0 & 1 & 11 & 5 \\
1 & 0 & 2 & 3 \\
11 & 2 & 0 & 4 \\
5 & 3 & 4 & 0
\end{bmatrix}.
\]
Applying the single-linkage hierarchical procedure twice, the two clusters that result are
\[
\text{(A)} \; \{2,3\} \text{ and } \{1,4\}
\quad
\text{(B)} \; \{1,2,3\} \text{ and } \{4\}
\quad
\text{(C)} \; \{1,3,4\} \text{ and } \{2\}
\quad
\text{(D)} \; \{2,3,4\} \text{ and } \{1\}.
\]


\item The maximum likelihood estimates of the mean vector and the variance-covariance matrix 
of a bivariate normal distribution based on the realization  \hfill \textbf{(GATE EE 2025)}\\
\[
\Big\{ \begin{pmatrix}1 \\ 2 \end{pmatrix}, 
       \begin{pmatrix}4 \\ 3 \end{pmatrix}, 
       \begin{pmatrix}4 \\ 4 \end{pmatrix} \Big\}
\]
of a random sample of size $3$, are given by
\[
\text{(A)} \; 
\begin{pmatrix}3 \\ 3\end{pmatrix}, \;
\begin{bmatrix}
2 & 1 \\
1 & 2/3
\end{bmatrix}
\quad
\text{(B)} \; 
\begin{pmatrix}3 \\ 3\end{pmatrix}, \;
\begin{bmatrix}
2 & 1 \\
1 & 3/2
\end{bmatrix}
\]
\[
\text{(C)} \; 
\begin{pmatrix}3 \\ 3\end{pmatrix}, \;
\begin{bmatrix}
3 & 3/2 \\
3/2 & 2/3
\end{bmatrix}
\quad
\text{(D)} \; 
\begin{pmatrix}3 \\ 3\end{pmatrix}, \;
\begin{bmatrix}
3 & 2/3 \\
2/3 & 1
\end{bmatrix}.
\]

\item Consider a fixed effects one-way analysis of variance model \hfill \textbf{(GATE EE 2025)}\\
\[
Y_{ij} = \mu + \tau_i + \epsilon_{ij}, \quad 
i=1,\dots,a, \; j=1,\dots,r,
\]
and $\epsilon_{ij}$'s are independent and identically distributed normal random variables with mean $0$ and variance $\sigma^2$. Here, $r$ and $a$ are positive integers.  

Let $\bar{Y}_{i\cdot} = \frac{1}{r}\sum_{j=1}^r Y_{ij}$. Then $\bar{Y}_{i\cdot}$ is the least squares estimator for
\[
\text{(A)} \; \mu + \frac{\tau_i}{2} 
\quad
\text{(B)} \; \tau_i
\quad
\text{(C)} \; \mu + \tau_i
\quad
\text{(D)} \; \mu.
\]

\item Let $A$ be a $n \times n$ positive semi-definite matrix with eigenvalues  \hfill \textbf{(GATE EE 2025)}\\
$\lambda_1 \ge \cdots \ge \lambda_n$ and with $\alpha$ as the maximum diagonal entry. 
We can find a vector $x$ such that $x^t x = 1$, where $t$ denotes the transpose, and
\[
\text{(A) } x^t A x > \lambda_1 \quad 
\text{(B) } x^t A x < \lambda_n \quad 
\text{(C) } \lambda_n \le x^t A x \le \lambda_1 \quad 
\text{(D) } x^t A x > n\alpha
\]


\item Let $X$ be a random variable with uniform distribution on the interval $(-1,1)$ and $Y = (X+1)^2$. 
Then the probability density function $f(y)$ of $Y$, over the interval $(0,4)$, is \hfill \textbf{(GATE EE 2025)}\\
\[
\text{(A) } \frac{3\sqrt{y}}{16} \quad 
\text{(B) } \frac{1}{4\sqrt{y}} \quad 
\text{(C) } \frac{1}{6\sqrt{y}} \quad 
\text{(D) } \frac{1}{\sqrt{y}}
\]
\item Let $S$ be the solid whose base is the region in the $xy$-plane bounded by the curves 
$y = x^2$ and $y = 8 - x^2$, and whose cross-sections perpendicular to the $x$-axis are squares. \hfill \textbf{(GATE EE 2025)}\\
Then the volume of $S$ (rounded off to two decimal places) is \dots
\item Consider the trinomial distribution with the probability mass function
\[
P(X = x, Y = y) = \frac{7!}{x! y! (7-x-y)!} (0.6)^x (0.2)^y (0.2)^{7-x-y}, 
\quad x \ge 0, y \ge 0, x+y \le 7.
\]
Then $E(Y|X=3)$ is equal to \dots


\item Let $Y_i = \alpha + \beta x_i + \epsilon_i$, where $i = 1,2,3,4$, $x_i$'s are fixed covariates \hfill \textbf{(GATE EE 2025)}\\
and $\epsilon_i$'s are independent and identically distributed standard normal random variables. 
Here, $\alpha$ and $\beta$ are unknown parameters. Let $\Phi$ be the cumulative distribution function 
of the standard normal distribution and $\Phi(1.96) = 0.975$. Given the following observations:
\[
\begin{array}{c|cccc}
Y_i & 3 & -2.5 & 5 & -5 \\
x_i & 1 & -2 & 3 & -2 \\
\end{array}
\]
the length (rounded off to two decimal places) of the shortest 95\% confidence interval for $\beta$ 
based on its least squares estimator is equal to \dots



\item Consider a discrete time Markov chain on the state space $\{1,2,3\}$ with one-step transition probability matrix \hfill \textbf{(GATE EE 2025)}\\
\[
\begin{bmatrix}
0 & 0.2 & 0.8 \\
0.5 & 0 & 0.5 \\
0.6 & 0.4 & 0
\end{bmatrix}.
\]
\item Then the period of the Markov chain is \dots \hfill \textbf{(GATE EE 2025)}\\

\item Suppose customers arrive at an ATM facility according to a Poisson process with rate 5 customers per hour. 
The probability (rounded off to two decimal places) that no customer arrives at the ATM facility 
from 1:00 pm to 1:18 pm is \dots \hfill \textbf{(GATE EE 2025)}\\

\item Let $X$ be a random variable with characteristic function $\phi_X(\cdot)$ such that $\phi_X(2\pi) = 1$. Let ${Z}$ denote the set of integers. Then $P(X \in {Z})$ is equal to \ldots \\[1em] \hfill \textbf{(GATE EE 2025)}\\

\item Let $X_1$ be a random sample of size 1 from uniform distribution over $(\theta, \theta^2)$, where $\theta > 1$. To test $H_0: \theta = 2$ against $H_1: \theta = 3$, reject $H_0$ if and only if $X_1 > 3.5$. Let $\alpha$ and $\beta$ be the size and the power, respectively, of this test. Then $\alpha + \beta$ (rounded off to two decimal places) is equal to \ldots \\[1em] \hfill \textbf{(GATE EE 2025)}\\

\item Let $Y_i = \beta_0 + \beta_1 x_i + \varepsilon_i$, $i=1,\ldots,n$, where $x_i$'s are fixed covariates, and $\varepsilon_i$'s are uncorrelated random variables with mean zero and constant variance. Suppose that $\hat{\beta}_0$ and $\hat{\beta}_1$ are the least squares estimators of the unknown parameters $\beta_0$ and $\beta_1$, respectively. If $\sum_{i=1}^n x_i = 0$, then the correlation between $\hat{\beta}_0$ and $\hat{\beta}_1$ is equal to \ldots \\[1em] \hfill \textbf{(GATE EE 2025)}\\

\item Let $f:{R} \to {R}$ be defined by $f(x) = (3x^2+4)\cos x$. Then \hfill \textbf{(GATE EE 2025)}\\
\[
\lim_{h \to 0} \frac{f(h) + f(-h) - 8}{h^2}
\]
is equal to \ldots \\[1em]
\item The maximum value of $(x-1)^2 + (y-2)^2$ subject to the constraint $x^2 + y^2 \leq 45$ is equal to \ldots \\[1em] \hfill \textbf{(GATE EE 2025)}\\

 Let $X_1, \ldots, X_{10}$ be independent and identically distributed normal random variables with mean $0$ and variance $2$. Then 
\[
E\!\left(\frac{X_1^2}{X_1^2 + \cdots + X_{10}^2}\right)
\]
is equal to \ldots \\[1em]
\item Let $I$ be the $4 \times 4$ identity matrix and $v = (1,2,3,4)^t$, where $t$ denotes the transpose. Then the determinant of $I + vv^t$ is equal to \ldots \\[2em] \hfill \textbf{(GATE EE 2025)}\\

\textbf{END OF THE QUESTION PAPER}

\end{enumerate}

\end{document}
