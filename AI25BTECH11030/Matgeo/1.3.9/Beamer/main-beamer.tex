\documentclass{beamer}
\mode<presentation>
\usepackage{amsmath}
\usepackage{amssymb}
%\usepackage{advdate}
\usepackage{adjustbox}
\usepackage{subcaption}
\usepackage{enumitem}
\usepackage{multicol}
\usepackage{mathtools}
\usepackage{listings}
\usepackage{url}
\def\UrlBreaks{\do\/\do-}
\usetheme{Boadilla}
\usecolortheme{lily}
\setbeamertemplate{footline}
{
  \leavevmode%
  \hbox{%
  \begin{beamercolorbox}[wd=\paperwidth,ht=2.25ex,dp=1ex,right]{author in head/foot}%
    \insertframenumber{} / \inserttotalframenumber\hspace*{2ex} 
  \end{beamercolorbox}}%
  \vskip0pt%
}

\providecommand{\nCr}[2]{\,^{#1}C_{#2}} % nCr
\providecommand{\nPr}[2]{\,^{#1}P_{#2}} % nPr
\providecommand{\mbf}{\mathbf}
\providecommand{\pr}[1]{\ensuremath{\Pr\left(#1\right)}}
\providecommand{\qfunc}[1]{\ensuremath{Q\left(#1\right)}}
\providecommand{\sbrak}[1]{\ensuremath{{}\left[#1\right]}}
\providecommand{\lsbrak}[1]{\ensuremath{{}\left[#1\right.}}
\providecommand{\rsbrak}[1]{\ensuremath{{}\left.#1\right]}}
\providecommand{\brak}[1]{\ensuremath{\left(#1\right)}}
\providecommand{\lbrak}[1]{\ensuremath{\left(#1\right.}}
\providecommand{\rbrak}[1]{\ensuremath{\left.#1\right)}}
\providecommand{\cbrak}[1]{\ensuremath{\left\{#1\right\}}}
\providecommand{\lcbrak}[1]{\ensuremath{\left\{#1\right.}}
\providecommand{\rcbrak}[1]{\ensuremath{\left.#1\right\}}}
\theoremstyle{remark}
\newtheorem{rem}{Remark}
\newcommand{\sgn}{\mathop{\mathrm{sgn}}}
\providecommand{\abs}[1]{\left\vert#1\right\vert}
\providecommand{\res}[1]{\Res\displaylimits_{#1}} 
\providecommand{\norm}[1]{\lVert#1\rVert}
\providecommand{\mtx}[1]{\mathbf{#1}}
\providecommand{\mean}[1]{E\left[ #1 \right]}
\providecommand{\fourier}{\overset{\mathcal{F}}{ \rightleftharpoons}}
%\providecommand{\hilbert}{\overset{\mathcal{H}}{ \rightleftharpoons}}
\providecommand{\system}{\overset{\mathcal{H}}{ \longleftrightarrow}}
	%\newcommand{\solution}[2]{\textbf{Solution:}{#1}}
%\newcommand{\solution}{\noindent \textbf{Solution: }}
\providecommand{\dec}[2]{\ensuremath{\overset{#1}{\underset{#2}{\gtrless}}}}
\newcommand{\myvec}[1]{\ensuremath{\begin{pmatrix}#1\end{pmatrix}}}
\let\vec\mathbf

\lstset{
%language=C,
frame=single, 
breaklines=true,
columns=fullflexible
}

\numberwithin{equation}{section}

\lstset{
  language=Python,
  basicstyle=\ttfamily\small,
  keywordstyle=\color{blue},
  stringstyle=\color{orange},
  numbers=left,
  numberstyle=\tiny\color{gray},
  breaklines=true,
  showstringspaces=false
}
\usepackage{listings}
\usepackage{xcolor}

\lstset{
  language=C,
  basicstyle=\ttfamily\footnotesize,
  keywordstyle=\color{blue}\bfseries,
  commentstyle=\color{gray}\itshape,
  stringstyle=\color{orange},
  numbers=left,
  numberstyle=\tiny\color{gray},
  breaklines=true,
  frame=single,
  showstringspaces=false,
  tabsize=4,
  captionpos=b
}

\numberwithin{equation}{section}
\title{1.3.9}
\author{AI25BTECH11030 - SARVESH TAMGADE}
% \maketitle
% \newpage
% \bigskip
\begin{document}

\begin{frame}
    \titlepage
\end{frame}

\begin{frame}
    \frametitle{Problem Statement}
    1.3.9 The center of a circle is at $(2,0)$. If one end of a diameter is at $(6,0)$, then find the other end.
\end{frame}

\section{Solution}
\begin{frame}
    \frametitle{Solution}
    Let the center be $\mathbf{C} = \myvec{2 \\ 0}$, one end of the diameter $\mathbf{A} = \myvec{6 \\ 0}$, and the other end be $\mathbf{B} = \myvec{x \\ y}$.
    \vspace{1em}
    
    Since the center is the midpoint of the diameter:
    \[
    \mathbf{C} = \frac{\mathbf{A} + \mathbf{B}}{2}
    \]
    Multiply both sides by 2:
    \[
    2\mathbf{C} = \mathbf{A} + \mathbf{B}
    \]
    Rearranging for $\mathbf{B}$:
    \[
    \mathbf{B} = 2\mathbf{C} - \mathbf{A} = 2\myvec{2 \\ 0} - \myvec{6 \\ 0} = \myvec{4 \\ 0} - \myvec{6 \\ 0} = \myvec{-2 \\ 0}
    \]
    \textbf{Answer:} The other end of the diameter is at $\myvec{-2 \\ 0}$.
\end{frame}

\begin{frame}
    \frametitle{Graph}
    \begin{figure}[h!]
        \centering
        \includegraphics[width=0.7\linewidth]{FIG/graph.png}
        \caption{Diameter of the circle with endpoints $\mathbf{A}(6,0)$ and $\mathbf{B}(-2,0)$, center at $(2,0)$.}
    \end{figure}
\end{frame}
\begin{frame}[fragile]
\frametitle{C Code }
\begin{lstlisting}[language=C]
#include <stdio.h>
#include <stdlib.h>
#include "matfun.h"

int main() {
    double **M, **k, **C;
    int cx = 2, cy = 0;  // Center of circle
    int ax = 6, ay = 0;  // One endpoint of diameter

    // Create matrices: M (2x2), k (2x1), C (2x1)
    M = createMat(2, 2);
    k = createMat(2, 1);

    // Arrange matrix M: columns are points C and A
    M[0][0] = (double)cx; M[1][0] = (double)cy;
    M[0][1] = (double)ax; M[1][1] = (double)ay;

    // Weights vector for B = 2*C - A
    k[0][0] = 2.0;
    k[1][0] = -1.0;

\end{lstlisting}
\end{frame}

\begin{frame}[fragile]
\frametitle{C Code }
\begin{lstlisting}[language=C]

    // Calculate B = M * k
    C = Matmul(M, k, 2, 2, 1);

    // Print result B
    printf("Coordinates of other end B = (%.2lf, %.2lf)\n", C[0][0], C[1][0]);

    // Free allocated matrices
    freeMat(M, 2);
    freeMat(k, 2);
    freeMat(C, 2);

    return 0;
}

\end{lstlisting}
\end{frame}
\begin{frame}[fragile]
\frametitle{Python Plot }
\begin{lstlisting}[language=Python]
import numpy as np
import matplotlib.pyplot as plt

def line_gen(A, B, num=100):
    """
    Generates points on a line segment between points A and B.
    A, B are 2x1 numpy arrays (column vectors).
    Returns 2 x num numpy array of points.
    """
    lam = np.linspace(0, 1, num)
    return (1 - lam) * A + lam * B

# Points as column vectors
C = np.array([2, 0]).reshape(-1,1)   # Center
A = np.array([6, 0]).reshape(-1,1)   # One end of diameter
B = 2*C - A                         # Other end of diameter calculated

coords = np.block([[A,B,C]])

# Generate line points for diameter AB
AB = line_gen(A, B)

\end{lstlisting}
\end{frame}
\begin{frame}[fragile]
\frametitle{Python Plot }
\begin{lstlisting}[language=Python]
# Plot line AB
plt.plot(AB[0,:], AB[1,:], label='Diameter')

# Plot points
plt.scatter(coords[0,:], coords[1,:], color=['blue', 'green', 'red'])

# Annotations
plt.text(A[0], A[1]+0.1, 'A (6,0)', ha='center', color='blue')
plt.text(B[0], B[1]+0.1, 'B (-2,0)', ha='center', color='green')
plt.text(C[0], C[1]+0.1, 'C (2,0)', ha='center', color='red')

# Draw the circle centered at C with radius = half the distance AB
radius = np.linalg.norm(A - B) / 2

circle = plt.Circle((C[0], C[1]), radius, fill=False, color='orange', linestyle='--', linewidth=2, label='Circle')

# Add circle to plot
ax = plt.gca()
ax.add_patch(circle)
\end{lstlisting}
\end{frame}
\begin{frame}[fragile]
\frametitle{Python Plot }
\begin{lstlisting}[language=Python]
# Labels and grid
plt.xlabel('$x$')
plt.ylabel('$y$')
plt.legend(loc='best')
plt.grid(True, linestyle='--')
plt.axis('equal')

# Save figure (adjust path as needed)
plt.savefig('circle_diameter_plot_with_circle.png', dpi=300)
plt.show() 
\end{lstlisting}
\end{frame}

\end{document}