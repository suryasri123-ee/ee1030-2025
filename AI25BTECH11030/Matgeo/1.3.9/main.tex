\let\negmedspace\undefined
\let\negthickspace\undefined
\documentclass[journal,12pt,onecolumn]{IEEEtran}
\usepackage{cite}
\usepackage{amsmath,amssymb,amsfonts,amsthm}
\usepackage{algorithmic}
\usepackage{graphicx}
\graphicspath{{./figs/}}
\usepackage{textcomp}
\usepackage{xcolor}
\usepackage{txfonts}
\usepackage{listings}
\usepackage{enumitem}
\usepackage{mathtools}
\usepackage{gensymb}
\usepackage{comment}
\usepackage{caption}
\usepackage[breaklinks=true]{hyperref}
\usepackage{tkz-euclide} 
\usepackage{listings}
\usepackage{gvv}                                        
%\def\inputGnumericTable{}                                 
\usepackage[latin1]{inputenc}     
\usepackage{xparse}
\usepackage{color}                                            
\usepackage{array}                                            
\usepackage{longtable}                                       
\usepackage{calc}                                             
\usepackage{multirow}
\usepackage{multicol}
\usepackage{hhline}                                           
\usepackage{ifthen}                                           
\usepackage{lscape}
\usepackage{tabularx}
\usepackage{array}
\usepackage{float}
%\newtheorem{theorem}{Theorem}[section]
%\newtheorem{theorem}{Theorem}[section]
%\newtheorem{problem}{Problem}
%\newtheorem{proposition}{Proposition}[section]
%\newtheorem{lemma}{Lemma}[section]
%\newtheorem{corollary}[theorem]{Corollary}
%\newtheorem{example}{Example}[section]
%\newtheorem{definition}[problem]{Definition}

\begin{document}


\title{1.3.9}
\author{AI25BTECH110030 - SARVESH TAMGADE}

{\let\newpage\relax\maketitle}


		\textbf{Question}:

The center of a circle is at $(2,0)$. If one end of a diameter is at $(6,0)$, then find the other end.

\textbf{Solution}:

Since the center $\mathbf{C}$ is the midpoint of the diameter endpoints $\mathbf{A}$ and $\mathbf{B}$,
\[
\mathbf{C} = \frac{\mathbf{A} + \mathbf{B}}{2}
\]

Multiply both sides by 2:
\[
2\mathbf{C} = \mathbf{A} + \mathbf{B}
\]

Rearranged for $\mathbf{B}$:
\[
\mathbf{B} = 2\mathbf{C} - \mathbf{A} = 2 \begin{bmatrix} 2 \\ 0 \end{bmatrix} - \begin{bmatrix} 6 \\ 0 \end{bmatrix} = \begin{bmatrix} 4 \\ 0 \end{bmatrix} - \begin{bmatrix} 6 \\ 0 \end{bmatrix} = \begin{bmatrix} -2 \\ 0 \end{bmatrix}
\]

\textbf{Answer:} The other end of the diameter is at
\[
\mathbf{B} = \begin{bmatrix} -2 \\ 0 \end{bmatrix}
\]
Graph:
\begin{figure}[H]
	\centering
	\includegraphics[width=0.8\columnwidth]{FIG/graph.png}
	\caption{Stem plot  of y(n)}
	\label{img}
\end{figure}
\end{document}

