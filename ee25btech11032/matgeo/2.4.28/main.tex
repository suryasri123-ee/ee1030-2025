\let\negmedspace\undefined
\let\negthickspace\undefined
\documentclass[journal]{IEEEtran}
\usepackage[a5paper, margin=10mm, onecolumn]{geometry}
\usepackage{tfrupee} 
\setlength{\headheight}{1cm} 
\setlength{\headsep}{0mm}     

\usepackage{gvv-book}
\usepackage{gvv}
\usepackage{cite}
\usepackage{amsmath,amssymb,amsfonts,amsthm}
\usepackage{algorithmic}
\usepackage{graphicx}
\usepackage{textcomp}
\usepackage{xcolor}
\usepackage{txfonts}
\usepackage{listings}
\usepackage{enumitem}
\usepackage{mathtools}
\usepackage{gensymb}
\usepackage{comment}
\usepackage[breaklinks=true]{hyperref}
\usepackage{tkz-euclide} 
\usepackage{listings}
\def\inputGnumericTable{}                                 
\usepackage[latin1]{inputenc}                                
\usepackage{color}                                            
\usepackage{array}                                            
\usepackage{longtable}                                       
\usepackage{calc}                                             
\usepackage{multirow}                                         
\usepackage{hhline}                                           
\usepackage{ifthen}                                           
\usepackage{lscape}
\usepackage{circuitikz}
\tikzstyle{block} = [rectangle, draw, fill=blue!20, 
    text width=4em, text centered, rounded corners, minimum height=3em]
\tikzstyle{sum} = [draw, fill=blue!10, circle, minimum size=1cm, node distance=1.5cm]
\tikzstyle{input} = [coordinate]
\tikzstyle{output} = [coordinate]
\renewcommand{\thefigure}{\theenumi}
\renewcommand{\thetable}{\theenumi}
\setlength{\intextsep}{10pt} % Space between text and floats
\numberwithin{equation}{enumi}
\numberwithin{figure}{enumi}
\renewcommand{\thetable}{\theenumi}

\begin{document}

\bibliographystyle{IEEEtran}
\vspace{3cm}

\title{2.4.28}
\author{EE25BTECH11032 - Kartik Lahoti}
\maketitle

\subsection*{Question: } 
Find the coordinates of the point $\vec{Q}$ on the x-axis which lies on the perpendicular bisector of the line segment joining the points $\vec{A} \brak{-5,-2} \text{ and } \vec{B} \brak{4,-2}$. Name the type of triangle formed by points $\vec{Q}, \vec{A} \text{ and } \vec{B}$.\\
\solution \\ 

\subsubsection*{Given }
\begin{align}
    \vec{A} = \myvec{-5\\-2} , \vec{B} = \myvec{4\\-2}      
\end{align}

Let $\vec{M}$ be the midpoint of $\vec{AB}$

\begin{align}
    \vec{M} &= \frac{1}{2}\brak{\vec{A} + \vec{B}}\\
     &= \frac{1}{2}\brak{\myvec{-5\\-2} + \myvec{4\\-2}}\\
     &= \myvec{-0.5 \\ -2}
\end{align}

To find the direction vector of perpendicular bisector , we can find the direction vector of $\vec{AB}$ and then rotate it by $90\degree$

Direction Vector of $\vec{AB}$ (represented by $\vec{V_{AB}}$): 
\begin{align}
    \vec{B} - \vec{A} = \vec{V_{AB}} = \myvec{4\\-2} - \myvec{-5\\-2} = \myvec{9 \\ 0 }
\end{align}

Rotation Matrix : 
\begin{align}
    R\brak{\theta} = \myvec{\cos{\theta} & -\sin{\theta} \\ \sin{\theta} & \cos{\theta}}
\end{align}

Direction Vector for perpendicular bisector (represented by $\vec{V}$) : 

\begin{align}
    \vec{V} = R\brak{90 \degree }\vec{V_{AB}} = \myvec{0 & -1 \\ 1 & 0}\myvec{9 \\ 0} = \myvec{0 \\ 9 }
\end{align}

Any arbitrary vector on perpendicular bisector can be given by : 

\begin{align}
    \vec{Q} = \vec{M} + t\vec{V} \text{ where } t \in \mathbb{R}
\end{align}
Finding $\vec{Q}$ , 
\begin{align}
    \vec{Q} = \myvec{-0.5 \\ -2 } + t\myvec{0 \\ 9} 
\end{align}
\begin{align}
    \vec{Q} = \myvec{-0.5 \\ -2 + 9t}
\end{align}
Since y-coordinate of $\vec{Q}$ is zero 
\begin{align}
    \vec{Q} = \myvec{-0.5 \\ 0 }
\end{align}

Since $\vec{Q}$ lies on perpendicular bisector of $\vec{AB}$ , it is equidistant from both $\vec{A}$ and $\vec{B}$
\begin{align}
    \norm{\vec{Q} - \vec{A}} = \norm{\vec{Q} - \vec{B}}
\end{align}
Hence $\triangle ABQ$ is an isosceles triangle.


\begin{figure}[H]
    \centering
    \includegraphics[width=1\columnwidth]{figs/perpbisector1.png}
    \caption*{}
    \label{fig:}
\end{figure}

\end{document}
