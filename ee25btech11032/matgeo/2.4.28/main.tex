\let\negmedspace\undefined
\let\negthickspace\undefined
\documentclass[journal]{IEEEtran}
\usepackage[a5paper, margin=10mm, onecolumn]{geometry}
\usepackage{tfrupee} 
\setlength{\headheight}{1cm} 
\setlength{\headsep}{0mm}     

\usepackage{gvv-book}
\usepackage{gvv}
\usepackage{cite}
\usepackage{amsmath,amssymb,amsfonts,amsthm}
\usepackage{algorithmic}
\usepackage{graphicx}
\usepackage{textcomp}
\usepackage{xcolor}
\usepackage{txfonts}
\usepackage{listings}
\usepackage{enumitem}
\usepackage{mathtools}
\usepackage{gensymb}
\usepackage{comment}
\usepackage[breaklinks=true]{hyperref}
\usepackage{tkz-euclide} 
\usepackage{listings}
\def\inputGnumericTable{}                                 
\usepackage[latin1]{inputenc}                                
\usepackage{color}                                            
\usepackage{array}                                            
\usepackage{longtable}                                       
\usepackage{calc}                                             
\usepackage{multirow}                                         
\usepackage{hhline}                                           
\usepackage{ifthen}                                           
\usepackage{lscape}
\usepackage{circuitikz}
\tikzstyle{block} = [rectangle, draw, fill=blue!20, 
    text width=4em, text centered, rounded corners, minimum height=3em]
\tikzstyle{sum} = [draw, fill=blue!10, circle, minimum size=1cm, node distance=1.5cm]
\tikzstyle{input} = [coordinate]
\tikzstyle{output} = [coordinate]
\renewcommand{\thefigure}{\theenumi}
\renewcommand{\thetable}{\theenumi}
\setlength{\intextsep}{10pt} % Space between text and floats
\numberwithin{equation}{enumi}
\numberwithin{figure}{enumi}
\renewcommand{\thetable}{\theenumi}

\begin{document}

\bibliographystyle{IEEEtran}
\vspace{3cm}

\title{2.4.28}
\author{EE25BTECH11032 - Kartik Lahoti}
\maketitle

\subsection*{Question: } 
Find the coordinates of the point $\vec{Q}$ on the $x$-axis which lies on the perpendicular bisector of the line segment joining the points $\vec{A} \brak{-5,-2} \text{ and } \vec{B} \brak{4,-2}$. Name the type of triangle formed by points $\vec{Q}, \vec{A} \text{ and } \vec{B}$.\\
\solution \\ 

\begin{table}[H]
\centering
\begin{center}
    \begin{tabular}{|c|c|} 
        \hline
            \textbf{Variable}  & \textbf{Formula} \\ 
        \hline
            $a$   & $a = \myvec{4 \\ -1 \\ 1}$ \\ 
        \hline
            $b$   &  $b = \myvec{2 \\ -2 \\ 1}$\\ 
        \hline
           \end{tabular}
\end{center}  

\caption*{Table:2.4.28}
\label{Table:2.4.28}	
\end{table}
%
  If $\vec{Q}$ lies on the  $x$-axis and on the perpendicular bisector of the points $\vec{A}$ and $\vec{B}$, i.e $\vec{Q}$ is equidistant from points $\vec{A}$ and $\vec{B}$
\begin{align}
 \norm{\vec{Q}-\vec{A}} &=
\norm{\vec{Q}-\vec{B}} 
\\
 \implies \norm{\vec{Q}-\vec{A}}^2 &=
\norm{\vec{Q}-\vec{B}}^2 
\\
 \implies \norm{\vec{Q}}^2-2{\vec{Q}}^{\top}\vec{A} + \norm{\vec{A}}^2
	&= \norm{\vec{Q}}^2-2{\vec{Q}}^{\top}\vec{B} + \norm{\vec{B}}^2,
\end{align}
which can be simplified to obtain, 
  \begin{align}
	  \brak{\vec{A}-\vec{B}}^\top   \vec{Q}&=\frac{\norm{\vec{A}}^2 -\norm{\vec{B}}^2 }{2}.
  \end{align}
  \begin{align}
  \because \vec{Q} &= x\vec{e}_1,
  \end{align}
  \begin{align}
   x &=\frac{\norm{\vec{A}}^2 -\norm{\vec{B}}^2 }{2\brak{\vec{A}-\vec{B}}^{\top }\vec{e}_1.}
  \end{align}
  \begin{align}
      \norm{\vec{A}}^2 = 29 , \norm{\vec{B}}^2 = 20   
  \end{align}
    \begin{align}
        \brak{\vec{A}-\vec{B}}^{\top} = \myvec{-9 & 0 } , \vec{e}_1 = \myvec{1 \\ 0} 
    \end{align}
    Substituting from $\brak{0.7}$ and $\brak{0.8}$,
 $x =  -0.5$.  Thus, 
		\begin{align}
        \vec{Q} = \myvec{-0.5 \\ 0}.
		\end{align}
		
Since $\vec{Q}$ lies on perpendicular bisector of $\vec{AB}$ , it is equidistant from both $\vec{A}$ and $\vec{B}$
\begin{align}
    \norm{\vec{Q} - \vec{A}} = \norm{\vec{Q} - \vec{B}}
\end{align}
Hence $\triangle ABQ$ is an isosceles triangle.

See
\figref{fig:2.4.28}.

\begin{figure}[H]
    \centering
    \includegraphics[width=1\columnwidth]{figs/perpbisector1.png}
    \caption{}
    \label{fig:2.4.28}
\end{figure}

\end{document}
