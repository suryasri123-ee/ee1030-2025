\documentclass{beamer}
\usepackage[utf8]{inputenc}

\usetheme{Madrid}
\usecolortheme{default}
\usepackage{amsmath,amssymb,amsfonts,amsthm}
\usepackage{txfonts}
\usepackage{tkz-euclide}
\usepackage{listings}
\usepackage{adjustbox}
\usepackage{array}
\usepackage{tabularx}
\usepackage{gvv}
\usepackage{lmodern}
\usepackage{circuitikz}
\usepackage{tikz}
\usepackage{graphicx}
\usepackage{gensymb}

\setbeamertemplate{page number in head/foot}[totalframenumber]

\usepackage{tcolorbox}
\tcbuselibrary{minted,breakable,xparse,skins}

\definecolor{bg}{gray}{0.95}
\DeclareTCBListing{mintedbox}{O{}m!O{}}{%
  breakable=true,
  listing engine=minted,
  listing only,
  minted language=#2,
  minted style=default,
  minted options={%
    linenos,
    gobble=0,
    breaklines=true,
    breakafter=,,
    fontsize=\small,
    numbersep=8pt,
    #1},
  boxsep=0pt,
  left skip=0pt,
  right skip=0pt,
  left=25pt,
  right=0pt,
  top=3pt,
  bottom=3pt,
  arc=5pt,
  leftrule=0pt,
  rightrule=0pt,
  bottomrule=2pt,
  toprule=2pt,
  colback=bg,
  colframe=orange!70,
  enhanced,
  overlay={%
    \begin{tcbclipinterior}
    \fill[orange!20!white] (frame.south west) rectangle ([xshift=20pt]frame.north west);
    \end{tcbclipinterior}},
  #3,
}
\lstset{
    language=C,
    basicstyle=\ttfamily\small,
    keywordstyle=\color{blue},
    stringstyle=\color{orange},
    commentstyle=\color{green!60!black},
    numbers=left,
    numberstyle=\tiny\color{gray},
    breaklines=true,
    showstringspaces=false,
}
%------------------------------------------------------------
%This block of code defines the information to appear in the
%Title page
\title %optional
{2.4.28}
\date{August 26,2025}


\author 
{Kartik Lahoti - EE25BTECH11032}



\begin{document}


\frame{\titlepage}
\begin{frame}{Question}
Find the coordinates of the point $\vec{Q}$ on the x-axis which lies on the perpendicular bisector of the line segment joining the points $\vec{A} \brak{-5,-2} \text{ and } \vec{B} \brak{4,-2}$. Name the type of triangle formed by points $\vec{Q}, \vec{A} \text{ and } \vec{B}$.
\end{frame}



\begin{frame}{Theoretical Solution}
Given,

\begin{align}
    \vec{A} = \myvec{-5\\-2} , \vec{B} = \myvec{4\\-2}      
\end{align}

Let $\vec{M}$ be the midpoint of $\vec{AB}$

\end{frame}

\begin{frame}{Theoretical Solution}

\begin{align}
    \vec{M} &= \frac{1}{2}\brak{\vec{A} + \vec{B}}\\
     &= \frac{1}{2}\brak{\myvec{-5\\-2} + \myvec{4\\-2}}\\
     &= \myvec{-0.5 \\ -2}
\end{align}

\end{frame}

\begin{frame}{Theoretical Solution}

To find the direction vector of perpendicular bisector , we can find the direction vector of $\vec{AB}$ and then rotate it by $90\degree$

Direction Vector of $\vec{AB}$ (represented by $\vec{V_{AB}}$): 
\begin{align}
    \vec{B} - \vec{A} = \vec{V_{AB}} = \myvec{4\\-2} - \myvec{-5\\-2} = \myvec{9 \\ 0 }
\end{align}

Rotation Matrix : 
\begin{align}
    R\brak{\theta} = \myvec{\cos{\theta} & -\sin{\theta} \\ \sin{\theta} & \cos{\theta}}
\end{align}
\end{frame}

\begin{frame}{Theoretical Solution}
Direction Vector for perpendicular bisector (represented by $\vec{V}$) : 

\begin{align}
    \vec{V} = R\brak{90 \degree } \vec{V_{AB}} = \myvec{0 & -1 \\ 1 & 0}\myvec{9 \\ 0} = \myvec{0 \\ 9 }
\end{align}

Any arbitrary vector on perpendicular bisector can be given by : 

\begin{align}
    \vec{Q} = \vec{M} + t\vec{V} \text{ where } t \in \mathbb{R}
\end{align}

\end{frame}

\begin{frame}{Theoretical Solution}
Finding $\vec{Q}$ , 
\begin{align}
    \vec{Q} = \myvec{-0.5 \\ -2 } + t\myvec{0 \\ 9} 
\end{align}
\begin{align}
    \vec{Q} = \myvec{-0.5 \\ -2 + 9t}
\end{align}
Since y-coordinate of $\vec{Q}$ is zero 
\begin{align}
    \vec{Q} = \myvec{-0.5 \\ 0 }
\end{align}   

\end{frame}

\begin{frame}{Theoretical Solution}
Since $\vec{Q}$ lies on perpendicular bisector of $\vec{AB}$ , it is equidistant from both $\vec{A}$ and $\vec{B}$
\begin{align}
    \norm{\vec{Q} - \vec{A}} = \norm{\vec{Q} - \vec{B}}
\end{align}
Hence $\triangle ABQ$ is an isosceles triangle.


\end{frame}

\begin{frame}[fragile]
    \frametitle{C Code (1) - Function to find Mid Point of Two given vectors }

    \begin{lstlisting}
    
#include <math.h>
void midpoint(double *A , double *B , double *M , int m )
{
    
    for ( int i = 0 ; i < m ; i++ )
    {
        M[i] = (A[i]+B[i])/ 2.0 ;
    }
    
}
    \end{lstlisting}
\end{frame}



\begin{frame}[fragile]
    \frametitle{C Code (2) - Function to rotate a Direction Vector by theta$\degree$}
    \begin{lstlisting}

void rotate(double *IN , double *OP , double theta )
{
    theta = M_PI / 180.0 * theta ; // converting to radian
    OP[0] = cos(theta)*IN[0] - sin(theta)*IN[1] ; 
    OP[1] = sin(theta) * IN[0] + cos(theta) * IN[1] ; 
}

    \end{lstlisting}
\end{frame}



\begin{frame}[fragile]
    \frametitle{C Code (3) - Function to generate points on Line }
    \begin{lstlisting}
void linegen(double *X, double *Y , double *A , double *B , int n , int m )
{
    double temp[m] ; 
    for (int i = 0 ; i < m ; i++)
    {
        temp [ i ] = (B[i]- A[i]) /(double) n ; 
    }
    for (int i = 0 ; i <= n ; i++ )
    {
        X[i] = A[0] + temp[0] * i ; 
        Y[i] = A[1] + temp[1] * i ;
    }
}

\end{lstlisting}
\end{frame}

\begin{frame}[fragile]
    \frametitle{Python Code - Using Shared Object}
    \begin{lstlisting}
import ctypes
import numpy as np
import matplotlib.pyplot as plt
handc1 = ctypes.CDLL("./func.so")

handc1.midpoint.argtypes = [
    ctypes.POINTER(ctypes.c_double),
    ctypes.POINTER(ctypes.c_double),
    ctypes.POINTER(ctypes.c_double),
    ctypes.c_int
]

handc1.midpoint.restype = None
A = np.array([[-5],[-2]], dtype=np.float64).reshape(-1,1)
B = np.array([[4],[-2]], dtype=np.float64).reshape(-1,1)
M = np.zeros(2,dtype=np.float64).reshape(-1,1)

\end{lstlisting}
\end{frame}

\begin{frame}[fragile]
    \frametitle{Python Code - Using Shared Object}
    \begin{lstlisting} 
handc1.midpoint (
    A.ctypes.data_as(ctypes.POINTER(ctypes.c_double)),
    B.ctypes.data_as(ctypes.POINTER(ctypes.c_double)),
    M.ctypes.data_as(ctypes.POINTER(ctypes.c_double)),2)
AB = np.array([[9],[0]],dtype=np.float64)
theta = 90
handc1.rotate.argtypes = [
    ctypes.POINTER(ctypes.c_double),
    ctypes.POINTER(ctypes.c_double),
    ctypes.c_double]

handc1.rotate.restype = None
per = np.zeros(2,dtype=np.float64).reshape(-1,1)
handc1.rotate(AB.ctypes.data_as(ctypes.POINTER(ctypes.c_double)),
    per.ctypes.data_as(ctypes.POINTER(ctypes.c_double)),theta)

Q = M + 2 / 9 * per


\end{lstlisting}
\end{frame}

\begin{frame}[fragile]
    \frametitle{Python Code - Using Shared Object}
    \begin{lstlisting}
def line_cre(P: np.ndarray , Q: np.ndarray, str):
    handc2 = ctypes.CDLL("./line_gen.so")

    handc2.linegen.argtypes = [
        ctypes.POINTER(ctypes.c_double),
        ctypes.POINTER(ctypes.c_double),
        ctypes.POINTER(ctypes.c_double),
        ctypes.POINTER(ctypes.c_double),
        ctypes.c_int , ctypes.c_int
    ]

    handc2.linegen.restype = None
\end{lstlisting}
\end{frame}
\begin{frame}[fragile]
    \frametitle{Python Code - Using Shared Object}
    \begin{lstlisting}
n = 200
    X_l = np.zeros(n,dtype=np.float64)
    Y_l = np.zeros(n,dtype=np.float64)

    handc2.linegen (
        X_l.ctypes.data_as(ctypes.POINTER(ctypes.c_double)),
        Y_l.ctypes.data_as(ctypes.POINTER(ctypes.c_double)),
        P.ctypes.data_as(ctypes.POINTER(ctypes.c_double)),
        Q.ctypes.data_as(ctypes.POINTER(ctypes.c_double)),
        n,2
    )
    plt.plot([X_l[0],X_l[-1]],[Y_l[0],Y_l[-1]],str)
    \end{lstlisting}
\end{frame}

\begin{frame}[fragile]
    \frametitle{Python Code - Using Shared Object}
    \begin{lstlisting}
plt.figure()
line_cre(A,B,"g-")
line_cre(Q,M,"r-")

coords = np.block([[A,B,M,Q]])
plt.scatter(coords[0,:],coords[1,:])
vert_labels = ['A','B','M','Q']
#for i , txt in enumerate(vert_labels):
#plt.annotate(txt,(coords[0,i],coords[1,i]),textcoords="offset points", xytext=(0,10),ha='center')

for i, txt in enumerate(vert_labels):
    plt.annotate(f'{txt}\n({coords[0,i]:.1f}, {coords[1,i]:.1f})',
                 (coords[0,i], coords[1,i]),
                 textcoords="offset points",
                 xytext=(20,0),ha='center', va = 'bottom')

\end{lstlisting}
\end{frame}

\begin{frame}[fragile]
    \frametitle{Python Code - Using Shared Object}
    \begin{lstlisting}
plt.xlabel('$x$')
plt.ylabel('$y$')
#plt.legend(loc='best')
plt.grid()

plt.title("Fig:2.4.28")
plt.axis('equal')

plt.savefig("../figs/perpbisector1.png")
plt.show()

#plt.savefig('figs/triangle/ang-bisect.pdf')
#subprocess.run(shlex.split("termux-open figs/triangle/ang-bisect.pdf"))

\end{lstlisting}
\end{frame}

\begin{frame}[fragile]
    \frametitle{Python Code}
    \begin{lstlisting}
import math
import sys 
sys.path.insert(0, '/home/kartik-lahoti/matgeo/codes/CoordGeo')
import numpy as np
import numpy.linalg as LA
import matplotlib.pyplot as plt
import matplotlib.image as mpimg

from line.funcs import *
#from triangle.funcs import *
#from conics.funcs import circ_gen

#if using termux
#import subprocess
#import shlex

\end{lstlisting}
\end{frame}

\begin{frame}[fragile]
    \frametitle{Python Code }
    \begin{lstlisting}
A = np.array([-5,-2]).reshape(-1,1)
B = np.array([4,-2]).reshape(-1,1)
M = (A+B)/2
AB = np.array([9,0]).reshape(-1,1)
theta = 90
theta = np.deg2rad(theta)
x,y = AB
x_1 = np.cos(theta)*x - np.sin(theta)*y
y_1 = np.sin(theta)*x + np.cos(theta)*y
per = np.array([x_1,y_1]).reshape(-1,1)

Q = M +2/9*per

def plot_it(P,Q,str):
    x_l = line_gen_num(P,Q,20)
    plt.plot(x_l[0,:],x_l[1,:] , str )
\end{lstlisting}
\end{frame}

\begin{frame}[fragile]
    \frametitle{Python Code }
    \begin{lstlisting}

plt.figure()
plot_it(A,B,"g-")
plot_it(M,Q,"r-")

coords = np.block([[A,B,M,Q]])
plt.scatter(coords[0,:],coords[1,:])
vert_labels = ['A','B','M','Q']
#for i , txt in enumerate(vert_labels):
 #   plt.annotate(txt,(coords[0,i],coords[1,i]),textcoords="offset points", xytext=(0,10),ha='center')
for i, txt in enumerate(vert_labels):
    plt.annotate(f'{txt}\n({coords[0,i]:.1f}, {coords[1,i]:.1f})',
                 (coords[0,i], coords[1,i]),
                 textcoords="offset points",
                 xytext=(20,0),
                 ha='center',va ='bottom')

\end{lstlisting}
\end{frame}

\begin{frame}[fragile]
    \frametitle{Python Code }
    \begin{lstlisting}
plt.xlabel('$x$')
plt.ylabel('$y$')
#plt.legend(loc='best')
plt.grid()

plt.title("Fig:2.4.28")
plt.axis('equal')

plt.savefig("../figs/perpbisector2.png")
plt.show()

#plt.savefig('figs/triangle/ang-bisect.pdf')
#subprocess.run(shlex.split("termux-open figs/triangle/ang-bisect.pdf"))
    \end{lstlisting}
\end{frame}


\begin{frame}{Plot}
    \centering
    \includegraphics[width=\columnwidth, height=0.8\textheight, keepaspectratio]{figs/perpbisector1.png}   
\end{frame}
\end{document}
