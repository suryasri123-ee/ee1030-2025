\documentclass{beamer}
\usepackage[utf8]{inputenc}

\usetheme{Madrid}
\usecolortheme{default}
\usepackage{amsmath,amssymb,amsfonts,amsthm}
\usepackage{txfonts}
\usepackage{tkz-euclide}
\usepackage{listings}
\usepackage{adjustbox}
\usepackage{array}
\usepackage{tabularx}
\usepackage{gvv}
\usepackage{lmodern}
\usepackage{circuitikz}
\usepackage{tikz}
\usepackage{graphicx}
\usepackage{gensymb}
\usepackage{gensymb}

\setbeamertemplate{page number in head/foot}[totalframenumber]

\usepackage{tcolorbox}
\tcbuselibrary{minted,breakable,xparse,skins}

\definecolor{bg}{gray}{0.95}
\DeclareTCBListing{mintedbox}{O{}m!O{}}{%
  breakable=true,
  listing engine=minted,
  listing only,
  minted language=#2,
  minted style=default,
  minted options={%
    linenos,
    gobble=0,
    breaklines=true,
    breakafter=,,
    fontsize=\small,
    numbersep=8pt,
    #1},
  boxsep=0pt,
  left skip=0pt,
  right skip=0pt,
  left=25pt,
  right=0pt,
  top=3pt,
  bottom=3pt,
  arc=5pt,
  leftrule=0pt,
  rightrule=0pt,
  bottomrule=2pt,
  toprule=2pt,
  colback=bg,
  colframe=orange!70,
  enhanced,
  overlay={%
    \begin{tcbclipinterior}
    \fill[orange!20!white] (frame.south west) rectangle ([xshift=20pt]frame.north west);
    \end{tcbclipinterior}},
  #3,
}
\lstset{
    language=C,
    basicstyle=\ttfamily\small,
    keywordstyle=\color{blue},
    stringstyle=\color{orange},
    commentstyle=\color{green!60!black},
    numbers=left,
    numberstyle=\tiny\color{gray},
    breaklines=true,
    showstringspaces=false,
}
%------------------------------------------------------------
%This block of code defines the information to appear in the
%Title page
\title %optional
{2.4.28}
\date{August 26,2025}


\author 
{Kartik Lahoti - EE25BTECH11032}



\begin{document}


\frame{\titlepage}
\begin{frame}{Question}
Find the coordinates of the point $\vec{Q}$ on the $x$-axis which lies on the perpendicular bisector of the line segment joining the points $\vec{A} \brak{-5,-2} \text{ and } \vec{B} \brak{4,-2}$. Name the type of triangle formed by points $\vec{Q}, \vec{A} \text{ and } \vec{B}$.
\end{frame}



% \begin{frame}{Theoretical Solution}
% \begin{table}[H]
% \centering
% \begin{center}
    \begin{tabular}{|c|c|} 
        \hline
            \textbf{Variable}  & \textbf{Formula} \\ 
        \hline
            $a$   & $a = \myvec{4 \\ -1 \\ 1}$ \\ 
        \hline
            $b$   &  $b = \myvec{2 \\ -2 \\ 1}$\\ 
        \hline
           \end{tabular}
\end{center}  

% \caption*{Table:2.4.28}
% \label{Table:2.4.28}	
% \end{table}
% \end{frame}

\begin{frame}{Theoretical Solution}
 If $\vec{Q}$ lies on the  $x$-axis and on the perpendicular bisector of the points $\vec{A}$ and $\vec{B}$, i.e $\vec{Q}$ is equidistant from points $\vec{A}$ and $\vec{B}$
\begin{align}
 \norm{\vec{Q}-\vec{A}} &=
\norm{\vec{Q}-\vec{B}} 
\\
 \implies \norm{\vec{Q}-\vec{A}}^2 &=
\norm{\vec{Q}-\vec{B}}^2 
\\
 \implies \norm{\vec{Q}}^2-2{\vec{Q}}^{\top}\vec{A} + \norm{\vec{A}}^2
	&= \norm{\vec{Q}}^2-2{\vec{Q}}^{\top}\vec{B} + \norm{\vec{B}}^2,
\end{align}
\end{frame}

\begin{frame}{Theoretical Solution}
which can be simplified to obtain, 
  \begin{align}
	  \brak{\vec{A}-\vec{B}}^\top   \vec{Q}&=\frac{\norm{\vec{A}}^2 -\norm{\vec{B}}^2 }{2}.
  \end{align}
  \begin{align}
  \because \vec{Q} &= x\vec{e}_1,
  \end{align}
  \begin{align}
   x &=\frac{\norm{\vec{A}}^2 -\norm{\vec{B}}^2 }{2\brak{\vec{A}-\vec{B}}^{\top }\vec{e}_1.}
  \end{align} 
  
\end{frame}

\begin{frame}{Theoretical Solution}
\begin{align}
      \norm{\vec{A}}^2 = 29 , \norm{\vec{B}}^2 = 20   
  \end{align}
    \begin{align}
        \brak{\vec{A}-\vec{B}}^{\top} = \myvec{-9 & 0 } , \vec{e}_1 = \myvec{1 \\ 0} 
    \end{align}
    Substituting from $\brak{0.7}$ and $\brak{0.8}$,
 $x =  -0.5$.  Thus, 
		\begin{align}
        \vec{Q} = \myvec{-0.5 \\ 0}.
		\end{align}
		
Since $\vec{Q}$ lies on perpendicular bisector of $\vec{AB}$ , it is equidistant from both $\vec{A}$ and $\vec{B}$
\begin{align}
    \norm{\vec{Q} - \vec{A}} = \norm{\vec{Q} - \vec{B}}
\end{align}
Hence $\triangle ABQ$ is an isosceles triangle.
\end{frame}

\begin{frame}[fragile]
    \frametitle{C Code (1) - Function to find norm square }
    \begin{lstlisting}  
#include <math.h>
double norm_vec_sq(double *A , int m )
{
    double sum = 0.0; 
    for ( int i = 0 ; i < m ; i++ )
    {
        sum += pow(A[i] , 2 );
    }
    return sum; 
}
    \end{lstlisting}
\end{frame}



\begin{frame}[fragile]
    \frametitle{C Code (2) - Function to $x$}
    \begin{lstlisting}

double x_cal(double *A , double *B , double *E , double na , double nb  )
{
    double x , k ; 
    k = (A[0]-B[0])*E[0] + (A[1]-B[1])*E[1];  
    x = (na - nb) / (2*k);
    return x; 
}
    \end{lstlisting}
\end{frame}



\begin{frame}[fragile]
    \frametitle{C Code (3) - Function to generate points on Line }
    \begin{lstlisting}
void linegen(double *X, double *Y , double *A , double *B , int n , int m )
{
    double temp[m] ; 
    for (int i = 0 ; i < m ; i++)
    {
        temp [ i ] = (B[i]- A[i]) /(double) n ; 
    }
    for (int i = 0 ; i <= n ; i++ )
    {
        X[i] = A[0] + temp[0] * i ; 
        Y[i] = A[1] + temp[1] * i ;
    }
}

\end{lstlisting}
\end{frame}

\begin{frame}[fragile]
    \frametitle{Python Code - Using Shared Object}
    \begin{lstlisting}
import ctypes
import numpy as np
import matplotlib.pyplot as plt
handc1 = ctypes.CDLL("./func.so")

def norm_cal ( A: np.ndarray , m) :
    handc1.norm_vec_sq.argtypes = [
        ctypes.POINTER(ctypes.c_double),
        ctypes.c_int
    ]

    handc1.norm_vec_sq.restype = ctypes.c_double

    return handc1.norm_vec_sq (
        A.ctypes.data_as(ctypes.POINTER(ctypes.c_double)),m
    )
\end{lstlisting}
\end{frame}

\begin{frame}[fragile]
    \frametitle{Python Code - Using Shared Object}
    \begin{lstlisting} 
A = np.array([[-5],[-2]], dtype=np.float64).reshape(-1,1)
B = np.array([[4],[-2]], dtype=np.float64).reshape(-1,1)
n1 = norm_cal(A,2)
n2 = norm_cal(B,2)

handc1.x_cal.argtypes = [
    ctypes.POINTER(ctypes.c_double),
    ctypes.POINTER(ctypes.c_double),
    ctypes.POINTER(ctypes.c_double),
    ctypes.c_double,
    ctypes.c_double
    ]

\end{lstlisting}
\end{frame}
\begin{frame}[fragile]
    \frametitle{Python Code - Using Shared Object}
    \begin{lstlisting}
e = np.array([[1],[0]],dtype=np.float64).reshape(-1,1)
handc1.x_cal.restype = ctypes.c_double

x = handc1.x_cal(
    A.ctypes.data_as(ctypes.POINTER(ctypes.c_double)),
    B.ctypes.data_as(ctypes.POINTER(ctypes.c_double)),
    e.ctypes.data_as(ctypes.POINTER(ctypes.c_double)),
    n1,n2)
Q = np.array([[x],[0]], dtype=np.float64).reshape(-1,1)
\end{lstlisting}
\end{frame}
\begin{frame}[fragile]
    \frametitle{Python Code - Using Shared Object}
    \begin{lstlisting}
def line_cre(P: np.ndarray , Q: np.ndarray, str):
    handc2 = ctypes.CDLL("./line_gen.so")

    handc2.linegen.argtypes = [
        ctypes.POINTER(ctypes.c_double),
        ctypes.POINTER(ctypes.c_double),
        ctypes.POINTER(ctypes.c_double),
        ctypes.POINTER(ctypes.c_double),
        ctypes.c_int , ctypes.c_int
    ]

    handc2.linegen.restype = None
\end{lstlisting}
\end{frame}
\begin{frame}[fragile]
    \frametitle{Python Code - Using Shared Object}
    \begin{lstlisting}
n = 200
    X_l = np.zeros(n,dtype=np.float64)
    Y_l = np.zeros(n,dtype=np.float64)

    handc2.linegen (
        X_l.ctypes.data_as(ctypes.POINTER(ctypes.c_double)),
        Y_l.ctypes.data_as(ctypes.POINTER(ctypes.c_double)),
        P.ctypes.data_as(ctypes.POINTER(ctypes.c_double)),
        Q.ctypes.data_as(ctypes.POINTER(ctypes.c_double)),
        n,2
    )
    plt.plot([X_l[0],X_l[-1]],[Y_l[0],Y_l[-1]],str)
    \end{lstlisting}
\end{frame}

\begin{frame}[fragile]
    \frametitle{Python Code - Using Shared Object}
    \begin{lstlisting}
plt.figure()
line_cre(A,B,"g-")
line_cre(Q,(A+B)/2,"r-")

coords = np.block([[A,B,Q]])
plt.scatter(coords[0,:],coords[1,:])
vert_labels = ['A','B','Q']
#for i , txt in enumerate(vert_labels):
#plt.annotate(txt,(coords[0,i],coords[1,i]),textcoords="offset points", xytext=(0,10),ha='center')

for i, txt in enumerate(vert_labels):
    plt.annotate(f'{txt}\n({coords[0,i]:.1f}, {coords[1,i]:.1f})',
                 (coords[0,i], coords[1,i]),
                 textcoords="offset points",
                 xytext=(20,0),ha='center', va = 'bottom')

\end{lstlisting}
\end{frame}

\begin{frame}[fragile]
    \frametitle{Python Code - Using Shared Object}
    \begin{lstlisting}
plt.xlabel('$x$')
plt.ylabel('$y$')
#plt.legend(loc='best')
plt.grid()

plt.title("Fig:2.4.28")
plt.axis('equal')

plt.savefig("../figs/perpbisector1.png")
plt.show()

#plt.savefig('figs/triangle/ang-bisect.pdf')
#subprocess.run(shlex.split("termux-open figs/triangle/ang-bisect.pdf"))

\end{lstlisting}
\end{frame}

\begin{frame}[fragile]
    \frametitle{Python Code}
    \begin{lstlisting}
import math
import sys 
sys.path.insert(0, '/home/kartik-lahoti/matgeo/codes/CoordGeo')
import numpy as np
import numpy.linalg as LA
import matplotlib.pyplot as plt
import matplotlib.image as mpimg

from line.funcs import *
#from triangle.funcs import *
#from conics.funcs import circ_gen

#if using termux
#import subprocess
#import shlex

\end{lstlisting}
\end{frame}

\begin{frame}[fragile]
    \frametitle{Python Code }
    \begin{lstlisting}

A = np.array([-5,-2]).reshape(-1,1)
B = np.array([4,-2]).reshape(-1,1)
e1 = np.array([1,0]).reshape(-1,1)

x = (LA.norm(A)**2 - LA.norm(B)**2)/(2*np.dot((A-B).T,e1))
#x = float(x)
x = np.squeeze(x)
Q = np.array([[x],[0]],dtype=np.float64).reshape(-1,1)

def plot_it(P,Q,str):
    x_l = line_gen_num(P,Q,20)
    plt.plot(x_l[0,:],x_l[1,:] , str )
\end{lstlisting}
\end{frame}

\begin{frame}[fragile]
    \frametitle{Python Code }
    \begin{lstlisting}

plt.figure()
plot_it(A,B,"g-")
plot_it((A+B)/2,Q,"r-")

coords = np.block([[A,B,Q]])
plt.scatter(coords[0,:],coords[1,:])
vert_labels = ['A','B','Q']
#for i , txt in enumerate(vert_labels):
 #   plt.annotate(txt,(coords[0,i],coords[1,i]),textcoords="offset points", xytext=(0,10),ha='center')
for i, txt in enumerate(vert_labels):
    plt.annotate(f'{txt}\n({coords[0,i]:.1f}, {coords[1,i]:.1f})',
                 (coords[0,i], coords[1,i]),
                 textcoords="offset points",
                 xytext=(20,0),
                 ha='center',va ='bottom')

\end{lstlisting}
\end{frame}

\begin{frame}[fragile]
    \frametitle{Python Code }
    \begin{lstlisting}
plt.xlabel('$x$')
plt.ylabel('$y$')
#plt.legend(loc='best')
plt.grid()

plt.title("Fig:2.4.28")
plt.axis('equal')

plt.savefig("../figs/perpbisector2.png")
plt.show()

#plt.savefig('figs/triangle/ang-bisect.pdf')
#subprocess.run(shlex.split("termux-open figs/triangle/ang-bisect.pdf"))
    \end{lstlisting}
\end{frame}


\begin{frame}{Plot}
    \centering
    \includegraphics[width=\columnwidth, height=0.8\textheight, keepaspectratio]{figs/perpbisector1.png}   
\end{frame}
\end{document}
