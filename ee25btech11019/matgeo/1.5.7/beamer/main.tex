\documentclass{beamer}

% Theme
\usetheme{Madrid}
\usecolortheme{default}

% Packages
\usepackage{amsmath,amssymb,amsfonts}
\usepackage{graphicx}
\usepackage{xcolor}
\usepackage{tikz}
\usepackage{tkz-euclide}
\usepackage{array}
\usepackage{multirow}
\usepackage{longtable}
\usepackage{lscape}

% Custom macros
\newcommand{\myvec}[1]{\begin{pmatrix}#1\end{pmatrix}}
\newcommand{\brak}[1]{\left( #1 \right)}

% Redefine \vec to bold letters only (no arrow)
\renewcommand{\vec}[1]{\mathbf{#1}}

\title{1.5.7}
\author{EE25BTECH11019 -- Darji Vivek M.}
\date{}

\begin{document}

% Title slide
\begin{frame}
  \titlepage
\end{frame}

% Question
\begin{frame}{Question}
If $\brak{\tfrac{a}{3},4}$ is the midpoint of the line segment joining 
the points $\brak{-6,5}$ and $\brak{-2,3}$, then the value of $a$ is
\hfill $\brak{10,2021}$
\end{frame}

% Variables table
\begin{frame}{Variables Used}
\begin{table}[h!]    
  \centering
  \begin{tabular}{|c|c|}
    \hline
    Symbol & Meaning \\ \hline
    $\vec{A}$ & Point $(-6,5)$ \\ \hline
    $\vec{B}$ & Point $(-2,3)$ \\ \hline
    $\vec{M}$ & Midpoint \\ \hline
    $a$ & Unknown to find \\ \hline
  \end{tabular}
\end{table}
\end{frame}

% Midpoint formula
\begin{frame}{Midpoint Formula}
\begin{align}
\vec{A} &= \myvec{-6\\5}, \quad 
\vec{B} = \myvec{-2\\3} \\[6pt]
\vec{M} &= \frac{\vec{A}+\vec{B}}{2}
\end{align}
\end{frame}

% Substitution
\begin{frame}{Substitution}
\begin{align}
\vec{M} &= \tfrac{1}{2}\brak{\myvec{-6\\5}+\myvec{-2\\3}} \\[6pt]
&= \tfrac{1}{2}\myvec{-8\\8} \\[6pt]
&= \myvec{-4\\4}
\end{align}
\end{frame}

% Given midpoint
\begin{frame}{Given Midpoint}
\begin{align}
\vec{M} = \myvec{\tfrac{a}{3}\\4}
\end{align}

Equating components:
\begin{align}
\tfrac{a}{3} = -4 \implies a = -12
\end{align}
\end{frame}

% Final Answer
\begin{frame}{Final Answer}
\[
\boxed{a = -12}
\]
\end{frame}
\begin{frame}[fragile]{C Code}
\begin{verbatim}
#include <stdio.h>

double find_a(int x1, int y1, int x2, int y2, int given_y)
{
    double mid_x = (x1 + x2) / 2.0;
    // given midpoint is (a/4, given_y)
    return 4 * mid_x;
}
\end{verbatim}
\end{frame}
\begin{frame}[fragile]{Python Code (Import and Setup)}
\begin{verbatim}
import ctypes
import numpy as np
import matplotlib.pyplot as plt

# Load the shared object file (compiled from your C code)
lib = ctypes.CDLL('./lib1.so')  # change name if your .so file is different

# Define argument and return types for the C function
lib.find_a.argtypes = [ctypes.c_int, ctypes.c_int, 
                       ctypes.c_int, ctypes.c_int, 
                       ctypes.c_int]
lib.find_a.restype = ctypes.c_double
\end{verbatim}
\end{frame}
\begin{frame}[fragile]{Python Code (Calling C Function)}
\begin{verbatim}
# Given data
x1, y1 = -6, 5
x2, y2 = -2, 3
given_y = 4

# Call the C function
a_value = lib.find_a(x1, y1, x2, y2, given_y)
print(f"Value of a: {a_value}")

# Midpoint coordinates from a
mid_x = a_value / 4
mid_y = given_y
\end{verbatim}
\end{frame}
\begin{frame}[fragile]{Python Code (Preparing Data)}
\begin{verbatim}
# Create numpy arrays for plotting
A = np.array([x1, y1])
B = np.array([x2, y2])
M = np.array([mid_x, mid_y])

# Generate line between A and B
line_AB = np.column_stack((A, B))
\end{verbatim}
\end{frame}
\begin{frame}[fragile]{Python Code (Plotting)}
\begin{verbatim}
# Plotting
plt.plot([A[0], B[0]], [A[1], B[1]], label='$AB$')
plt.scatter([A[0], B[0], M[0]], 
            [A[1], B[1], M[1]], 
            color=['red', 'blue', 'green'])

# Annotate points
labels = ['A', 'B', 'Midpoint']
coords = [A, B, M]
for label, coord in zip(labels, coords):
    plt.annotate(f'{label}\n({coord[0]:.2f}, {coord[1]:.2f})',
                 (coord[0], coord[1]),
                 textcoords="offset points",
                 xytext=(10, -10),
                 ha='center')
\end{verbatim}
\end{frame}
\begin{frame}[fragile]{Python Code (Finalizing The Plot)}
\begin{verbatim}
plt.legend()
plt.grid(True)
plt.axis('equal')
plt.savefig('1.png')
plt.show()
\end{verbatim}
\end{frame}

\begin{frame}{Plot}
\begin{figure}[H]
\centering
\includegraphics[width=0.75\columnwidth]{figs/1.png}
\caption{\centering plot}
\label{fig:placeholder_125}
\end{figure}
\end{frame}

\end{document}