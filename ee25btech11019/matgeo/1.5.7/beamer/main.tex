\documentclass{beamer}

% Theme
\usetheme{Madrid}
\usecolortheme{default}

% Packages
\usepackage{amsmath,amssymb,amsfonts}
\usepackage{graphicx}
\usepackage{xcolor}
\usepackage{tikz}
\usepackage{tkz-euclide}
\usepackage{array}
\usepackage{multirow}
\usepackage{longtable}
\usepackage{lscape}

% Custom macros
\newcommand{\myvec}[1]{\begin{pmatrix}#1\end{pmatrix}}
\newcommand{\brak}[1]{\left( #1 \right)}

% Redefine \vec to bold letters only (no arrow)
\renewcommand{\vec}[1]{\mathbf{#1}}

\title{1.5.7}
\author{EE25BTECH11019 -- Darji Vivek M.}
\date{}

\begin{document}

% Title slide
\begin{frame}
  \titlepage
\end{frame}

% Question
\begin{frame}{Question}
If $\brak{\tfrac{a}{3},4}$ is the midpoint of the line segment joining 
the points $\brak{-6,5}$ and $\brak{-2,3}$, then the value of $a$ is
\hfill $\brak{10,2021}$
\end{frame}

% Variables table
\begin{frame}{Variables Used}
\begin{table}[h!]    
  \centering
  \begin{tabular}{|c|c|}
    \hline
    Symbol & Meaning \\ \hline
    $\vec{A}$ & Point $(-6,5)$ \\ \hline
    $\vec{B}$ & Point $(-2,3)$ \\ \hline
    $\vec{M}$ & Midpoint \\ \hline
    $a$ & Unknown to find \\ \hline
  \end{tabular}
\end{table}
\end{frame}

% Midpoint formula
\begin{frame}{Midpoint Formula}
\begin{align}
\vec{A} &= \myvec{-6\\5}, \quad 
\vec{B} = \myvec{-2\\3} \\[6pt]
\vec{M} &= \frac{\vec{A}+\vec{B}}{2}
\end{align}
\end{frame}

% Substitution
\begin{frame}{Substitution}
\begin{align}
\vec{M} &= \tfrac{1}{2}\brak{\myvec{-6\\5}+\myvec{-2\\3}} \\[6pt]
&= \tfrac{1}{2}\myvec{-8\\8} \\[6pt]
&= \myvec{-4\\4}
\end{align}
\end{frame}

% Given midpoint
\begin{frame}{Given Midpoint}
\begin{align}
\vec{M} = \myvec{\tfrac{a}{3}\\4}
\end{align}

Equating components:
\begin{align}
\tfrac{a}{3} = -4 \implies a = -12
\end{align}
\end{frame}

% Final Answer
\begin{frame}{Final Answer}
\[
\boxed{a = -12}
\]
\end{frame}
\begin{figure}[H]
\centering
\includegraphics[width=0.75\columnwidth]{figs/1.png}
\caption{\centering plot}
\label{fig:placeholder_125}
\end{figure}

\end{document}