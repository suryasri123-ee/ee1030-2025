\let\negmedspace\undefined
\let\negthickspace\undefined
\documentclass[journal]{IEEEtran}
\usepackage[a5paper, margin=10mm, onecolumn]{geometry}
%\usepackage{lmodern} % Ensure lmodern is loaded for pdflatex
\usepackage{tfrupee} % Include tfrupee package

\setlength{\headheight}{1cm} % Set the height of the header box
\setlength{\headsep}{0mm}     % Set the distance between the header box and the top of the text

\usepackage{gvv-book}
%\usepackage{gvv}
\usepackage{cite}
\usepackage{amsmath,amssymb,amsfonts,amsthm}
\usepackage{algorithmic}
\usepackage{graphicx}
\usepackage{textcomp}
\usepackage{xcolor}
\usepackage{txfonts}
\usepackage{listings}
\usepackage{enumitem}
\usepackage{mathtools}
\usepackage{gensymb}
\usepackage{comment}
\usepackage[breaklinks=true]{hyperref}
\usepackage{tkz-euclide} 
\usepackage{listings}
\usepackage{gvv}                                        
\def\inputGnumericTable{}                                 
\usepackage[latin1]{inputenc}                                
\usepackage{color}                                            
\usepackage{array}                                            
\usepackage{longtable}                                       
\usepackage{calc}                                             
\usepackage{multirow}                                         
\usepackage{hhline}                                           
\usepackage{ifthen}                                           
\usepackage{lscape}
\begin{document}

\bibliographystyle{IEEEtran}

\title{1.5.7}
\author{EE25BTECH11019 - Darji Vivek M.}
% \maketitle
% \newpage
% \bigskip
{\let\newpage\relax\maketitle}

\renewcommand{\thefigure}{\theenumi}
\renewcommand{\thetable}{\theenumi}
\setlength{\intextsep}{10pt} % Space between text and floats


%\numberwithin{equation}{enumi}
\numberwithin{figure}{enumi}
\renewcommand{\thetable}{\theenumi}


\textbf{Question}:\\
If $\brak{\frac{a}{3},4}$ is the midpoint of the line segment joining the points $\brak{-6,5}$ and $\brak{-2,3}$, then the value of $a$ is \hfill $\brak{10,2021}$
\\
\solution \\
\begin{table}[h!]    
  \centering
  \begin{center}
    \begin{tabular}{|c|c|} 
        \hline
            \textbf{Variable}  & \textbf{Formula} \\ 
        \hline
            $a$   & $a = \myvec{4 \\ -1 \\ 1}$ \\ 
        \hline
            $b$   &  $b = \myvec{2 \\ -2 \\ 1}$\\ 
        \hline
           \end{tabular}
\end{center}  

  \caption{Variables Used}
  \label{tab10.5.7.1}
\end{table}

\begin{align}
\vec{A} &= \myvec{-6\\5}, \quad 
\vec{B} = \myvec{-2\\3}
\end{align}

The midpoint formula is
\begin{align}
\vec{M} = \frac{\vec{A}+\vec{B}}{2}
\end{align}

Substituting values,
\begin{align}
\vec{M} &= \frac{1}{2}\brak{\myvec{-6\\5}+\myvec{-2\\3}}
= \frac{1}{2}\myvec{-8\\8} 
= \myvec{-4\\4}
\end{align}

But given midpoint is
\begin{align}
\vec{M} = \myvec{\tfrac{a}{3}\\4}
\end{align}

Equating first components,
\begin{align}
\frac{a}{3} = -4 \implies a = -12
\end{align}

Hence, the value of $a$ is $\boxed{-12}$ 
\begin{figure}[H]
\centering
\includegraphics[width=0.75\columnwidth]{figs/1.png}
\caption{\centering plot}
\label{fig:placeholder_125}
\end{figure}
\end{document}