\let\negmedspace\undefined
\let\negthickspace\undefined
\documentclass[journal]{IEEEtran}
\usepackage[a5paper, margin=10mm, onecolumn]{geometry}

\usepackage{tfrupee} 
\setlength{\headheight}{1cm} 
\setlength{\headsep}{0mm}       

\usepackage{gvv-book}
\usepackage{gvv}
\usepackage{cite}
\usepackage{amsmath,amssymb,amsfonts,amsthm}
\usepackage{algorithmic}
\usepackage{graphicx}
\usepackage{textcomp}
\usepackage{xcolor}
\usepackage{txfonts}
\usepackage{listings}
\usepackage{enumitem}
\usepackage{mathtools}
\usepackage{gensymb}
\usepackage{comment}
\usepackage[breaklinks=true]{hyperref}
\usepackage{tkz-euclide} 
\usepackage{listings}

\def\inputGnumericTable{}                                 
\usepackage[latin1]{inputenc}                                
\usepackage{color}                                            
\usepackage{array}                                            
\usepackage{longtable}                                       
\usepackage{calc}                                             
\usepackage{multirow}                                         
\usepackage{hhline}                                           
\usepackage{ifthen}                                           
\usepackage{lscape}
\begin{document}
	
	\bibliographystyle{IEEEtran}
	\vspace{3cm}
	
	\title{4.3.43}
	\author{EE25BTECH11052 - Shriyansh Kalpesh Chawda}
	
	{\let\newpage\relax\maketitle}
	
	\renewcommand{\thefigure}{\theenumi}
	\renewcommand{\thetable}{\theenumi}
	\setlength{\intextsep}{10pt} 
	
	\numberwithin{equation}{enumi}
	\numberwithin{figure}{enumi}
	\renewcommand{\thetable}{\theenumi}
	\textbf{Question}:\\
	Find the equation of the line that passes through the points (3, -2, -5), (3, -2, 6).
	\\
	\solution

\[
\textbf{Problem.}\;
\text{Find the equation of the line through }P_1(3,-2,-5)\text{ and }P_2(3,-2,6).
\]

\[
\textbf{Solution.}
\]

\[
P_1=\begin{pmatrix}3\\-2\\-5\end{pmatrix},\quad
P_2=\begin{pmatrix}3\\-2\\ 6\end{pmatrix},\quad
\vec d=P_2-P_1=\begin{pmatrix}0\\0\\11\end{pmatrix}\sim\begin{pmatrix}0\\0\\1\end{pmatrix}.
\]
Hence a parametric (affine) description of the line is :
\[
\mathbf{x}=\mathbf{p}+\lambda\vec d
=\begin{pmatrix}3\\-2\\-5\end{pmatrix}
+\lambda\begin{pmatrix}0\\0\\1\end{pmatrix},\qquad \lambda\in\mathbb{R}.
\]
\begin{figure}[H]
	\centering
	\includegraphics[width=0.6\linewidth]{Figs/line3d}
	\caption{}
	\label{fig:line3d}
\end{figure}




\end{document}