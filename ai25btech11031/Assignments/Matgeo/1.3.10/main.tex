\let\negmedspace\undefined
\let\negthickspace\undefined
\documentclass[journal]{IEEEtran}
\usepackage[a5paper, margin=10mm, onecolumn]{geometry}
%\usepackage{lmodern} % Ensure lmodern is loaded for pdflatex
\usepackage{tfrupee} % Include tfrupee package

\setlength{\headheight}{1cm} % Set the height of the header box
\setlength{\headsep}{0mm}     % Set the distance between the header box and the top of the text

\usepackage{gvv-book}
\usepackage{gvv}
\usepackage{cite}
\usepackage{amsmath,amssymb,amsfonts,amsthm}
\usepackage{algorithmic}
\usepackage{graphicx}
\usepackage{textcomp}
\usepackage{xcolor}
\usepackage{txfonts}
\usepackage{listings}
\usepackage{enumitem}
\usepackage{mathtools}
\usepackage{gensymb}
\usepackage{comment}
\usepackage[breaklinks=true]{hyperref}
\usepackage{tkz-euclide} 
\usepackage{listings}
% \usepackage{gvv}                                        
\def\inputGnumericTable{}                                 
\usepackage[latin1]{inputenc}                                
\usepackage{color}                                            
\usepackage{array}                                            
\usepackage{longtable}                                       
\usepackage{calc}                                             
\usepackage{multirow}                                         
\usepackage{hhline}                                           
\usepackage{ifthen}                                           
\usepackage{lscape}
\usepackage{circuitikz}
\tikzstyle{block} = [rectangle, draw, fill=blue!20, 
    text width=4em, text centered, rounded corners, minimum height=3em]
\tikzstyle{sum} = [draw, fill=blue!10, circle, minimum size=1cm, node distance=1.5cm]
\tikzstyle{input} = [coordinate]
\tikzstyle{output} = [coordinate]


\begin{document}

\bibliographystyle{IEEEtran}
\vspace{3cm}

\title{1.3.10}
\author{AI25BTECH110031 \\ Shivam Sawarkar}
 \maketitle
% \newpage
% \bigskip
{\let\newpage\relax\maketitle}

\renewcommand{\thefigure}{\theenumi}
\renewcommand{\thetable}{\theenumi}
\setlength{\intextsep}{10pt} % Space between text and floats


\numberwithin{equation}{enumi}
\numberwithin{figure}{enumi}
\renewcommand{\thetable}{\theenumi}

\textbf{Question(1.3.10)}

Find the ratio in which the point $P = (8, y)$ divides the line segment joining
$A = (1, 2)$ and $B = (2, 3)$. Also, find the value of $y$.

\textbf{Solution}:

Let the given points be $A$ and $B$
\begin{align*}
\vec{A} = \myvec{1 \\ 2}, \quad \vec{B} = \myvec{2 \\ 3}
\end{align*}

Let the point $P$ divide the line segment $\overline{AB}$ in the ratio $k:1$.

Given $P = \myvec{8 \\ y}$

The points $A$, $B$, $P$ are collinear.

\begin{align}
\implies \text{rank} \myvec{ \vec{B}-\vec{A} & \vec{P}-\vec{A} }^T = 1
\end{align}

\begin{align}
\vec{B}-\vec{A} &= \myvec{2-1 \\ 3-2} = \myvec{1 \\ 1} \\
\vec{P}-\vec{A} &= \myvec{8-1 \\ y-2} = \myvec{7 \\ y-2}
\end{align}

Therefore, our matrix is:
\begin{align}
\myvec{
1 & 1 \\
7 & y-2
}
\end{align}

Row reducing:
\begin{align}
R_2 \rightarrow R_2 - 7R_1 \implies
\myvec{
1 & 1 \\
0 & y-9
}
\end{align}

For the above matrix to be of rank 1,
\begin{align}
y-9 = 0 \implies y = 9
\end{align}

$\therefore$ The coordinates of the point of division are
\begin{align*}
\vec{P} = \myvec{8 \\ 9}
\end{align*}

\begin{align}
    \vec{P} = \frac{k\vec{B}+\vec{A}}{k+1}
\end{align}

\begin{align}
k &= \frac{(\myvec{A} - \myvec{P})^T (\myvec{P} - \myvec{B})}{\|\myvec{P} - \myvec{B}\|^2}
\end{align}

Substituting the values of $\vec{A}$, $\vec{B}$ and $\vec{P}$,
\begin{align}
k=\frac{\myvec{-7 & -7}\myvec{6 \\ 6}}{\norm{\myvec{6 \\ 6}}^2}=\frac{-7}{6}
\end{align}

Thus, the ratio in which the point $P$ divides the line segment $AB$ is $\boxed{-7:6}$.


\begin{figure}[H]
    \centering
    \includegraphics[width=0.8\columnwidth]{figs/fig1.jpg}
    \caption{}
    \label{fig:placeholder}
\end{figure}




\end{document}