\let\negmedspace\undefined
\let\negthickspace\undefined
\documentclass[journal,12pt,onecolumn]{IEEEtran}
\usepackage{cite}
\usepackage{amsmath,amssymb,amsfonts,amsthm}
\usepackage{algorithmic}
\usepackage{graphicx}
\graphicspath{{./figs/}}
\usepackage{textcomp}
\usepackage{xcolor}
\usepackage{txfonts}
\usepackage{listings}
\usepackage{enumitem}
\usepackage{mathtools}
\usepackage{gensymb}
\usepackage{comment}
\usepackage{caption}
\usepackage[breaklinks=true]{hyperref}
\usepackage{tkz-euclide} 
\usepackage{listings}
\usepackage{gvv}                                        
%\def\inputGnumericTable{}                                 
\usepackage[latin1]{inputenc}     
\usepackage{xparse}
\usepackage{color}                                            
\usepackage{array}                                            
\usepackage{longtable}                                       
\usepackage{calc}                                             
\usepackage{multirow}
\usepackage{multicol}
\usepackage{hhline}                                           
\usepackage{ifthen}                                           
\usepackage{lscape}
\usepackage{tabularx}
\usepackage{array}
\usepackage{float}
%\newtheorem{theorem}{Theorem}[section]
%\newtheorem{theorem}{Theorem}[section]
%\newtheorem{problem}{Problem}
%\newtheorem{proposition}{Proposition}[section]
%\newtheorem{lemma}{Lemma}[section]
%\newtheorem{corollary}[theorem]{Corollary}
%\newtheorem{example}{Example}[section]
%\newtheorem{definition}[problem]{Definition}

\begin{document}

%\textbf{\Large 1.2.1} \\
%\textbf{\large AI25BTECH11001 - Abhisek Mohapatra} \\
\title{2.5.34}
\author{AI25BTECH11001 - ABHISEK MOHAPATRA}
% \maketitle
% \newpage
% \bigskip
%\begin{document}
{\let\newpage\relax\maketitle}
%\renewcommand{\thefigure}{\theenumi}
%\renewcommand{\thetable}{\theenumi}
	 	\textbf{Question}:
		Show that the points (-2,3),(8,3) and (6,7) are the vertices of a right triangle.

		\textbf{Solution:}Let 
		\begin{align}
			\vec{A} = \myvec{-2\\3}, \vec{B} = \myvec{8\\3}, \vec{C} = \myvec{6\\7}
		\end{align}
		
		Finding out the dot product of the vectors representing the sides,
		\begin{align}
			(\vec{A}-\vec{B})^T(\vec{C}-\vec{B}) = \myvec{-10\\0}^T\myvec{-2\\4} = 20 + 0 = 20
		\end{align}
		\begin{align}
			(\vec{A}-\vec{C})^T(\vec{B}-\vec{C}) = \myvec{-8\\-4}^T\myvec{2\\-4} = -16 + 16 = 0
		\end{align}
		\begin{align}
			(\vec{B}-\vec{A})^T(\vec{C}-\vec{A}) = \myvec{10\\0}^T\myvec{8\\4} = 80 + 0 = 80 
		\end{align}

		So, it is a right angle triangle with right angle at $\vec{C}$ as the second statement is zero. 
	
	Graph:
\begin{figure}[h!]
	\centering
	\includegraphics[width=0.7\linewidth]{img.png}
\end{figure}
\end{document}

