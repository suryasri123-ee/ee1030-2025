\let\negmedspace\undefined
\let\negthickspace\undefined
\documentclass[journal]{IEEEtran}
\usepackage[a4paper,margin=10mm, onecolumn]{geometry}
\usepackage{tfrupee}

\setlength{\headheight}{1cm}
\setlength{\headsep}{0mm}

\usepackage{gvv-book}
\usepackage{gvv}
\usepackage{cite}
\usepackage{amsmath,amssymb,amsfonts,amsthm}
\usepackage{algorithmic}
\usepackage{graphicx}
\graphicspath{{./figs/}}
\usepackage{textcomp}
\usepackage{xcolor}
\usepackage{txfonts}
\usepackage{listings}
\usepackage{enumitem}
\usepackage{mathtools}
\usepackage{gensymb}
\usepackage{comment}
\usepackage[breaklinks=true]{hyperref}
\usepackage{tkz-euclide} 
\usepackage{listings}
% \usepackage{gvv}                                        
\def\inputGnumericTable{}                                 
\usepackage[latin1]{inputenc}                                
\usepackage{color}                                            
\usepackage{array}                                            
\usepackage{longtable}                                       
\usepackage{calc}                                             
\usepackage{multirow}                                         
\usepackage{hhline}                                           
\usepackage{ifthen}                                           
\usepackage{lscape}
\usepackage{circuitikz}


\renewcommand{\thefigure}{\theenumi}
\renewcommand{\thetable}{\theenumi}
\setlength{\intextsep}{10pt} % Space between text and floats


\numberwithin{equation}{enumi}
\numberwithin{figure}{enumi}
\renewcommand{\thetable}{\theenumi}


% Marks the beginning of the document
\begin{document}
\bibliographystyle{IEEEtran}
\vspace{3cm}

\title{1.6.16}
\author{AI25BTECH11001 - ABHISEK MOHAPATRA}
% \maketitle
% \newpage
% \bigskip
{\let\newpage\relax\maketitle}
\renewcommand{\thefigure}{\theenumi}
\renewcommand{\thetable}{\theenumi}

		\textbf{Question}:

		\noindent Find the values of $k$ if the points A(k +1,2k), B(3k, 2k +3) and C(5k-1,5k) are collinear.

		\textbf{Solution:} From the given information,


		\begin{align}
			A = \myvec{k+1\\2k},B = \myvec{3k\\2k+3},C = \myvec{5k-1\\5k} 
		\end{align}
		To check if the points are collinear, we can use 
		\begin{align}
			rank\myvec{B-A & C-A} = 1	
		\end{align}
		So,
		\begin{align}
			\myvec{B-A & C-A} = \myvec{2k-1 & 4k-2 \\ 3 & 3k}	
			\\
			\xleftrightarrow[]{C_2 = C_2 - 2C_1} 
			\myvec{2k-1 & 0 \\ 3 & 3k-6} 
		\end{align}
		The rank of the matrix will be 1 when 
		\begin{align}
		3k-6 = 0
		\end{align}
		\begin{align}
		\Rightarrow k = 2
		\end{align}
		Graph:
\begin{figure}[H]
	\centering
     \includegraphics{img}
	\caption*{}
	\label{img}
\end{figure}


		Therefore, k = 2.
\end{document}


