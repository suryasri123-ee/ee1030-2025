\let\negmedspace\undefined
\let\negthickspace\undefined
\documentclass[journal]{IEEEtran}
\usepackage[a5paper, margin=10mm, onecolumn]{geometry}
\usepackage{lmodern} % Ensure lmodern is loaded for pdflatex
\usepackage{tfrupee} % Include tfrupee package

\setlength{\headheight}{1cm} % Set the height of the header box
\setlength{\headsep}{0mm}     % Set the distance between the header box and the top of the text

\usepackage{gvv-book}
\usepackage{gvv}
\usepackage{cite}
\usepackage{amsmath,amssymb,amsfonts,amsthm}
\usepackage{algorithmic}
\usepackage{graphicx}
\usepackage{textcomp}
\usepackage{xcolor}
\usepackage{txfonts}
\usepackage{listings}
\usepackage{enumitem}
\usepackage{mathtools}
\usepackage{gensymb}
\usepackage{comment}
\usepackage[breaklinks=true]{hyperref}
\usepackage{tkz-euclide} 
\usepackage{listings}
 \usepackage{gvv}                                        
\def\inputGnumericTable{}                                 
\usepackage[latin1]{inputenc}                                
\usepackage{color}                                            
\usepackage{array}                                            
\usepackage{longtable}                                       
\usepackage{calc}                                             
\usepackage{multirow}                                         
\usepackage{hhline}                                           
\usepackage{ifthen}                                           
\usepackage{lscape}
\begin{document}

\bibliographystyle{IEEEtran}


\title{1.5.9}
\author{EE25BTECH11021 - Dhanush Sagar
}
% \maketitle
% \newpage
% \bigskip
{\let\newpage\relax\maketitle}

\renewcommand{\thefigure}{\theenumi}
\renewcommand{\thetable}{\theenumi}
\setlength{\intextsep}{10pt} % Space between text and floats


\numberwithin{equation}{enumi}
\numberwithin{figure}{enumi}
\renewcommand{\thetable}{\theenumi}



\textbf{Problem (1.5.9).} Find the ratio in which the Y-axis divides the line segment joining
\[
A=\myvec{5\\-6} \quad \text{and} \quad B=\myvec{-1\\-4}.
\]
Also, find the coordinates of the point of intersection.


\textbf{Solution.}  
Let the intersection point be 
\[
P=\myvec{0\\y}.
\]
Assume \(P\) divides \(\overline{AB}\) internally in the ratio \(k:1\) (so \(AP:PB=k:1\)).  
Using the section formula with vectors,
\[
P=\frac{kB+A}{k+1}.
\]

So,
\[
\myvec{0\\y} 
= \frac{k\myvec{-1\\-4}+\myvec{5\\-6}}{k+1}.
\]

From the first row (x-component),
\[
0=\frac{-k+5}{k+1} \;\;\;\Longrightarrow\;\;\; k=5.
\]

Therefore the Y-axis divides \(\overline{AB}\) in the ratio
\[
AP:PB=5:1.
\]

Now, for the second row (y-component),
\[
y=\frac{5(-4)+(-6)}{6} 
=\frac{-20-6}{6} 
=\frac{-26}{6} 
=-\frac{13}{3}.
\]

Thus,
\[
P=\myvec{0\\[4pt]-\dfrac{13}{3}}.
\]

\textbf{Answer:} The Y-axis divides \(\overline{AB}\) in the ratio \(5:1\) (internally), and the intersection point is
\[
\boxed{\;P=\myvec{0\\[4pt]-\dfrac{13}{3}}\;}
\]

\end{document}


