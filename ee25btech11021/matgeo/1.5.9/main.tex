\let\negmedspace\undefined
\let\negthickspace\undefined
\documentclass[journal]{IEEEtran}
\usepackage[a5paper, margin=10mm, onecolumn]{geometry}
\usepackage{lmodern} % Ensure lmodern is loaded for pdflatex
\usepackage{tfrupee} % Include tfrupee package

\setlength{\headheight}{1cm} % Set the height of the header box
\setlength{\headsep}{0mm}     % Set the distance between the header box and the top of the text

\usepackage{gvv-book}
\usepackage{gvv}
\usepackage{cite}
\usepackage{amsmath,amssymb,amsfonts,amsthm}
\usepackage{algorithmic}
\usepackage{graphicx}
\usepackage{textcomp}
\usepackage{xcolor}
\usepackage{txfonts}
\usepackage{listings}
\usepackage{enumitem}
\usepackage{mathtools}
\usepackage{gensymb}
\usepackage{comment}
\usepackage[breaklinks=true]{hyperref}
\usepackage{tkz-euclide} 
\usepackage{listings}
 \usepackage{gvv}                                        
\def\inputGnumericTable{}                                 
\usepackage[latin1]{inputenc}                                
\usepackage{color}                                            
\usepackage{array}                                            
\usepackage{longtable}                                       
\usepackage{calc}                                             
\usepackage{multirow}                                         
\usepackage{hhline}                                           
\usepackage{ifthen}                                           
\usepackage{lscape}
\begin{document}

\bibliographystyle{IEEEtran}


\title{1.5.9}
\author{EE25BTECH11021 - Dhanush Sagar
}
% \maketitle
% \newpage
% \bigskip
{\let\newpage\relax\maketitle}

\renewcommand{\thefigure}{\theenumi}
\renewcommand{\thetable}{\theenumi}
\setlength{\intextsep}{10pt} % Space between text and floats


\numberwithin{equation}{enumi}
\numberwithin{figure}{enumi}
\renewcommand{\thetable}{\theenumi}




\textbf{Question:}  
Find the ratio in which the $Y$-axis divides the line segment joining the points  
$A = \myvec{5 \\ -6}$ and $B = \myvec{-1 \\ -4}$.  
Also, find the coordinates of the point of intersection.  

\textbf{Solution:}  

The given points are
\begin{align}
A &= \myvec{5 \\ -6},  
B = \myvec{-1 \\ -4}.
\end{align}

Let the point of intersection be
\begin{align}
P &= \myvec{0 \\ y}.
\end{align}

Since $P$ lies on the line segment $AB$, we can write
\begin{align}
P &= \lambda A + \mu B,  \lambda + \mu = 1.
\end{align}

From the $x$-coordinate of $P$,
\begin{align}
\myvec{5 & -1 \\ 1 & 1}\myvec{\lambda \\ \mu} &= \myvec{0 \\ 1}.
\end{align}

This gives
\begin{align}
5\lambda - \mu &= 0, \\
\lambda + \mu &= 1.
\end{align}

Solving,
\begin{align}
\mu &= 5\lambda, \\
6\lambda &= 1, \\
\lambda &= \tfrac{1}{6},  \mu = \tfrac{5}{6}.
\end{align}

Now,
\begin{align}
P &= \tfrac{1}{6}\myvec{5 \\ -6} + \tfrac{5}{6}\myvec{-1 \\ -4}, \\
P &= \myvec{\tfrac{5-5}{6} \\ \tfrac{-6-20}{6}}, \\
P &= \myvec{0 \\ -\tfrac{13}{3}}.
\end{align}

Hence, the $Y$-axis divides $AB$ internally in the ratio
\begin{align}
AP:PB &= 5:1,
\end{align}
and the coordinates of the point of intersection are
\begin{align}
P &= \myvec{0 \\ -\tfrac{13}{3}}.
\end{align}

\end{document}
\end{document}
