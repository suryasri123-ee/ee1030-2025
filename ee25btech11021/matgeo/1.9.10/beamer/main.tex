\documentclass{beamer}
\mode<presentation>
\usepackage{amsmath,amssymb,mathtools}
\usepackage{textcomp}
\usepackage{gensymb}
\usepackage{adjustbox}
\usepackage{subcaption}
\usepackage{enumitem}
\usepackage{multicol}
\usepackage{listings}
\usepackage{url}
\usepackage{graphicx} % <-- needed for images
\def\UrlBreaks{\do\/\do-}

\usetheme{Boadilla}
\usecolortheme{lily}
\setbeamertemplate{footline}{
  \leavevmode%
  \hbox{%
  \begin{beamercolorbox}[wd=\paperwidth,ht=2ex,dp=1ex,right]{author in head/foot}%
    \insertframenumber{} / \inserttotalframenumber\hspace*{2ex}
  \end{beamercolorbox}}%
  \vskip0pt%
}
\setbeamertemplate{navigation symbols}{}

\lstset{
  frame=single,
  breaklines=true,
  columns=fullflexible,
  basicstyle=\ttfamily\tiny   % tiny font so code fits
}

\numberwithin{equation}{section}

% ---- your macros ----
\providecommand{\nCr}[2]{\,^{#1}C_{#2}}
\providecommand{\nPr}[2]{\,^{#1}P_{#2}}
\providecommand{\mbf}{\mathbf}
\providecommand{\pr}[1]{\ensuremath{\Pr\left(#1\right)}}
\providecommand{\qfunc}[1]{\ensuremath{Q\left(#1\right)}}
\providecommand{\sbrak}[1]{\ensuremath{{}\left[#1\right]}}
\providecommand{\lsbrak}[1]{\ensuremath{{}\left[#1\right.}}
\providecommand{\rsbrak}[1]{\ensuremath{\left.#1\right]}}
\providecommand{\brak}[1]{\ensuremath{\left(#1\right)}}
\providecommand{\lbrak}[1]{\ensuremath{\left(#1\right.}}
\providecommand{\rbrak}[1]{\ensuremath{\left.#1\right)}}
\providecommand{\cbrak}[1]{\ensuremath{\left\{#1\right\}}}
\providecommand{\lcbrak}[1]{\ensuremath{\left\{#1\right.}}
\providecommand{\rcbrak}[1]{\ensuremath{\left.#1\right\}}}
\theoremstyle{remark}
\newtheorem{rem}{Remark}
\newcommand{\sgn}{\mathop{\mathrm{sgn}}}
\providecommand{\abs}[1]{\left\vert#1\right\vert}
\providecommand{\res}[1]{\Res\displaylimits_{#1}}
\providecommand{\norm}[1]{\lVert#1\rVert}
\providecommand{\mtx}[1]{\mathbf{#1}}
\providecommand{\mean}[1]{E\left[ #1 \right]}
\providecommand{\fourier}{\overset{\mathcal{F}}{ \rightleftharpoons}}
\providecommand{\system}{\overset{\mathcal{H}}{ \longleftrightarrow}}
\providecommand{\dec}[2]{\ensuremath{\overset{#1}{\underset{#2}{\gtrless}}}}
\newcommand{\myvec}[1]{\ensuremath{\begin{pmatrix}#1\end{pmatrix}}}
\let\vec\mathbf
% ---------------------

\title{Matgeo Presentation - Problem 1.9.10}
\author{ee25btech11021 - Dhanush sagar}

\begin{document}
	

		




%---------------- Title Page ----------------
\begin{frame}
  \titlepage
\end{frame}

%---------------- Problem Statement ----------------
\begin{frame}{Problem Statement}
  \begin{itemize}
    \item Given two points
    \begin{align*}
    \vec{A} = \myvec{0\\6},  \vec{B} = \myvec{0\\-2}.
    \end{align*}
    \item find the  distance between them.
    \item Verify results using:
    \begin{itemize}
      \item C implementation
      \item Python (ctypes + numpy)
      \item Visualization with matplotlib
    \end{itemize}
  \end{itemize}
\end{frame}

%---------------- Mathematical Formula ----------------
\begin{frame}{solution}
 

\textbf{solution :}
\begin{align}
\text{Let } 
\vec{A} = \myvec{0\\6}, \quad 
\vec{B} = \myvec{0\\-2}.
\end{align}

\begin{align}
\text{The distance between } \vec{A} \text{ and } \vec{B} \text{ is }
d(\vec{A},\vec{B}) = \|\vec{A}-\vec{B}\|_2.
\end{align}

\begin{align}
\vec{A}-\vec{B} =
\myvec{0\\6} - \myvec{0\\-2} = \myvec{0\\8}.
\end{align}
\end{frame}
\begin{frame}{solution}
\begin{align}
\|\vec{A}-\vec{B}\|_2
= \sqrt{(\vec{A}-\vec{B})^T (\vec{A}-\vec{B})}.
\end{align}

\begin{align}
= \sqrt{\myvec{0 & 8}\myvec{0\\8}}
= \sqrt{0^2 + 8^2}
= \sqrt{64}.
\end{align}

\begin{align}
\textbf{conclusion} : \text{The distance between } \vec{A} \text{ and } \vec{B} \text{ is } = 8.
\end{align}
\end{frame}

%---------------- C Source Code ----------------
\begin{frame}[fragile]{C Source Code: points.c}
\begin{verbatim}
#include <stdio.h>
#include <math.h>

double distance(int x1, int y1, int x2, int y2) {
    return sqrt((x1-x2)*(x1-x2) + (y1-y2)*(y1-y2));
}

void get_points(int *points) {
    points[0] = 0; points[1] = 6;   // A
    points[2] = 0; points[3] = -2;  // B
}
\end{verbatim}
\end{frame}

%---------------- Python solve.py ----------------
\begin{frame}[fragile]{Python Script: solve.py}
\begin{verbatim}
import ctypes
# Load shared library
lib = ctypes.CDLL("./libpoints.so")
\# Define argument and return types
lib.distance.argtypes=[ctypes.c_int,ctypes.c_int,ctypes.c_int,ctypes.c_int]
lib.distance.restype = ctypes.c_double
lib.get_points.argtypes = [ctypes.POINTER(ctypes.c_int)]
lib.get_points.restype = None
# Prepare array for points
points = (ctypes.c_int * 4)()
lib.get_points(points)
x1, y1, x2, y2 = points
print(f"Point A = ({x1}, {y1})")
print(f"Point B = ({x2}, {y2})")
\# Get distance
dist = lib.distance(x1, y1, x2, y2)
print(f"Distance between A and B = {dist}")
\end{verbatim}
\end{frame}

%---------------- Python plot.py ----------------
\begin{frame}[fragile]{Python Script: plot.py}
\begin{verbatim}
import numpy as np
import matplotlib.pyplot as plt

# Points
A = np.array([0, 6])
B = np.array([0, -2])

# Distance using numpy
dist = np.linalg.norm(A - B)

# Plotting
plt.scatter(A[0], A[1], color="red", label=f"A{tuple(A)}")
plt.scatter(B[0], B[1], color="blue", label=f"B{tuple(B)}")
plt.plot([A[0], B[0]], [A[1], B[1]],
         color="green", linestyle="--",
         label=f"Distance = {dist:.2f}")
\end{verbatim}
\end{frame}
\begin{frame}[fragile]{Python Script: plot.py}
\begin{verbatim}

plt.xlabel("X-axis")
plt.ylabel("Y-axis")
plt.title("Distance between A and B")
plt.legend()
plt.grid(True)
plt.axis("equal")

plt.savefig("points_plot.png", dpi=300)
plt.show()
\end{verbatim}
\end{frame}

%---------------- Result Plot ----------------
\begin{frame}{Result Plot}
 \begin{figure}[H]
     \centering
     \includegraphics[width=0.5\columnwidth]{figs/fig1.png}
     \caption*{}
     \label{fig:fig1}
 \end{figure}
  \begin{align}
    \text{Distance } &= 8.00
  \end{align}
\end{frame}

\end{document}
