\let\negmedspace\undefined
\let\negthickspace\undefined
\documentclass[journal]{IEEEtran}
\usepackage[a5paper, margin=10mm, onecolumn]{geometry}

\usepackage{tfrupee} 
\setlength{\headheight}{1cm} 
\setlength{\headsep}{0mm}     

\usepackage{gvv-book}
\usepackage{gvv}
\usepackage{cite}
\usepackage{amsmath,amssymb,amsfonts,amsthm}
\usepackage{algorithmic}
\usepackage{graphicx}
\graphicspath{{figs/}}
\usepackage{textcomp}
\usepackage{xcolor}
\usepackage{txfonts}
\usepackage{listings}
\usepackage{enumitem}
\usepackage{mathtools}
\usepackage{gensymb}
\usepackage{comment}
\usepackage[breaklinks=true]{hyperref}
\usepackage{tkz-euclide} 
\usepackage{listings}

\def\inputGnumericTable{}                                 
\usepackage[latin1]{inputenc}                                
\usepackage{color}                                            
\usepackage{array}                                            
\usepackage{longtable}                                       
\usepackage{calc}                                             
\usepackage{multirow}                                         
\usepackage{hhline}                                           
\usepackage{ifthen}                                           
\usepackage{lscape}

\begin{document}

\bibliographystyle{IEEEtran}

\title{1.2.21}
\author{AI25btech11014- Suhas}

{\let\newpage\relax\maketitle}

\renewcommand{\thefigure}{\theenumi}
\renewcommand{\thetable}{\theenumi}
\setlength{\intextsep}{10pt}

\numberwithin{equation}{enumi}
\numberwithin{figure}{enumi}
\renewcommand{\thetable}{\theenumi}





\textbf{Question}:\par
The centroid of a triangle $ABC$ is at the point $(1,1,1)$. If the coordinates of $A$ and $B$ are $(3,-5,7)$ and $(-1,7,-6)$ respectively, find the coordinates of the point $C$.


\vspace{1cm}

\textbf{Solution:}\\
\vspace{0.1cm}



Let the position vectors of points $A$, $B$, and $C$ be:
\begin{align}
\vec{A} = \myvec{3\\-5\\7},\quad
\vec{B} = \myvec{-1\\7\\-6},\quad
\vec{C} = \vec{C}
\end{align}

The centroid $\vec{G}$ of triangle $ABC$ is given by:
\begin{align}
\vec{G} = \frac{1}{3}(\vec{A} + \vec{B} + \vec{C})
\end{align}

Given:
\begin{align}
\vec{G} = \myvec{1\\1\\1}
\end{align}

Substitute and solve:
\begin{align}
\frac{1}{3}\left(\myvec{3\\-5\\7} + \myvec{-1\\7\\-6} + \vec{C}\right) = \myvec{1\\1\\1}
\end{align}

Add vectors:
\begin{align}
\frac{1}{3}\left(\myvec{2\\2\\1} + \vec{C}\right) = \myvec{1\\1\\1}
\end{align}

Multiply both sides by 3:
\begin{align}
\myvec{2\\2\\1} + \vec{C} = \myvec{3\\3\\3}
\end{align}

Subtract:
\begin{align}
\vec{C} = \myvec{1\\1\\2}
\end{align}

\textbf{Therefore, the coordinates of point $C$ are $(1,1,2)$.}



\begin{figure}[h!]
   \centering
   \includegraphics[width=1\linewidth]{figs/fig1.png}
   \caption{3D plot of triangle ABC and centroid G}
   \label{}
\end{figure}

\end{document}