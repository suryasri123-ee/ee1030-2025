\let\negmedspace\undefined
\let\negthickspace\undefined
\documentclass[journal]{IEEEtran}
\usepackage[a5paper, margin=10mm, onecolumn]{geometry}
\usepackage{lmodern} % Ensure lmodern is loaded for pdflatex

\setlength{\headheight}{1cm} % Set the height of the header box
\setlength{\headsep}{0mm}     % Set the distance between the header box and the top of the text

\usepackage{gvv-book}
\usepackage{gvv}
\usepackage{cite}
\usepackage{amsmath,amssymb,amsfonts,amsthm}
\usepackage{algorithmic}
\usepackage{graphicx}
\graphicspath{{./figs/}}
\usepackage{textcomp}
\usepackage{xcolor}
\usepackage{txfonts}
\usepackage{listings}
\usepackage{enumitem}
\usepackage{mathtools}
\usepackage{gensymb}
\usepackage{comment}
\usepackage[breaklinks=true]{hyperref}
\usepackage{tkz-euclide} 
\usepackage{listings}
\usepackage{gvv}                                        
\def\inputGnumericTable{}                                 
\usepackage[latin1]{inputenc}                                
\usepackage{color}                                            
\usepackage{array}                                            
\usepackage{longtable}                                       
\usepackage{calc}                                             
\usepackage{multirow}                                         
\usepackage{hhline}                                           
\usepackage{ifthen}                                           
\usepackage{lscape}
\usepackage{circuitikz}
\tikzstyle{block} = [rectangle, draw, fill=blue!20, 
text width=4em, text centered, rounded corners, minimum height=3em]
\tikzstyle{sum} = [draw, fill=blue!10, circle, minimum size=1cm, node distance=1.5cm]
\tikzstyle{input} = [coordinate]
\tikzstyle{output} = [coordinate]


\begin{document}
	
	\bibliographystyle{IEEEtran}
	\vspace{3cm}
	
	\title{1.6.4}
	\author{EE25BTECH11054 - S.Harsha Vardhan Reddy}
	\maketitle
	% \newpage
	% \bigskip
	{\let\newpage\relax\maketitle}
	
	\renewcommand{\thefigure}{\theenumi}
	\renewcommand{\thetable}{\theenumi}
	\setlength{\intextsep}{10pt} % Space between text and floats
	
	
	\numberwithin{equation}{enumi}
	\numberwithin{figure}{enumi}
	\renewcommand{\thetable}{\theenumi}
	
	\textbf{Question}:
To show that the vectors 
$\vec{v}_1 = 2\hat{\vec{i}} - 3\hat{\vec{j}} + 4\hat{\vec{k}}, \quad \vec{v}_2 = -4\hat{\vec{i}} + 6\hat{\vec{j}} - 8\hat{\vec{k}}$
are collinear.
\textbf{Solution}:

Given vectors $\vec{v_1}$ and $\vec{v_2}$ can be represented as

\begin{align}
\vec{v_1}=\myvec{2 \\ -3 \\ 4 } , \vec{v_2}=\myvec{-4 \\ 6 \\8 }
\end{align}

We write the vectors as the rows of a matrix:
\begin{align}
\myvec{
2 & -3 & 4 \\
-4 & 6 & -8
}
\end{align}

 To show 2 vectors should be collinear,
\begin{align}
\text{rank}\myvec{\vec{v_1} & \vec{v_2}} = 1
\end{align}

\begin{align}
\myvec{
2 & -3 & 4 \\ -4 & 6 & -8 } \xleftrightarrow{R_2 \rightarrow R_2 + 2R_1} \myvec{2 & -3 & 4 \\ 0 & 0 & 0}
\end{align}

So the matrix becomes:
\begin{align}
\myvec{
2 & -3 & 4 \\
0 & 0 & 0
}
\end{align}


\textbf{Conclusion:}

Because the rank is 1, the vectors are linearly dependent. Therefore:
\begin{align}
\vec{v}_1 \text{ and } \vec{v}_2 \text{ are collinear.}
\end{align}


\begin{figure}
    \centering
    \includegraphics[width=0.8\columnwidth]{figs/fig1.png}
    \caption{}
    \label{fig:placeholder}
\end{figure}
\end{document}




















\end{document}
