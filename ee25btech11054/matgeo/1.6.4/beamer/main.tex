\documentclass{beamer}
\mode<presentation>
\usepackage{amsmath}
\usepackage{amssymb}
%\usepackage{advdate}
\usepackage{adjustbox}
\usepackage{subcaption}
\usepackage{enumitem}
\usepackage{multicol}
\usepackage{mathtools}
\usepackage{listings}
\usepackage{url}
\def\UrlBreaks{\do\/\do-}
\usetheme{Boadilla}
\usecolortheme{lily}
\setbeamertemplate{footline}
{
  \leavevmode%
  \hbox{%
  \begin{beamercolorbox}[wd=\paperwidth,ht=2.25ex,dp=1ex,right]{author in head/foot}%
    \insertframenumber{} / \inserttotalframenumber\hspace*{2ex} 
  \end{beamercolorbox}}%
  \vskip0pt%
}

\providecommand{\nCr}[2]{\,^{#1}C_{#2}} % nCr
\providecommand{\nPr}[2]{\,^{#1}P_{#2}} % nPr
\providecommand{\mbf}{\mathbf}
\providecommand{\pr}[1]{\ensuremath{\Pr\left(#1\right)}}
\providecommand{\qfunc}[1]{\ensuremath{Q\left(#1\right)}}
\providecommand{\sbrak}[1]{\ensuremath{{}\left[#1\right]}}
\providecommand{\lsbrak}[1]{\ensuremath{{}\left[#1\right.}}
\providecommand{\rsbrak}[1]{\ensuremath{{}\left.#1\right]}}
\providecommand{\brak}[1]{\ensuremath{\left(#1\right)}}
\providecommand{\lbrak}[1]{\ensuremath{\left(#1\right.}}
\providecommand{\rbrak}[1]{\ensuremath{\left.#1\right)}}
\providecommand{\cbrak}[1]{\ensuremath{\left\{#1\right\}}}
\providecommand{\lcbrak}[1]{\ensuremath{\left\{#1\right.}}
\providecommand{\rcbrak}[1]{\ensuremath{\left.#1\right\}}}
\theoremstyle{remark}
\newtheorem{rem}{Remark}
\newcommand{\sgn}{\mathop{\mathrm{sgn}}}
\providecommand{\abs}[1]{\left\vert#1\right\vert}
\providecommand{\res}[1]{\Res\displaylimits_{#1}} 
\providecommand{\norm}[1]{\lVert#1\rVert}
\providecommand{\mtx}[1]{\mathbf{#1}}
\providecommand{\mean}[1]{E\left[ #1 \right]}
\providecommand{\fourier}{\overset{\mathcal{F}}{ \rightleftharpoons}}
%\providecommand{\hilbert}{\overset{\mathcal{H}}{ \rightleftharpoons}}
\providecommand{\system}{\overset{\mathcal{H}}{ \longleftrightarrow}}
	%\newcommand{\solution}[2]{\textbf{Solution:}{#1}}
%\newcommand{\solution}{\noindent \textbf{Solution: }}
\providecommand{\dec}[2]{\ensuremath{\overset{#1}{\underset{#2}{\gtrless}}}}
\newcommand{\myvec}[1]{\ensuremath{\begin{pmatrix}#1\end{pmatrix}}}
\let\vec\mathbf

\lstset{
%language=C,
frame=single, 
breaklines=true,
columns=fullflexible
}

\numberwithin{equation}{section}

\lstset{
  language=Python,
  basicstyle=\ttfamily\small,
  keywordstyle=\color{blue},
  stringstyle=\color{orange},
  numbers=left,
  numberstyle=\tiny\color{gray},
  breaklines=true,
  showstringspaces=false
}

\title{Problem 1.6.4}
\author{EE25BTECH11054 - S.Harsha Vardhan Reddy}

\date{\today} 
\begin{document}

\begin{frame}
\titlepage
\end{frame}

\section*{Outline}
\begin{frame}
\tableofcontents
\end{frame}
\section{Problem}
\begin{frame}
\frametitle{Problem Statement}
%
Show that the vectors 
$\vec{v}_1 = 2\hat{\vec{i}} - 3\hat{\vec{j}} + 4\hat{\vec{k}}, \quad \vec{v}_2 = -4\hat{\vec{i}} + 6\hat{\vec{j}} - 8\hat{\vec{k}}$
are collinear.
 
\end{frame}

%\subsection{Literature}
\section{Solution}
\subsection{Requirement}
\setcounter{section}{1}
\begin{frame}
\frametitle{Requirement}
%\framesubtitle{Literature}
To show 2 vectors should be collinear,
\begin{align}
\text{rank}\myvec{\vec{v_1} & \vec{v_2}} = 1
\end{align}
\end{frame}
\subsection{Converting into Echelon form and Finding Rank}
\begin{frame}
\frametitle{Converting into Echelon form and Finding Rank}
Given vectors $\vec{v_1}$ and $\vec{v_2}$ can be represented as

\begin{align}
\vec{v_1}=\myvec{2 \\ -3 \\ 4 } , \vec{v_2}=\myvec{-4 \\ 6 \\8 }
\end{align}

We write the vectors as the rows of a matrix:
\begin{align}
   \myvec{
2 & -3 & 4 \\ -4 & 6 & -8 } \xleftrightarrow{R_2 \rightarrow R_2 + 2R_1} \myvec{2 & -3 & 4 \\ 0 & 0 & 0}
\end{align}
As the last row is entirely zero,
\\ We can say that Rank of the matrix is 1.
\end{frame}
\subsection{Conclusion}
\begin{frame}
\frametitle{Conclusion}

Because the rank is 1, the vectors are linearly dependent. Therefore:
\begin{align}
\vec{v}_1 \text{ and } \vec{v}_2 \text{ are collinear.}
\end{align}

\end{frame}

\subsection{Plot}
\begin{frame}[fragile]
\frametitle{Plot}

\begin{figure}[H]
   \centering
   \includegraphics[width=0.8\columnwidth]{figs/fig1.png}
	\caption{}
   \label{stemplot}
\end{figure}
\end{frame}

\section{C Code}
\begin{frame}[fragile]
\frametitle{C Code for generating points on line}
\begin{lstlisting}[language=C]

#include <stdio.h>

static double v1[3] = {2.0, -3.0, 4.0};
static double v2[3] = {-4.0, 6.0, -8.0};

// Function to get v1
double* get_v1() {
    return v1;
}

// Function to get v2
double* get_v2() {
    return v2;
}

 
    \end{lstlisting}
\end{frame}

\section{Python Code}
\begin{frame}[fragile]
\frametitle{Python Code for Calling}
\begin{lstlisting}[language=Python]
import ctypes

# Load shared library
lib = ctypes.CDLL("./libvectors.so")

# Set return types
lib.get_v1.restype = ctypes.POINTER(ctypes.c_double)
lib.get_v2.restype = ctypes.POINTER(ctypes.c_double)

# Extract vectors
v1 = [lib.get_v1()[i] for i in range(3)]
v2 = [lib.get_v2()[i] for i in range(3)]

print("Vector v1 =", v1)
print("Vector v2 =", v2)

\end{lstlisting}
\end{frame}

\begin{frame}[fragile]
\frametitle{Python Code for Plotting}
\begin{lstlisting}[language=Python]
import ctypes
import numpy as np
import sys
import matplotlib.pyplot as plt
sys.path.insert(0, '/home/soma-harsha/Desktop/matgeo/codes/CoordGeo')
# Load shared library
lib = ctypes.CDLL("./libvectors.so")

# Set return types
lib.get_v1.restype = ctypes.POINTER(ctypes.c_double)
lib.get_v2.restype = ctypes.POINTER(ctypes.c_double)

from line.funcs import *
from triangle.funcs import *
from conics.funcs import circ_gen

\end{lstlisting}
\end{frame}

\begin{frame}[fragile]
\frametitle{Python Code for Plotting}
\begin{lstlisting}[language=Python]
# Get vectors
v1 = np.array([lib.get_v1()[i] for i in range(3)])
v2 = np.array([lib.get_v2()[i] for i in range(3)])

print("v1 =", v1)
print("v2 =", v2)

# Plot
fig = plt.figure(figsize=(8,6))
ax = fig.add_subplot(111, projection='3d')

origin = np.array([0,0,0])

# Plot vectors
ax.quiver(*origin, *v1, color='r', label="v1")
ax.quiver(*origin, *v2, color='b', label="v2")

# Line of collinearity
t = np.linspace(-2, 2, 100)

\end{lstlisting}
\end{frame}

\begin{frame}[fragile]
\frametitle{Python Code for Plotting}
\begin{lstlisting}[language=Python]
line = np.outer(t, v1)
ax.plot(line[:,0], line[:,1], line[:,2], 'g--', label="Line of Collinearity")
# Labels
ax.set_xlabel("X")
ax.set_ylabel("Y")
ax.set_zlabel("Z")
ax.set_title("Collinear Vectors from C .so File")
ax.legend()
plt.savefig('../figs/fig1.png')
plt.show()
\end{lstlisting}
\end{frame}


\end{document}