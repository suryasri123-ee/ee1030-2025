\documentclass{beamer}
\usepackage[utf8]{inputenc}

\usetheme{Madrid}
\usecolortheme{default}
\usepackage{amsmath,amssymb,amsfonts,amsthm}
\usepackage{txfonts}
\usepackage{tkz-euclide}
\usepackage{listings}
\usepackage{adjustbox}
\usepackage{array}
\usepackage{tabularx}
\usepackage{gvv}
\usepackage{lmodern}
\usepackage{circuitikz}
\usepackage{tikz}
\usepackage{graphicx}
\usepackage{mathtools}
\setbeamertemplate{page number in head/foot}[totalframenumber]

\usepackage{tcolorbox}
\tcbuselibrary{minted,breakable,xparse,skins}



\definecolor{bg}{gray}{0.95}
\DeclareTCBListing{mintedbox}{O{}m!O{}}{%
  breakable=true,
  listing engine=minted,
  listing only,
  minted language=#2,
  minted style=default,
  minted options={%
    linenos,
    gobble=0,
    breaklines=true,
    breakafter=,,
    fontsize=\small,
    numbersep=8pt,
    #1},
  boxsep=0pt,
  left skip=0pt,
  right skip=0pt,
  left=25pt,
  right=0pt,
  top=3pt,
  bottom=3pt,
  arc=5pt,
  leftrule=0pt,
  rightrule=0pt,
  bottomrule=2pt,
  toprule=2pt,
  colback=bg,
  colframe=orange!70,
  enhanced,
  overlay={%
    \begin{tcbclipinterior}
    \fill[orange!20!white] (frame.south west) rectangle ([xshift=20pt]frame.north west);
    \end{tcbclipinterior}},
  #3,
}
\lstset{
    language=C,
    basicstyle=\ttfamily\small,
    keywordstyle=\color{blue},
    stringstyle=\color{orange},
    commentstyle=\color{green!60!black},
    numbers=left,
    numberstyle=\tiny\color{gray},
    breaklines=true,
    showstringspaces=false,
}

\title 
{1.2.16}
\date{August 27, 2025}


\author 
{Dhanush Kumar A - AI25BTECH11010}



\begin{document}
\frame{\titlepage}
\begin{frame}{Question}
 Show that $(-1,2,1), \; (1,-2,5), \; (4,-7,8)$ and $(2,-3,4)$ are the vertices of a parallelogram.
\end{frame}
\begin{frame}{Variables used}
\begin{table}[H]    
  \centering
  \begin{center}
    \begin{tabular}{|c|c|} 
        \hline
            \textbf{Variable}  & \textbf{Formula} \\ 
        \hline
            $a$   & $a = \myvec{4 \\ -1 \\ 1}$ \\ 
        \hline
            $b$   &  $b = \myvec{2 \\ -2 \\ 1}$\\ 
        \hline
           \end{tabular}
\end{center}  

  \caption{Variables Used}
  \label{tab:1.2.16}
\end{table}
\end{frame}
\begin{frame}{Solution}

	Now,
  \begin{align}
 \vec{B - A}= \myvec{1 - (-1) \\ -2 - 2 \\ 5 - 1} = \myvec{2 \\ -4 \\ 4}, \\[6pt]
\vec{C - D} = \myvec{4 - 2 \\ -7 - (-3) \\ 8 - 4}= \myvec{2 \\ -4 \\ 4}.
 \end{align}

 Hence,
 \begin{align}
        \vec{B - A} = \vec{C - D}
 \end{align}
  $\implies$


 \begin{align}
        \vec{C-B} = \vec{D-A}
 \end{align}
	Therefore, $A, B, C, D$ are the vertices of a parallelogram.

\end{frame}

\begin{frame}[fragile]
    \frametitle{Python code- Checking whether the points are vertices of parallelogram}
    \begin{lstlisting}
import numpy as np
 import itertools

 def is_parallel(v, w, tol=1e-9):
    """Check if vectors v and w are parallel (cross pro    duct = 0)."""
     v = np.array(v, dtype=float)
   w = np.array(w, dtype=float)
   if np.allclose(v, 0, atol=tol) or np.allclose(w, 0,     atol=tol):
       return np.allclose(v, w, atol=tol)
    return np.allclose(np.cross(v, w), 0, atol=tol)

 def is_parallelogram(points):
     """
   Check if 4 points form a parallelogram using only the parallel-sides test.
     Returns True if yes, else False.
"""
     \end{lstlisting}
\end{frame}

\begin{frame}[fragile]                            
	\frametitle{Python code - Checking whether the points ar    e vertices of parallelogram}                
	\begin{lstlisting}

     P = [np.array(p, dtype=float) for p in points]

    for perm in itertools.permutations(range(4)):
         A, B, C, D = [P[i] for i in perm]

       AB, BC, CD, DA = B - A, C - B, D - C, A - D

   # Check opposite sides are parallel and adjacen    t sides not parallel
    if is_parallel(AB, CD) and is_parallel(BC, DA)     and not is_parallel(AB, BC):
          return True
     return False
	\end{lstlisting}                            
\end{frame}

\begin{frame}[fragile]                           
	\frametitle{Python code - Checking whether the points ar    e vertices of parallelogram}             
	\begin{lstlisting}
  # Example points
  A = (-1,  2,  1)
 B = ( 1, -2,  5)
  C = ( 4, -7,  8)
 D = ( 2, -3,  4)

 points = [A, B, C, D]

 if is_parallelogram(points):
   print("The given points form a parallelogram.")
  else:
   print(" The given points do NOT form a parallelogra    m.")
    \end{lstlisting}
\end{frame}
\begin{frame}[fragile]                              
	\frametitle{Python code - plotting the points}
	\begin{lstlisting}
	import numpy as np
import matplotlib.pyplot as plt
from mpl_toolkits.mplot3d.art3d import Poly3DCollection
import os

# Given points as column vectors (x,y,z)
A = np.array([-1, 2, 1]).reshape(-1,1)
B = np.array([1, -2, 5]).reshape(-1,1)
C = np.array([4, -7, 8]).reshape(-1,1)
D = np.array([2, -3, 4]).reshape(-1,1)

# Stack coordinates
coords = np.block([A,B,C,D])

# Create 3D plot
fig = plt.figure()
ax = fig.add_subplot(111, projection='3d')

\end{lstlisting}                               
\end{frame}
\begin{frame}[fragile]                              
	\frametitle{Python code - plotting the points}
	\begin{lstlisting}
# Scatter points
ax.scatter(coords[0,:], coords[1,:], coords[2,:], color='r', s=50)

# Draw parallelogram edges
edges = [(A,B), (B,C), (C,D), (D,A)]
for edge in edges:
    pts = np.hstack(edge)
    ax.plot(pts[0,:], pts[1,:], pts[2,:], color='b')

# Fill parallelogram face
verts = [[A.flatten(), B.flatten(), C.flatten(), D.flatten()]]
ax.add_collection3d(Poly3DCollection(verts, alpha=0.3, facecolor='cyan'))

\end{lstlisting}                               
\end{frame}
\begin{frame}[fragile]                              
	\frametitle{Python code - plotting the points}
	\begin{lstlisting}
# Labels
ax.text(A[0,0], A[1,0], A[2,0], "A(-1,2,1)")
ax.text(B[0,0], B[1,0], B[2,0], "B(1,-2,5)")
ax.text(C[0,0], C[1,0], C[2,0], "C(4,-7,8)")
ax.text(D[0,0], D[1,0], D[2,0], "D(2,-3,4)")

# Axes
ax.set_xlabel('$x$')
ax.set_ylabel('$y$')
ax.set_zlabel('$z$')
plt.title("Parallelogram in 3D")

# Save figure
save_path = '../figs/img.png'
os.makedirs(os.path.dirname(save_path), exist_ok=True)
plt.savefig(save_path, dpi=300)

print(f"Image saved at: {save_path}")

plt.show()
\end{lstlisting}                               
\end{frame}

\begin{frame}{Plot-Using  Python}
    \centering
    \includegraphics[width=\columnwidth, height=0.8\textheight, keepaspectratio]{../figs/img.png}     
\end{frame}

\begin{frame}[fragile]                              
	\frametitle{C code- writing the points}
	\begin{lstlisting}
#include <stdio.h>
typedef struct { double x,y,z; } Point;
void save_points() {
    Point pts[4]={{-1,2,1},{1,-2,5},{4,-7,8},{2,-3,4}};
    FILE *f=fopen("points.dat","w");
    for(int i=0;i<4;i++) fprintf(f,"%f %f %f\n",pts[i].x,pts[i].y,pts[i].z);
    fclose(f);
}

\end{lstlisting}                               
\end{frame}
	
\begin{frame}[fragile]                              
	\frametitle{Python code -Ploting the points using c function} 
	\begin{lstlisting}

import numpy as np, matplotlib.pyplot as plt, os
from mpl_toolkits.mplot3d.art3d import Poly3DCollection

points=np.loadtxt("points.dat")
fig=plt.figure(); ax=fig.add_subplot(111,projection="3d")
ax.add_collection3d(Poly3DCollection([points],alpha=0.3,facecolor="cyan"))
ax.scatter(points[:,0],points[:,1],points[:,2],color="red",s=50)
for i,(x,y,z) in enumerate(points): ax.text(x,y,z,f"P{i+1}",color="black")
ax.set_xlabel("X"); ax.set_ylabel("Y"); ax.set_zlabel("Z")

save_path="../figs/img1.png"
os.makedirs(os.path.dirname(save_path),exist_ok=True)
plt.savefig(save_path,dpi=300)
plt.show()

\end{lstlisting}                               
\end{frame}

\begin{frame}{Plot-Using  Python and C}
    \centering
    \includegraphics[width=\columnwidth, height=0.8\textheight, keepaspectratio]{../figs/img1.png}     
\end{frame}

\end{document}
