\let\negmedspace\undefined
\let\negthickspace\undefined
\documentclass[journal]{IEEEtran}
\usepackage[a5paper, margin=10mm, onecolumn]{geometry}
%\usepackage{lmodern} % Ensure lmodern is loaded for pdflatex
\usepackage{tfrupee} % Include tfrupee package

\setlength{\headheight}{1cm} % Set the height of the header box
\setlength{\headsep}{0mm}     % Set the distance between the header box and the top of the text

\usepackage{gvv-book}
\usepackage{gvv}
\usepackage{cite}
\usepackage{amsmath,amssymb,amsfonts,amsthm}
\usepackage{algorithmic}
\usepackage{graphicx}
\usepackage{textcomp}
\usepackage{xcolor}
\usepackage{txfonts}
\usepackage{listings}
\usepackage{enumitem}
\usepackage{mathtools}
\usepackage{gensymb}
\usepackage{comment}
%\usepackage{multiclo}
\usepackage[breaklinks=true]{hyperref}
\usepackage{tkz-euclide} 
\usepackage{listings}
% \usepackage{gvv} 
\graphicspath{ {./figs/} }



\begin{document}
\title{
1.2.16}
\author{AI25BTECH11010 - Dhanush Kumar}
\maketitle
\renewcommand{\thefigure}{\theenumi}
\renewcommand{\thetable}{\theenumi}


\noindent
\textbf{Question:} \\
Show that $(-1,2,1), \; (1,-2,5), \; (4,-7,8)$ and $(2,-3,4)$ are the vertices of a parallelogram.

\bigskip
\textbf{Solution:} \\
Let
\begin{align}
	\vec{A} = \myvec{-1 \\ 2 \\ 1},
	\vec{B}= \myvec{1 \\ -2 \\ 5}, 
\vec{C} = \myvec{4 \\ -7 \\ 8},
\vec{D} = \myvec{2 \\ -3 \\ 4}.
\end{align}

Now,
\begin{align}
	\vec{B - A}= \myvec{1 - (-1) \\ -2 - 2 \\ 5 - 1}
      = \myvec{2 \\ -4 \\ 4}, \\[6pt]
	\vec{C - D} = \myvec{4 - 2 \\ -7 - (-3) \\ 8 - 4}
      = \myvec{2 \\ -4 \\ 4}.
\end{align}

Hence,
\begin{align}
	\vec{B - A} = \vec{C - D}.
\end{align}
$\implies$ 
\begin{align}
	\vec{C-B} = \vec{D -A} .
\end{align}
Therefore, $A, B, C, D$ are the vertices of a parallelogram.

\begin{figure}[H]
    \centering
	\includegraphics[width=1\textwidth]{img.png}
	    \caption{}
    \label{img}
    \end{figure}
\end{document}

