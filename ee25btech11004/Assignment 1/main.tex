\documentclass[12pt]{article}
\usepackage{graphicx} 
\usepackage{float}
\usepackage[margin=2cm]{geometry}
\usepackage{fancyhdr}
\usepackage{multicol}
\pagestyle{fancy}
\graphicspath{ {./images/} }
\usepackage{gensymb}
\usepackage{siunitx}
\usepackage{array}
\usepackage{amsmath}



\title{Geomatics Engineering (GE) 2024}
\author{Aditya Appana}
\date{August 2025}

\begin{document}
\lhead{\includegraphics[scale=0.3]{GateHeader.png}}
\rhead{\textbf{Geomatics Engineering (GE)}}
\cfoot{Page {\thepage } of 46}
\lfoot{Organizing Institute: IISc Bengaluru} 
\maketitle

\section*{\underline{General Aptitude (GA)}}

\textbf{Q.1 – Q.5 Carry ONE mark Each}

\vspace{1cm}



\begin{table}[H]
\renewcommand{\arraystretch}{2}
\setlength{\tabcolsep}{8pt}
\begin{tabular}{|l|p{15cm}|}
\hline
Q.1 & If ‘→’ denotes increasing order of intensity, then the meaning of the words [smile → giggle → laugh] is analogous to [disapprove → \rule{1.5cm}{0.15mm} → chide]. Which one of the given options is appropriate to fill the blank? \\ \hline
 & \\ \hline
(A)&reprove \\ \hline
(B)&praise \\ \hline
(C)&reprise \\ \hline
(D)&grieve \\ \hline
 & \\ \hline
Q.2 & Find the odd one out in the set: \{19, 37, 21, 17, 23, 29, 31, 11\}\\ \hline
 & \\ \hline
(A)&21 \\ \hline
(B)&29 \\ \hline
(C)&37 \\ \hline
(D)&23 \\ \hline
 



\end{tabular}
\end{table}


\newpage

\begin{table}[H]
\renewcommand{\arraystretch}{3}
\setlength{\tabcolsep}{8pt}
\begin{tabular}{|l|p{15cm}|}
\hline
Q.3 & In the following series, identify the number that needs to be changed to form the
Fibonacci series. 1, 1, 2, 3, 6, 8, 13, 21 \\ \hline
 & \\ \hline
(A)&8\\ \hline
(B)&21\\ \hline
(C)&6 \\ \hline
(D)&13\\ \hline
 & \\ \hline
Q.4 & The real variables $x,y,z$ and the real constants $p,q,r$ satisfy  \\ & \begin{Large} $\frac{x}{pq-r^2}$ =     $\frac{y}{qr-p^2}$ = $\frac{z}{rp-q^2}$ \end{Large} \\ \hline 
 & \\ \hline
(A)&0 \\ \hline
(B)&1 \\ \hline
(C)&$pqr$ \\ \hline
(D)&$p^2 + q^2 + r^2$  \\ \hline
 



\end{tabular}
\end{table}




\newpage

\begin{table}[H]
\renewcommand{\arraystretch}{3}
\setlength{\tabcolsep}{8pt}
\begin{tabular}{|l|p{15cm}|}
\hline
Q.5 &Take two long dice (rectangular parallelepiped), each having four rectangular faces
labelled as 2, 3, 5, and 7. If thrown, the long dice cannot land on the square faces
and has 1/4 probability of landing on any of the four rectangular faces. The label on
the top face of the dice is the score of the throw. 

If thrown together, what is the probability of getting the sum of the two long dice
scores greater than 11? \\ \hline
 & \\ \hline
(A)&3/8\\ \hline
(B)&1/8\\ \hline
(C)&1/16 \\ \hline
(D)&3/16\\ \hline
 & \\ \hline
\end{tabular}
\end{table}


\newpage

\textbf{Q.6 – Q.10 Carry TWO marks Each}


\begin{table}[H]
\renewcommand{\arraystretch}{2.8}
\setlength{\tabcolsep}{8pt}
\begin{tabular}{|l|p{15cm}|}
\hline
Q.6 &In the given text, the blanks are numbered (i)---(iv). Select the best match for all the
blanks.

Prof. P \rule{0.5cm}{0.15mm}(i)\rule{0.5cm}{0.15mm} merely a man who narrated funny stories. \rule{0.5cm}{0.15mm}(ii)\rule{0.5cm}{0.15mm} in his blackest moments he was capable of self-deprecating humor.

Prof. Q \rule{0.5cm}{0.15mm}(iii)\rule{0.5cm}{0.15mm} a man who hardly narrated funny stories. \rule{0.5cm}{0.15mm}(iv)\rule{0.5cm}{0.15mm} in his blackest moments was he able to find humor.
 \\ \hline
 & \\ \hline
(A)& (i) was (ii) Only (iii) wasn’t (iv) Even \\ \hline
(B)&(i) wasn't (ii) Even (iii) was (iv) Only\\ \hline
(C)&(i) was (ii) Even (iii) wasn't (iv) Only \\ \hline
(D)&(i) wasn't (ii) Only (iii) was (iv) Even\\ \hline
 & \\ \hline


Q.7 & How many combinations of non-null sets A, B, C are possible from the subsets of
{2, 3, 5} satisfying the conditions: (i) A is a subset of B, and (ii) B is a subset of C? \\ \hline
 & \\ \hline
(A)&28\\ \hline
(B)&27\\ \hline
(C)&18 \\ \hline
(D)&19\\ \hline
\end{tabular}
\end{table}

\newpage


\begin{table}[H]
\renewcommand{\arraystretch}{2.8}
\setlength{\tabcolsep}{8pt}
\begin{tabular}{|l|p{15cm}|}
\hline
 & \\ \hline
Q.8 & The bar chart gives the batting averages of VK and RS for 11 calendar years from
2012 to 2022. Considering that 2015 and 2019 are world cup years, which one of
the following options is true?

\includegraphics[scale=0.7]{LatexGraph.png}
 \\ \hline
 & \\ \hline
(A)& RS has a higher yearly batting average than that of VK in every world cup year. \\ \hline
(B)&VK has a higher yearly batting average than that of RS in every world cup year.\\ \hline
(C)&VK’s yearly batting average is consistently higher than that of RS between the two
world cup years. \\ \hline
(D)& RS’s yearly batting average is consistently higher than that of VK in the last three
years.\\ \hline
 & \\ 
 & \\ 
 & \\  \hline







\end{tabular}
\end{table}

\newpage

\begin{table}[H]
\renewcommand{\arraystretch}{2.8}
\setlength{\tabcolsep}{8pt}
\begin{tabular}{|l|p{15cm}|}
\hline
Q.9& A planar rectangular paper has two V-shaped pieces attached as shown below.

\hspace{1cm}\includegraphics[scale = 0.8]{LatexImage1.png} 

This piece of paper is folded to make the following closed three-dimensional object.

\hspace{4cm}\includegraphics[scale = 0.9]{LatexImage2.png} 


 \\ \hline
 & \\ \hline
(A)& 9\\ \hline
(B)&7\\ \hline
(C)&11 \\ \hline
(D)& 18.\\ \hline
 & \\  
  & \\  \hline

\end{tabular}
\end{table}

\newpage
\begin{table}[H]


\renewcommand{\arraystretch}{2.8}
\setlength{\tabcolsep}{8pt}
\begin{tabular}{|l|p{15cm}|}
\hline

Q.10 & Four equilateral triangles are used to form a regular closed three-dimensional object
by joining along the edges. The angle between any two faces is\\ \hline
 & \\ \hline
(A)&30\degree\\ \hline
(B)&60\degree\\ \hline
(C)&45\degree\\ \hline
(D)&90\degree\\ \hline
 & \\ \hline

\end{tabular}
\end{table}

\newpage

\textbf{ PART A: Common FOR ALL CANDIDATES} \\

\textbf{ Q.11 – Q.27 Carry ONE mark each} 

\begin{table}[H]
\renewcommand{\arraystretch}{3}
\setlength{\tabcolsep}{8pt}
\begin{tabular}{|l|p{15cm}|}
\hline
Q.11 & Which of the following options best describes the “uncertainty” in a
measurement?\\ \hline
 & \\ \hline
(A)&It includes both random and gross errors\\ \hline
(B)&It includes only systematic errors\\ \hline
(C)&It includes both systematic and gross errors \\ \hline
(D)&It includes both random and systematic errors\\ \hline
 & \\ \hline
Q.12 & A distance was measured as 200 m ± 0.1 m. The relative precision of this
measurement is\\ \hline 
 & \\ \hline
(A)&1:20 \\ \hline
(B)&1:200 \\ \hline
(C)&1:2000\\ \hline
(D)&1:20000 \\ \hline


\end{tabular}
\end{table}

\begin{table}[H]
\renewcommand{\arraystretch}{3}
\setlength{\tabcolsep}{8pt}
\begin{tabular}{|l|p{15cm}|}
\hline

Q.13 &Which of the following options describes the CORRECT relationship for a
Gaussian distributed random error?\\ \hline 
 & \\ \hline
(A)&Probable error < Average error < Standard error < 90\% error\\ \hline
(B)&Standard error < Average error < Probable error < 90\% error \\ \hline
(C)&Average error < Probable error < 90\% error < Standard error\\ \hline
(D)&Probable error < 90\% error < Average error < Standard error \\ \hline
 & \\ \hline

Q.14 & The Chi-square distribution is used for comparing the\\ \hline
 & \\ \hline
(A)&population variance with the sample variance for a given degree of freedom\\ \hline
(B)&population mean with the sample mean for a given degree of freedom\\ \hline
(C)&population median with the sample median for a given degree of freedom \\ \hline
(D)&population mean and standard deviation with the sample mean and standard
deviation for a given degree of freedom\\ \hline
 & \\ 
 & \\ \hline

\end{tabular}
\end{table}


\begin{table}[H]
\renewcommand{\arraystretch}{3}
\setlength{\tabcolsep}{8pt}
\begin{tabular}{|l|p{15cm}|}
\hline

Q.15 &Water bodies appear in dark tone in Near Infrared (NIR) image, because water
\rule{2cm}{0.15mm} most of the NIR radiations incident on it.\\ \hline 
 & \\ \hline
(A)&absorbs\\ \hline
(B)&emits \\ \hline
(C)&reflects\\ \hline
(D)&scatters \\ \hline
 & \\ \hline

Q.16 & The approximate altitude (above earth surface) of polar sun-synchronous orbits of
ISRO’s remote sensing satellites is\\ \hline
 & \\ \hline
(A)&< 90 km\\ \hline
(B)&90 km to 200 km\\ \hline
(C)&200 km to 400 km \\ \hline
(D)& > 400 km\\ \hline
 & \\ \hline
 & \\ \hline

\end{tabular}
\end{table}

\begin{table}[H]
\renewcommand{\arraystretch}{3}
\setlength{\tabcolsep}{8pt}
\begin{tabular}{|l|p{15cm}|}
\hline

Q.17 &Hyperspectral sensor consists of.\\ \hline 
 & \\ \hline
(A)&large number of wide and discrete bands\\ \hline
(B)&small number of wide and contiguous bands \\ \hline
(C)&large number of narrow and contiguous bands\\ \hline
(D)&small number of narrow and discrete bands \\ \hline
 & \\ \hline

Q.18 & Part of the solar radiation incident on the water surface gets refracted as per\\ \hline
 & \\ \hline
(A)&Rayleigh’s law\\ \hline
(B)&Snell’s law\\ \hline
(C)&Moore’s law \\ \hline
(D)&Newton’s law\\ \hline
 & \\ 
 & \\ \hline

 

\end{tabular}
\end{table}

\begin{table}[H]
\renewcommand{\arraystretch}{3}
\setlength{\tabcolsep}{8pt}
\begin{tabular}{|l|p{15cm}|}
\hline

Q.19 &Which of the following mathematical principles is applied for finding a
geographic position on Earth’s surface using GPS?\\ \hline 
 & \\ \hline
(A)&Triangulation\\ \hline
(B)&Analytical traversing \\ \hline
(C)&Trilateration\\ \hline
(D)&Analytical leveling \\ \hline
 & \\ \hline

Q.20 & Which of the following is NOT a segment of GPS to determine position and time?\\ \hline
 & \\ \hline
(A)&Space segment\\ \hline
(B)&Control segment\\ \hline
(C)&Launch segment\\ \hline
(D)&User segment\\ \hline
 & \\ 
 & \\ \hline

 

\end{tabular}
\end{table}


\begin{table}[H]
\renewcommand{\arraystretch}{3}
\setlength{\tabcolsep}{8pt}
\begin{tabular}{|l|p{15cm}|}
\hline

Q.21 &Dilution of Precision (DOP) in GPS based survey is primarily used to assess the
quality of\\ \hline 
 & \\ \hline
(A)&satellite’s altitude\\ \hline
(B)&satellite’s geometry \\ \hline
(C)&satellite’s atomic clocks\\ \hline
(D)&satellite’s velocity \\ \hline
 & \\ \hline

Q.22 & How many NAVSTAR GPS satellites in standard constellation are operational
and provide uninterrupted service?\\ \hline
 & \\ \hline
(A)&4\\ \hline
(B)&12\\ \hline
(C)&24\\ \hline
(D)&36\\ \hline
 & \\ 
 & \\ \hline

 

\end{tabular}
\end{table}

\begin{table}[H]
\renewcommand{\arraystretch}{3}
\setlength{\tabcolsep}{8pt}
\begin{tabular}{|l|p{15cm}|}
\hline

Q.23 &Identify the type of digitizing error in the following figure.\\ \hline 

 &\vspace{0.3cm} \hspace{3cm} \includegraphics[scale = 0.8]{LatexImage3.png} \\ \hline
(A)&satellite’s altitude\\ \hline
(B)&satellite’s geometry \\ \hline
(C)&satellite’s atomic clocks\\ \hline
(D)&satellite’s velocity \\ \hline
 & \\ \hline

Q.24 & Which of the following is NOT a derivative of digital elevation model (DEM)?\\ \hline
 & \\ \hline
(A)&Slope\\ \hline
(B)&Aspect\\ \hline
(C)&Contour\\ \hline
(D)&Emissivity\\ \hline

 

\end{tabular}
\end{table}

\begin{table}[H]
\renewcommand{\arraystretch}{3}
\setlength{\tabcolsep}{8pt}
\begin{tabular}{|l|p{15cm}|}
\hline

Q.25 &Which of the following is a core vector GIS operation?\\ \hline 
& \\ \hline
(A)&Overlaying\\ \hline
(B)&Contrast stretching\\ \hline
(C)&Histogram equalization\\ \hline
(D)&Band ratioing \\ \hline
 & \\ \hline

Q.26 & The wavelength at which maximum energy is radiated or emitted from the forest
fire at temperature of 700 °C is \rule{2cm}{0.15mm} (rounded off to one decimal place). \\ \hline
& \\ \hline

Q.27 & The standard error of a unit weight for a set of angle observations is 10\textquotedblright.
The minimum number of observations required to reduce the standard error of the
mean for this set of observations to 2\textquotedblright\ is  \rule{2cm}{0.15mm} \textit{(in integer)}.\\ \hline

 & \\ 
 & \\
 & \\\hline
 & \\ \hline
 

 

\end{tabular}
\end{table}


\begin{small}\textbf{Q.28 – Q.46 Carry TWO marks Each}\end{small}


\begin{table}[H]
\renewcommand{\arraystretch}{3}
\setlength{\tabcolsep}{8pt}
\begin{tabular}{|l|p{15cm}|}
\hline

Q.28 &Which of the following is a core vector GIS operation?\\
    & \ang{60;30;10} ± 10''\\  
    & \ang{60;30;20} ± 20''\\
    &The most probable value (MPV) of the angle is: \\ \hline
    & \\\hline
(A)&\ang{60;30;12}\\ \hline
(B)&\ang{60;30;15}\\ \hline
(C)&\ang{60;30;18}\\ \hline
(D)&\ang{60;30;14}\\ \hline

\end{tabular}
\end{table} 
\newpage

\begin{table}[H]
\renewcommand{\arraystretch}{3}
\setlength{\tabcolsep}{8pt}
\begin{tabular}{|l|p{15cm}|}
\hline


 Q.29 &In the figure, $d_1, d_2, d_3$ are three independently measured distances for estimating
the unknown distances $x$ and $y$. The correlation coefficient between the unknown
estimates approximately equals to \\ \hline
& $d_1$ = 100m ± 1cm \includegraphics[scale = 0.9]{LatexImage4.png} \\
& $d_1$ = 150m ± 2cm \\
& $d_1$ = 175m ± 3cm \\ \hline
(A)&+ 0.325\\ \hline
(B)&$-$ 0.496\\ \hline
(C)&+ 0.755\\ \hline
(D)&$-$ 0.592 \\ \hline
 & \\ \hline


\end{tabular}
\end{table}



\newpage

\begin{table}[H]
\renewcommand{\arraystretch}{2.5}
\setlength{\tabcolsep}{8pt}
\begin{tabular}{|l|p{15cm}|}
\hline

& \\ \hline
 Q.30 &Independent angles $AOB, BOC$ and $AOC$ were observed as shown in figure. The
standard error of all observations is same. The adjusted values of these angles
using the least squares adjustment are\\ \hline
& $AOB$ = \ang{30;00;20}\includegraphics[scale = 0.9]{LatexImage5.png} \\
& $BOC$ = \ang{30;00;05}\\
& $AOC$ = \ang{60;00;10} \\ \hline
(A)&$AOB$ = \ang{30;00;15} , $BOC$ = \ang{30;00;00}, $AOC$ = \ang{60;00;15}\\ \hline
(B)&$AOB$ = \ang{30;00;10} , $BOC$ = \ang{30;00;05}, $AOC$ = \ang{60;00;15}\\ \hline
(C)&$AOB$ = \ang{30;00;05} , $BOC$ = \ang{30;00;10}, $AOC$ = \ang{60;00;15}\\ \hline
(D)&$AOB$ = \ang{30;00;10} , $BOC$ = \ang{30;00;10}, $AOC$ = \ang{60;00;20} \\ \hline
 & \\
 & \\
 & \\
  & \\
 & \\ \hline


\end{tabular}
\end{table}



\begin{table}[H]
\renewcommand{\arraystretch}{2.5}
\setlength{\tabcolsep}{8pt}
\begin{tabular}{|l|p{15cm}|}
\hline

& \\ \hline
 Q.31 &To reduce the slope distance $(S)$ to an equivalent horizontal distance $(H)$ as shown in the figure given below, the following independent observations were taken.\\
 &S = 29.95 m ± 0.01 m; $\theta = 4\degree 30' ± 10'$ . \\
 &The required precision of computed horizontal distance is ± 0.005 m. Assume a “balanced accuracy” where the contribution to precision of the horizontal distance comes equally from the slope distance and angle measurements. The minimum number of angle observations to achieve the desired precision is\\ 
 & (Given 1 radian = 206265 seconds) \\
\hline
& \includegraphics[]{LatexImage6.png}\\ \hline
(A)&1\\ \hline
(B)&2\\ \hline
(C)&3\\ \hline
(D)&4 \\ \hline
 & \\
 & \\
 & \\ \hline


\end{tabular}
\end{table}


\begin{table}[H]
\renewcommand{\arraystretch}{2.5}
\setlength{\tabcolsep}{8pt}
\begin{tabular}{|l|p{15cm}|}
\hline

& \\ \hline
 Q.32 &Find the best match between remote sensing sensors (Column A) with their
characteristics (Column B) \\
\hline

& \includegraphics[scale = 0.7]{LatexImage14.png}\\ \hline

    
(A)&P\textemdash1, Q\textemdash5, R\textemdash2, S\textemdash3\\ \hline
(B)&P\textemdash3, Q\textemdash2, R\textemdash4, S\textemdash1\\ \hline
(C)&P\textemdash2, Q\textemdash3, R\textemdash1, S\textemdash5\\ \hline
(D)&P\textemdash1, Q\textemdash3, R\textemdash4, S\textemdash5 \\ \hline
& \\ 
& \\ 
& \\ 
& \\ 
& \\ 

 & \\ \hline

\end{tabular}
\end{table}

\newpage
\begin{table}[H]
\renewcommand{\arraystretch}{2.5}
\setlength{\tabcolsep}{8pt}
\begin{tabular}{|l|p{15cm}|}
\hline






\end{tabular}
\end{table}


\begin{table}[H]
\renewcommand{\arraystretch}{2.5}
\setlength{\tabcolsep}{8pt}
\begin{tabular}{|l|p{15cm}|}
\hline
 
Q.33&Find the best match between Column A and Column B \\ \hline
 & \includegraphics[scale = 0.5]{LatexImage15.png}\\ \hline

(A)&P\textemdash5, Q\textemdash4, R\textemdash3, S\textemdash1\\ \hline
(B)&P\textemdash5, Q\textemdash4, R\textemdash2, S\textemdash3\\ \hline
(C)&P\textemdash3, Q\textemdash1, R\textemdash2, S\textemdash4\\ \hline
(D)&P\textemdash2, Q\textemdash3, R\textemdash4, S\textemdash1 \\ \hline
& \\ \hline
 

Q.34&Which of the following factors is/are responsible for ionospheric delay in GNSS
observations? \\ \hline

 (A)&Total electron count in the ionosphere\\ \hline
(B)&Carrier signal frequency\\ \hline
(C)&Size of GPS receivers\\ \hline
(D)&Size and accuracy of atomic clocks \\ \hline


\end{tabular}
\end{table}


\begin{table}[H]
\renewcommand{\arraystretch}{3}
\setlength{\tabcolsep}{8pt}
\begin{tabular}{|l|p{15cm}|}
\hline
 & \\\hline
 
Q.35 &Which of the following statements is/are CORRECT in the context of GPS data
collection methods?\\ \hline 
 & \\ \hline
(A)&CORS (Continuously Operating Reference Station) can be used as a reference
(base) GPS receiver\\ \hline
(B)&Reference (base) receiver should record the observations for longer period as
compared to remote (rover) GPS receiver for applying corrections \\ \hline
(C)&Remote (rover) GPS receiver must always be placed on a known location for
applying the corrections of reference (base) GPS receiver\\ \hline
(D)&Reference (base) and remote (rover) GPS receivers must be placed on top of each
other for applying corrections\\ \hline
 & \\ \hline

Q.36 & Which of the following errors is/are corrected in Differential GPS (DGPS)?\\ \hline
 & \\ \hline
(A)&Tropospheric delays\\ \hline
(B)&Orbital errors\\ \hline
(C)&Ionospheric delays\\ \hline
(D)&Ambiguity in atomic clocks\\ \hline

 

\end{tabular}
\end{table}

\begin{table}[H]
\renewcommand{\arraystretch}{3}
\setlength{\tabcolsep}{8pt}
\begin{tabular}{|l|p{15cm}|}
\hline
 & \\\hline
 
Q.37 &Which of the following statements is/are CORRECT?\\ \hline 
 & \\ \hline
(A)&Network analysis can be done with vector data.\\ \hline
(B)&Linear features are clearly identified as discrete features in vector database.\\ \hline
(C)&Satellite images are in vector format.\\ \hline
(D)&Digital elevation model is in raster format.\\ \hline
 & \\ \hline

Q.38 & In GIS, buffer is a zone with a specified width surrounding a spatial feature. Which
of the following statements regarding buffer is/are CORRECT?\\ \hline
 & \\ \hline
(A)&For a point feature, buffer is an ellipse with minor and major axes as buffer
distances\\ \hline
(B)&For a line feature, buffer is a band with a specified distance created around the line
conforming to the line’s curve\\ \hline
(C)&Buffer zones are polylines\\ \hline
(D)&For a polygon feature, buffer is a belt of a specified distance from the edge of the
polygon and conforming to its shape\\ \hline

 

\end{tabular}
\end{table}

\begin{table}[H]
\renewcommand{\arraystretch}{3}
\setlength{\tabcolsep}{8pt}
\begin{tabular}{|l|p{15cm}|}
\hline
 
Q.39 & Which of the following statements about the Triangulated Irregular Network
(TIN) model is/are INCORRECT?\\ \hline 
 & \\ \hline
(A)&TIN contains irregularly spaced sampled points\\ \hline
(B)&Triangulation is performed to form network of triangles.\\ \hline
(C)&In the TIN model, the edges represent features such as peaks and depression.\\ \hline
(D)&In the TIN model, the vertices represent features such as peaks and depression.\\ \hline
 & \\ \hline

Q.40 & Which of the following statements is/are INCORRECT in the context of GIS?\\ \hline
 & \\ \hline
(A)&CLIP erases a part of one of the input layers.\\ \hline
(B)&SPLIT overlays polygons and keeps all areas in both layers.\\ \hline
(C)&INTERSECT overlays polygons and keeps only the common portions of both
layers.\\ \hline
(D)&UNION overlays polygons and keeps all areas in both layers.\\ \hline
 & \\ 
 
 & \\ \hline

 

\end{tabular}
\end{table}

\begin{table}[H]
\renewcommand{\arraystretch}{3}
\setlength{\tabcolsep}{8pt}
\begin{tabular}{|l|p{15cm}|}
\hline
 
Q.41 & Which of the following is/are method(s) used for compact storage of raster GIS
data?\\ \hline 
 & \\ \hline
(A)&Chain code\\ \hline
(B)&Run-length code\\ \hline
(C)&Quadtree\\ \hline
(D)&Decision-tree\\ \hline
 & \\ \hline

Q.42& Which of the following statements is/are CORRECT?\\ \hline
 & \\ \hline
(A)&CARTOSAT-1 satellite can acquire across-track stereoscopic pairs of images of a
geographical region on the same day.\\ \hline
(B)&CARTOSAT-1 satellite can acquire across-track stereoscopic pairs of images of a
geographical region on successive days.\\ \hline
(C)&CARTOSAT-1 satellite can acquire along-track stereoscopic pairs of images of a
geographical region on the same day.\\ \hline
(D)&CARTOSAT-1 satellite can acquire along-track stereoscopic pairs of images of a
geographical region on successive days.\\ \hline
 
 & \\ \hline

 

\end{tabular}
\end{table}


\begin{table}[H]
\renewcommand{\arraystretch}{3}
\setlength{\tabcolsep}{8pt}
\begin{tabular}{|l|p{15cm}|}
\hline
 
Q.43 & Which of the following statements is/are CORRECT for satellite image
interpretation?\\ \hline 
 & \\ \hline
(A)&SWIR band is sensitive to moisture in soil and vegetation\\ \hline
(B)&Blue band is not useful to discriminate between water and snow\\ \hline
(C)&NIR band is useful to discriminate between land and water\\ \hline
(D)&Green band is useful to discriminate between cloud and snow\\ \hline
 & \\ \hline



Q.44 & Which of the following CANNOT be used as visual interpretation key(s) for
satellite images?\\ \hline 
 & \\ \hline
(A)&Texture\\ \hline
(B)&Projection\\ \hline
(C)&Pattern\\ \hline
(D)&Association\\ \hline
\end{tabular}
\end{table}



\begin{table}[H]
\renewcommand{\arraystretch}{3}
\setlength{\tabcolsep}{8pt}
\begin{tabular}{|l|p{15cm}|}
\hline
 
Q.45 & Which of the following parts of the electromagnetic spectrum is/are used in
satellite remote sensing for earth observation?\\ \hline 
 & \\ \hline
(A)&Visible wavelengths\\ \hline
(B)&Thermal Infrared wavelengths\\ \hline
(C)&Radio wavelengths\\ \hline
(D)&Gamma wavelengths\\ \hline
 & \\ \hline
Q.46& Which of the following CANNOT be used as visual interpretation key(s) for
satellite images?\\\hline
 & \\ \hline
 & Orbital altitude (above earth surface) = 1000 km\\
 &Number of spectral bands = 5\\
 &Number of detectors/CCDs (charged coupled devices) in a row = 4000\\
 &Ground swath = 20 km\\\hline



\end{tabular}
\end{table}

\newpage

\textbf{PART B1: FOR Surveying and Mapping CANDIDATES ONLY} 

\textbf{Q.47 – Q.54 Carry ONE mark Each} \\

\begin{table}[H]
\renewcommand{\arraystretch}{3}
\setlength{\tabcolsep}{8pt}
\begin{tabular}{|l|p{15cm}|}
\hline
 
Q.47& If the plotting accuracy of a map is 0.25 mm and the scale of the same map is
1:100000, what will be the minimum ground distance that can be plotted on the
map?\\ \hline 
 & \\ \hline
(A)&2.5 m\\ \hline
(B)&25 m\\ \hline
(C)&250 m\\ \hline
(D)&2500\\ \hline
 & \\ \hline

Q.48 & The Survey of India toposheet number 43$\frac{D}{6}$ \\ \hline
 & \\ \hline
(A)&1° by 1°\\ \hline
(B)&25' by 25'\\ \hline
(C)&15' by 15'\\ \hline
(D)&7.5' by 7.5'\\ \hline

\end{tabular}
\end{table}

\begin{table}[H]
\renewcommand{\arraystretch}{3}
\setlength{\tabcolsep}{8pt}
\begin{tabular}{|l|p{15cm}|}
\hline
 
Q.49& Universal Transverse Mercator (UTM) is a\\ \hline 
 & \\ \hline
(A)&conical projection\\ \hline
(B)&azimuthal projection\\ \hline
(C)&azimuthal projection\\ \hline
(D)&cylindrical projection\\ \hline
 & \\ \hline

Q.50 & Change Point (CP) in levelling refers to a location where\\ \hline
 & \\ \hline
(A)&only backsight reading is taken\\ \hline
(B)&both backsight and foresight readings are taken\\ \hline
(C)&survey work ends\\ \hline
(D)&staff reading is taken on a benchmark\\ \hline

& \\ 
& \\ \hline

\end{tabular}
\end{table}


\begin{table}[H]
\renewcommand{\arraystretch}{3}
\setlength{\tabcolsep}{8pt}
\begin{tabular}{|l|p{15cm}|}
\hline
 
Q.51& 1 At a fixed instrument location in levelling, if the backsight reading at a point P is
more than the foresight reading at a point Q, then\\ \hline 
 & \\ \hline
(A)&point P has lower elevation than point Q\\ \hline
(B)&point P has higher elevation than point Q\\ \hline
(C)&the elevation difference between P and Q depends on height of the instrument\\ \hline
(D)&the elevation difference between P and Q depends on benchmark elevation\\ \hline
 & \\ \hline

Q.52 &``Transit the telescope'' of a theodolite involves\\ \hline
 & \\ \hline
(A)&rotating the theodolite about its vertical axis\\ \hline
(B)&rotating the telescope about its trunnion axis\\ \hline
(C)&rotating the telescope about its line of collimation\\ \hline
(D)&rotating the theodolite by 90° in horizontal plane\\ \hline
& \\ 
& \\ \hline

\end{tabular}
\end{table}


\begin{table}[H]
\renewcommand{\arraystretch}{3}
\setlength{\tabcolsep}{8pt}
\begin{tabular}{|l|p{15cm}|}
\hline
 
Q.53& Scale of a vertical aerial photograph of an undulating terrain is\\ \hline 
 & \\ \hline
(A)&directly proportional to the height of terrain\\ \hline
(B)&inversely proportional to the focal length of camera lens\\ \hline
(C)&directly proportional to the flying height of aircraft\\ \hline
(D)&uniform throughout the photograph\\ \hline
 & \\ \hline

Q.54 &Isocentre of a tilted photograph is\\ \hline
 & \\ \hline
(A)&intersection of the optical axis of the aerial camera with the plane of the photograph\\ \hline
(B)&the point of aerial photograph where a plumb line dropped from exposure station
pierces the photograph\\ \hline
(C)&angle of tilt of the photograph\\ \hline
(D)&the point on the photograph where the bisector of the angle of tilt meets the
photograph\\ \hline
& \\ \hline

\end{tabular}
\end{table}


\newpage

\textbf{Q.55– Q.65 Carry TWO marks Each}

\begin{table}[H]
\renewcommand{\arraystretch}{3}
\setlength{\tabcolsep}{8pt}
\begin{tabular}{|l|p{15cm}|}
\hline
 
Q.55& The magnetic bearing of a line in the year 1990 was found to be N 40°30' W and
magnetic declination was 3°30' E. If the present magnetic declination is 2°10' W,
the magnetic bearing now (in reduced bearing system) would be\\ \hline 
 & \\ \hline
(A)&S 30°50' W\\ \hline
(B)&N 30°50' W\\ \hline
(C)&S 34°50' W\\ \hline
(D)&N 34°50' W\\ \hline
 & \\ \hline

Q.56 &Map (A) represents all the roads, street lights, trees and buildings of a campus of
5 ${km}^2$. Another map (B) represents the forest and agricultural area of a district of
10000 ${km}^2$. Considering the physical size of both the maps (A) \& (B) same, which
of the following statements is/are CORRECT?\\ \hline
 & \\ \hline
(A)&Map (A) is at relatively large scale\\ \hline
(B)&Map (B) is at relatively large scale\\ \hline
(C)&Both maps are at same scale\\ \hline
(D)&Both maps are not at same scale\\ \hline


\end{tabular}
\end{table}

\begin{table}[H]
\renewcommand{\arraystretch}{3}
\setlength{\tabcolsep}{8pt}
\begin{tabular}{|l|p{15cm}|}
\hline
 
Q.57& Which of the following statements is/are CORRECT?\\ \hline 
 & \\ \hline
(A)&Triangulation is preferred in plain areas, whereas trilateration is preferred in hilly
areas\\ \hline
(B)&Triangulation is preferred in hilly areas, whereas trilateration is preferred in plain
areas\\ \hline
(C)&In triangulation, the angles are measured with greater accuracy, while in
trilateration, sides are measured with greater accuracy\\ \hline
(D)&In trilateration, the angles are measured with greater accuracy, while in
triangulation, sides of triangles are measured with greater accuracy\\ \hline
 & \\ \hline

Q.58 &Which of the following statements is/are CORRECT?\\ \hline
 & \\ \hline
(A)&Bowditch rule in traverse adjustment is particularly useful, where angular and linear
measurements are equally precise\\ \hline
(B)&Transit rule in traverse adjustment is particularly useful, where angular
measurements are more precise than linear measurements\\ \hline
(C)&In Bowditch rule, the traverse adjustment is done using arithmetic sum of latitudes
or departures of the traverse\\ \hline
(D)&In Transit rule, the traverse adjustment is done using perimeter of the traverse\\ \hline


\end{tabular}
\end{table}

\begin{table}[H]
\renewcommand{\arraystretch}{3}
\setlength{\tabcolsep}{8pt}
\begin{tabular}{|l|p{15cm}|}
\hline
 
Q.59& Consider a point A on the surface of Earth, its elevation with respect to EGM2008
(geoid) is 95.5 m. The geoidal undulation at point A is 4.5 m. The orthometric height
of point A is \rule{2cm}{0.15mm} m (rounded off to one decimal place).\\ \hline 
 & \\ \hline
Q.60&If the longitudinal overlap in aerial photographs is kept as 65\%, the common overlap
(superlap) between three successive photographs is \rule{2cm}{0.15mm}\% \textit{(in integer)}.\\ \hline
& \\ \hline
Q.61&The Representative Fraction (RF) of the graphical scale given below is 1/X, where X is \rule{2cm}{0.15mm}($in$ $integer$).\\ \hline
&\includegraphics[scale=0.6]{LatexImage7.png}\\ \hline
 & \\ \hline
Q.62 &The combined correction for curvature of Earth and refraction in levelling for a
distance of 6 km would be \rule{2cm}{0.15mm}m \textit{(rounded off to two decimal places)}.\\ 
 & Assume the radius of earth is 6370 km. \\ \hline
 & \\ \hline
Q.63&In tangential method of tacheometry, two vanes in a staff were fixed at a distance
of 1.0 m with the bottom vane fixed at 1.0 m. The levelling staff was held vertical
at a point P and the vertical angles of the vanes observed were 5°30' and 3°15',
respectively. The vertical distance between the instrument axis and the bottom vane
would be \rule{2cm}{0.15mm} \textit{(rounded off to two decimal places)} \\ \hline
& \\ \hline


\end{tabular}
\end{table}


\begin{table}[H]
\renewcommand{\arraystretch}{3}
\setlength{\tabcolsep}{8pt}
\begin{tabular}{|l|p{15cm}|}
\hline
 
Q.64& A line measures 15 cm on an aerial photograph, while it measures 5 cm on a map
at 1:24000 scale. The photograph was taken using a camera lens of 20 cm focal
length. Average elevation of terrain is 240 m above mean sea level. The flying
height of the aircraft above mean sea level is\rule{2cm}{0.15mm} m \textit{(in integer)}.\\ \hline 
 & \\ \hline
Q.65&A high tower appeared on an aerial photograph taken at 1000 m above mean sea
level with a camera lens of 15 cm focal length. The radial distances of the top and
bottom images of the tower from principal point of photograph are 92.6 mm and
78.3 mm, respectively. If the average elevation of terrain is 300 m above mean sea
level, then the height of the tower above ground is \rule{2cm}{0.15mm} m \textit{(rounded off to the nearest integer)} \\ \hline
& \\ \hline

\end{tabular}
\end{table}


\newpage

\textbf{PART B2: FOR Image Processing and Analysis CANDIDATES ONLY} 

\textbf{Q.66 – Q.73 Carry ONE mark Each}

\begin{table}[H]
\renewcommand{\arraystretch}{3}
\setlength{\tabcolsep}{8pt}
\begin{tabular}{|l|p{15cm}|}
\hline
 
Q.66& A four-band multispectral image of size 64 × 64 pixels has 560 header bytes. The
per pixel depth of the image is 2 bytes. The total number of bytes required to store
this image on the disk in the Band Interleaved by Line (BIL) format will be\\ \hline 
(A)&33328\\ \hline
(B)&32338\\ \hline
(C)&33823\\ \hline
(D)&33283\\ \hline
 & \\ \hline

Q.67 &A one-dimensional normalized kernel $ \frac{1}{4} [\begin{smallmatrix}1 & 2 & 1\\ \end{smallmatrix} ] $ is convolved with an image to produce an intermediate result. The intermediate image of this operation is again convolved with the same kernel to produce a final result. The equivalent kernel to
achieve the same final result in one step from the original image is given as\\ \hline
 & \\ \hline
(A)&$ \frac{1}{16}[\begin{smallmatrix}1 & 4 & 6&4&1\\ \end{smallmatrix} ] $\\ \hline
(B)&$ \frac{1}{16}[ \begin{smallmatrix}1 & 2 & 1&2&1\\ \end{smallmatrix} ] $\\ \hline
(C)&$ \frac{1}{8}[ \begin{smallmatrix}1 & 2 & 4&2&1\\ \end{smallmatrix} ] $\\ \hline
(D)&$ \frac{1}{10}[ \begin{smallmatrix}1 & 2 & 4&2&1\\ \end{smallmatrix} ] $\\ \hline

\end{tabular}
\end{table}
 \newpage
 
\begin{table}[H]
\renewcommand{\arraystretch}{3}
\setlength{\tabcolsep}{8pt}
\begin{tabular}{|l|p{15cm}|}
\hline

Q.68 &The histogram equalization applied to a digital image generally DOES NOT yield
a truly uniform histogram of the transformed image due to\\ \hline
    & \\ \hline
(A)&discrete nature of pixel values\\ \hline
(B)&poor contrast of the original image\\ \hline
(C)&low frequency image information\\ \hline
(D)&presence of edges\\ \hline
& \\ 
& \\ 
& \\ 
& \\ 
& \\ 
& \\ 
& \\ 
& \\ 
& \\ \hline

\end{tabular}
\end{table} 
\newpage

\begin{table}[H]
\renewcommand{\arraystretch}{3}
\setlength{\tabcolsep}{8pt}
\begin{tabular}{|l|p{15cm}|}
\hline
 
Q.69&Which type of contrast stretching is represented by the following figure?\\ \hline 
& \includegraphics[scale = 0.6]{LatexImage8.png} \\ \hline
(A)&Linear contrast stretch\\ \hline
(B)&Multiple linear stretch\\ \hline
(C)&Logarithmic stretch\\ \hline
(D)&Gaussian stretch\\ \hline
 & \\ \hline

Q.70 &Contrast enhancement is a type of \rule{2cm}{0.15mm} enhancement.\\ \hline
 & \\ \hline
(A)&spectral\\ \hline
(B)&spatial\\ \hline
(C)&radiometric\\ \hline
(D)&temporal\\ \hline

\end{tabular}
\end{table}
 \newpage

 \begin{table}[H]
\renewcommand{\arraystretch}{3}
\setlength{\tabcolsep}{8pt}
\begin{tabular}{|l|p{15cm}|}
\hline
 
Q.71&\rule{2cm}{0.15mm} is a raster image resampling technique that DOES NOT alter any of the
output cell values from the input raster dataset.\\ \hline 
&  \\ \hline
(A)&Nearest neighbor\\ \hline
(B)&Cubic convolution\\ \hline
(C)&Bilinear\\ \hline
(D)&Kriging\\ \hline
 & \\ \hline

Q.72 &De-stripping in radiometric correction is used to correct a type of\\ \hline
 & \\ \hline
(A)&sensor defect\\ \hline
(B)&atmospheric effect\\ \hline
(C)&path radiance\\ \hline
(D)&geometric error\\ \hline
& \\
& \\ \hline

\end{tabular}
\end{table}
 \newpage


\begin{table}[H]
\renewcommand{\arraystretch}{3}
\setlength{\tabcolsep}{8pt}
\begin{tabular}{|l|p{15cm}|}
\hline

Q.73 &The figure given below shows the Fourier spectrum obtained by applying filter on
a remote sensing image in frequency domain. Zone A represents the location of
\rule{2cm}{0.15mm} components.\\ \hline
    & \includegraphics[scale = 1]{LatexImage9.png}\\ \hline
(A)&low frequency\\ \hline
(B)&mid frequency\\ \hline
(C)&mid to high frequency\\ \hline
(D)&high frequency\\ \hline
& \\ 
& \\ 
& \\ 
& \\ 
& \\ \hline

\end{tabular}
\end{table} 
\newpage

\textbf{Q.74– Q.84 Carry TWO marks Each}

 \begin{table}[H]
\renewcommand{\arraystretch}{3}
\setlength{\tabcolsep}{8pt}
\begin{tabular}{|l|p{15cm}|}
\hline
 
Q.74&For the following covariance matrix ($\Sigma$) of a multispectral image, which of the
statements is/are INCORRECT?\\ \hline 
&  \includegraphics[scale=1]{LatexImage10.png}\\ \hline
(A)&band-1 and band-2 have maximum correlation\\ \hline
(B)&band-2 and band-3 are least correlated\\ \hline
(C)&band-3 conveys the maximum information content\\ \hline
(D)&band-1 conveys the minimum information content\\ \hline
& \\ \hline

Q.75 &Which of the following statistical measures CANNOT be computed from the
multispectral image histograms?\\ \hline
 
(A)&Mean, skewness, kurtosis\\ \hline
(B)&Covariance matrix\\ \hline
(C)&Co-occurrence matrix\\ \hline
(D)&Correlation matrix\\ \hline


\end{tabular}
\end{table}


 \begin{table}[H]
\renewcommand{\arraystretch}{3}
\setlength{\tabcolsep}{8pt}
\begin{tabular}{|l|p{15cm}|}
\hline
 
Q.76&Which of the following statements about Principal Component Analysis (PCA)
is/are CORRECT?\\ \hline 
&  \\ \hline
(A)&A two-dimensional data set can have up to four principal components.\\ \hline
(B)&The first principal component accounts for the majority of conceivable data
variation.\\ \hline
(C)&The second principal component attempts to encapsulate the mode of the data.\\ \hline
(D)&The transformed principal components are linear combinations of the original
variables and are orthogonal.\\ \hline
 & \\ \hline

Q.77 &In the context of satellite image classification, which of the following statements
is/are CORRECT?\\ \hline
 & \\ \hline
(A)&Both ANN and Fuzzy C-means clustering are parametric classifiers\\ \hline
(B)&Both ANN and Fuzzy C-means clustering are non-parametric classifiers\\ \hline
(C)&ANN can be both supervised and unsupervised classification method\\ \hline
(D)&Fuzzy C-means clustering is a supervised classification method\\ \hline

& \\ \hline

\end{tabular}
\end{table}


\begin{table}[H]
\renewcommand{\arraystretch}{3}
\setlength{\tabcolsep}{8pt}
\begin{tabular}{|l|p{15cm}|}
\hline

Q.78 &Which of the following filters can be used to suppress the low frequency component
of a raster image?\\ \hline
    & \includegraphics[scale = 0.7]{LatexImage11.png}\\ \hline
(A)&(i)\\ \hline
(B)&(ii)\\ \hline
(C)&(iii)\\ \hline
(D)&(iv)\\ \hline
& \\ 
& \\ 
& \\ 
& \\ \hline

\end{tabular}
\end{table} 
\newpage


\begin{table}[H]
\renewcommand{\arraystretch}{3}
\setlength{\tabcolsep}{8pt}
\begin{tabular}{|l|p{15cm}|}
\hline
 
Q.79&Which of the following statements about image ratio is/are CORRECT?\\ \hline 
&  \includegraphics[scale=1]{LatexImage10.png}\\ \hline
(A)&It cannot be used to suppress the effects of topography\\ \hline
(B)&It cannot be used to suppress the effects of differential sun-illumination\\ \hline
(C)&It helps in suppressing the effects of differential sun-illumination\\ \hline
(D)&It helps in suppressing the effects of topography\\ \hline
& \\ \hline
\end{tabular}
\end{table} 
\newpage

\begin{table}[H]
\renewcommand{\arraystretch}{3}
\setlength{\tabcolsep}{8pt}
\begin{tabular}{|l|p{15cm}|}
\hline

Q.80 &Which of the following statistical classification algorithms is/are represented by the
figure given below?\\ \hline
& \includegraphics[]{LatexImage12.png} \\ \hline
(A)&Minimum distance to mean classification\\ \hline
(B)&Parallelepiped classification\\ \hline
(C)&Maximum likelihood classification\\ \hline
(D)&k-means clustering\\ \hline


\end{tabular}
\end{table}

\newpage

\begin{table}[H]
\renewcommand{\arraystretch}{3}
\setlength{\tabcolsep}{8pt}
\begin{tabular}{|l|p{15cm}|}
\hline
 & \\ \hline
Q.81& Using the given 3 × 3 pixel kernel and original image and applying the concept of
convolution, the value of central pixel of the output image is \rule{2cm}{0.15mm} \textit{(in integer)} \\ \hline
& \includegraphics[]{LatexImage13.png} \\ \hline
& \\ \hline
Q.82&A four-band multispectral image with pixel size of 50 m × 50 m covers a ground
area of 20 km × 20 km. If the radiometric resolution of the satellite data is 8 bits,
then the uncompressed satellite image contains\rule{2cm}{0.15mm}  kilobytes (kB) of data \textit{(in integer)}.\\ \hline

& \\ \hline
Q.83&In spatial interpolation using coordinate transformations for image-to-map
rectification, the minimum number of ground control points (GCPs) required to
perform a third-order transformation is \rule{2cm}{0.15mm}($in$ $integer$).\\ \hline
 & \\ \hline
 
Q.84&IIn an image with 6-bit quantization level, the pixel values of a scene are between 25
and 55. A linear contrast stretch is applied to the image covering the full dynamic
range. A pixel value 40 in the original image will be mapped to \rule{2cm}{0.15mm} \textit{(rounded off to nearest integer)} in the stretched image. \\ \hline
& \\ \hline


\end{tabular}
\end{table}



\end{document}
