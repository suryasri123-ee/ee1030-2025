\documentclass{beamer}
\usepackage[utf8]{inputenc}

\usetheme{Madrid}
\usecolortheme{default}
\usepackage{amsmath,amssymb,amsfonts,amsthm}
\usepackage{mathtools}
\usepackage{txfonts}
\usepackage{tkz-euclide}
\usepackage{listings}
\usepackage{adjustbox}
\usepackage{array}
\usepackage{tabularx}
\usepackage{gvv}
\usepackage{lmodern}
\usepackage{circuitikz}
\usepackage{tikz}
\usepackage{graphicx}

\setbeamertemplate{page number in head/foot}[totalframenumber]

\usepackage{tcolorbox}
\tcbuselibrary{minted,breakable,xparse,skins}



\definecolor{bg}{gray}{0.95}
\DeclareTCBListing{mintedbox}{O{}m!O{}}{%
  breakable=true,
  listing engine=minted,
  listing only,
  minted language=#2,
  minted style=default,
  minted options={%
    linenos,
    gobble=0,
    breaklines=true,
    breakafter=,,
    fontsize=\small,
    numbersep=8pt,
    #1},
  boxsep=0pt,
  left skip=0pt,
  right skip=0pt,
  left=25pt,
  right=0pt,
  top=3pt,
  bottom=3pt,
  arc=5pt,
  leftrule=0pt,
  rightrule=0pt,
  bottomrule=2pt,
  toprule=2pt,
  colback=bg,
  colframe=orange!70,
  enhanced,
  overlay={%
    \begin{tcbclipinterior}
    \fill[orange!20!white] (frame.south west) rectangle ([xshift=20pt]frame.north west);
    \end{tcbclipinterior}},
  #3,
}
\lstset{
    language=C,
    basicstyle=\ttfamily\small,
    keywordstyle=\color{blue},
    stringstyle=\color{orange},
    commentstyle=\color{green!60!black},
    numbers=left,
    numberstyle=\tiny\color{gray},
    breaklines=true,
    showstringspaces=false,
}
%------------------------------------------------------------
%This block of code defines the information to appear in the
%Title page
\title %optional
{2.10.55}
\date{August 26,2025}
%\subtitle{A short story}

\author % (optional)
{Harsha-EE25BTECH11026}



\begin{document}


\frame{\titlepage}
\begin{frame}{Question}
The edges of a parallelopiped are of unit length and are parallel to non-coplanar unit vectors $\hat{a},\hat{b},\hat{c}$ such that $\hat{a}.\hat{b}=\hat{b}.\hat{c}=\hat{c}.\hat{a}=\frac{1}{2}$ . Then, the volume of the parallelopiped is

\begin{enumerate}
    \item $\frac{1}{\sqrt{2}}$
    \item $\frac{1}{2\sqrt{2}}$
    \item $\frac{\sqrt{3}}{2}$
    \item $\frac{1}{\sqrt{3}}$
\end{enumerate}
\end{frame}



\begin{frame}{Theoretical Solution}
According to the question,the edges of the parallelopiped are parallel to the unit vectors $\hat{a},\hat{b},\hat{c}$ and\\
\begin{align*}
    \hat{a}^T\hat{b}=\hat{b}^T\hat{c}=\hat{c}^T\hat{a}=\frac{1}{2}
\end{align*}
\end{frame}

\begin{frame}{Equation}

As we know that the volume of parallelopiped is given by
\begin{align*}
    V=[\vec{a}\;\vec{b}\;\vec{c}]
\end{align*}
and
\begin{align*}
    [\vec{a}\;\vec{b}\;\vec{c}][\vec{a}\;\vec{b}\;\vec{c}]^T=\vec{G}
\end{align*}
where $\vec{G}$ is the Gram Matrix.
\end{frame}

\begin{frame}{Theoretical Solution}
\begin{align*}
    \therefore \vec{G}=\myvec{\hat{a}^T\hat{a}&&\hat{a}^T\hat{b}&&\hat{a}^T\hat{c} \\ \hat{b}^T\hat{a}&& \hat{b}^T\hat{b}&& \hat{b}^T\hat{c} \\ \hat{c}^T\hat{a}&& \hat{c}^T\hat{b}&& \hat{c}^T\hat{c}}=\myvec{1&&\frac{1}{2}&&\frac{1}{2}\\ \frac{1}{2}&&1&&\frac{1}{2}\\ \frac{1}{2}&&\frac{1}{2}&&1}
\end{align*}
For calculating the det($\vec{G}$), we can use the concept of eigen values.
\end{frame}

\begin{frame}{Definition}

Eigen values are those scalars which satisfies the following condition,
For any non-zero eigen-vector $\vec{v}$ and coefficient matrix $\vec{M}$,
\begin{align*}
    \vec{M}\vec{v}=\lambda \vec{v} \; , where \; \lambda \;is\; an \;eigen\; value.
\end{align*}
\end{frame}

\begin{frame}{Theoretical solution}
\begin{align*}
     \vec{G}=(1-\rho)\vec{I}\;+\rho\;\vec{1}\vec{1}^T, \; where\, \rho=\frac{1}{2}\;and\;\vec{1}=\myvec{1&&1&&1}
\end{align*}
Let $\vec{1}\vec{1}^T=\vec{J}$. As we could see that the eigen-vector of $\vec{J}$ is $\vec{1}$ and by the rule,
\begin{align*}
    \myvec{1&&1&&1\\1&&1&&1\\1&&1&&1}\myvec{1\\1\\1}=\myvec{3\\3\\3}=3\vec{1}
\end{align*}


\end{frame}


\begin{frame}{Theoretical solution}

So, 3 is a eigen value of $\vec{J}$. Also, we know that the sum of eigen values is equal to trace of a matrix,we can say that the sum of the other eigen values would be 0. Also, we know that any orthogonal vector to $\vec{1}$ ,say $\myvec{-1&&1&&0}^T$ ,
\begin{align*}
    \myvec{1&&1&&1\\1&&1&&1\\1&&1&&1}\myvec{-1\\1\\0}=\vec{0}
\end{align*}
yields 0.Thus we can say that 0 is also one of the eigen value of J. As sum of the other eigen values other than 3 is zero, the other eigen value must be zero.
\begin{align*}
    \therefore eigen\;values\;of\;\vec{J}\;are\;\{3,0,0\}
\end{align*}
\end{frame}


\begin{frame}{Theoretical Solution}
Modifying the above equation on $\vec{G}$,
\begin{align*}
    \therefore \vec{G}\vec{v}=\frac{1}{2}\vec{I}\vec{v}+\frac{1}{2}\vec{J}\vec{v}
\end{align*}
\begin{align*}
    \implies G\vec{v}=\frac{(1+\mu)}{2}\vec{v}
\end{align*}
where $\mu$ is the eigen value of $\vec{J}$. Here the eigen value of $\vec{G}$ is $\frac{1+\mu}{2}$ and substituting the obtained eigen values of $\vec{J}$ in this equation, we get the eigen values of $\vec{G}$ to be $\{2,\frac{1}{2},\frac{1}{2}\}$
\end{frame}

\begin{frame}{Theoretical Solution}
As we know that for eigen values of $\vec{G}$ being $\{\mu_1,\mu_2,\mu_3\}$
\begin{align*}
    det(\vec{G})=\mu_1\mu_2\mu_3
\end{align*}
\begin{align*}
    \therefore det(\vec{G})=2\times\frac{1}{2}\times\frac{1}{2}=\frac{1}{2}
\end{align*}
\begin{align*}
    \implies V=\sqrt{det(\vec{G})}=\frac{1}{\sqrt{2}} \; units
\end{align*}
\end{frame}

\begin{frame}[fragile]
    \frametitle{C Code - Volume of parallelopiped}

    \begin{lstlisting}
#include <stdio.h>
#include <math.h>

// Jacobi eigenvalue algorithm for 3x3 symmetric matrix
// Finds eigenvalues of G
void jacobi_eigenvalues(double G[3][3], double eigenvalues[3]) {
    double A[3][3];
    for(int i=0;i<3;i++) {
        for(int j=0;j<3;j++) A[i][j] = G[i][j];
    }


        
    \end{lstlisting}
\end{frame}

\begin{frame}[fragile]
    \frametitle{C Code - Volume of parallelopiped}

    \begin{lstlisting}

    double eps = 1e-12;
    for(int iter=0; iter<100; iter++) {
        // find largest off-diagonal element
        int p=0,q=1;
        double max = fabs(A[0][1]);
        for(int i=0;i<3;i++) {
            for(int j=i+1;j<3;j++) {
                if(fabs(A[i][j]) > max) { max=fabs(A[i][j]); p=i; q=j; }
            }
        }
        
    \end{lstlisting}
\end{frame}

\begin{frame}[fragile]
    \frametitle{C Code - Volume of parallelopiped}

    \begin{lstlisting}

        if(max < eps) break;

        double theta = 0.5 * atan2(2*A[p][q], A[q][q]-A[p][p]);
        double c = cos(theta), s = sin(theta);

        // rotate
        double app = c*c*A[p][p] - 2*s*c*A[p][q] + s*s*A[q][q];
        double aqq = s*s*A[p][p] + 2*s*c*A[p][q] + c*c*A[q][q];
        A[p][q] = A[q][p] = 0.0;
        A[p][p] = app;
        A[q][q] = aqq;

    \end{lstlisting}
\end{frame}
\begin{frame}[fragile]
    \frametitle{C Code - Volume of parallelopiped}

    \begin{lstlisting}
        for(int k=0;k<3;k++) {
            if(k!=p && k!=q) {
                double akp = c*A[k][p] - s*A[k][q];
                double akq = s*A[k][p] + c*A[k][q];
                A[k][p] = A[p][k] = akp;
                A[k][q] = A[q][k] = akq;
            }
        }
    }

    // diagonal entries are eigenvalues
    for(int i=0;i<3;i++) eigenvalues[i] = A[i][i];
}


    \end{lstlisting}
\end{frame}
\begin{frame}[fragile]
    \frametitle{C Code - Volume of parallelopiped}

    \begin{lstlisting}

// Volume from eigenvalues
double parallelepiped_volume(double G[3][3]) {
    double eigenvalues[3];
    jacobi_eigenvalues(G, eigenvalues);

    double det = eigenvalues[0] * eigenvalues[1] * eigenvalues[2];
    if(det < 0) det = -det;
    return sqrt(det);
}

    \end{lstlisting}
\end{frame}



\begin{frame}[fragile]
    \frametitle{Python+C Code}
    \begin{lstlisting}
import ctypes
import matplotlib as mp
mp.use('TkAgg')
from mpl_toolkits.mplot3d.art3d import Poly3DCollection
import numpy as np
import matplotlib.pyplot as plt

lib = ctypes.CDLL("./libvolume.so")
lib.parallelepiped_volume.restype = ctypes.c_double
lib.parallelepiped_volume.argtypes = [ctypes.c_double * 3 * 3]

G = np.array([[1,0.5,0.5],
              [0.5,1,0.5],
              [0.5,0.5,1]], dtype=np.float64)

G_ctypes = (ctypes.c_double * 3 * 3)(*[(ctypes.c_double * 3)(*row) for row in G])
    \end{lstlisting}
\end{frame}

\begin{frame}[fragile]
    \frametitle{Python+C Code}
    \begin{lstlisting}
volume = lib.parallelepiped_volume(G_ctypes)
print("Volume =", round(volume,4),"units")

# Define three unit vectors with given dot products = 0.5
a = np.array([1, 0, 0])
b = np.array([0.5, np.sqrt(3)/2, 0])                 
c = np.array([0.5, 1/2*np.sqrt(3), np.sqrt(2/3)])   

    \end{lstlisting}
\end{frame}

\begin{frame}[fragile]
    \frametitle{Python+C Code}
    \begin{lstlisting}
# Vertices
O = np.array([0,0,0])
A = a
B = b
C = c
AB = a+b
AC = a+c
BC = b+c
ABC = a+b+c

vertices = [O, A, B, C, AB, AC, BC, ABC]
labels = ["O","A","B","C","A+B","A+C","B+C","A+B+C"]

    \end{lstlisting}
\end{frame}

\begin{frame}[fragile]
    \frametitle{Python+C Code}
    \begin{lstlisting}
# Define faces
faces = [
    [O, A, AB, B],
    [O, A, AC, C],
    [O, B, BC, C],
    [A, AB, ABC, AC],
    [B, AB, ABC, BC],
    [C, AC, ABC, BC]
]

# Plot
fig = plt.figure(figsize=(9,7))
ax = fig.add_subplot(111, projection='3d')

    \end{lstlisting}
\end{frame}

\begin{frame}[fragile]
    \frametitle{Python+C Code}
    \begin{lstlisting}
# Draw faces
poly3d = [[list(v) for v in face] for face in faces]
ax.add_collection3d(Poly3DCollection(poly3d, alpha=0.3, facecolor='cyan'))

# Draw vectors
ax.quiver(0,0,0, a[0],a[1],a[2], color='r', linewidth=2, label='a')
ax.quiver(0,0,0, b[0],b[1],b[2], color='g', linewidth=2, label='b')
ax.quiver(0,0,0, c[0],c[1],c[2], color='b', linewidth=2, label='c')

# Label vertices
for i, v in enumerate(vertices):
    ax.text(v[0], v[1], v[2], labels[i], fontsize=10, color='black')


    \end{lstlisting}
\end{frame}

\begin{frame}[fragile]
    \frametitle{Python+C Code}
    \begin{lstlisting}
# Labels
ax.set_xlabel("X-axis")
ax.set_ylabel("Y-axis")
ax.set_zlabel("Z-axis")
ax.set_title("Parallelepiped formed by unit vectors a, b, c")

ax.set_box_aspect([1,1,1])  # equal aspect
ax.legend()
plt.savefig("/home/user/Matrix/Matgeo_assignments/2.10.55/figs/Figure-1.png")
plt.show()
    \end{lstlisting}
\end{frame}


\begin{frame}[fragile]
    \frametitle{Python Code}
    \begin{lstlisting}
import numpy as np
import matplotlib.pyplot as plt
import matplotlib as mp
mp.use('TkAgg')
from mpl_toolkits.mplot3d.art3d import Poly3DCollection

# Define three unit vectors with given dot products = 0.5
a = np.array([1, 0, 0])
b = np.array([0.5, np.sqrt(3)/2, 0])                 
c = np.array([0.5, 1/2*np.sqrt(3), np.sqrt(2/3)])   
# Gram matrix
G = np.array([[np.dot(a,a), np.dot(a,b), np.dot(a,c)],
              [np.dot(b,a), np.dot(b,b), np.dot(b,c)],
              [np.dot(c,a), np.dot(c,b), np.dot(c,c)]])

    \end{lstlisting}
\end{frame}

\begin{frame}[fragile]
    \frametitle{Python Code}
    \begin{lstlisting}
# Volume of parallelepiped
volume = np.sqrt(np.linalg.det(G))
print("Volume of parallelepiped =", round(volume,4))

# Vertices
O = np.array([0,0,0])
A = a
B = b
C = c
AB = a+b
AC = a+c
BC = b+c
ABC = a+b+c

vertices = [O, A, B, C, AB, AC, BC, ABC]
labels = ["O","A","B","C","A+B","A+C","B+C","A+B+C"]


    \end{lstlisting}
\end{frame}

\begin{frame}[fragile]
    \frametitle{Python Code}
    \begin{lstlisting}
# Define faces
faces = [
    [O, A, AB, B],
    [O, A, AC, C],
    [O, B, BC, C],
    [A, AB, ABC, AC],
    [B, AB, ABC, BC],
    [C, AC, ABC, BC]
]

# Plot
fig = plt.figure(figsize=(9,7))
ax = fig.add_subplot(111, projection='3d')

# Draw faces
poly3d = [[list(v) for v in face] for face in faces]
ax.add_collection3d(Poly3DCollection(poly3d, alpha=0.3, facecolor='cyan'))

    \end{lstlisting}
\end{frame}

\begin{frame}[fragile]
    \frametitle{Python Code}
    \begin{lstlisting}
# Draw vectors
ax.quiver(0,0,0, a[0],a[1],a[2], color='r', linewidth=2, label='a')
ax.quiver(0,0,0, b[0],b[1],b[2], color='g', linewidth=2, label='b')
ax.quiver(0,0,0, c[0],c[1],c[2], color='b', linewidth=2, label='c')

# Label vertices
for i, v in enumerate(vertices):
    ax.text(v[0], v[1], v[2], labels[i], fontsize=10, color='black')
    \end{lstlisting}
\end{frame}

\begin{frame}[fragile]
    \frametitle{Python Code}
    \begin{lstlisting}
# Labels
ax.set_xlabel("X-axis")
ax.set_ylabel("Y-axis")
ax.set_zlabel("Z-axis")
ax.set_title("Parallelepiped formed by unit vectors a, b, c")

ax.set_box_aspect([1,1,1])  # equal aspect
ax.legend()
plt.savefig("/home/user/Matrix/Matgeo_assignments/2.10.55/figs/Figure-1.png")
plt.show()

    \end{lstlisting}
\end{frame}

\begin{frame}{Plot}
    \begin{figure}[H]
    \centering
    \includegraphics[width=0.8\columnwidth]{figs/Figure-1.png}
    \label{fig:1}
\end{figure}
\end{frame}


\end{document}