\let\negmedspace\undefined
\let\negthickspace\undefined
\documentclass[journal]{IEEEtran}
\usepackage[a5paper, margin=10mm, onecolumn]{geometry}
%\usepackage{lmodern} % Ensure lmodern is loaded for pdflatex
\usepackage{tfrupee} % Include tfrupee package

\setlength{\headheight}{1cm} % Set the height of the header box
\setlength{\headsep}{0mm}     % Set the distance between the header box and the top of the text

\usepackage{gvv-book}
\usepackage{gvv}
\usepackage{cite}
\usepackage{amsmath,amssymb,amsfonts,amsthm}
\usepackage{algorithmic}
\usepackage{graphicx}
\usepackage{textcomp}
\usepackage{xcolor}
\usepackage{txfonts}
\usepackage{listings}
\usepackage{enumitem}
\usepackage{mathtools}
\usepackage{gensymb}
\usepackage{comment}
\usepackage[breaklinks=true]{hyperref}
\usepackage{tkz-euclide} 
\usepackage{listings}
% \usepackage{gvv}                                        
\def\inputGnumericTable{}                                 
\usepackage[latin1]{inputenc}                                
\usepackage{color}                                            
\usepackage{array}                                            
\usepackage{longtable}                                       
\usepackage{calc}                                             
\usepackage{multirow}                                         
\usepackage{hhline}                                           
\usepackage{ifthen}                                           
\usepackage{lscape}
\usepackage{circuitikz}
\tikzstyle{block} = [rectangle, draw, fill=blue!20, 
    text width=4em, text centered, rounded corners, minimum height=3em]
\tikzstyle{sum} = [draw, fill=blue!10, circle, minimum size=1cm, node distance=1.5cm]
\tikzstyle{input} = [coordinate]
\tikzstyle{output} = [coordinate]


\begin{document}

\bibliographystyle{IEEEtran}
\vspace{3cm}

\title{2.10.55}
\author{EE25BTECH11026-Harsha}
 \maketitle
% \newpage
% \bigskip
{\let\newpage\relax\maketitle}

\renewcommand{\thefigure}{\theenumi}
\renewcommand{\thetable}{\theenumi}
\setlength{\intextsep}{10pt} % Space between text and floats


\numberwithin{equation}{enumi}
\numberwithin{figure}{enumi}
\renewcommand{\thetable}{\theenumi}

\textbf{Question}:\\
The edges of a parallelopiped are of unit length and are parallel to non-coplanar unit vectors $\hat{a},\hat{b},\hat{c}$ such that $\hat{a}.\hat{b}=\hat{b}.\hat{c}=\hat{c}.\hat{a}=\frac{1}{2}$ . Then, the volume of the parallelopiped is
\begin{multicols}{4}
\begin{enumerate}
    \item $\frac{1}{\sqrt{2}}$
    \item $\frac{1}{2\sqrt{2}}$
    \item $\frac{\sqrt{3}}{2}$
    \item $\frac{1}{\sqrt{3}}$
\end{enumerate}
\end{multicols}
\solution \\
Let us solve the given equation theoretically and then verify the solution computationally.\\
\\
According to the question,the edges of the parallelopiped are parallel to the unit vectors $\hat{a},\hat{b},\hat{c}$ and\\
\begin{align*}
    \hat{a}.\hat{b}=\hat{b}.\hat{c}=\hat{c}.\hat{a}=\frac{1}{2}
\end{align*}
As we know that the volume of parallelopiped is given by
\begin{align*}
    V=|[\vec{a}\;\vec{b}\;\vec{c}]|
\end{align*}
and
\begin{align*}
    [\vec{a}\;\vec{b}\;\vec{c}][\vec{a}\;\vec{b}\;\vec{c}]^T=G
\end{align*}
where G is the Gram Matrix.
\begin{align*}
    \therefore G=\myvec{\hat{a}.\hat{a}&&\hat{a}.\hat{b}&&\hat{a}.\hat{c} \\ \hat{b}.\hat{a}&& \hat{b}.\hat{b}&& \hat{b}.\hat{c} \\ \hat{c}.\hat{a}&& \hat{c}.\hat{b}&& \hat{c}.\hat{c}}=\myvec{1&&\frac{1}{2}&&\frac{1}{2}\\ \frac{1}{2}&&1&&\frac{1}{2}\\ \frac{1}{2}&&\frac{1}{2}&&1}
\end{align*}
For calculating the det(G), we can use the concept of eigen values.\\
\\
Eigen values are those scalars which satisfies the following condition,
For any non-zero eigen-vector $\vec{v}$ and coefficient matrix A,
\begin{align*}
    M\vec{v}=\lambda \vec{v} \; , where \; \lambda \;is\; an \;eigen\; value.
\end{align*}
\begin{align*}
     G=(1-\rho)I\;+\rho\;\vec{1}\vec{1}^T, \; where\, \rho=\frac{1}{2}\;and\;\vec{1}=\myvec{1&&1&&1}
\end{align*}
Let $\vec{1}\vec{1}^T=J$. As we could see that the eigen-vector of J is $\vec{1}$ and by the rule,
\newpage
\vspace*{0.25cm}

\begin{align*}
    \myvec{1&&1&&1\\1&&1&&1\\1&&1&&1}\myvec{1\\1\\1}=\myvec{3\\3\\3}=3\vec{1}
\end{align*}
So, 3 is a eigen value of J. Also we can observe that any vector orthogonal to J has eigen value 0 and since the eigen vector has only two degrees of freedom,
\begin{align*}
    \therefore eigen\;values\;of\;J\;are\;\{3,0,0\}
\end{align*}
Modifying the above equation on G,
\begin{align*}
    \therefore G\vec{v}=\frac{1}{2}I\vec{v}+\frac{1}{2}J\vec{v}
\end{align*}
\begin{align*}
    \implies G\vec{v}=\frac{(1+\mu)}{2}\vec{v}
\end{align*}
where $\mu$ is the eigen value of J. Here the eigen value of G is $\frac{1+\mu}{2}$ and substituting the obtained eigen values of J in this equation, we get the eigen values of G to be $\{2,\frac{1}{2},\frac{1}{2}\}$\\
\\
As we know that for eigen values of G being $\{\mu_1,\mu_2,\mu_3\}$
\begin{align*}
    det(G)=\mu_1\mu_2\mu_3
\end{align*}
\begin{align*}
    \therefore det(G)=2\times\frac{1}{2}\times\frac{1}{2}=\frac{1}{2}
\end{align*}
\begin{align*}
    \implies V=\sqrt{det(G)}=\frac{1}{\sqrt{2}} \; units
\end{align*}

From the figure, taking an example of vectors $\vec{a}$ and $\vec{b}$ ,it is clearly verified that the theoretical solution matches with the computational solution.\\
\newpage
\vspace*{0.25cm}
\begin{figure}[H]
    \centering
    \includegraphics[width=1.0\columnwidth]{figs/Figure-1.png}
    \label{fig:1}
\end{figure}


\end{document}