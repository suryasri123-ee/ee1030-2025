\documentclass{beamer}
\usepackage[utf8]{inputenc}

\usetheme{Madrid}
\usecolortheme{default}
\usepackage{amsmath,amssymb,amsfonts,amsthm}
\usepackage{mathtools}
\usepackage{txfonts}
\usepackage{tkz-euclide}
\usepackage{listings}
\usepackage{adjustbox}
\usepackage{array}
\usepackage{gensymb}
\usepackage{tabularx}
\usepackage{gvv}
\usepackage{lmodern}
\usepackage{circuitikz}
\usepackage{tikz}
\usepackage{graphicx}

\setbeamertemplate{page number in head/foot}[totalframenumber]

\usepackage{tcolorbox}
\tcbuselibrary{minted,breakable,xparse,skins}



\definecolor{bg}{gray}{0.95}
\DeclareTCBListing{mintedbox}{O{}m!O{}}{%
  breakable=true,
  listing engine=minted,
  listing only,
  minted language=#2,
  minted style=default,
  minted options={%
    linenos,
    gobble=0,
    breaklines=true,
    breakafter=,,
    fontsize=\small,
    numbersep=8pt,
    #1},
  boxsep=0pt,
  left skip=0pt,
  right skip=0pt,
  left=25pt,
  right=0pt,
  top=3pt,
  bottom=3pt,
  arc=5pt,
  leftrule=0pt,
  rightrule=0pt,
  bottomrule=2pt,
  toprule=2pt,
  colback=bg,
  colframe=orange!70,
  enhanced,
  overlay={%
    \begin{tcbclipinterior}
    \fill[orange!20!white] (frame.south west) rectangle ([xshift=20pt]frame.north west);
    \end{tcbclipinterior}},
  #3,
}
\lstset{
    language=C,
    basicstyle=\ttfamily\small,
    keywordstyle=\color{blue},
    stringstyle=\color{orange},
    commentstyle=\color{green!60!black},
    numbers=left,
    numberstyle=\tiny\color{gray},
    breaklines=true,
    showstringspaces=false,
}
%------------------------------------------------------------
%This block of code defines the information to appear in the
%Title page
\title %optional
{4.7.12}
\date{August 29,2025}
%\subtitle{A short story}

\author % (optional)
{Harsha-EE25BTECH11026}



\begin{document}


\frame{\titlepage}


\begin{frame}{Question}
Find the distance of the line $4x-y=0$ from the point P$\brak{4,1}$ measured along the line making an angle of $135\degree$ with the positive x-axis.
\end{frame}

\begin{frame}{Theoretical Solution}
According to the question,\\
\begin{align}
    \text{Equation of target line:}\myvec{4&&-1}\myvec{x\\y}=0
\end{align}
and
\begin{align}
    \vec{P}=\myvec{4\\1}
\end{align}
As the direction of line makes an angle of $135\degree$ with the $+x$ axis, the unit direction vector of the line is given by
\begin{align}
    \vec{m_0}=\myvec{\cos{135\degree}\\\sin{135\degree}}=\myvec{-\frac{1}{\sqrt{2}}\\\frac{1}{\sqrt{2}}}
\end{align}
\end{frame}

\begin{frame}{Equation}
To calculate the distance $\kappa$ of a vector $\vec{P}$ from the target line $\vec{n}^{\top}\vec{x}=c$ along a line with direction vector $\vec{m_0}$,
\begin{align}
    \text{Parametric form:\;}\vec{x}=\vec{P}+\kappa\vec{m_0} 
\end{align}
\begin{align}
    \implies \vec{n}^{\top}\brak{\vec{P}+\kappa\vec{m_0}}=c
\end{align}
\begin{align}
    \therefore \kappa=\frac{c-\vec{n}^{\top}\vec{P}}{\vec{n}^{\top}\vec{m_0}}\\
\end{align}
\end{frame}

\begin{frame}{Theoretical Solution}
\begin{align}
     \implies \kappa=\frac{-\myvec{4&&-1}\myvec{4\\1}}{\myvec{4&&-1}\myvec{-\frac{1}{\sqrt{2}}\\\frac{1}{\sqrt{2}}}}
\end{align}
\begin{align}
    \implies \kappa=3\sqrt{2}\; \text{units}
\end{align}

\end{frame}



\begin{frame}[fragile]
    \frametitle{C Code -Finding Equations of diagonals of a square}

    \begin{lstlisting}
#include <stdio.h>
#include <math.h>

// Function to compute intersection point Q
// Inputs: Px, Py = point P
// Outputs: Qx, Qy (via pointers)
void find_intersection(double Px, double Py, double *Qx, double *Qy) {
    // direction vector for 135° = (-1, 1)
    double dx = -1.0, dy = 1.0;

    \end{lstlisting}
\end{frame}

\begin{frame}[fragile]
    \frametitle{C Code -Finding Equations of diagonals of a square}

    \begin{lstlisting}
    // Parametric line: (x,y) = (Px + t dx, Py + t dy)
    // Line: 4x - y = 0  -> y = 4x
    // Substitute: Py + t dy = 4(Px + t dx)
    // => Py + t = 4Px + 4t dx
    // => Py + t = 4Px + 4t(-1)
    // => Py + t = 4Px - 4t
    // => t + 4t = 4Px - Py
    // => 5t = 4Px - Py
    double t = (4*Px - Py) / 5.0;

    *Qx = Px + t*dx;
    *Qy = Py + t*dy;
}

// Function to compute distance between two points
double distance(double x1, double y1, double x2, double y2) {
    return sqrt((x2-x1)*(x2-x1) + (y2-y1)*(y2-y1));
}
    \end{lstlisting}
\end{frame}

\begin{frame}[fragile]
    \frametitle{Python + C code}

    \begin{lstlisting}
import ctypes
import math
import matplotlib.pyplot as plt
import matplotlib as mp
mp.use("TkAgg")
# Load shared library
lib = ctypes.CDLL("./libnorm.so")
# Define function signatures
lib.find_intersection.argtypes = [ctypes.c_double, ctypes.c_double,
                                  ctypes.POINTER(ctypes.c_double), ctypes.POINTER(ctypes.c_double)]
lib.distance.argtypes = [ctypes.c_double, ctypes.c_double, ctypes.c_double, ctypes.c_double]
lib.distance.restype = ctypes.c_double
    \end{lstlisting}
\end{frame}

\begin{frame}[fragile]
    \frametitle{Python + C code}

    \begin{lstlisting}
# Given point P
Px, Py = 4.0, 1.0

# Call C function to get Q
Qx = ctypes.c_double()
Qy = ctypes.c_double()
lib.find_intersection(Px, Py, ctypes.byref(Qx), ctypes.byref(Qy))
Qx, Qy = Qx.value, Qy.value

# Distance from C
dist = lib.distance(Px, Py, Qx, Qy)
print(f"distance: {dist:.3f} units")

    \end{lstlisting}
\end{frame}

\begin{frame}[fragile]
    \frametitle{Python + C code}

    \begin{lstlisting}
# -----------------------------
# Plot same as before
# -----------------------------
fig, ax = plt.subplots(figsize=(6, 6))

# Line y = 4x
x_line = [-1, 5]
y_line = [4*x for x in x_line]
ax.plot(x_line, y_line, label="Line: 4x - y = 0 (y=4x)")

# Direction line through P (slope -1)
x_dir = [Px - 4, Px + 4]
y_dir = [Py + 4, Py - 4]
ax.plot(x_dir, y_dir, linestyle="--", label="Direction 135° (slope -1)")


    \end{lstlisting}
\end{frame}

\begin{frame}[fragile]
    \frametitle{Python + C code}

    \begin{lstlisting}
# Points P and Q
ax.plot(Px, Py, 'ro')
ax.annotate(f'P({Px:.0f},{Py:.0f})', xy=(Px,Py), xytext=(Px+0.2,Py-0.5))
ax.plot(Qx, Qy, 'mo')
ax.annotate(f'Q({Qx:.0f},{Qy:.0f})', xy=(Qx,Qy), xytext=(Qx+0.2,Qy+0.2))

# Segment PQ
ax.plot([Px, Qx], [Py, Qy], 'r-', linewidth=2)

# Distance label
mid = ((Px+Qx)/2, (Py+Qy)/2)
ax.text(mid[0], mid[1]+0.3, f'distance = {dist:.3f}', ha='center', color='red')
    \end{lstlisting}
\end{frame}

\begin{frame}[fragile]
    \frametitle{Python + C code}

    \begin{lstlisting}
# Formatting
ax.set_aspect('equal', 'box')
ax.set_xlabel("x")
ax.set_ylabel("y")
ax.grid(True, alpha=0.4)
ax.legend(loc="upper right")


plt.savefig("/home/user/Matrix/Matgeo_assignments/4.7.12/figs/Figure_1.png")
plt.show()
    \end{lstlisting}
\end{frame}

\begin{frame}[fragile]
    \frametitle{Python code}

    \begin{lstlisting}
import math
import matplotlib.pyplot as plt
import matplotlib as mp
mp.use("TkAgg")
# -----------------------------
# Data
# -----------------------------
P = (4.0, 1.0) # given point
m_dir = -1.0 # slope for 135° direction (tan 135° = -1)
d = (-1.0, 1.0) # unit direction up to scale
# Given line: 4x - y = 0 -> y = 4x
# Parametric line through P in direction d: (x, y) = P + t d = (4 - t, 1 + t)
# Intersect with y = 4x:
# 1 + t = 4(4 - t) -> 1 + t = 16 - 4t -> 5t = 15 -> t = 3
t = 3.0
Q = (P[0] + t*d[0], P[1] + t*d[1]) # (1, 4)

    \end{lstlisting}
\end{frame}

\begin{frame}[fragile]
    \frametitle{Python code}

    \begin{lstlisting}
# Distance along the 135° direction
DX = Q[0] - P[0]
DY = Q[1] - P[1]
distance = math.hypot(DX, DY) # 3*sqrt(2)

print(f"distance: {distance:.3f} units")

# -----------------------------
# Plot
# -----------------------------
fig, ax = plt.subplots(figsize=(6, 6))

# Plot given line y = 4x
x_line = [ -1, 5 ]
y_line = [ 4*x for x in x_line ]
ax.plot(x_line, y_line, label='Line: 4x - y = 0 (y=4x)')
    \end{lstlisting}
\end{frame}

\begin{frame}[fragile]
    \frametitle{Python code}

    \begin{lstlisting}
# Plot direction line through P (slope -1)
x_dir = [P[0] - 4, P[0] + 4]
y_dir = [P[1] + 4, P[1] - 4]
ax.plot(x_dir, y_dir, linestyle='--', label='Direction 135° (slope -1)')
# Points P and Q
ax.plot(P[0], P[1], 'ro')
ax.annotate('P(4,1)', xy=P, xytext=(P[0]+0.2, P[1]-0.5))
ax.plot(Q[0], Q[1], 'mo')
ax.annotate(f'Q{Q}', xy=Q, xytext=(Q[0]+0.2, Q[1]+0.2))
# Segment PQ (the measured distance)
ax.plot([P[0], Q[0]], [P[1], Q[1]], 'r-', linewidth=2)
# Annotate distance directly above the segment
mid = ((P[0]+Q[0])/2, (P[1]+Q[1])/2)
ax.text(mid[0], mid[1]+0.3, f'distance = {distance:.3f}', ha='center', va='bottom', fontsize=10, color='red')

    \end{lstlisting}
\end{frame}

\begin{frame}[fragile]
    \frametitle{Python code}

    \begin{lstlisting}
# Axes formatting
ax.set_aspect('equal', 'box')
ax.set_xlabel('x')
ax.set_ylabel('y')
ax.grid(True, alpha=0.4)
ax.legend(loc='upper right')


# Set sensible limits around all geometry
xs = [-1, 5, P[0], Q[0]]
ys = [min(y_line), max(y_line), P[1], Q[1]]
ax.set_xlim(min(xs)-1, max(xs)+1)
ax.set_ylim(min(ys)-1, max(ys)+1)

plt.savefig("/home/user/Matrix/Matgeo_assignments/4.7.12/figs/Figure_1.png")
plt.show()
    \end{lstlisting}
\end{frame}

\begin{frame}{Plot}
    \begin{figure}[H]
    \centering
    \includegraphics[width=0.6\columnwidth]{figs/Figure_1.png}
    \label{fig:1}
    \end{figure}
\end{frame}





\end{document}