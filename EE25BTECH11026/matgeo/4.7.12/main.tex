\let\negmedspace\undefined
\let\negthickspace\undefined
\documentclass[journal]{IEEEtran}
\usepackage[a5paper, margin=10mm, onecolumn]{geometry}
%\usepackage{lmodern} % Ensure lmodern is loaded for pdflatex
\usepackage{tfrupee} % Include tfrupee package

\setlength{\headheight}{1cm} % Set the height of the header box
\setlength{\headsep}{0mm}     % Set the distance between the header box and the top of the text

\usepackage{gvv-book}
\usepackage{gvv}
\usepackage{cite}
\usepackage{amsmath,amssymb,amsfonts,amsthm}
\usepackage{algorithmic}
\usepackage{graphicx}
\usepackage{textcomp}
\usepackage{xcolor}
\usepackage{txfonts}
\usepackage{listings}
\usepackage{enumitem}
\usepackage{mathtools}
\usepackage{gensymb}
\usepackage{comment}
\usepackage[breaklinks=true]{hyperref}
\usepackage{tkz-euclide} 
\usepackage{listings}
% \usepackage{gvv}                                        
\def\inputGnumericTable{}                                 
\usepackage[latin1]{inputenc}                                
\usepackage{color}                                            
\usepackage{array}                                            
\usepackage{longtable}                                       
\usepackage{calc}                                             
\usepackage{multirow}                                         
\usepackage{hhline}                                           
\usepackage{ifthen}                                           
\usepackage{lscape}
\usepackage{circuitikz}
\tikzstyle{block} = [rectangle, draw, fill=blue!20, 
    text width=4em, text centered, rounded corners, minimum height=3em]
\tikzstyle{sum} = [draw, fill=blue!10, circle, minimum size=1cm, node distance=1.5cm]
\tikzstyle{input} = [coordinate]
\tikzstyle{output} = [coordinate]


\begin{document}

\bibliographystyle{IEEEtran}
\vspace{3cm}

\title{4.3.13}
\author{EE25BTECH11026-Harsha}
 \maketitle
% \newpage
% \bigskip
{\let\newpage\relax\maketitle}

\renewcommand{\thefigure}{\theenumi}
\renewcommand{\thetable}{\theenumi}
\setlength{\intextsep}{10pt} % Space between text and floats


\numberwithin{equation}{enumi}
\numberwithin{figure}{enumi}
\renewcommand{\thetable}{\theenumi}

\textbf{Question}:\\
Find the distance of the line $4x-y=0$ from the point P$\brak{4,1}$ measured along the line making an angle of $135\degree$ with the positive x-axis.\\
\solution \\
Let us solve the given question theoretically and then verify the solution computationally.\\
\\
According to the question,\\
\begin{align}
    Equation\;of\;target\;line\;:\;\myvec{4&&-1}\myvec{x\\y}=0
\end{align}
and
\begin{align}
    \vec{P}=\myvec{4\\1}
\end{align}
As the direction of line makes an angle of $135\degree$ with the $+x$ axis, the unit direction vector of the line is given by
\begin{align}
    \vec{m}=\myvec{\cos{135\degree}\\\sin{135\degree}}=\myvec{-\frac{1}{\sqrt{2}}\\\frac{1}{\sqrt{2}}}
\end{align}
Parametrize the required line using $\vec{P}$,yielding
\begin{align}
    \vec{x}=\vec{P}+\kappa\vec{m}
\end{align}
Inserting the parametric form in the equation of target line,
\begin{align}
    \myvec{4&&-1}\brak{\vec{P}+\kappa\vec{m}}=0
\end{align}
\begin{align}
    \therefore \kappa=\frac{-\myvec{4&&-1}\myvec{4\\1}}{\myvec{4&&-1}\myvec{-\frac{1}{\sqrt{2}}\\\frac{1}{\sqrt{2}}}}
\end{align}
\begin{align}
    \implies \kappa=3\sqrt{2}
\end{align}
\newpage
\vspace*{0.25cm}
Since $\vec{m}$ is a unit vector, the norm of vector $\vec{P}$ from the given line along the line with $\vec{m}=\myvec{-\frac{1}{\sqrt{2}}&&\frac{1}{\sqrt{2}}}^{\top}$ is
\begin{align}
    \kappa=3\sqrt{2}\;units
\end{align}
\\
From the figure, it is clearly verified that the theoretical solution matches with the computational solution.\\
\begin{figure}[H]
    \centering
    \includegraphics[width=0.6\columnwidth]{figs/Figure_1.png}
    \label{fig:1}
\end{figure}


\end{document}