\documentclass{beamer}
\usepackage[utf8]{inputenc}

\usetheme{Madrid}
\usecolortheme{default}
\usepackage{amsmath,amssymb,amsfonts,amsthm}
\usepackage{mathtools}
\usepackage{txfonts}
\usepackage{tkz-euclide}
\usepackage{listings}
\usepackage{adjustbox}
\usepackage{array}
\usepackage{tabularx}
\usepackage{gvv}
\usepackage{lmodern}
\usepackage{circuitikz}
\usepackage{tikz}
\usepackage{graphicx}

\setbeamertemplate{page number in head/foot}[totalframenumber]

\usepackage{tcolorbox}
\tcbuselibrary{minted,breakable,xparse,skins}



\definecolor{bg}{gray}{0.95}
\DeclareTCBListing{mintedbox}{O{}m!O{}}{%
  breakable=true,
  listing engine=minted,
  listing only,
  minted language=#2,
  minted style=default,
  minted options={%
    linenos,
    gobble=0,
    breaklines=true,
    breakafter=,,
    fontsize=\small,
    numbersep=8pt,
    #1},
  boxsep=0pt,
  left skip=0pt,
  right skip=0pt,
  left=25pt,
  right=0pt,
  top=3pt,
  bottom=3pt,
  arc=5pt,
  leftrule=0pt,
  rightrule=0pt,
  bottomrule=2pt,
  toprule=2pt,
  colback=bg,
  colframe=orange!70,
  enhanced,
  overlay={%
    \begin{tcbclipinterior}
    \fill[orange!20!white] (frame.south west) rectangle ([xshift=20pt]frame.north west);
    \end{tcbclipinterior}},
  #3,
}
\lstset{
    language=C,
    basicstyle=\ttfamily\small,
    keywordstyle=\color{blue},
    stringstyle=\color{orange},
    commentstyle=\color{green!60!black},
    numbers=left,
    numberstyle=\tiny\color{gray},
    breaklines=true,
    showstringspaces=false,
}
%------------------------------------------------------------
%This block of code defines the information to appear in the
%Title page
\title %optional
{4.3.13}
\date{August 27,2025}
%\subtitle{A short story}

\author % (optional)
{Harsha-EE25BTECH11026}



\begin{document}


\frame{\titlepage}
\begin{frame}{Question}
Equations of the diagonals of the square formed by the lines x = 0, y = 0, x = 1 and y = 1 are \underline{\hspace{2cm}}.
\end{frame}

\begin{frame}{Theoretical Solution}
According to the question,\\
The vertices of the square are ,
\begin{align}
    \vec{a}=\myvec{0\\0}\;\;\vec{b}=\myvec{1\\0}\;\;\vec{c}=\myvec{1\\1}\;\;\vec{d}=\myvec{0\\1}
\end{align}
\end{frame}

\begin{frame}{Equation}
To compute the equation of the diagnols , we can use the normal form of the equation,which is given by
\begin{align}
    \vec{n}^{\top}\vec{x}=0\; \text{for the lines passing through the origin}
\end{align}
\begin{align}
    \vec{n}^{\top}\vec{x}=1 \;\text{for the lines not passing through the origin}
\end{align}
\end{frame}

\begin{frame}{Theoretical Solution}
For diagonal $\vec{c}-\vec{a}$, as it passes through the origin,
\begin{align}
    \therefore \vec{n}^{\top}\vec{x}=0
\end{align}
By substituting the vector through which it passes through,
\begin{align}
    \vec{n}^{\top}\myvec{1\\1}=0
\end{align}
\begin{align}
    \implies \vec{n}=\myvec{-1\\1}
\end{align}
\begin{align}
    \therefore \myvec{-1&&1}\vec{x}=0
\end{align}
\end{frame}

\begin{frame}{Theoretical Solution}
But, for diagonal $\vec{d}-\vec{b}$, as the diagonal doesn't pass through the origin,
\begin{align}
    \vec{n}^{\top}\vec{x}=1
\end{align}
\begin{align}
    \therefore \vec{n}^{\top}\myvec{1&&0\\0&&1}=\myvec{1\\1}
\end{align}
\begin{align}
    \implies \vec{n}=\myvec{1\\1}
\end{align}
\begin{align}
    \therefore \myvec{1&&1}\vec{x}=1
\end{align}
\end{frame}

\begin{frame}[fragile]
    \frametitle{C Code -Finding Equations of diagonals of a square}

    \begin{lstlisting}
#include <stdio.h>

void diagonal_equations(double vertices[4][2], double n1[2], double *c1, double n2[2], double *c2) {
    // A, B, C, D are the vertices
    double *A = vertices[0];
    double *B = vertices[1];
    double *C = vertices[2];
    double *D = vertices[3];

    // Diagonal AC
    double dx1 = C[0] - A[0];
    double dy1 = C[1] - A[1];
    n1[0] = dy1;
    n1[1] = -dx1;
    *c1 = n1[0]*A[0] + n1[1]*A[1];


    \end{lstlisting}
\end{frame}

\begin{frame}[fragile]
    \frametitle{C Code -Finding Equations of diagonals of a square}

    \begin{lstlisting}

    // Diagonal BD
    double dx2 = D[0] - B[0];
    double dy2 = D[1] - B[1];
    n2[0] = dy2;
    n2[1] = -dx2;
    *c2 = n2[0]*B[0] + n2[1]*B[1];
}
    \end{lstlisting}
\end{frame}


\begin{frame}[fragile]
    \frametitle{Python+C code}

    \begin{lstlisting}
import ctypes
import numpy as np
import matplotlib as mp
mp.use("TkAgg")
import matplotlib.pyplot as plt

# Load C shared library
lib = ctypes.CDLL("./libdiagonals.so")

# Define argument types for the function
lib.diagonal_equations.argtypes = [
    (ctypes.c_double * 2) * 4,
    ctypes.POINTER(ctypes.c_double),
    ctypes.POINTER(ctypes.c_double),
    ctypes.POINTER(ctypes.c_double),
    ctypes.POINTER(ctypes.c_double)
]

    \end{lstlisting}
\end{frame}

\begin{frame}[fragile]
    \frametitle{Python+C code}

    \begin{lstlisting}
# Define square vertices
vertices = [(0.0,0.0), (1.0,0.0), (1.0,1.0), (0.0,1.0)]
vert_array = ((ctypes.c_double*2)*4)(*[(ctypes.c_double*2)(*v) for v in vertices])

# Output containers
n1 = (ctypes.c_double*2)()
c1 = ctypes.c_double()
n2 = (ctypes.c_double*2)()
c2 = ctypes.c_double()

# Call C function
lib.diagonal_equations(vert_array, n1, ctypes.byref(c1), n2, ctypes.byref(c2))

n1 = np.array([n1[0], n1[1]])
n2 = np.array([n2[0], n2[1]])

    \end{lstlisting}
\end{frame}

\begin{frame}[fragile]
    \frametitle{Python+C code}

    \begin{lstlisting}
# Convert to Cartesian equation: ax + by + d = 0
def cartesian_eq(n, c):
    a, b = n
    d = -c
    eq = []
    if a != 0:
        eq.append(f"{a}x")
    if b != 0:
        sign = "+" if b > 0 and eq else ""
        eq.append(f"{sign}{b}y")
    if d != 0:
        sign = "+" if d > 0 and eq else ""
        eq.append(f"{sign}{d}")
    return " ".join(eq) + " = 0"

eq1 = cartesian_eq(n1, c1.value)
eq2 = cartesian_eq(n2, c2.value)


    \end{lstlisting}
\end{frame}

\begin{frame}[fragile]
    \frametitle{Python+C code}

    \begin{lstlisting}
print("Diagonal AC (normal form):", f"[{n1[0]} {n1[1]}] · [x y]^{\top} = {c1.value}")
print("Diagonal BD (normal form):", f"[{n2[0]} {n2[1]}] · [x y]^{\top} = {c2.value}")

# ---- PLOT ----
A, B, C, D = vertices
square_x = [A[0], B[0], C[0], D[0], A[0]]
square_y = [A[1], B[1], C[1], D[1], A[1]]

plt.plot(square_x, square_y, 'b-', label='Square')

# Plot diagonals
plt.plot([A[0], C[0]], [A[1], C[1]], 'r--', label=eq1)
plt.plot([B[0], D[0]], [B[1], D[1]], 'g--', label=eq2)

    \end{lstlisting}
\end{frame}

\begin{frame}[fragile]
    \frametitle{Python+C code}

    \begin{lstlisting}
plt.gca().set_aspect('equal', adjustable='box')
plt.legend(loc="upper right")
plt.grid(True)
plt.savefig("/home/user/Matrix/Matgeo_assignments/4.3.13/figs/Figure_1.png")
plt.show()
    \end{lstlisting}
\end{frame}


\begin{frame}[fragile]
    \frametitle{Python code}

    \begin{lstlisting}
import matplotlib as mp
mp.use("TkAgg")
import numpy as np
import matplotlib.pyplot as plt

def line_equation_normal(point1, point2):
    """Returns line equation in normal form: n^{\top} x = c
       where n is the normal vector and x = [x y]^{\top}."""
    x1, y1 = point1
    x2, y2 = point2
    # Direction vector
    dx, dy = x2 - x1, y2 - y1
    # Normal vector
    n = np.array([dy, -dx])
    # Constant term
    c = n @ np.array([x1, y1])
    return n, c
    \end{lstlisting}
\end{frame}

\begin{frame}[fragile]
    \frametitle{Python code}

    \begin{lstlisting}
def diagonals_of_square(vertices):
    """
    Given 4 vertices of a square (in order), compute equations of diagonals.
    """
    A, B, C, D = vertices
    
    # Diagonals are AC and BD
    line1 = line_equation_normal(A, C)
    line2 = line_equation_normal(B, D)
    
    return line1, line2

def format_normal_form(n, c):
    """Format equation in normal form."""
    return f"[{n[0]} {n[1]}] · [x y]^{\top} = {c}"

    \end{lstlisting}
\end{frame}


\begin{frame}[fragile]
    \frametitle{Python code}

    \begin{lstlisting}
def format_cartesian(n, c):
    """Convert n^{\top} x = c into Cartesian form ax + by + d = 0
    where n = [a b]."""
    a, b = n
    d = -c
    terms = []
    if a != 0:
        terms.append(f"{'' if a == 1 else '-' if a == -1 else a}x")
    if b != 0:
        sign = "+" if b > 0 and terms else ""
        terms.append(f"{sign}{'' if abs(b) == 1 else b}y" if b not in [1, -1] else f"{sign}{'y' if b == 1 else '-y'}")
    if d != 0:
        sign = "+" if d > 0 and terms else ""
        terms.append(f"{sign}{d}")
    return " ".join(terms) + " = 0"

    \end{lstlisting}
\end{frame}

\begin{frame}[fragile]
    \frametitle{Python code}

    \begin{lstlisting}
    def plot_square_and_diagonals(vertices, line1, line2):
    """Plot square and its diagonals with Cartesian equations on the plot."""
    A, B, C, D = vertices
    square_x = [A[0], B[0], C[0], D[0], A[0]]
    square_y = [A[1], B[1], C[1], D[1], A[1]]
    
    plt.plot(square_x, square_y, 'b-', label='Square')
    
    # Plot diagonals
    plt.plot([A[0], C[0]], [A[1], C[1]], 'r--', label='Diagonal AC')
    plt.plot([B[0], D[0]], [B[1], D[1]], 'g--', label='Diagonal BD')
    # Equations in Cartesian form for plot
    eq1 = format_cartesian(*line1)
    eq2 = format_cartesian(*line2)
   
    \end{lstlisting}
\end{frame}


\begin{frame}[fragile]
    \frametitle{Python code}

    \begin{lstlisting}
 # Midpoints of diagonals
    mid_AC = ((A[0]+C[0])/2, (A[1]+C[1])/2)
    mid_BD = ((B[0]+D[0])/2, (B[1]+D[1])/2)
    # Place texts
    plt.text(mid_AC[0]+0.05, mid_AC[1]+0.05, eq1, color='red', fontsize=10, ha='left')
    plt.text(mid_BD[0]-0.15, mid_BD[1]-0.1, eq2, color='green', fontsize=10, ha='right')
    
    plt.gca().set_aspect('equal', adjustable='box')
    plt.legend(loc="upper right")
    plt.grid(True)
    plt.savefig("/home/user/Matrix/Matgeo_assignments/4.3.13/figs/Figure_1")
    plt.show()
    \end{lstlisting}
\end{frame}

\begin{frame}[fragile]
    \frametitle{Python code}

    \begin{lstlisting}
vertices = [(0,0), (1,0), (1,1), (0,1)]

# Compute diagonal equations
line1, line2 = diagonals_of_square(vertices)

# Print normal forms
print("Diagonal AC equation (normal form):", format_normal_form(*line1))
print("Diagonal BD equation (normal form):", format_normal_form(*line2))

# Plot with Cartesian equations
plot_square_and_diagonals(vertices, line1, line2)


    \end{lstlisting}
\end{frame}

\begin{frame}{Plot}
    \begin{figure}[H]
    \centering
    \caption{Plot of square and its diagonals}
    \includegraphics[width=0.6\columnwidth]{figs/Figure_1.png}
    \label{fig:1}
    \end{figure}
\end{frame}





\end{document}