\documentclass{beamer}
\usepackage[utf8]{inputenc}

\usetheme{Madrid}
\usecolortheme{default}
\usepackage{amsmath,amssymb,amsfonts,amsthm}
\usepackage{mathtools}
\usepackage{txfonts}
\usepackage{tkz-euclide}
\usepackage{listings}
\usepackage{adjustbox}
\usepackage{array}
\usepackage{tabularx}
\usepackage{gvv}
\usepackage{lmodern}
\usepackage{circuitikz}
\usepackage{tikz}
\usepackage{graphicx}

\setbeamertemplate{page number in head/foot}[totalframenumber]

\usepackage{tcolorbox}
\tcbuselibrary{minted,breakable,xparse,skins}



\definecolor{bg}{gray}{0.95}
\DeclareTCBListing{mintedbox}{O{}m!O{}}{%
  breakable=true,
  listing engine=minted,
  listing only,
  minted language=#2,
  minted style=default,
  minted options={%
    linenos,
    gobble=0,
    breaklines=true,
    breakafter=,,
    fontsize=\small,
    numbersep=8pt,
    #1},
  boxsep=0pt,
  left skip=0pt,
  right skip=0pt,
  left=25pt,
  right=0pt,
  top=3pt,
  bottom=3pt,
  arc=5pt,
  leftrule=0pt,
  rightrule=0pt,
  bottomrule=2pt,
  toprule=2pt,
  colback=bg,
  colframe=orange!70,
  enhanced,
  overlay={%
    \begin{tcbclipinterior}
    \fill[orange!20!white] (frame.south west) rectangle ([xshift=20pt]frame.north west);
    \end{tcbclipinterior}},
  #3,
}
\lstset{
    language=C,
    basicstyle=\ttfamily\small,
    keywordstyle=\color{blue},
    stringstyle=\color{orange},
    commentstyle=\color{green!60!black},
    numbers=left,
    numberstyle=\tiny\color{gray},
    breaklines=true,
    showstringspaces=false,
}
%------------------------------------------------------------
%This block of code defines the information to appear in the
%Title page
\title %optional
{4.3.13}
\date{August 27,2025}
%\subtitle{A short story}

\author % (optional)
{Harsha-EE25BTECH11026}



\begin{document}


\frame{\titlepage}
\begin{frame}{Question}
Equations of the diagonals of the square formed by the lines x = 0, y = 0, x = 1 and y = 1 are \underline{\hspace{2cm}}.
\end{frame}

\begin{frame}{Theoretical Solution}
According to the question,\\
The vertices of the square are ,
\begin{align*}
    \vec{a}=\myvec{0\\0}\;\;\vec{b}=\myvec{1\\0}\;\;\vec{c}=\myvec{1\\1}\;\;\vec{d}=\myvec{0\\1}
\end{align*}
\end{frame}

\begin{frame}{Equation}
To compute the equation of the diagonals , we can use the normal form of the equation,which is given by
\begin{align*}
    \vec{n}^T\vec{x}=0 \;for \;the \;lines \;passing\;through \;the \;origin
\end{align*}
\begin{align*}
    \vec{n}^T\vec{x}=1 \;for \;the \;lines \;not \;passing\;through \;the \;origin
\end{align*}
where,\\
\hspace*{4em}  $\vec{n}$-vector orthogonal to the direction vector\\
\\
\hspace*{4em}  $\vec{x}=\myvec{x&&y}^T$
\end{frame}

\begin{frame}{Theoretical Solution}
For diagonal $\vec{c}-\vec{a}$,
\begin{align*}
    \vec{n}=\myvec{0&&-1\\1&&0}\vec{d}
\end{align*}
where $\vec{d}$ is the direction vector of diagonal.
\begin{align*}
    \therefore \vec{n}=\myvec{0&&-1\\1&&0}\brak{\myvec{1\\1}-\myvec{0\\0}}
\end{align*}
\begin{align*}
    \implies \vec{n}=\myvec{-1\\1}
\end{align*}
As we know that the diagonal $\vec{c}-\vec{a}$ passes through the origin,
\begin{align*}
    \therefore \myvec{-1&&1}\myvec{x\\y}=0
\end{align*}
\end{frame}

\begin{frame}{Theoretical Solution}
But, for diagonal $\vec{d}-\vec{b}$, as the diagonal doesn't pass through the origin,
\begin{align*}
    \vec{n}^T\vec{x}=1
\end{align*}
As we know that the diagonal $\vec{d}-\vec{b}$ pass through the vectors $\myvec{1&&0}^T$ and $\myvec{0&&1}^T$, we can say that $\vec{n}$ would be $\myvec{1&&1}^T$ 
\begin{align*}
    \therefore \myvec{1&&1}\myvec{x\\y}=1
\end{align*}
\end{frame}

\begin{frame}[fragile]
    \frametitle{C Code -Finding Diagonals of a square}

    \begin{lstlisting}
    #include <stdio.h>

typedef struct {
    double A, B, C;  // Line coefficients Ax + By + C = 0
} Line;

// Function to compute line equation given two points (x1,y1), (x2,y2)
Line line_from_points(double x1, double y1, double x2, double y2) {
    Line l;
    l.A = y1 - y2;
    l.B = x2 - x1;
    l.C = (x1 * y2) - (x2 * y1);
    return l;
}


    \end{lstlisting}
\end{frame}

\begin{frame}[fragile]
    \frametitle{C Code -Finding Diagonals of a square}

    \begin{lstlisting}
// Function to compute diagonals of a square
void diagonals_of_square(double vertices[4][2], Line *diag1, Line *diag2) {
    // vertices order: A, B, C, D
    // Diagonals: AC and BD
    *diag1 = line_from_points(vertices[0][0], vertices[0][1],
                              vertices[2][0], vertices[2][1]);
    *diag2 = line_from_points(vertices[1][0], vertices[1][1],
                              vertices[3][0], vertices[3][1]);
}


    \end{lstlisting}
\end{frame}

\begin{frame}[fragile]
    \frametitle{C Code -Finding Diagonals of a square}

    \begin{lstlisting}
// Export function for Python
__attribute__((visibility("default"))) 
void get_square_diagonals(double vertices[4][2], double *out) {
    Line d1, d2;
    diagonals_of_square(vertices, &d1, &d2);


    // Store results in array: [A1, B1, C1, A2, B2, C2]
    out[0] = d1.A;
    out[1] = d1.B;
    out[2] = d1.C;
    out[3] = d2.A;
    out[4] = d2.B;
    out[5] = d2.C;
}

    \end{lstlisting}
\end{frame}

\begin{frame}[fragile]
    \frametitle{Python+C code}

    \begin{lstlisting}
import ctypes
import numpy as np
import matplotlib.pyplot as plt
import matplotlib as mp
mp.use('TkAgg')

# Load the shared C library
lib = ctypes.CDLL("./libdiagonals.so")

# Define function signature for C function
lib.get_square_diagonals.argtypes = [ctypes.c_double * 8, ctypes.c_double * 6]

    \end{lstlisting}
\end{frame}

\begin{frame}[fragile]
    \frametitle{Python+C code}

    \begin{lstlisting}

def get_diagonals(vertices):
    """Call C function to compute diagonals of a square"""
    verts = (ctypes.c_double * 8)(*np.array(vertices).flatten())
    out = (ctypes.c_double * 6)()
    lib.get_square_diagonals(verts, out)
    return np.array(out[:]).reshape(2,3)  # [[A1,B1,C1],[A2,B2,C2]]


    \end{lstlisting}
\end{frame}

\begin{frame}[fragile]
    \frametitle{Python+C code}

    \begin{lstlisting}
def format_equation(A, B, C):
    """
    Beautify line equation into readable string.
    Example: -1x + 1y + 0 -> -x + y = 0
    """
    terms = []

    # Ax term
    if A != 0:
        if A == 1:
            terms.append("x")
        elif A == -1:
            terms.append("-x")
        else:
            terms.append(f"{A:g}x")

    \end{lstlisting}
\end{frame}

\begin{frame}[fragile]
    \frametitle{Python+C code}

    \begin{lstlisting}
 # By term
    if B != 0:
        sign = "+" if B > 0 and terms else ""
        if B == 1:
            terms.append(f"{sign}y")
        elif B == -1:
            terms.append(f"{sign}-y")
        else:
            terms.append(f"{sign}{B:g}y")
  # Constant term
    if C != 0:
        sign = "+" if C > 0 and terms else ""
        terms.append(f"{sign}{C:g}")
    if not terms:
        return "0 = 0"
    return " ".join(terms) + " = 0"
    \end{lstlisting}
\end{frame}

\begin{frame}[fragile]
    \frametitle{Python+C code}

    \begin{lstlisting}
# Example: Square vertices (A,B,C,D)
vertices = [(0,0), (1,0), (1,1), (0,1)]
lines = get_diagonals(vertices)

eq1 = format_equation(*lines[0])
eq2 = format_equation(*lines[1])

print("Diagonal AC:", eq1)
print("Diagonal BD:", eq2)

# Plotting
square_x, square_y = zip(*vertices, vertices[0])
plt.plot(square_x, square_y, "b-", label="Square")
    \end{lstlisting}
\end{frame}

\begin{frame}[fragile]
    \frametitle{Python+C code}

    \begin{lstlisting}
# Diagonal AC
plt.plot([vertices[0][0], vertices[2][0]], [vertices[0][1], vertices[2][1]], "r--", label=f"AC: {eq1}")

# Diagonal BD
plt.plot([vertices[1][0], vertices[3][0]], [vertices[1][1], vertices[3][1]], "g--", label=f"BD: {eq2}")

plt.legend(loc="upper right")
plt.gca().set_aspect("equal", adjustable="box")
plt.grid(True)
plt.savefig("/home/user/Matrix/Matgeo_assignments/4.3.13/figs/Figure_1")
plt.show()
    \end{lstlisting}
\end{frame}

\begin{frame}[fragile]
    \frametitle{Python code}

    \begin{lstlisting}
import matplotlib as mp
mp.use("TkAgg")
import numpy as np
import matplotlib.pyplot as plt

def line_equation_cartesian(point, direction):
    """
    Returns line equation in Cartesian form: Ax + By + C = 0
    given a point and a direction vector.
    """
    x0, y0 = point
    a, b = direction
    
    # Normal vector
    A, B = -b, a
    C = -(A*x0 + B*y0)
    return A, B, C
    \end{lstlisting}
\end{frame}

\begin{frame}[fragile]
    \frametitle{Python code}

    \begin{lstlisting}
def diagonals_of_square(vertices):
    """
    Given 4 vertices of a square (in order), compute equations of diagonals in Cartesian form.
    """
    A, B, C, D = vertices
    
    # Diagonals are AC and BD
    AC_dir = (C[0]-A[0], C[1]-A[1])
    BD_dir = (D[0]-B[0], D[1]-B[1])
    
    line1 = line_equation_cartesian(A, AC_dir)
    line2 = line_equation_cartesian(B, BD_dir)
    
    return line1, line2
    \end{lstlisting}
\end{frame}


\begin{frame}[fragile]
    \frametitle{Python code}

    \begin{lstlisting}
def format_equation(A, B, C):
    """
    Beautify line equation into readable string.
    Example: -1x + 1y + 0 -> -x + y = 0
    """
    terms = []

    # Handle Ax term
    if A != 0:
        if A == 1:
            terms.append("x")
        elif A == -1:
            terms.append("-x")
        else:
            terms.append(f"{A}x")

    \end{lstlisting}
\end{frame}

\begin{frame}[fragile]
    \frametitle{Python code}

    \begin{lstlisting}
    # Handle By term
    if B != 0:
        sign = "+" if B > 0 and terms else ""
        if B == 1:
            terms.append(f"{sign}y")
        elif B == -1:
            terms.append(f"{sign}-y")
        else:
            terms.append(f"{sign}{B}y")
            # Handle C constant term
    if C != 0:
        sign = "+" if C > 0 and terms else ""
        terms.append(f"{sign}{C}")

    # In case all are zero
    if not terms:
        return "0 = 0"
    return " ".join(terms) + " = 0"
    \end{lstlisting}
\end{frame}


\begin{frame}[fragile]
    \frametitle{Python code}

    \begin{lstlisting}
def plot_square_and_diagonals(vertices, line1, line2):
    """
    Plot square and its diagonals with equations shown on the plot.
    """
    A, B, C, D = vertices
    square_x = [A[0], B[0], C[0], D[0], A[0]]
    square_y = [A[1], B[1], C[1], D[1], A[1]]
    
    plt.plot(square_x, square_y, 'b-', label='Square') 
    # Plot diagonals
    plt.plot([A[0], C[0]], [A[1], C[1]], 'r--', label='Diagonal AC')
    plt.plot([B[0], D[0]], [B[1], D[1]], 'g--', label='Diagonal BD')
    # Equations
    eq1 = format_equation(*line1)
    eq2 = format_equation(*line2)
    \end{lstlisting}
\end{frame}

\begin{frame}[fragile]
    \frametitle{Python code}

    \begin{lstlisting}
  # Midpoints of diagonals
    mid_AC = ((A[0]+C[0])/2, (A[1]+C[1])/2)
    mid_BD = ((B[0]+D[0])/2, (B[1]+D[1])/2)
    
    # Place texts with slight offsets to avoid overlap
    plt.text(mid_AC[0]+0.05, mid_AC[1]+0.05, eq1, color='red', fontsize=10, ha='left')
    plt.text(mid_BD[0]-0.15, mid_BD[1]-0.1, eq2, color='green', fontsize=10, ha='right')
    
    plt.gca().set_aspect('equal', adjustable='box')
    plt.legend(loc="upper right")
    plt.grid(True)
    plt.savefig("/home/user/Matrix/Matgeo_assignments/4.3.13/figs/Figure_1")
    plt.show()
    \end{lstlisting}
\end{frame}

\begin{frame}{Plot}
    \begin{figure}[H]
    \centering
    \caption{Plot of square and its diagonals}
    \includegraphics[width=0.6\columnwidth]{figs/Figure_1.png}
    \label{fig:1}
    \end{figure}
\end{frame}





\end{document}