\let\negmedspace\undefined
\let\negthickspace\undefined
\documentclass[journal]{IEEEtran}
\usepackage[a5paper, margin=10mm, onecolumn]{geometry}
%\usepackage{lmodern} % Ensure lmodern is loaded for pdflatex
\usepackage{tfrupee} % Include tfrupee package

\setlength{\headheight}{1cm} % Set the height of the header box
\setlength{\headsep}{0mm}     % Set the distance between the header box and the top of the text

\usepackage{gvv-book}
\usepackage{gvv}
\usepackage{cite}
\usepackage{amsmath,amssymb,amsfonts,amsthm}
\usepackage{algorithmic}
\usepackage{graphicx}
\usepackage{textcomp}
\usepackage{xcolor}
\usepackage{txfonts}
\usepackage{listings}
\usepackage{enumitem}
\usepackage{mathtools}
\usepackage{gensymb}
\usepackage{comment}
\usepackage[breaklinks=true]{hyperref}
\usepackage{tkz-euclide} 
\usepackage{listings}
% \usepackage{gvv}                                        
\def\inputGnumericTable{}                                 
\usepackage[latin1]{inputenc}                                
\usepackage{color}                                            
\usepackage{array}                                            
\usepackage{longtable}                                       
\usepackage{calc}                                             
\usepackage{multirow}                                         
\usepackage{hhline}                                           
\usepackage{ifthen}                                           
\usepackage{lscape}
\usepackage{circuitikz}
\tikzstyle{block} = [rectangle, draw, fill=blue!20, 
    text width=4em, text centered, rounded corners, minimum height=3em]
\tikzstyle{sum} = [draw, fill=blue!10, circle, minimum size=1cm, node distance=1.5cm]
\tikzstyle{input} = [coordinate]
\tikzstyle{output} = [coordinate]


\begin{document}

\bibliographystyle{IEEEtran}
\vspace{3cm}

\title{4.3.13}
\author{EE25BTECH11026-Harsha}
 \maketitle
% \newpage
% \bigskip
{\let\newpage\relax\maketitle}

\renewcommand{\thefigure}{\theenumi}
\renewcommand{\thetable}{\theenumi}
\setlength{\intextsep}{10pt} % Space between text and floats


\numberwithin{equation}{enumi}
\numberwithin{figure}{enumi}
\renewcommand{\thetable}{\theenumi}

\textbf{Question}:\\
Equations of the diagonals of the square formed by the lines x = 0, y = 0, x = 1 and y = 1 are \underline{\hspace{2cm}}.\\
\solution \\
Let us solve the given equation theoretically and then verify the solution computationally.\\
\\
According to the question,\\
The vertices of the square are ,
\begin{align}
    \vec{a}=\myvec{0\\0}\;\;\vec{b}=\myvec{1\\0}\;\;\vec{c}=\myvec{1\\1}\;\;\vec{d}=\myvec{0\\1}
\end{align}
To compute the equation of the diagnols , we can use the normal form of the equation,which is given by
\begin{align}
    \vec{n}^{\top}\vec{x}=0\; \text{for the lines passing through the origin}
\end{align}
\begin{align}
    \vec{n}^{\top}\vec{x}=1 \;\text{for the lines not passing through the origin}
\end{align}
\\
For diagonal $\vec{c}-\vec{a}$, as it passes through the origin,
\begin{align}
    \therefore \vec{n}^{\top}\vec{x}=0
\end{align}
By substituting the vector through which it passes through,
\begin{align}
    \vec{n}^{\top}\myvec{1\\1}=0
\end{align}
\begin{align}
    \implies \vec{n}=\myvec{-1\\1}
\end{align}
\begin{align}
    \therefore \myvec{-1&&1}\vec{x}=0
\end{align}
\newpage
\vspace*{0.25cm}

But, for diagonal $\vec{d}-\vec{b}$, as the diagonal doesn't pass through the origin,
\begin{align}
    \vec{n}^{\top}\vec{x}=1
\end{align}
\begin{align}
    \therefore \vec{n}^{\top}\myvec{1&&0\\0&&1}=\myvec{1\\1}
\end{align}
\begin{align}
    \implies \vec{n}=\myvec{1\\1}
\end{align}
\begin{align}
    \therefore \myvec{1&&1}\vec{x}=1
\end{align}
\\
From the figure, it is clearly verified that the theoretical solution matches with the computational solution.\\
\begin{figure}[H]
    \centering
    \includegraphics[width=0.6\columnwidth]{figs/Figure_1.png}
    \caption*{Plot of Square with diagonals}
    \label{fig:1}
\end{figure}
\end{document}
