\let\negmedspace\undefined
\let\negthickspace\undefined
\documentclass[journal]{IEEEtran}
\usepackage[a5paper, margin=10mm, onecolumn]{geometry}
\usepackage{tfrupee}

\setlength{\headheight}{1cm}
\setlength{\headsep}{0mm}

\usepackage{cite}
\usepackage{amsmath,amssymb,amsfonts}
\usepackage{algorithmic}
\usepackage{graphicx}
\usepackage{textcomp}
\usepackage{xcolor}
\usepackage{float}
\usepackage{txfonts}
\usepackage{listings}
\usepackage{enumitem}
\usepackage{mathtools}
\usepackage{gensymb}
\usepackage{comment}
\usepackage[breaklinks=true]{hyperref}
\usepackage{tkz-euclide} 
\usepackage{caption}
\usepackage{longtable,multirow,array,hhline}
\newcommand{\solution}{\textbf{Solution: }}

% Define brak for matrices
\newcommand{\brak}[1]{\begin{pmatrix}#1\end{pmatrix}}

\begin{document}

\bibliographystyle{IEEEtran}
\vspace{3cm}

\title{1.4.25}
\author{EE25BTECH11009 - Anshu kumar ram}
{\let\newpage\relax\maketitle}

\renewcommand{\thefigure}{\theenumi}
\renewcommand{\thetable}{\theenumi}
\setlength{\intextsep}{10pt}
\numberwithin{equation}{enumi}
\numberwithin{figure}{enumi}

\parindent 0px
\textbf{Question:} \\
Find the position vector of a point \( R \) which divides the line joining two points \( P \) and \( Q \) whose position vectors are \(2\mathbf a + \mathbf b\) and \(\mathbf a - 3\mathbf b\) externally in the ratio \(1:2\).

\solution \\

\begin{align}
P &= 2\mathbf a + \mathbf b
   = \brak{2\\1}, \\
Q &= \mathbf a - 3\mathbf b
   = \brak{1\\-3}.
\end{align}

For external division of \(PQ\) in ratio \(1:2\),  
the point \(R\) is given by
\[
R = \frac{1 \cdot Q - 2 \cdot P}{1-2}.
\]

\begin{align}
R &= \frac{1}{-1}\brak{\brak{1\\-3} - 2\brak{2\\1}} \\[6pt]
  &= -\brak{1-4\\-3-2} \\[6pt]
  &= -\brak{-3\\-5} \\[6pt]
  &= \brak{3\\5}.
\end{align}

So the position vector is
\[
\boxed{\,R = 3\mathbf a + 5\mathbf b\,}
\]

\begin{figure}[ht!]
    \centering
    \includegraphics[width=\columnwidth, keepaspectratio]{figs/section_graph.png}
    \captionsetup{justification=centering}
    \caption{Graph for Question 2}
    \label{fig:section_graph}
\end{figure}


\end{document}
