\documentclass{beamer}
\usetheme{Madrid}

% Packages
\usepackage{amsmath,amssymb,amsfonts}
\usepackage{mathtools}
\usepackage{graphicx}
\usepackage{gensymb}
\usepackage{caption}
\usepackage{tikz}
\usetikzlibrary{calc}
\usepackage{listings}
\usepackage{xcolor}

% Custom command for matrices
\newcommand{\brak}[1]{\begin{pmatrix}#1\end{pmatrix}}

% C code style
\lstdefinestyle{CStyle}{
    language=C,
    basicstyle=\ttfamily\tiny,
    keywordstyle=\color{blue}\bfseries,
    stringstyle=\color{red},
    commentstyle=\color{green!60!black}\itshape,
    numbers=left,
    numberstyle=\tiny,
    stepnumber=1,
    numbersep=8pt,
    showstringspaces=false,
    frame=single,
    breaklines=true,
    tabsize=4,
    captionpos=b
}

% Python code style
\lstdefinestyle{PyStyle}{
    language=Python,
    basicstyle=\ttfamily\tiny,
    keywordstyle=\color{blue}\bfseries,
    stringstyle=\color{red!70!black},
    commentstyle=\color{green!60!black}\itshape,
    numbers=left,
    numberstyle=\tiny,
    stepnumber=1,
    numbersep=8pt,
    showstringspaces=false,
    frame=single,
    breaklines=true,
    tabsize=4,
    captionpos=b
}

% Title info
\title{Problem 1.4.25}
\author{EE25BTECH11009 – Anshu Kumar Ram}
\date{\today}

\begin{document}

% ---------------- Title Page ----------------
\begin{frame}
    \titlepage
\end{frame}

% ---------------- Question ----------------
\begin{frame}{Question}
\textbf{Find the position vector of a point $R$ which divides the line joining two points $P$ and $Q$ whose position vectors are $2\vec{a} + \vec{b}$ and $\vec{a} - 3\vec{b}$ externally in the ratio $1:2$.}
\end{frame}

% ---------------- Solution ----------------
\begin{frame}{Solution Step 1}
\textbf{Step 1: Represent points in coordinates}  
\begin{equation}
\vec{P} = 2\vec{a} + \vec{b} = \brak{2\\1}
\end{equation}
\begin{equation}
\vec{Q} = \vec{a} - 3\vec{b} = \brak{1\\-3}
\end{equation}
\end{frame}

\begin{frame}{Solution Step 2}
\textbf{Step 2: Apply section formula (external division)}
\begin{equation}
\vec{R} = \frac{1 \cdot \vec{Q} - 2 \cdot \vec{P}}{1 - 2}
\end{equation}
\begin{equation}
= \frac{1}{-1}\left(\brak{1\\-3} - 2\brak{2\\1}\right)
\end{equation}
\begin{equation}
= -\brak{-3\\-5}
\end{equation}
\begin{equation}
= \brak{3\\5}
\end{equation}
\begin{equation}
\vec{R} = 3\vec{a} + 5\vec{b}
\end{equation}
\end{frame}

% ---------------- Graph ----------------
\begin{frame}{Graphical Representation}
    \centering
    \includegraphics[width=0.8\linewidth]{figs/section_graph.png}
    \captionsetup{justification=centering}
    \captionof{figure}{Graph for Question 2}
\end{frame}

% ---------------- C Code ----------------
\begin{frame}[fragile]{C Code: section() Function}
\begin{lstlisting}[style=CStyle]
void section(double* P, double* Q, double* R, int m) {
    for (int i = 0; i < m; i++) {
        R[i] = (Q[i] - 2 * P[i]) / (1 - 2);
    }
}
\end{lstlisting}
\end{frame}

\begin{frame}[fragile]{C Code: line\_gen() Function}
\begin{lstlisting}[style=CStyle]
void line_gen(double* X, double* Y, const double* A, const double* B, int n, int m) {
    double temp[2];
    for (int i = 0; i < 2; i++) {
        temp[i] = (B[i] - A[i]) / (double)n;
    }
    for (int i = 0; i <= n; i++) {
        X[i] = A[0] + temp[0] * i;
        Y[i] = A[1] + temp[1] * i;
    }
}
\end{lstlisting}
\end{frame}

% ---------------- Python with C Integration ----------------
\begin{frame}[fragile]{Python + C: Load Library}
\begin{lstlisting}[style=PyStyle]
import ctypes
import numpy as np
import matplotlib.pyplot as plt

handc = ctypes.CDLL("./func.so")

# section function
handc.section.argtypes = [
    ctypes.POINTER(ctypes.c_double),
    ctypes.POINTER(ctypes.c_double),
    ctypes.POINTER(ctypes.c_double),
    ctypes.c_int
]
handc.section.restype = None

# line_gen function
handc.line_gen.argtypes = [
    ctypes.POINTER(ctypes.c_double),
    ctypes.POINTER(ctypes.c_double),
    ctypes.POINTER(ctypes.c_double),
    ctypes.POINTER(ctypes.c_double),
    ctypes.c_int,
    ctypes.c_int
]
handc.line_gen.restype = None
\end{lstlisting}
\end{frame}

\begin{frame}[fragile]{Python + C: Compute \& Plot}
\begin{lstlisting}[style=PyStyle]
m = 2
a = np.array([1,0], dtype=np.float64)
b = np.array([0,1], dtype=np.float64)
P = 2*a + b
Q = a - 3*b
R = np.zeros(m, dtype=np.float64)

handc.section(P.ctypes.data_as(ctypes.POINTER(ctypes.c_double)),
              Q.ctypes.data_as(ctypes.POINTER(ctypes.c_double)),
              R.ctypes.data_as(ctypes.POINTER(ctypes.c_double)),
              m)

n = 20
X_l = np.zeros(n, dtype=np.float64)
Y_l = np.zeros(n, dtype=np.float64)
handc.line_gen(X_l.ctypes.data_as(ctypes.POINTER(ctypes.c_double)),
               Y_l.ctypes.data_as(ctypes.POINTER(ctypes.c_double)),
               Q.ctypes.data_as(ctypes.POINTER(ctypes.c_double)),
               R.ctypes.data_as(ctypes.POINTER(ctypes.c_double)),
               n, m)

plt.plot(X_l, Y_l, "g--", label="Line PQ")
plt.scatter(P[0], P[1], color="blue", s=50)
plt.scatter(Q[0], Q[1], color="green", s=50)
plt.scatter(R[0], R[1], color="red", s=50, label="R")
plt.show()
\end{lstlisting}
\end{frame}

% ---------------- Pure Python ----------------
\begin{frame}[fragile]{Pure Python: Functions \& Setup}
\begin{lstlisting}[style=PyStyle]
import sys
sys.path.insert(0, '/home/anshu-ram/matgeo/codes/CoordGeo')
import numpy as np
import matplotlib.pyplot as plt

from line.funcs import *
from triangle.funcs import *
from conics.funcs import circ_gen

def section_point(P, Q, m, n, external=True):
    if external:
        return (m*Q - n*P)/(m-n)
    else:
        return (m*Q + n*P)/(m+n)
\end{lstlisting}
\end{frame}

\begin{frame}[fragile]{Pure Python: Compute \& Plot}
\begin{lstlisting}[style=PyStyle]
a = np.array([1,0]).reshape(-1,1)
b = np.array([0,1]).reshape(-1,1)
P = 2*a + b
Q = a - 3*b
R = section_point(P, Q, 1, 2, external=True)

x_PQ = line_gen_num(P, Q, 20)
x_PR = line_gen_num(P, R, 20)
x_QR = line_gen_num(Q, R, 20)

plt.plot(x_PQ[0,:], x_PQ[1,:], "g--", label="Line PQ")
plt.plot(x_PR[0,:], x_PR[1,:], "r--", label="Line PR")
plt.plot(x_QR[0,:], x_QR[1,:], "b--", label="Line QR")
tri_coords = np.hstack((P,Q,R))
plt.scatter(tri_coords[0,:], tri_coords[1,:])
plt.show()
\end{lstlisting}
\end{frame}
\end{document}
