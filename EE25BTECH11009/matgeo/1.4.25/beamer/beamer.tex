\documentclass{beamer}

% Theme (you can change to Warsaw, Madrid, CambridgeUS, etc.)
\usetheme{Madrid}

% Packages
\usepackage{amsmath,amssymb,amsfonts}
\usepackage{mathtools}
\usepackage{graphicx}
\usepackage{gensymb}
\usepackage{caption}
\usepackage{tikz}
\usetikzlibrary{calc}

% Custom command for matrices
\newcommand{\brak}[1]{\begin{pmatrix}#1\end{pmatrix}}

% Title info
\title{Problem 1.4.25}
\author{EE25BTECH11009 – Anshu Kumar Ram}
\date{\today}

\begin{document}

% Title Page
\begin{frame}
    \titlepage
\end{frame}

% Question Frame
\begin{frame}{Question}
    \textbf{Find the position vector of a point \( R \) which divides the line joining two points \( P \) and \( Q \) whose position vectors are \(2\mathbf a + \mathbf b\) and \(\mathbf a - 3\mathbf b\) externally in the ratio \(1:2\).}
\end{frame}

% Solution Frame (step 1)
\begin{frame}{Solution}
\textbf{Step 1: Represent points in coordinates}  
\begin{align*}
P &= 2\mathbf a + \mathbf b = \brak{2\\1}, \\
Q &= \mathbf a - 3\mathbf b = \brak{1\\-3}.
\end{align*}
\end{frame}

% Solution Frame (step 2)
\begin{frame}{Solution}
\textbf{Step 2: Apply section formula (external division)}

\begin{align*}
R &= \frac{1 \cdot Q - 2 \cdot P}{1 - 2} \\[6pt]
  &= \frac{1}{-1}\left(\begin{pmatrix}1\\-3\end{pmatrix} - 2\begin{pmatrix}2\\1\end{pmatrix}\right) \\[6pt]
  &= -\begin{pmatrix}1-4\\-3-2\end{pmatrix} \\[6pt]
  &= -\begin{pmatrix}-3\\-5\end{pmatrix} \\[6pt]
  &= \begin{pmatrix}3\\5\end{pmatrix}.
\end{align*}

So, the position vector is
\[
\boxed{R = 3\mathbf{a} + 5\mathbf{b}}
\]
\end{frame}


% Graph Frame
\begin{frame}{Graphical Representation}
    \centering
    \includegraphics[width=0.8\linewidth, keepaspectratio]{figs/section_graph.png}
    \captionsetup{justification=centering}
    \captionof{figure}{Graph for Question 2}
\end{frame}

\end{document}
