\let\negmedspace\undefined
\let\negthickspace\undefined
\documentclass[journal]{IEEEtran}
\usepackage[a5paper, margin=10mm, onecolumn]{geometry}
%\usepackage{lmodern} % Ensure lmodern is loaded for pdflatex
\usepackage{tfrupee} % Include tfrupee package

\setlength{\headheight}{1cm} % Set the height of the header box
\setlength{\headsep}{0mm}     % Set the distance between the header box and the top of the text

\usepackage{gvv-book}
\usepackage{gvv}
\usepackage{cite}
\usepackage{amsmath,amssymb,amsfonts,amsthm}
\usepackage{algorithmic}
\usepackage{graphicx}
\usepackage{textcomp}
\usepackage{xcolor}
\usepackage{txfonts}
\usepackage{listings}
\usepackage{enumitem}
\usepackage{mathtools}
\usepackage{gensymb}
\usepackage{comment}
\usepackage[breaklinks=true]{hyperref}
\usepackage{tkz-euclide} 
\usepackage{listings}
\def\inputGnumericTable{}                                 
\usepackage[latin1]{inputenc}                                
\usepackage{color}                                            
\usepackage{array}                                            
\usepackage{longtable}                                       
\usepackage{calc}                                             
\usepackage{multirow}                                         
\usepackage{hhline}                                           
\usepackage{ifthen}                                           
\usepackage{lscape}
\begin{document}

\bibliographystyle{IEEEtran}
\vspace{3cm}

\title{2.6.21}
\author{AI25BTECH11012 - GARIGE UNNATHI}
% \maketitle
% \newpage
% \bigskip
{\let\newpage\relax\maketitle}


\renewcommand{\thefigure}{\theenumi}
\renewcommand{\thetable}{\theenumi}
\setlength{\intextsep}{10pt} % Space between text and floats


\numberwithin{equation}{enumi}
\numberwithin{figure}{enumi}
\renewcommand{\thetable}{\theenumi}


\textbf{Question}:\\
Find the area of the triangle whose vertices are $(3,8)$ , $(-4,2)$ and $(5,1)$.\\
\textbf{Solution: }

 \begin{table}[h!]    
      \centering
      \begin{center}
    \begin{tabular}{|c|c|} 
        \hline
            \textbf{Variable}  & \textbf{Formula} \\ 
        \hline
            $a$   & $a = \myvec{4 \\ -1 \\ 1}$ \\ 
        \hline
            $b$   &  $b = \myvec{2 \\ -2 \\ 1}$\\ 
        \hline
           \end{tabular}
\end{center}  

      \caption{Variables Used}
      \label{}
    \end{table}

The area of a triangle ABC is given by :

\begin{align*}
\frac{1}{2}\,\lVert \vec{(A-B)}\times\vec{(A-C)}\rVert
\end{align*}


\begin{align}
   \vec{A}-\vec{B} = \myvec{3 \\ 8} - \myvec{-4 \\ 2} = \myvec{7 \\ 6}
\end{align}

\begin{align}
   \vec{A}-\vec{C} = \myvec{3 \\ 8} - \myvec{5 \\ 1} = \myvec{-2 \\ 7}
\end{align}


\begin{align}
\frac{1}{2}\,\lVert \vec{(A-B)}\times\vec{(A-C)}\rVert = 14
\end{align}

\bigskip
The area of the triangle ABC is 14


\begin{figure}[h!]
   \centering
   \includegraphics[width=0.7\linewidth]{/Users/unnathi/Documents/ee1030-2025/ai25btech11012/matgeo/2.6.21/figs/figs.png}
   \caption{}
   \label{stemplot}
\end{figure}


\end{document}
