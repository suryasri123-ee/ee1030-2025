\documentclass{article}
\usepackage{graphicx} % Required for inserting images
\usepackage[a4paper,margin=1in]{geometry}
\usepackage{enumitem}
\usepackage{setspace}
\usepackage{titlesec}
\usepackage{lmodern} % Better font rendering
\usepackage{parskip} % Add space between paragraphs
\usepackage{ulem}
\setlength{\parindent}{1pt} % No paragraph indentation
\usepackage{bm}
\usepackage{amsmath,amssymb}
\usepackage{mathtools}
\usepackage{fancyhdr}

\begin{document}

\begin{center}
    \textbf{2007} \\
        \vspace{1em}
	    {\LARGE \textbf{MA: Mathematics}}
	    \end{center}

	    \vspace{1em}

	    \textbf{Duration :} Three Hours \hfill \textbf{Maximum Marks : 150}

	    \vspace{2em}

	    \begin{center}
	        \textbf{Read the following instructions carefully.}
		\end{center}

		\vspace{1em}

		\begin{enumerate}[leftmargin=1.5em, label=\arabic*.]
		    \item This question paper contains 85 objective type questions. Q.1 to Q.20 carry \textit{one} mark each and Q.21 to Q.85 carry two marks each.

		        \item Attempt all the questions.

			    \item Questions must be answered on \textbf{Objective Response Sheet (ORS)} by darkening the appropriate bubble (marked A, B, C, D) using HB pencil against the question number on the left hand side of the \textbf{ORS}. \textit{Each question has only one correct answer}.

			        \item Wrong answers will carry \textbf{NEGATIVE} marks. In Q.1 to Q.20, \textbf{0.25} mark will be deducted for each wrong answer. In Q.21 to Q.76, Q.78, Q.80, Q.82 and in Q.84, \textbf{0.5} mark will be deducted for each wrong answer. However, there is no negative marking in Q.77, Q.79, Q.81, Q.83 and in Q.85. More than one answer bubbled against a question will be taken as an incorrect response. Unattempted questions will not carry any marks.

				    \item Write your registration number, your name and name of the examination centre at the specified locations on the right half of the \textbf{ORS}.

				        \item Using HB pencil, darken the appropriate bubble under each digit of your registration number and the letters corresponding to your paper code.

					    \item Calculator is allowed in the examination hall.

					        \item Charts, graph sheets or tables are NOT allowed in the examination hall.

						    \item Rough work can be done on the question paper itself. Additionally blank pages are given at the end of the question paper for rough work.

						        \item This question paper contains \textbf{24} printed pages including pages for rough work. Please check all pages and report, if there is any discrepancy.
							\end{enumerate}

							\vspace{25em}
							\begin{center}
							    {MA 1/24}
							    \end{center}
							    \vspace{1em}
							    \fancyfoot{\small S/121 Food/06-MA-1A}

							    \newpage

							    \begin{center}
							        \uline {\textbf{Notations and Definitions used in the paper}}
								\end{center}

								\vspace{1em}
								\begin{itemize}
								    \item $\mathbb{R}$ : The set of real numbers.

								        \item $\mathbb{R}^n = \{(x_1, x_2, \ldots, x_n): x_i \in \mathbb{R},\ i = 1,2,\ldots,n\}$

									    \item $\mathbb{C}$ : The set of complex numbers.

									        \item $\phi$ : The empty set.

										    \item For any subset $E$ of $X$ (or a topological space $X$):
										        \begin{itemize}
											        \item $\bar{E}$ : The closure of $E$ in $X$.
												        \item $E^\circ$ : The interior of $E$ in $X$.
													        \item $E^c$ : The complement of $E$ in $X$.
														    \end{itemize}

														        \item $\mathbb{Z}_n = \{0, 1, 2, \ldots, n-1\}$

															    \item $A^t$ : The transpose of a matrix $A$.
															    \end{itemize}
															    \vspace{45em}
															    \begin{center}
															        {MA 2/24}
																\end{center}
																\newpage

																\begin{center}
																    \textbf{Q.1-Q.20 carry one mark each.}
																    \end{center}
																    \vspace{1em}

																    \noindent{Q.1} \quad Consider $\mathbb{R}^2$ with the usual topology. Let $S = \{(x, y) \in \mathbb{R}^2 : x \text{ is an integer} \}$. Then $S$ is
																    \vspace{1em}
																    \begin{enumerate} [label=(\Alph*), leftmargin=2em]
																        \item open but NOT closed
																	    \item both open and closed
																	        \item neither open nor closed
																		    \item closed but NOT open
																		    \end{enumerate}
																		    \vspace{1em}

																		    Q.2 \quad Suppose $X = \{ \alpha, \beta, \delta \}$. Let
																		    \[
																		    \mathcal{T}_1 = \{\emptyset, X, \{\alpha\}, \{\alpha, \beta\} \} \quad \text{and} \quad \mathcal{T}_2 = \{\emptyset, X, \{\alpha\}, \{\beta, \delta\} \}.
																		    \]
																		    Then
																		    \begin{itemize}
																		        \item[(A)] both $\mathcal{T}_1 \cap \mathcal{T}_2$ and $\mathcal{T}_1 \cup \mathcal{T}_2$ are topologies
																			    \item[(B)] neither $\mathcal{T}_1 \cap \mathcal{T}_2$ nor $\mathcal{T}_1 \cup \mathcal{T}_2$ is a topology
																			        \item[(C)] $\mathcal{T}_1 \cup \mathcal{T}_2$ is a topology but $\mathcal{T}_1 \cap \mathcal{T}_2$ is NOT a topology
																				    \item[(D)] $\mathcal{T}_1 \cap \mathcal{T}_2$ is a topology but $\mathcal{T}_1 \cup \mathcal{T}_2$ is NOT a topology
																				    \end{itemize}
																				    \vspace{1em}
																				    Q.3 \quad For a positive integer $n$, let $f_n : \mathbb{R} \to \mathbb{R}$ be defined by
																				    \[
																				    f_n(x) = 
																				    \begin{cases}
																				    \frac{1}{4n + 5} & \text{if } 0 \leq x \leq n, \\
																				    0 & \text{otherwise}.
																				    \end{cases}
																				    \]

																				    Then $\{f_n(x)\}$ converges to zero
																				    \vspace{1em}
																				    \begin{enumerate}
																				        \item[(A)] uniformly but NOT in $L^1$ norm
																					    \item[(B)] uniformly and also in $L^1$ norm
																					        \item[(C)] pointwise but NOT uniformly
																						    \item[(D)] in $L^1$ norm but NOT pointwise
																						    \end{enumerate}
																						    \vspace{1em}
																						    Q.4 \quad Let $P_1$ and $P_2$ be two projection operators on a vector space. Then
																						    \begin{enumerate}
																						        \item[(A)] $P_1 + P_2$ is a projection if $P_1 P_2 = P_2 P_1 = 0$
																							    \item[(B)] $P_1 - P_2$ is a projection if $P_1 P_2 = P_2 P_1 = 0$
																							        \item[(C)] $P_1 + P_2$ is a projection
																								    \item[(D)] $P_1 - P_2$ is a projection
																								    \end{enumerate}

																								    \vspace{1em}

																								    Q.5 \quad Consider the system of linear equations
																								    \[
																								    \begin{aligned}
																								    x + y + z &= 3 \\
																								    x - y - z &= 4 \\
																								    -5y + kz &= 6
																								    \end{aligned}
																								    \]
																								    Then the value of $k$ for which this system has an infinite number of solutions is
																								    \vspace{1em}
																								    \newline 
																								    \noindent (A) $k = -5$ \hspace{2cm} (B) $k = 0$ \hspace{2cm}(C) $k = 1$ \hspace{2cm} (D) $k = 3$

																								    \vspace{1em}

																								    \begin{center}
																								        {MA 3/24}
																									\end{center}
																									\newpage

																									Q.6 \quad Let 
																									\[
																									A = \begin{bmatrix}
																									1 & 1 & 1 \\
																									2 & 2 & 3 \\
																									x & y & z
																									\end{bmatrix}
																									\]
																									and let $V = \{(x, y, z) \in \mathbb{R}^3 : \det(A) = 0\}$. Then the dimension of $V$ equals:
																									\vspace{1em}
																									\newline \noindent (A)$0$ \hspace{2cm} (B) $1$\hspace{2cm}(C) $2$ \hspace{2cm} (D) $3$

																									\vspace{1em}
																									Q.7 \quad Let $S = \{0\} \cup \left\{ \frac{1}{4n + 7} : n = 1, 2, \ldots \right\}$. Then the number of analytic functions which vanish only on $S$ is:
																									\vspace{1em}
																									\newline \noindent (A) infinite \hspace{2cm} (B) $0$\hspace{2cm}(C) $1$ \hspace{2cm} (D) $2$

																									\vspace{1em}
																									{Q.8} \quad It is given that $\sum_{n=0}^\infty a_n z^n$ converges at $z = 3 + i4$. Then the radius of convergence of the power series $\sum_{n=0}^\infty a_n z^n$ is:
																									\vspace{1em}
																									\newline \noindent (A) $\leq 5$ \hspace{2cm} (B) $\geq 5$ \hspace{2cm} (C) $< 5$ \hspace{2cm} (D) $> 5$
																									\vspace{1em}

																									Q.9 \quad The value of $\alpha$ for which $G = \langle \alpha, 1, 3, 9, 19, 27 \rangle$ is a cyclic group under multiplication modulo $56$ is:
																									\vspace{1em}
																									\newline 
																									\noindent (A) $5$ \hspace{2cm} (B) $15$ \hspace{2cm}(C) $25$ \hspace{2cm} (D) $35$



																									\vspace{1em}

																									\textbf{Q.10} \quad Consider $\mathbb{Z}_{24}$ as the additive group modulo $24$. Then the number of elements of order $8$ in the group $\mathbb{Z}_{24}$ is:

																									\vspace{1em}

																									\noindent  (A)$1$ \hspace{2cm} (B) $2$ \hspace{2cm}(C) $3$ \hspace{2cm} (D) $4$

																									\vspace{1em}


																									Q.11 \quad Define $f : \mathbb{R}^2 \to \mathbb{R}$ by
																									\[
																									f(x, y) =
																									\begin{cases}
																									1, & \text{if } xy = 0, \\
																									2, & \text{otherwise}.
																									\end{cases}
																									\]
																									If $S = \{(x, y) : f \text{ is continuous at the point } (x, y)\}$, then:

																									\noindent (A) $S$ is open \hspace{5cm} (B) $S$ is connected
																									\newline \noindent (C) $S = \emptyset$ \hspace{5.5cm}  (D) $S$ is closed
																									\vspace{20em}
																									\begin{center}
																									    {MA 4/24}
																									    \end{center}
																									    \newpage
																									    Q.12 \quad Consider the linear programming problem
																									    \[
																									    \begin{aligned}
																									    & \text{Maximize } z = c_1 x_1 + c_2 x_2, \quad c_1, c_2 > 0, \\
																									    & \text{subject to} \\
																									    & x_1 + x_2 \le 3 \\
																									    & 2x_1 + 3x_2 \le 4 \\
																									    & x_i \ge 0
																									    \end{aligned}
																									    \]
																									    Then:
																									    \vspace{1em}
																									    \noindent \newline (A) the primal has an optimal solution but the dual does NOT have an optimal solution
																									    \noindent \newline (B) both the primal and the dual have optimal solutions
																									    \noindent \newline (C)  the dual has an optimal solution but the primal does NOT have an optimal solution
																									    \noindent \newline (D) neither the primal nor the dual have optimal solutions
																									    \vspace{1em}

																									    Q.13 \quad Let $f(x) = x^{10} + x - 1$, $x \in \mathbb{R}$ and let $x_k = k$, $k=0,1,2,\dots,10$. Then the value of the divided difference
																									    \[
																									    f[x_0, x_1, x_2, \dots, x_{10}]
																									    \]
																									    is:
																									    \newline 
																									    \noindent (A) $-1$ \hspace{2cm} (B) $0$ \hspace{2cm}(C) $1$ \hspace{2cm} (D) $10$
																									    \vspace{1em}
																									    \newline
																									    Q.14 \quad Let $X, Y$ be jointly distributed random variables having the joint probability density function
																									    \[
																									    f(x,y) = \begin{cases}
																									    1, & \text{if } 0 < x + y < 1, \\
																									    0, & \text{otherwise.}
																									    \end{cases}
																									    \]
																									    Then $P(Y \ge \max(X, 1 - X))$ is
																									    \vspace{1em}
																									    \newline

																									    \noindent (A) $\tfrac{1}{2}$ \hspace{2cm} (B) $1$ \hspace{2cm}(C) $\tfrac{1}{4}$ \hspace{2cm} (D) $\tfrac{1}{6}$
																									    \vspace{1em}

																									    Q.15 \quad Let $X_1, X_2, \dots$ be a sequence of independent and identically distributed chi-square random variables, each having 4 degrees of freedom. Define
																									    \[
																									    S_n = \sum_{i=1}^n X_i
																									    \]
																									    If $\frac{S_n}{n} \to \mu$ as $n \to \infty$, then $\mu$ =
																									    \vspace{1em}
																									    \newline 
																									    \noindent (A) $8$ \hspace{2cm} (B) $16$ \hspace{2cm}(C) $24$ \hspace{2cm} (D) $32$
																									    \vspace{15em}
																									    \begin{center}
																									        {MA 5/24}
																										\end{center}
																										\newpage
																										Q.16\quad Let $\{E_n : n = 1, 2, \dots\}$ be a decreasing sequence of Lebesgue measurable sets on $\mathbb{R}$ and let $F$ be a Lebesgue measurable set on $\mathbb{R}$ such that $E_n \cap F = \emptyset$. Suppose that $F$ has Lebesgue measure 2 and the Lebesgue measure of $E_n$ equals $\dfrac{2n + 2}{3n + 1}$, $n = 1, 2, \dots$\\
																										\vspace{1.5em}
																										Then the Lebesgue measure of the set $\left(\bigcap_{n=1}^{\infty} E_n\right) \cup F$ equals


																										\noindent (A) $\tfrac{5}{3}$ \hspace{2cm} (B) $2$ \hspace{2cm}(C) $\tfrac{7}{3}$ \hspace{2cm} (D) $\tfrac{8}{3}$
																										\vspace{1em}
																										\newline
																										Q.17\quad The extremum for the variational problem
																										\[
																										\int_0^{\frac{\pi}{8}} ((y')^2 + 2yy' - 16y^2) \, dx, \quad y(0) = 0, \; y\left(\frac{\pi}{8}\right) = 1,
																										\]
																										occurs for the curve
																										\newline
																										\noindent (A) $y = \sin(4x)$ \hspace{6cm} (B)  $y = \sqrt{2} \sin(2x)$
																										\newline \noindent (C) $y = 1 - \cos(4x)$ \hspace{5.5cm}  (D) $y = \dfrac{1 - \cos(8x)}{2}$
																										\vspace{1em}
																										\newline
																										Q.18\quad Suppose $y_p(x) = x \cos(2x)$ is a particular solution of 
																										\[
																										y'' + \alpha y = \sin(2x).
																										\]
																										Then the constant $\alpha$ equals
																										\vspace{1em}
																										\newline 
																										\noindent (A) $-4$ \hspace{2cm} (B) $-2$ \hspace{2cm}(C) $2$ \hspace{2cm} (D) $4$

																										\vspace{1em}
																										Q.19\quad If $F(s) = \tan^{-1}(s) + k$ is the Laplace transform of some function $f(t)$, $t \ge 0$, then $k =$
																										\vspace{1em}
																										\newline 
																										\noindent (A) $-\pi$ \hspace{2cm} (B) $\dfrac{\pi}{2}$ \hspace{2cm}(C) $0$ \hspace{2cm} (D) $\dfrac{\pi}{2}$

																										\vspace{1em}

																										Q.20\quad Let $S = \{(0,1,1), (1,0,1), (-1,2,1)\} \subseteq \mathbb{R}^3$. Suppose $\mathbb{R}^3$ is endowed with the standard inner product $\langle \cdot , \cdot \rangle$. Define $M = \{x \in \mathbb{R}^3 : \langle x, y \rangle = 0 \text{ for all } y \in S\}$. Then the dimension of $M$ equals
																										\vspace{1em}
																										\newline 
																										\noindent (A) $0$ \hspace{2cm} (B) $1$ \hspace{2cm}(C) $2$ \hspace{2cm} (D) $3$

																										\vspace{25em}
																										\begin{center}
																										    {MA 6/24}
																										    \end{center}
																										    \newpage
																										    \begin{center}
																										        \textbf{Q.21-Q.75 carry one mark each.}
																											\end{center}
																											\vspace{2em}
																											Q.21\quad Let $X$ be an uncountable set and let
																											\[
																											\tau = \{U \subseteq X : X \setminus U \text{ is countable or } X \setminus U \text{ is finite} \}.
																											\]
																											Then the topological space $(X, \tau)$
																											\begin{itemize}
																											    \item[(A)] is separable
																											        \item[(B)] is Hausdorff
																												    \item[(C)] has a countable basis
																												        \item[(D)] has a countable basis at each point
																													\end{itemize}

																													\vspace{1em}

																													Q.22\quad Suppose $(X, \tau)$ is a topological space. Let $\{S_\alpha\}_{\alpha \in A}$ be a sequence of subsets of $X$.\newline Then
																													\begin{itemize}
																													    \item[(A)] $\left(S_1\bigcup S_2\right)^" = S_1^" \bigcup S_2^" $
																													        \item[(B)] $\left(\bigcap S_"\right)^" = \bigcap S_"$
																														    \item[(C)] $\overline{\bigcup S_"} = \bigcup_\alpha \overline{S_"}$
																														        \item[(D)] $\overline{S_1\bigcup S_2} = \overline{S_1} \cup \overline{S_2}$
																															\end{itemize}
																															\vspace{1cm}

																															Q.23\quad Let $(X, d)$ be a metric space. Consider the metric $\rho$ on $X$ defined by
																															\[
																															\rho(x,y) = \min(d(x,y), 1), \quad x,y \in X.
																															\]
																															Suppose $\tau$ and $\tau_1$ are topologies on $X$ defined by $d$ and $\rho$ respectively. Then

																															\noindent (A)  $\tau_1$ is a proper subset of $\tau_2$ \hspace{6cm} (B)  $\tau_2$ is a proper subset of $\tau_1$
																															\newline \noindent (C) neither $\tau_2$ nor $\tau_1$ is a subset of the other \hspace{4cm}  (D) $\tau_1 = \tau_2$

																															\vspace{1em}


																															Q.24\quad A basis of the vector space $W = \{(x,y,z,w) \in \mathbb{R}^4 : x + y + z = 0, y + z + w = 0, 2x + y - z + w = 0\}$ is
																															\newline
																															\noindent (A) $\{(1,1,1,1), (2,1,1,1)\}$ \hspace{6cm} (B) $\{(1,-1,0,1), (0,1,-1,0)\}$
																															\newline \noindent (C) $\{(1,0,-1,0), (2,1,1,1)\}$\hspace{5.75cm}  (D) $\{(1,0,-1,0), (0,1,-1,0)\}$
																															\vspace{1em}

																															Q.25\quad Consider $\mathbb{R}^3$ with the standard inner product. Let\\
																															\[
																															S = \{(1,1,1), (2,-1,2), (-1,2,1)\}.
																															\]
																															For a subset $W$ of $\mathbb{R}^3$, let $L(W)$ denote the linear span of $W$ in $\mathbb{R}^3$. Then an orthonormal set $T$ with $L(S) = L(T)$ is
																															\newline
																															\noindent (A)  $\left\{ \frac{1}{\sqrt{3}}(1,1,1), \frac{1}{\sqrt{6}}(1,0,-2), \frac{1}{\sqrt{2}}(1,-1,0) \right\}$ \hspace{4cm} (B)$\{(0,0,0), (0,1,0), (0,0,1)\}$
																															\newline \noindent (C) $\left\{ \frac{1}{\sqrt{3}}(1,1,1), \frac{1}{\sqrt{2}}(1,0,-1) \right\}$\hspace{6cm}  (D) $\left\{ \frac{1}{\sqrt{3}}(1,1,1), \frac{1}{\sqrt{2}}(1,-1,0) \right\}$
																															\vspace{10em}
																															\begin{center}
																															    {MA 7/24}
																															    \end{center}

																															    \newpage
																															    Q.26\quad Let $A$ be a $3 \times 3$ matrix. Suppose that the eigenvalues of $A$ are $-1, 0, 1$ with respective eigenvectors $(1, -1, 0)^T$, $(1, 1, -2)^T$ and $(1, 1, 1)^T$. Then $6A$ equals
																															    \newline
																															    \noindent (A) $\begin{bmatrix} -1 & 5 & 2 \\ 5 & -1 & 2 \\ 2 & 2 & -1 \end{bmatrix}$\hspace{6cm} (B) $\begin{bmatrix} 1 & 0 & 0 \\ 0 & -1 & 0 \\ 0 & 0 & 0 \end{bmatrix}$
																															    \newline \noindent (C) $\begin{bmatrix} 1 & 5 & 3 \\ 5 & 1 & 3 \\ 3 & 3 & 3 \end{bmatrix}$\hspace{6.75cm}  (D) $\begin{bmatrix} -3 & 9 & 0 \\ 9 & -3 & 0 \\ 0 & 0 & 6 \end{bmatrix}$
																															    \vspace{1em}
																															    \newline
																															    Q.27\quad Let $T:\mathbb{R}^3 \rightarrow \mathbb{R}^3$ be a linear transformation defined by
																															    \[
																															    T((x, y, z)) = (x + y - z, x + y + z, y - z).
																															    \]
																															    Then the matrix of the linear transformation $T$ with respect to the ordered basis $B = \{(0,1,0), (0,0,1), (1,0,0)\}$ of $\mathbb{R}^3$ is
																															    \vspace{1em}

																															    \newline
																															    \noindent (A) $\begin{bmatrix} 1 & 1 & 0 \\ 0 & -1 & 1 \\ 1 & 1 & 1 \end{bmatrix}$ \hspace{6.75cm} (B)  $\begin{bmatrix} 1 & 0 & 1 \\ 1 & 1 & 1 \\ -1 & 0 & 1 \end{bmatrix}$
																															    \newline \noindent (C) $\begin{bmatrix} 1 & -1 & 0 \\ 1 & -1 & 1 \\ 1 & 1 & 1 \end{bmatrix}$\hspace{6.75cm}  (D) $\begin{bmatrix} 1 & 1 & 1 \\ -1 & 0 & 1 \\ 1 & -1 & 0 \end{bmatrix}$
																															    \vspace{1em}
																															    \newline
																															    Q.28\quad Let $Y(x) = (y_1(x), y_2(x))^T$ and let
																															    \[
																															    A = \begin{bmatrix} -3 & 1 \\ k & -1 \end{bmatrix}.
																															    \]
																															    Further, let $S$ be the set of values of $k$ for which all the solutions of the system of equations $Y'(x) = A Y(x)$ tend to zero as $x \rightarrow \infty$. Then $S$ is given by

																															    \vspace{1em} 
																															    \noindent (A) $\{k : k \leq -1\}$ \hspace{6cm} (B)  $\{k : k \leq 3\}$
																															    \newline \noindent (C) $\{k : k < -1\}$ \hspace{6cm}  (D) $\{k : k < 3\}$
																															    \vspace{1em}


																															    Q.29\quad Let
																															    \[
																															    u(x,y) = f(xe^y) + g(y^2 \cos y),
																															    \]
																															    where $f$ and $g$ are infinitely differentiable functions. Then the partial differential equation of minimum order satisfied by $u$ is
																															    \newline
																															    \noindent (A)  $u_x + x u_{xx} = u_y$ \hspace{6cm} (B)  $u_y + x u_{xx} = x u_y$
																															    \newline \noindent (C) $u_y - x u_{xx} = u_x$ \hspace{6cm}  (D) $u_y - x u_{xx} = x u_y$
																															    \vspace{20em}
																															    \begin{center}
																															        {MA 8/24}
																																\end{center}

																																\newpage
																																Q.30\quad Let $C$ be the boundary of the triangle formed by the points $(1,0,0)$, $(0,1,0)$, $(0,0,1)$.\\
																																Then the value of the line integral
																																\[
																																\oint_C -2y\,dx + (3x - 4y^2)\,dy + (z^2 + 3y)\,dz
																																\]
																																is
																																\newline \noindent (A)$0$ \hspace{2cm} (B) $1$\hspace{2cm}(C) $2$ \hspace{2cm} (D) $4$


																																\vspace{0.5cm}

																																Q.31\quad Let $X$ be a complete metric space and let $E \subset X$. Consider the following statements:
																																\begin{itemize}
																																  \item[$(S_1)$] $E$ is compact,
																																    \item[$(S_2)$] $E$ is closed and bounded,
																																      \item[$(S_3)$] $E$ is closed and totally bounded,
																																        \item[$(S_4)$] Every sequence in $E$ has a subsequence converging in $E$.
																																	\end{itemize}
																																	Which one of the above statements does \textbf{NOT} imply all the other statements?
																																	\vspace{0.5cm}
																																	 \newline \noindent (A)$S_1$ \hspace{2cm} (B) $S_2$\hspace{2cm}(C) $S_3$ \hspace{2cm} (D) $S_4$
																																	 \vspace{0.5cm}

																																	 Q.32\quad Consider the series
																																	 \[
																																	 \sum_{n=1}^{\infty} \frac{1}{n^2} \sin(nx).
																																	 \]
																																	 Then the series
																																	 \begin{itemize}
																																	   \item[(A)] converges uniformly on $\mathbb{R}$
																																	     \item[(B)] converges pointwise but NOT uniformly on $\mathbb{R}$
																																	       \item[(C)] converges in $L^1$ norm to an integrable function on $[0, 2\pi]$ but does NOT converge uniformly on $\mathbb{R}$
																																	         \item[(D)] does NOT converge pointwise
																																		 \end{itemize}

																																		 \vspace{0.5cm}

																																		 Q.33\quad Let $f(z)$ be an analytic function. Then the value of
																																		 \[
																																		 \int_0^{2\pi} f(e^{it}) \cos(t)\,dt
																																		 \]
																																		 equals
																																		 \newline \noindent (A)$0$ \hspace{2cm} (B) $2\pi f(0)$\hspace{2cm}(C) $2\pi f'(0)$ \hspace{2cm} (D) $\pi f'(0)$


																																		 \vspace{0.5cm}

																																		 Q.34\quad Let $G_1$ and $G_2$ be the images of the disc $\{ z \in \mathbb{C} : |z + 1| < 1 \}$ under the transformations
																																		 \[
																																		 w = \frac{(1 - i)z + 2}{(1 + i)z + 2} \quad \text{and} \quad w = \frac{(1 + i)z + 2}{(1 - i)z + 2}
																																		 \]
																																		 respectively. Then
																																		 \begin{itemize}
																																		   \item[(A)] $G_1 = \{w \in \mathbb{C} : \operatorname{Im}(w) < 0\}$ and $G_2 = \{w \in \mathbb{C} : \operatorname{Im}(w) > 0\}$
																																		     \item[(B)] $G_1 = \{w \in \mathbb{C} : \operatorname{Im}(w) > 0\}$ and $G_2 = \{w \in \mathbb{C} : \operatorname{Im}(w) < 0\}$
																																		       \item[(C)] $G_1 = \{w \in \mathbb{C} : |w| > 2\}$ and $G_2 = \{w \in \mathbb{C} : |w| < 2\}$
																																		         \item[(D)] $G_1 = \{w \in \mathbb{C} : |w| < 2\}$ and $G_2 = \{w \in \mathbb{C} : |w| > 2\}$
																																			 \end{itemize}
																																			 \vspace{2em}
																																			 \begin{center}
																																			     {MA 9/24}
																																			     \end{center}

																																			     \newpage
																																			     Q.35 Let $f(z) = 2^z - 2^{-z}$. Then the maximum value of $|f(z)|$ on the unit disc $D = \{ z \in \mathbb{C} : |z| \leq 1 \}$ equals\\
																																			     (A) 1 \hspace{2cm} (B) 2 \hspace{2cm} (C) 3 \hspace{2cm} (D) 4

																																			     \vspace{1em}

																																			     Q.36 Let
																																			     \[
																																			     f(z) = \frac{1}{z^2 - 3z + 2}
																																			     \]
																																			     Then the coefficient of $\frac{1}{z}$ in the Laurent series expansion of $f(z)$ for $|z| > 2$ is\\
																																			     \vspace{0.125cm} \newline
																																			     (A) 0 \hspace{2cm} (B) 1 \hspace{2cm} (C) 3 \hspace{2cm} (D) 5

																																			     \vspace{1em}

																																			     Q.37 Let $f: \mathbb{C} \rightarrow \mathbb{C}$ be an arbitrary analytic function satisfying $f(0) = 0$ and $f(1) = 2$. Then\\
																																			     \newline
																																			     (A) there exists a sequence $\{z_n\}$ such that $|z_n| > n$ and $|f(z_n)| > n$\\
																																			     (B) there exists a sequence $\{z_n\}$ such that $|z_n| > n$ and $|f(z_n)| < n$\\
																																			     (C) there exists a bounded sequence $\{z_n\}$ such that $|f(z_n)| > n$\\
																																			     (D) there exists a sequence $\{z_n\}$ such that $z_n \rightarrow 0$ and $f(z_n) \rightarrow 2$

																																			     \vspace{0.5cm}

																																			     Q.38 Define $f : \mathbb{C} \rightarrow \mathbb{C}$ by
																																			     \[
																																			     f(z) =
																																			     \begin{cases}
																																			     0, & \text{if } \text{Re}(z) = 0 \text{ or } \text{Im}(z) = 0,\\
																																			     \frac{1}{z}, & \text{otherwise}.
																																			     \end{cases}
																																			     \]
																																			     Then the set of points where $f$ is analytic is
																																			     \vspace{1em} \newline

																																			     \noindent (A) $\{z : \text{Re}(z) \neq 0 \text{ and } \text{Im}(z) \neq 0 \}$ \hspace{5.5cm} (B)  $\{z : \text{Re}(z) \neq 0 \}$
																																			     \newline \noindent (C) $\{z : \text{Re}(z) \neq 0 \text{ or } \text{Im}(z) \neq 0 \}$\hspace{6cm}  (D) $\{z : \text{Im}(z) \neq 0 \}$
																																			     \vspace{1em}

																																			     Q.39 Let $U(n)$ be the set of all positive integers less than $n$ and relatively prime to $n$. Then $U(n)$ is a group under multiplication modulo $n$. For $n = 248$, the number of elements in $U(n)$ is\\
																																			     \newline
																																			     (A) 60 \hspace{2cm}(B) 120 \hspace{2cm} (C) 180 \hspace{2cm} (D) 240

																																			     \vspace{1em}

																																			     Q.40 \quad Let $\mathbb{R}[x]$ be the polynomial ring in $x$ with real coefficients and let $I = \langle x^2 + 1 \rangle$ be the ideal generated by the polynomial $x^2 + 1$ in $\mathbb{R}[x]$. Then\\
																																			     \newline
																																			     (A) $I$ is a maximal ideal\\
																																			     (B) $I$ is a prime ideal but NOT a maximal ideal\\
																																			     (C) $I$ is NOT a prime ideal\\
																																			     (D) $\mathbb{R}[x]/I$ has zero divisors
																																			     \vspace{15em}
																																			     \begin{center}
																																			         {MA 10/24}
																																				 \end{center}

																																				 \newpage

																																				 Q.41 \quad  Consider $\mathbb{Z}5$ and $\mathbb{Z}{20}$ as rings modulo 5 and 20, respectively. Then the number of homomorphisms $\varphi: \mathbb{Z}5 \to \mathbb{Z}{20}$ is
																																				 \newline
																																				 (A) 1 \hspace{2cm}(B) 2 \hspace{2cm} (C) 4 \hspace{2cm} (D) 5


																																				 Let $\mathbb{Q}$ be the field of rational numbers and consider $\mathbb{Z}_2$ as a field modulo 2. Let
																																				     \[
																																				         f(x) = x^3 - 9x^2 + 9x + 3.
																																					     \]
																																					         Then $f(x)$ is
																																						     \begin{itemize}
																																						       \item[(A)] irreducible over $\mathbb{Q}$ but reducible over $\mathbb{Z}_2$
																																						         \item[(B)] irreducible over both $\mathbb{Q}$ and $\mathbb{Z}_2$
																																							   \item[(C)] reducible over $\mathbb{Q}$ but irreducible over $\mathbb{Z}_2$
																																							     \item[(D)] educible over both $\mathbb{Q}$ and $\mathbb{Z}_2$
																																							       \end{itemize}


																																							       \vspace{0.5em}

																																							       Q.42 \quad Let $\mathbb{Q}$ be the field of rational numbers and consider $\mathbb{Z}_2$ as a field modulo 2. Let
																																							           \[
																																								       f(x) = x^3 - 9x^2 + 9x + 3.
																																								           \]
																																									       Then $f(x)$ is
																																									           \begin{itemize}
																																										           
																																											           \item[(A)] irreducible over $\mathbb{Q}$ but reducible over $\mathbb{Z}_2$
																																												           \item[(B)] irreducible over both $\mathbb{Q}$ and $\mathbb{Z}_2$
																																													           \item[(C)] reducible over $\mathbb{Q}$ but irreducible over $\mathbb{Z}_2$
																																														           \item[(D)] reducible over both $\mathbb{Q}$ and $\mathbb{Z}_2$
																																															       \end{itemize}
																																															        \vspace{0.5em}

																																																Q.43 \quad Consider $\mathbb{Z}_5$ as a field modulo 5 and let
																																																    \[
																																																        f(x) = x^4 + 4x^3 + 4x^2 + 4x + 1.
																																																	    \]
																																																	        Then the zeros of $f(x)$ over $\mathbb{Z}_5$ are 1 and 3 with respective multiplicity
																																																		    \begin{itemize}
																																																		            
																																																			            \item[(A)] 1 and 4
																																																				            \item[(B)] 2 and 3
																																																					            \item[(C)] 2 and 2
																																																						            \item[(D)] 1 and 2
																																																							        \end{itemize}
																																																								 \vspace{0.5em}
																																																								 Q.44 \quad Consider the Hilbert space
																																																								     \[
																																																								         \ell^2 = \left\{ x = \{x_n\};\ x_n \in \mathbb{R},\ \sum x_n^2 < \infty \right\}.
																																																									     \]
																																																									         Let
																																																										     \[
																																																										         E = \left\{ x = \{x_n\} \mid |x_n| < \frac{1}{n} \text{ for all } n \right\}
																																																											     \]
																																																											         be a subset of $\ell^2$. Then
																																																												     \begin{itemize}
																																																												             
																																																													             \item[(A)] $E^\circ = \left\{ x \mid |x_n| < \frac{1}{n} \text{ for all } n \right\}$
																																																														             \item[(B)] $E^\circ = E$
																																																															             \item[(C)] $E^\circ = \left\{ x \mid |x_n| < \frac{1}{n} \text{ for all but finitely many } n \right\}$
																																																																             \item[(D)] $E^\circ = \emptyset$
																																																																	         \end{itemize}
																																																																		     
																																																																		     Q.45 \quad Let $X$ and $Y$ be normed linear spaces and let $T: X \to Y$ be a linear map. Then $T$ is continuous if
																																																																		         \begin{itemize}
																																																																			         \item[(A)] $Y$ is finite dimensional
																																																																				         \item[(B)] $X$ is finite dimensional
																																																																					         \item[(C)] $T$ is one to one
																																																																						         \item[(D)] $T$ is onto
																																																																							     \end{itemize}
																																																																							         \newpage
																																																																								 Q.46 \quad Let $X$ be a normed linear space and let $E_1, E_2 \subseteq X$. Define
																																																																								     \[
																																																																								         E_1 + E_2 = \{x + y : x \in E_1, y \in E_2\}.
																																																																									     \]
																																																																									         Then $E_1 + E_2$ is:
																																																																										     \begin{itemize}
																																																																										             
																																																																											             \item[(A)] open if $E_1$ or $E_2$ is open
																																																																												             \item[(B)] NOT open unless both $E_1$ and $E_2$ are open
																																																																													             \item[(C)] closed if $E_1$ or $E_2$ is closed
																																																																														             \item[(D)] closed if both $E_1$ and $E_2$ are closed
																																																																															         \end{itemize}
																																																																																 \vspace{1em}
																																																																																 Q.47 \quad For each $a \in \mathbb{R}$, consider the linear programming problem:
																																																																																    \begin{flushleft}
																																																																																    \hspace{3cm} Max. $z = x_1 + 2x_2 + 3x_3 + 4x_4$
																																																																																       \end{flushleft}
																																																																																           \hspace{3cm} subject to
																																																																																	       \begin{align*}
																																																																																	               ax_1 + 2x_2 &\leq 1 \\
																																																																																		               x_1 + 2x_2 + 3x_3 &\leq 2 \\
																																																																																			               x_1, x_2, x_3, x_4 &\geq 0
																																																																																				           \end{align*}
																																																																																					       Let $S = \{a \in \mathbb{R} : \text{the given LP problem has a basic feasible solution}\}$. Then:
																																																																																					           \newline
																																																																																						   \noindent (A)  $S = \emptyset$ \hspace{5.75cm} (B)  $S = \mathbb{R}$
																																																																																						   \newline \noindent (C) $S = (0, \infty)$ \hspace{5cm}  (D) $S = (-\infty, 0)$
																																																																																						   \vspace{1em}
																																																																																						   \newline 
																																																																																						   Q.48 \quad Consider the linear programming problem:
																																																																																						      \begin{flushleft}
																																																																																						      \hspace{3cm} Max. $z = x_1 + 5x_2 + 3x_3$
																																																																																						         \end{flushleft} 
																																																																																							     \hspace{3cm} subject to
																																																																																							         \begin{align*}
																																																																																								         2x_1 - 3x_2 + 5x_3 &\leq 3 \\
																																																																																									         x_1 - x_2 &\leq 5 \\
																																																																																										         x_1, x_2, x_3 &\geq 0
																																																																																											     \end{align*}
																																																																																											         Then the dual of this LP problem:
																																																																																												     \begin{itemize}
																																																																																												         
																																																																																													         \item[(A)] has a feasible solution but does NOT have a basic feasible solution
																																																																																														         \item[(B)] has a basic feasible solution
																																																																																															         \item[(C)] has infinite number of feasible solutions
																																																																																																         \item[(D)] has no feasible solution
																																																																																																	  \end{itemize}
																																																																																																	  \vspace{1em}
																																																																																																	  Q.49 \quad Consider a transportation problem with two warehouses and two markets. The warehouse capacities are $a_1 = 2$ and $a_2 = 4$, and the market demands are $b_1 = 3$ and $b_2 = 3$. Let $x_{ij}$ be the quantity shipped from warehouse $i$ to market $j$, and $c_{ij}$ be the corresponding unit cost. Suppose that $c_{11} = 1$, $c_{21} = 1$, and $c_{22} = 2$. Then $(x_{11}, x_{12}, x_{21}, x_{22}) = (2, 0, 1, 3)$ is optimal for every:
																																																																																																	   \vspace{1em}  \newline
																																																																																																	   \noindent (A)  $c_{12} \in [1, 2]$ \hspace{5.75cm} (B)  $c_{12} \in [0, 3]$
																																																																																																	   \newline \noindent (C)  $c_{12} \in [1, 3]$ \hspace{5cm}  (D) $c_{12} \in [2, 4]$
																																																																																																	   \vspace{8em}
																																																																																																	   \begin{center}
																																																																																																	       {MA 12/24}
																																																																																																	       \end{center}

																																																																																																	       \newpage
																																																																																																	       Q.50 \quad The smallest degree of the polynomial that interpolates the data
																																																																																																	          \begin{center}
																																																																																																		  \begin{tabular}{|c|c|c|c|c|c|c|}
																																																																																																		  \hline
																																																																																																		  $x$     & $-2$ & $-1$ & $0$ & $1$ & $2$ & $3$ \\
																																																																																																		  \hline
																																																																																																		  $f(x)$  & $-58$ & $-21$ & $-12$ & $-13$ & $-6$ & $27$ \\
																																																																																																		  \hline
																																																																																																		  \end{tabular}
																																																																																																		  \end{center}

																																																																																																		      is:
																																																																																																		       \newline
																																																																																																		       (A) 3 \hspace{2cm}(B) 4 \hspace{2cm} (C) 5 \hspace{2cm} (D) 6
																																																																																																		       \vspace{1em}
																																																																																																		       \newline
																																																																																																		       Q.51 \quad Suppose that $x_n$ is sufficiently close to 3. Which of the following iterations $x_{n+1} = g(x_n)$ will converge to the fixed point $x = 3$?
																																																																																																		          \vspace{1em}  \newline
																																																																																																			  \noindent (A)  $x_{n+1} = -16 + 6x_n + \dfrac{3}{x_n}$ \hspace{4cm} (B)  $x_{n+1} = \sqrt{3 + 2x_n}$
																																																																																																			  \newline \noindent (C)  $x_{n+1} = \dfrac{3}{x_n} - \dfrac{x_n}{2}$ \hspace{5cm}  (D) $x_{n+1} = \dfrac{x_n^2 - 3}{2}$
																																																																																																			  \vspace{1em}
																																																																																																			  \newline
																																																																																																			  Q.52 \quad Consider the quadrature formula:
																																																																																																			      \[
																																																																																																			          \int_{x_1}^{x_2} f(x)\, dx \approx \dfrac{1}{2} \left[f(x_1) + f(x_2)\right],
																																																																																																				      \]
																																																																																																				          where $x_1$ and $x_2$ are quadrature points. Then the highest degree of the polynomial for which the above formula is exact equals:
																																																																																																					  \vspace{1em}   \newline
																																																																																																					  (A) 1 \hspace{2cm}(B) 2 \hspace{2cm} (C) 3 \hspace{2cm} (D) 4   
																																																																																																					  \vspace{1em}
																																																																																																					  \newline
																																																																																																					  Q.53 \quad Let $A$, $B$ and $C$ be three events such that:
																																																																																																					      \[
																																																																																																					          P(A) = 0.4,\quad P(B) = 0.5,\quad P(A \cup B) = 0.6,\quad P(C) = 0.6,\quad \text{and } P(A \cap B \cap C^c) = 0.1.
																																																																																																						      \]
																																																																																																						          Then $P(A \cap B \cap C) =$
																																																																																																							      \vspace{1em}  \newline
																																																																																																							      (A) $\dfrac{1}{2}$ \hspace{2cm}(B) $\dfrac{1}{3}$ \hspace{2cm} (C) $\dfrac{1}{4}$ \hspace{2cm} (D) $\dfrac{1}{5}$  
																																																																																																							      \vspace{2em}
																																																																																																							      \newline
																																																																																																							      Q.54 \quad Consider two identical boxes $B_1$ and $B_2$, where the box $B_i$ ($i=1,2$) contains $i+1$ red and $5 - i + 1$ white balls. A fair die is cast. Let the number of dots shown on the top face of the die be $N$. If $N$ is even or 5, then two balls are drawn with replacement from the box $B_1$; otherwise, two balls are drawn with replacement from the box $B_2$. The probability that the two drawn balls are of different colours is:
																																																																																																							       \vspace{1em}   \newline 
																																																																																																							       (A) $\dfrac{7}{25}$ \hspace{2cm}(B) $\dfrac{9}{25}$ \hspace{2cm} (C) $\dfrac{12}{25}$ \hspace{2cm} (D) $\dfrac{16}{25}$  

																																																																																																							       \vspace{20em}
																																																																																																							       \begin{center}
																																																																																																							           {MA 13/24}
																																																																																																								   \end{center}
																																																																																																								   \newpage
																																																																																																								   Q.55 \quad Let $X_1, X_2, \ldots$ be a sequence of independent and identically distributed random variables with
																																																																																																								   \[
																																																																																																								   P(X_i = 1) = P(X_i = -1) = \frac{1}{2}.
																																																																																																								   \]
																																																																																																								   Suppose for the standard normal random variable $Z$, $P(-0.1 < Z \leq 0.1) = 0.08$. If $S_n = \sum_{i=1}^{n} X_i$, then
																																																																																																								   \[
																																																																																																								   \lim P\left(\frac{S_n}{\sqrt{n}} > \frac{n}{10}\right) =
																																																																																																								   \]

																																																																																																								   (A) 0.42 \hspace{2cm} (B) 0.46 \hspace{2cm} (C) 0.50 \hspace{2cm} (D) 0.54

																																																																																																								   \vspace{0.5cm}

																																																																																																								   Q.56 \quad Let $X_1, X_2, \ldots, X_5$ be a random sample of size 5 from a population having standard normal distribution. Let
																																																																																																								   \[
																																																																																																								   \bar{X} = \frac{1}{5} \sum_{i=1}^{5} X_i \quad \text{and} \quad T = \sum_{i=1}^{5} (X_i - \bar{X})^2.
																																																																																																								   \]
																																																																																																								   Then $E(T^2 \bar{X}^2) =$

																																																																																																								   (A) 3 \hspace{2cm} (B) 3.6 \hspace{2cm} (C) 4.8 \hspace{2cm} (D) 5.2

																																																																																																								   \vspace{0.5cm}

																																																																																																								   Q.57 \quad Let $x_1 = 3.5$, $x_2 = 7.5$ and $x_3 = 5.2$ be observed values of a random sample of size three from a population having uniform distribution over the interval $(\theta, \theta + 5)$, where $\theta \in (0, \infty)$ is unknown and is to be estimated. Then which of the following is NOT a maximum likelihood estimate of $\theta$?

																																																																																																								   (A) 2.4 \hspace{2cm} (B) 2.7 \hspace{2cm} (C) 3.0 \hspace{2cm} (D) 3.3

																																																																																																								   \vspace{0.5cm}

																																																																																																								   Q.58 \quad  The value of 
																																																																																																								   \[
																																																																																																								   \int_0^1 \int_y^1 x^2 e^{x^2} \, dx \, dy
																																																																																																								   \]
																																																																																																								   equals

																																																																																																								   (A) $\frac{1}{4}$ \hspace{2cm} (B) $\frac{1}{3}$ \hspace{2cm} (C) $\frac{1}{2}$ \hspace{2cm} (D) 1

																																																																																																								   \vspace{0.5cm}

																																																																																																								   Q.59 \quad 
																																																																																																								   \[
																																																																																																								   \lim_{n \to \infty} \left[ (n+1) \int_0^1 x^n \ln(1+x) \, dx \right] =
																																																																																																								   \]

																																																																																																								   (A) 0 \hspace{2cm} (B) $\ln 2$ \hspace{2cm} (C) $\ln 3$ \hspace{2cm} (D) $\infty$

																																																																																																								   \vspace{0.5cm}

																																																																																																								   Q.60 \quad Consider the function $f \colon \mathbb{R} \to \mathbb{R}$ defined by
																																																																																																								   \[
																																																																																																								   f(x) = 
																																																																																																								   \begin{cases}
																																																																																																								   x^4, & \text{if } x \text{ is rational}, \\
																																																																																																								   2x^4 - 1, & \text{if } x \text{ is irrational}.
																																																																																																								   \end{cases}
																																																																																																								   \]
																																																																																																								   Let $S$ be the set of points where $f$ is continuous. Then

																																																																																																								   (A) $S = \{1\}$ \hspace{2cm} (B) $S = \{-1\}$ \hspace{2cm} (C) $S = \{-1, 1\}$ \hspace{2cm} (D) $S = \emptyset$
																																																																																																								   \vspace{10em}
																																																																																																								   \begin{center}
																																																																																																								       {MA 14/24}
																																																																																																								       \end{center}
																																																																																																								       \newpage

																																																																																																								       Q.61 \quad For a positive real number $p$, let $\{f_n: n=1,2,\dots\}$ be a sequence of functions defined on $[0,1]$ by
																																																																																																								       \[
																																																																																																								       f_n(x) =
																																																																																																								       \begin{cases}
																																																																																																								       n^{p+1} x, & 0 \leq x \leq \frac{1}{n} \\
																																																																																																								       \frac{1}{n^p}, & \frac{1}{n} < x \leq 1.
																																																																																																								       \end{cases}
																																																																																																								       \]
																																																																																																								       Let $f(x) = \lim\limits_{n \to \infty} f_n(x),\ x \in [0,1]$. Then, on $[0,1]$,

																																																																																																								       (A) $f$ is Riemann integrable \hspace{2cm}\\ 
																																																																																																								       (B) the improper integral $\int_0^1 f(x) dx$ converges for $p \geq 1$ \\
																																																																																																								       (C) the improper integral $\int_0^1 f(x) dx$ converges for $p < 1$ \hspace{2cm}\\
																																																																																																								       (D) $f_n$ converges uniformly

																																																																																																								       \vspace{0.5cm}

																																																																																																								       Q.62 \quad Which of the following inequality is NOT true for $x \in \left[ \frac{1}{4}, \frac{3}{4} \right]$

																																																																																																								       (A) $e^{-x} > \sum_{j=0}^{\infty} \frac{(-x)^j}{j!}$ \hspace{2cm}
																																																																																																								       (B) $e^{-x} < \sum_{j=0}^{\infty} \frac{(-x)^j}{j!}$ \\
																																																																																																								       (C) $e^{-x} = \sum_{j=0}^{\infty} \frac{(-x)^j}{j!}$ \hspace{2cm}
																																																																																																								       (D) $e^{-x} > \sum_{j=0}^{10} \frac{(-x)^j}{j!}$

																																																																																																								       \vspace{0.5cm}

																																																																																																								       Q.63 \quad Let $u(x,y)$ be the solution to the Cauchy problem
																																																																																																								       \[
																																																																																																								       x u_x + u_y = 1, \quad u(x,0) = 2 \ln(x),\ x > 1.
																																																																																																								       \]
																																																																																																								       Then $u(e,1) =$

																																																																																																								       (A) $-1$ \hspace{2cm} (B) $0$ \hspace{2cm} (C) $1$ \hspace{2cm} (D) $e$

																																																																																																								       \vspace{0.5cm}

																																																																																																								       Q.64 \quad Suppose
																																																																																																								       \[
																																																																																																								       y(x) = \lambda \int_0^{2\pi} y(t) \sin(x + t)\, dt,\ x \in [0,2\pi]
																																																																																																								       \]
																																																																																																								       has eigenvalues $\lambda = \frac{1}{\pi}$ and $\lambda = -\frac{1}{\pi}$ with corresponding eigenfunctions \\
																																																																																																								       $y_1(x) = \sin(x) + \cos(x)$ and $y_2(x) = \sin(x) - \cos(x)$, respectively. Then the integral equation
																																																																																																								       \[
																																																																																																								       y(x) = f(x) + \frac{1}{\pi} \int_0^{2\pi} y(t) \sin(x + t)\, dt,\ x \in [0,2\pi]
																																																																																																								       \]
																																																																																																								       has a solution when $f(x) =$
																																																																																																								       \vspace{1em} \newline
																																																																																																								       (A) $1$ \hspace{2cm} (B) $\cos(x)$ \hspace{2cm} (C) $\sin(x)$ \hspace{2cm} (D) $1 + \sin(x) + \cos(x)$
																																																																																																								       \vspace{20em}
																																																																																																								       \begin{center}
																																																																																																								           {MA 15/24}
																																																																																																									   \end{center}
																																																																																																									   \newpage

																																																																																																									   Q.65 \quad Consider the Neumann problem
																																																																																																									   \[
																																																																																																									   u_{xx} + u_{yy} = 0,\quad 0 < x < \pi,\ -1 < y < 1,
																																																																																																									   \]
																																																																																																									   \[
																																																																																																									   u_y(0, y) = u_y(\pi, y) = 0,
																																																																																																									   \]
																																																																																																									   \[
																																																																																																									   u_y(x, -1) = 0,\quad u_y(x, 1) = \alpha + \beta \sin(x).
																																																																																																									   \]
																																																																																																									   The problem admits solution for
																																																																																																									   \newline
																																																																																																									   (A) $\alpha = 0,\ \beta = 1$ \hspace{2cm} (B) $\alpha = -1,\ \beta = \dfrac{\pi}{2}$ \hspace{2cm} \\ (C) $\alpha = 1,\ \beta = \dfrac{\pi}{2}$ \hspace{2cm} (D) $\alpha = 1,\ \beta = -\pi$
																																																																																																									   \vspace{1em} \newline
																																																																																																									   Q.66 \quad The functional
																																																																																																									   \[
																																																																																																									   \int_0^1 (1+x)(y')^2 \, dx,\quad y(0) = 0,\ y(1) = 1,
																																																																																																									   \]
																																																																																																									   possesses
																																																																																																									   \begin{itemize}
																																																																																																									       \item[(A)] strong maxima
																																																																																																									           \item[(B)] strong minima
																																																																																																										       \item[(C)] weak maxima but NOT a strong maxima
																																																																																																										           \item[(D)] weak minima but NOT a strong minima
																																																																																																											   \end{itemize}

																																																																																																											   \bigskip

																																																																																																											   Q.67 \quad The value of $\alpha$ for which the integral equation
																																																																																																											   \[
																																																																																																											   u(x) = \alpha \int_0^1 e^{xt} u(t)\,dt,
																																																																																																											   \]
																																																																																																											   has a non-trivial solution is
																																																																																																											   \newline
																																																																																																											   (A) $-2$ \hspace{2cm} (B) $-1$ \hspace{2cm} (C) $1$ \hspace{2cm} (D) $2$
																																																																																																											   \vspace{1em} \newline


																																																																																																											   Q.68 \quad Let $P_n(x)$ be the Legendre polynomial of degree $n$ and let
																																																																																																											   \[
																																																																																																											   P_{n+1}(0) = -\frac{m}{m+1} P_{n-1}(0), \quad m = 1, 2, \ldots
																																																																																																											   \]
																																																																																																											   If $P_2(0) = -\dfrac{5}{16}$ then $\int_{-1}^1 \left[P_2^2(x)\right] dx =$
																																																																																																											   \newline
																																																																																																											   (A) $\dfrac{2}{13}$ \hspace{2cm} (B) $\dfrac{2}{9}$ \hspace{2cm} (C) $\dfrac{5}{16}$ \hspace{2cm} (D) $\dfrac{2}{5}$
																																																																																																											   \vspace{1em} \newline




																																																																																																											   Q.69 \quad For which of the following pair of functions $y_1(x)$ and $y_2(x)$, continuous functions $p(x)$ and $q(x)$ can be determined on $[-1, 1]$ such that $y_1(x)$ and $y_2(x)$ give two linearly independent solutions of
																																																																																																											   \[
																																																																																																											   y'' + p(x)y' + q(x)y = 0,\quad x \in [-1, 1].
																																																																																																											   \]
																																																																																																											   \newline \vspace{1em} 
																																																																																																											   (A) $y_1(x) = x \sin(x),\ y_2(x) = \cos(x)$ \hspace{2cm}(B) $y_1(x) = x e^x,\ y_2(x) = \sin(x)$ \hspace{2cm} \\  (C) $y_1(x) = e^{-x},\ y_2(x) = e^{-1}$ \hspace{3cm} (D) $y_1(x) = x^2,\ y_2(x) = \cos(x)$
																																																																																																											   \vspace{5em}
																																																																																																											   \begin{center}
																																																																																																											       {MA 16/24}
																																																																																																											       \end{center}
																																																																																																											       \newpage

																																																																																																											       Q.70 \quad Let $J_0(s)$ and $J_1(s)$ be the Bessel functions of the first kind of orders zero and one, respectively. If
																																																																																																											       \[
																																																																																																											       \mathcal{L}(J_0)(s) = \frac{1}{\sqrt{s^2 + 1}},
																																																																																																											       \]
																																																																																																											       then $\mathcal{L}(J_1)(s) =$
																																																																																																											       \begin{itemize}
																																																																																																											           \item[(A)] $\dfrac{s}{\sqrt{s^2 + 1}}$ \hspace{6cm} (B) $\dfrac{1}{\sqrt{s^2 + 1}}$
																																																																																																												       \item[(C)] $1 - \dfrac{1}{\sqrt{s^2 + 1}}$ \hspace{5.5cm} (D) $\dfrac{1}{\sqrt{s^2 + 1}} - 1$
																																																																																																												           
																																																																																																													   \end{itemize}

																																																																																																													   \bigskip
																																																																																																													   \begin{center}
																																																																																																													       \textbf{Common Data Questions}
																																																																																																													       \end{center}
																																																																																																													       \bigskip
																																																																																																													       \textbf{Common Data for Questions 71, 72, 73:}

																																																																																																													       Let $P[0,1] = \{p : p \text{ is a polynomial function on } [0,1]\}$. For $p \in P[0,1]$, define
																																																																																																													       \[
																																																																																																													       \|p\| = \sup \{|p(x)| : 0 \leq x \leq 1\}.
																																																																																																													       \]
																																																																																																													       Consider the map $T : P[0,1] \rightarrow P[0,1]$ defined by
																																																																																																													       \[
																																																																																																													       (Tp)(x) = \frac{d}{dx} \left(p(x)\right).
																																																																																																													       \]
																																																																																																													       Then $P[0,1]$ is a normed linear space and $T$ is a linear map. The map $T$ is said to be closed if the set $G = \{(p, Tp) : p \in P[0,1]\}$ is a closed subset of $P[0,1] \times P[0,1]$.

																																																																																																													       \bigskip

																																																																																																													       Q.71 \quad The linear map $T$ is
																																																																																																													       \begin{itemize}
																																																																																																													           \item[(A)] one to one and onto \hspace{6cm} (B) one to one but NOT onto
																																																																																																														       
																																																																																																														           \item[(C)] onto but NOT one to one \hspace{5cm} (D) neither one to one nor onto
																																																																																																															       
																																																																																																															       \end{itemize}

																																																																																																															       \bigskip

																																																																																																															       Q.72 \quad The normed linear space $P[0,1]$ is
																																																																																																															       \begin{itemize}
																																																																																																															           \item[(A)] a finite dimensional normed linear space which is NOT a Banach space
																																																																																																																       \item[(B)] a finite dimensional Banach space
																																																																																																																           \item[(C)] an infinite dimensional normed linear space which is NOT a Banach space
																																																																																																																	       \item[(D)] an infinite dimensional Banach space
																																																																																																																	       \end{itemize}

																																																																																																																	       \bigskip

																																																																																																																	       Q.73 \quad The map $T$ is
																																																																																																																	       \begin{itemize}
																																																																																																																	           \item[(A)] closed and continuous \hspace{5cm} (B) neither continuous nor closed
																																																																																																																		       
																																																																																																																		           \item[(C)] continuous but NOT closed \hspace{4cm} (D) closed but NOT continuous
																																																																																																																			       
																																																																																																																			       \end{itemize}

																																																																																																																			       \vspace{1em}

																																																																																																																			       \textbf{Common Data for Questions 74, 75:}

																																																																																																																			       Let $X$ and $Y$ be jointly distributed random variables such that the conditional distribution of $Y$, given $X = x$, is uniform on the interval $(x - 1, x + 1)$. Suppose $\mathbb{E}(X) = 1$ and $\text{Var}(X) = \dfrac{5}{3}$.

																																																																																																																			       \vspace{1em}

																																																																																																																			       Q.74 \quad The mean of the random variable $Y$ is
																																																																																																																			       \newline
																																																																																																																			       (A) $\dfrac{1}{2}$ \hspace{2cm} (B) $1$ \hspace{2cm} (C) $\dfrac{3}{2}$ \hspace{2cm} (D) $2$
																																																																																																																			       \vspace{1em} \newline
																																																																																																																			       \vspace{5em}
																																																																																																																			       \begin{center}
																																																																																																																			           {MA 17/24}
																																																																																																																				   \end{center}
																																																																																																																				   \newpage


																																																																																																																				   Q.75 \quad The variance of the random variable $Y$ is
																																																																																																																				   \vspace{1em} \newline 
																																																																																																																				   (A) $\dfrac{1}{2}$ \hspace{2cm} (B) $\dfrac{2}{3}$ \hspace{2cm} (C) $1$ \hspace{2cm} (D) $2$
																																																																																																																				   \vspace{1em} \newline
																																																																																																																				   \vspace{1em}


																																																																																																																				   \begin{center}
																																																																																																																				    \textbf{Linked Answer Questions: Q.76 to Q.85 carry two marks each.}   
																																																																																																																				    \end{center}
																																																																																																																				    \vspace{1em}

																																																																																																																				    \textbf{Statement for Linked Answer Questions 76 \& 77:}  
																																																																																																																				    \vspace{1em} \newline
																																																																																																																				    Suppose the equation  
																																																																																																																				    \[
																																																																																																																				    x^2 y'' - x y' + (1 + x^2) y = 0
																																																																																																																				    \]
																																																																																																																				    has a solution of the form  
																																																																																																																				    \[
																																																																																																																				    y = x^r \sum_{n=0}^{\infty} c_n x^n,\quad c_0 \neq 0.
																																																																																																																				    \]

																																																																																																																				    Q.76 \quad The indicial equation for $r$ is
																																																																																																																				    \vspace{1em} \newline 
																																																																																																																				    (A) $r^2 - 1 = 0$ \hspace{2cm} (B) $(r - 1)^2 = 0$ \hspace{2cm} \\ (C) $(r + 1)^2 = 0$ \hspace{2cm} (D) $r^2 + 1 = 0$
																																																																																																																				    \vspace{1em} \newline


																																																																																																																				    Q.77 \quad For $n \geq 2$, the coefficients $c_n$ will satisfy the relation
																																																																																																																				    \vspace{1em} \newline 
																																																																																																																				    (A) $n^2 c_n - c_{n-2} = 0$ \hspace{2cm} (B) $n^2 c_n + c_{n-2} = 0$ \hspace{2cm} \\ (C) $c_n - n^2 c_{n-2} = 0$ \hspace{2cm} (D) $c_n + n^2 c_{n-2} = 0$
																																																																																																																				    \vspace{1em} \newline

																																																																																																																				    \textbf{Statement for Linked Answer Questions 78 \& 79:} 
																																																																																																																				    \newline \vspace{1em}
																																																																																																																				    A particle of mass $m$ slides down without friction along a curve $z = 1 + \dfrac{x^2}{2}$ in the $xz$-plane under the action of constant gravity. Suppose the $z$-axis points vertically upwards. Let $\dot{x}$ and $\ddot{x}$ denote $\dfrac{dx}{dt}$ and $\dfrac{d^2x}{dt^2}$ respectively.

																																																																																																																				    Q.78 \quad The Lagrangian of the motion is
																																																																																																																				    \begin{itemize}
																																																																																																																				        \item[(A)] $\dfrac{1}{2} m \dot{x}^2 (1 + x^2) - mg \left(1 + \dfrac{x^2}{2} \right)$ \hspace{4cm} (B) $\dfrac{1}{2} m \dot{x}^2 (1 + x^2) + mg \left(1 + \dfrac{x^2}{2} \right)$
																																																																																																																					   
																																																																																																																					       \item[(C)] $\dfrac{1}{2} m x^2 \dot{x}^2 - mg \left(1 + \dfrac{x^2}{2} \right)$ \hspace{5cm}(D) $\dfrac{1}{2} m \dot{x}^2 (1 - x^2) - mg \left(1 + \dfrac{x^2}{2} \right)$
																																																																																																																					           
																																																																																																																						   \end{itemize}

																																																																																																																						   Q.79 \quad The Lagrangian equation of motion is
																																																																																																																						   \begin{itemize}
																																																																																																																						       \item[(A)] $\ddot{x}(1 + x^2) = -x(g + \dot{x}^2)$ \hspace{3cm} (B) $\ddot{x}(1 + x^2) = x(g - \dot{x}^2)$
																																																																																																																						          
																																																																																																																							      \item[(C)] $\ddot{x} = -g x$ \hspace{5cm} (D) $\ddot{x}(1 - x^2) = -x(g - \dot{x}^2)$
																																																																																																																							          
																																																																																																																								  \end{itemize}
																																																																																																																								  \vspace{10em}
																																																																																																																								  \begin{center}
																																																																																																																								      {MA 18/24}
																																																																																																																								      \end{center}
																																																																																																																								      \newpage



																																																																																																																								      \textbf{Statement for Linked Answer Questions 80 \& 81:}

																																																																																																																								      Let $u(x,t)$ be the solution of the one dimensional wave equation
																																																																																																																								      \[
																																																																																																																								      u_{tt} = 4u_{xx}, \quad -\infty < x < \infty,\ t > 0,
																																																																																																																								      \]
																																																																																																																								      \[
																																																																																																																								      u(x,0) =
																																																																																																																								      \begin{cases}
																																																																																																																								      16 - x^2, & |x| \leq 4, \\
																																																																																																																								      0, & \text{otherwise},
																																																																																																																								      \end{cases}
																																																																																																																								      \quad \text{and} \quad
																																																																																																																								      u_t(x,0) =
																																																																																																																								      \begin{cases}
																																																																																																																								      1, & |x| \leq 2, \\
																																																																																																																								      0, & \text{otherwise}.
																																																																																																																								      \end{cases}
																																																																																																																								      \]

																																																																																																																								      Q.80 \quad For $1 < t < 3$, $u(2,t) =$  
																																																																																																																								      \begin{itemize}
																																																																																																																								          \item[(A)] $\left[16 - (2 - 2t)^2\right]^+ + \dfrac{1}{2}\left[1 - \min\{1, t - 1\}\right]$
																																																																																																																									      \item[(B)] $\left[32 - (2 - 2t)^2 - (2 + 2t)^2\right]^+ + t$
																																																																																																																									          \item[(C)] $\left[32 - (2 - 2t)^2 - (2 + 2t)^2\right]^+ + 1$
																																																																																																																										      \item[(D)] $\left[16 - (2 - 2t)^2\right]^+ + \dfrac{1}{2}\left[1 - \max\{1, t - 1\}\right]$
																																																																																																																										      \end{itemize}

																																																																																																																										      \bigskip

																																																																																																																										      Q.81 \quad The value of $u(2,2)$  
																																																																																																																										      \begin{itemize}
																																																																																																																										          \item[(A)] equals $15$
																																																																																																																											      \item[(B)] equals $16$
																																																																																																																											          \item[(C)] equals $0$
																																																																																																																												      \item[(D)] does NOT exist
																																																																																																																												      \end{itemize}

																																																																																																																												      \bigskip

																																																																																																																												      \textbf{Statement for Linked Answer Questions 82 \& 83:}

																																																																																																																												      Suppose $E = \{(x, y): xy \ne 0\}$. Let $f: \mathbb{R}^2 \rightarrow \mathbb{R}$ be defined by  
																																																																																																																												      \[
																																																																																																																												      f(x, y) =
																																																																																																																												      \begin{cases}
																																																																																																																												      0, & \text{if } xy = 0, \\
																																																																																																																												      y \sin\left(\frac{1}{x}\right) + x \sin\left(\frac{1}{y}\right), & \text{otherwise}.
																																																																																																																												      \end{cases}
																																																																																																																												      \]

																																																																																																																												      Let $S_1$ be the set of points in $\mathbb{R}^2$ where $f_x$ exists and $S_2$ be the set of points in $\mathbb{R}^2$ where $f_y$ exists. Also, let $E_1$ be the set of points where $f_x$ is continuous and $E_2$ be the set of points where $f_y$ is continuous.

																																																																																																																												      Q.82 \quad $S_1$ and $S_2$ are given by  
																																																																																																																												      \begin{itemize}
																																																																																																																												          \item[(A)] $S_1 = E \cup \{(x, y): y = 0\},\quad S_2 = E \cup \{(x, y): x = 0\}$
																																																																																																																													      \item[(B)] $S_1 = E \cup \{(x, y): x = 0\},\quad S_2 = E \cup \{(x, y): y = 0\}$
																																																																																																																													          \item[(C)] $S_1 = S_2 = \mathbb{R}^2$
																																																																																																																														      \item[(D)] $S_1 = S_2 = E \cup \{(0, 0)\}$
																																																																																																																														      \end{itemize}
																																																																																																																														      \vspace{15em}
																																																																																																																														      \begin{center}
																																																																																																																														          {MA 19/24}
																																																																																																																															  \end{center}
																																																																																																																															  \newpage


																																																																																																																															  Q.83 \quad $E_1$ and $E_2$ are given by  
																																																																																																																															  \begin{itemize}
																																																																																																																															      \item[(A)] $E_1 = S_1,\quad E_2 = S_1 \cap S_2$
																																																																																																																															          \item[(B)] $E_1 = S_1 \cap S_2 \setminus \{(0,0)\},\quad E_2 = S_1$
																																																																																																																																      \item[(C)] $E_1 = S_2,\quad E_2 = S_1$
																																																																																																																																          \item[(D)] $E_1 = S_2,\quad E_2 = S_2$
																																																																																																																																	  \end{itemize}

																																																																																																																																	  \bigskip

																																																																																																																																	  \textbf{Statement for Linked Answer Questions 84 \& 85:}

																																																																																																																																	  Let
																																																																																																																																	  \[
																																																																																																																																	  A = \begin{bmatrix}
																																																																																																																																	  3 & 0 & 0 \\
																																																																																																																																	  0 & 6 & 2 \\
																																																																																																																																	  0 & 2 & 6
																																																																																																																																	  \end{bmatrix}
																																																																																																																																	  \]
																																																																																																																																	  and let $\lambda_1 \geq \lambda_2 \geq \lambda_3$ be the eigenvalues of $A$.

																																																																																																																																	  Q.84 \quad The triple $(\lambda_1, \lambda_2, \lambda_3)$ equals  
																																																																																																																																	  \newline
																																																																																																																																	  (A) $(9, 4, 2)$ \hspace{2cm} (B) $(8, 4, 3)$ \hspace{2cm} \\ (C) $(9, 3, 3)$ \hspace{2cm} (D) $(7, 5, 3)$
																																																																																																																																	  \vspace{1em} \newline

																																																																																																																																	  \bigskip

																																																																																																																																	  Q.85 \quad The matrix $P$ such that
																																																																																																																																	  \[
																																																																																																																																	  P^{-1} A P = \begin{bmatrix}
																																																																																																																																	  \lambda_1 & 0 & 0 \\
																																																																																																																																	  0 & \lambda_2 & 0 \\
																																																																																																																																	  0 & 0 & \lambda_3
																																																																																																																																	  \end{bmatrix}
																																																																																																																																	  \]
																																																																																																																																	  is
																																																																																																																																	  \newline \vspace{1em}
																																																																																																																																	  (A) $\begin{bmatrix}
																																																																																																																																	      \dfrac{1}{\sqrt{3}} & 0 & \dfrac{-2}{\sqrt{6}} \\
																																																																																																																																	          \dfrac{1}{\sqrt{3}} & \dfrac{1}{\sqrt{2}} & \dfrac{1}{\sqrt{6}} \\
																																																																																																																																		      \dfrac{1}{\sqrt{3}} & \dfrac{-1}{\sqrt{2}} & \dfrac{1}{\sqrt{6}}
																																																																																																																																		          \end{bmatrix}$ \hspace{2cm} (B) $\begin{bmatrix}
																																																																																																																																			      \dfrac{1}{\sqrt{3}} & \dfrac{-2}{\sqrt{6}} & 0 \\
																																																																																																																																			          \dfrac{1}{\sqrt{3}} & \dfrac{1}{\sqrt{6}} & \dfrac{1}{\sqrt{2}} \\
																																																																																																																																				      \dfrac{1}{\sqrt{3}} & \dfrac{1}{\sqrt{6}} & \dfrac{-1}{\sqrt{2}}
																																																																																																																																				          \end{bmatrix}$
																																																																																																																																					       \hspace{2cm} \\ (C) $\begin{bmatrix}
																																																																																																																																					           0 & 0 & 1 \\
																																																																																																																																						       \dfrac{1}{\sqrt{2}} & \dfrac{1}{\sqrt{2}} & 0 \\
																																																																																																																																						           \dfrac{1}{\sqrt{2}} & \dfrac{-1}{\sqrt{2}} & 0
																																																																																																																																							       \end{bmatrix}$ \hspace{2cm} (D) $\begin{bmatrix}
																																																																																																																																							           0 & 1 & 0 \\
																																																																																																																																								       \dfrac{1}{\sqrt{2}} & 0 & \dfrac{1}{\sqrt{2}} \\
																																																																																																																																								           \dfrac{1}{\sqrt{2}} & 0 & \dfrac{-1}{\sqrt{2}}
																																																																																																																																									       \end{bmatrix}$
																																																																																																																																									       \vspace{1em} \newline


																																																																																																																																									       \vspace{5em}
																																																																																																																																									       \begin{center}
																																																																																																																																									           \textbf{END OF THE QUESTION PAPER}
																																																																																																																																										   \end{center}
																																																																																																																																										   \vspace{15em}
																																																																																																																																										   \begin{center}
																																																																																																																																										       {MA 20/24}
																																																																																																																																										       \end{center}

																																																																																																																																										       \end{document}

