\let\negmedspace\undefined
\let\negthickspace\undefined
\documentclass[journal]{IEEEtran}
\usepackage[a5paper, margin=10mm, onecolumn]{geometry}
%\usepackage{lmodern} % Ensure lmodern is loaded for pdflatex
\usepackage{tfrupee} % Include tfrupee package

\setlength{\headheight}{1cm} % Set the height of the header box
\setlength{\headsep}{0mm}     % Set the distance between the header box and the top of the text
\usepackage{gvv-book}
\usepackage{gvv}
\usepackage{cite}
\usepackage{amsmath,amssymb,amsfonts,amsthm}
\usepackage{algorithmic}
\usepackage{graphicx}
\usepackage{textcomp}
\usepackage{xcolor}
\usepackage{txfonts}
\usepackage{listings}
\usepackage{enumitem}
\usepackage{mathtools}
\usepackage{gensymb}
\usepackage{comment}
\usepackage[breaklinks=true]{hyperref}
\usepackage{tkz-euclide} 
\usepackage{listings}
% \usepackage{gvv}                                        
\def\inputGnumericTable{}                                 
\usepackage[latin1]{inputenc}                                
\usepackage{color}                                            
\usepackage{array}                                            
\usepackage{longtable}                                       
\usepackage{calc}                                             
\usepackage{multirow}                                         
\usepackage{hhline}                                           
\usepackage{ifthen}                                           
\usepackage{lscape}



\usepackage{amsmath,amssymb}
\usepackage{booktabs}
\usepackage{tikz}
\usetikzlibrary{arrows.meta,angles,quotes}

\begin{document}

\bibliographystyle{IEEEtran}
\vspace{3cm}

\title{1.5.2}
\author{EE25BTECH11014 - Bhoomika Lokesh}
% \maketitle
% \newpage
% \bigskip
{\let\newpage\relax\maketitle}

\renewcommand{\thefigure}{\theenumi}
\renewcommand{\thetable}{\theenumi}
\setlength{\intextsep}{10pt} % Space between text and floats


\numberwithin{equation}{enumi}
\numberwithin{figure}{enumi}
\renewcommand{\thetable}{\theenumi}

\textbf{Question:}  Find the ratio in which the \textit{Y} axis divides the line segment joining the points $\brak{6,-4}$ and $\brak{-2,-7}$. Also find the point of intersection.\\

\textbf{Solution:}
Given the points,
\begin{align}
    \vec{A}=\begin{myvec}{6\\-4}\end{myvec}
    \vec{B}=\begin{myvec}{-2\\-7}\end{myvec}
\end{align}

Let the vector $\vec{P}$ be 
\begin{align}
    \vec{P}=\begin{myvec}{0\\y}\end{myvec} \;, 
\end{align}
WKT points \textbf{A},\textbf{P},\textbf{B} are collinear.\\
The points to be collinear,
\begin{align}
	\label{eq:line-rank-2}
	\rank{\myvec{\vec{P}-\vec{A}& \vec{B}-\vec{A}}} = 1\\
        \vec{P}-\vec{A} = \myvec{-6\\y+4}\\
         \vec{B}-\vec{A} = \myvec{-8\\-3}\\
        \myvec{\vec{P}-\vec{A}& \vec{B}-\vec{A}} = \myvec{-6 & -8\\ y+4&-3 }
      \end{align}
 \begin{center}
 Conversion to Row Echelon form,
 $R_2 \rightarrow R_2 + \frac{y+4}{6}R_1 :$\\
\end{center}
\begin{align}
\myvec{ -6 & -8\\0 & -3+ \frac{y+4}{6}(-8)}\implies
\myvec {-6 & -8\\0 & \frac{-4y-25}{3}}\\
\end{align}
 \begin{align}
  \frac{-4y-25}{3}=0\implies y=-\frac{25}{4}
 \end{align}
 \begin{center}
      $\therefore \vec{P}=\begin{myvec}{0\\-\frac{25}{4}}\end{myvec}$
 \end{center}
Vector \textbf{P} divides the line joining vectors \textbf{A} and \textbf{B} in the ratio k:1
\begin{align}
    by\ using \ section\ formula,
    \textbf{P}=\frac{k\textbf{B}+\textbf{A}}{k+1}
\end{align}
 \begin{align}
			k\brak{\vec{P}-\vec{B}}&= \vec{A}-\vec{P}
\end{align}
\begin{align}
    \implies k &=
		\frac{\brak{\vec{A}-\vec{P}}^{\top}\brak{\vec{P}-\vec{B}}}{\norm{\vec{P}-\vec{B}}^2}
			\label{eq:section_formula-k}
\end{align}
\begin{align}
\brak{\vec{A}-\vec{P}}^{\top}\brak{\vec{P}-\vec{B}} = \myvec{6 & \frac{9}{4}}\myvec{2\\ \frac{3}{4}
}=\frac{219}{16}\\
{\norm{\vec{P}-\vec{B}}^2} = \brak{\sqrt{2^2 + {\brak{\frac{3}{4}}}^2}}^2 = \frac{73}{16}\\
k=\frac{\frac{219}{16}}{\frac{73}{16}}
\end{align}

\begin{align}
\implies k &= 3
\end{align}
 Therefore the ratio in which point \textbf{P} divides the line segment joining \textbf{A} and \textbf{B} is 3:1

See Fig.0.1,
\begin{figure}[H]
\begin{center}
\includegraphics[width=0.7\columnwidth]{Figs/graph3d.png}
\end{center}
\caption{}
\label{fig:fig.py}
\end{figure}


\end{document}
