\documentclass{beamer}
\usepackage[utf8]{inputenc}

\usetheme{Madrid}
\usecolortheme{default}
\usepackage{amsmath,amssymb,amsfonts,amsthm}
\usepackage{txfonts}
\usepackage{tkz-euclide}
\usepackage{listings}
\usepackage{adjustbox}
\usepackage{array}
\usepackage{tabularx}
\usepackage{gvv}
\usepackage{lmodern}
\usepackage{circuitikz}
\usepackage{tikz}
\usepackage{graphicx}

\setbeamertemplate{page number in head/foot}[totalframenumber]

\usepackage{tcolorbox}
\tcbuselibrary{minted,breakable,xparse,skins}



\definecolor{bg}{gray}{0.95}
\DeclareTCBListing{mintedbox}{O{}m!O{}}{%
  breakable=true,
  listing engine=minted,
  listing only,
  minted language=#2,
  minted style=default,
  minted options={%
    linenos,
    gobble=0,
    breaklines=true,
    breakafter=,,
    fontsize=\small,
    numbersep=8pt,
    #1},
  boxsep=0pt,
  left skip=0pt,
  right skip=0pt,
  left=25pt,
  right=0pt,
  top=3pt,
  bottom=3pt,
  arc=5pt,
  leftrule=0pt,
  rightrule=0pt,
  bottomrule=2pt,
  toprule=2pt,
  colback=bg,
  colframe=orange!70,
  enhanced,
  overlay={%
    \begin{tcbclipinterior}
    \fill[orange!20!white] (frame.south west) rectangle ([xshift=20pt]frame.north west);
    \end{tcbclipinterior}},
  #3,
}
\lstset{
    language=C,
    basicstyle=\ttfamily\small,
    keywordstyle=\color{blue},
    stringstyle=\color{orange},
    commentstyle=\color{green!60!black},
    numbers=left,
    numberstyle=\tiny\color{gray},
    breaklines=true,
    showstringspaces=false,
}


\title 
{1.2.29}
\date{August 27,2025}


\author 
{Abhiram Reddy-AI25BTECH11021}



\begin{document}


\frame{\titlepage}
\begin{frame}{Question}
In a harbour, wind is blowing at the speed of $72\ \mathrm{km/h}$ and the flag on the mast of a boat anchored in the harbour flutters along the N--E direction. If the boat starts moving at a speed of $51\ \mathrm{km/h}$ to the north, what is the direction of the flag on the mast of the boat?
\end{frame}



\begin{frame}{Represent given velocities as vectors}

The wind velocity (ground frame) is along the NE direction with speed $72 \,\text{km/h}$:
\[
W = \begin{bmatrix} 72 \cos 45^\circ \\ 72 \sin 45^\circ \end{bmatrix}
= \begin{bmatrix} 50.91 \\ 50.91 \end{bmatrix} \ \text{km/h}.
\]

The boat velocity (ground frame) is northward with speed $51 \,\text{km/h}$:
\[
V = \begin{bmatrix} 0 \\ 51 \end{bmatrix} \ \text{km/h}.
\]\\

\end{frame}

\begin{frame}{Relative wind (wind as seen from the boat)}
\[
R = W - V
= \begin{bmatrix} 50.91 \\ 50.91 \end{bmatrix} -
\begin{bmatrix} 0 \\ 51 \end{bmatrix}
= \begin{bmatrix} 50.91 \\ -0.09 \end{bmatrix}.
\]
\end{frame}
\begin{frame}{Direction of the relative wind}
\[
\theta = \tan^{-1}\left(\frac{-0.09}{50.91}\right) \approx -0.1^\circ
\]

Thus, the relative wind is almost exactly eastward, slightly south of east.

\[
\boxed{\text{The flag on the mast points nearly East, slightly tilted South.}}
\]
\end{frame}


\begin{frame}[fragile]
    \frametitle{C Code }

    \begin{lstlisting}

#include <stdio.h>
#include <math.h>

int main() {
    // Wind (NE at 72 km/h)
    double W_x = 72 * cos(M_PI/4);
    double W_y = 72 * sin(M_PI/4);

    // Boat (North at 51 km/h)
    double V_x = 0;
    double V_y = 51;

    // Relative wind = Wind - Boat
    double R_x = W_x - V_x;
    double R_y = W_y - V_y;

    printf("Relative wind: (%.2f, %.2f)\n", R_x, R_y);
    return 0;
}
    \end{lstlisting}
\end{frame}

\begin{frame}[fragile]
    \frametitle{Python Code}
    \begin{lstlisting}
import numpy as np
import matplotlib.pyplot as plt

# Wind velocity (ground frame), NE direction at 72 km/h
W = np.array([72/np.sqrt(2), 72/np.sqrt(2)])  # [East, North]

# Boat velocity (ground frame), north at 51 km/h
V = np.array([0, 51])  # [East, North]

# Relative wind (wind seen from boat)
R = W - V


    \end{lstlisting}
\end{frame}

\begin{frame}[fragile]
    \frametitle{Python Code}
    \begin{lstlisting}

# Calculate angle of relative wind
angle_deg = np.degrees(np.arctan2(R[1], R[0]))

print("Wind vector W =", W)
print("Boat vector V =", V)
print("Relative wind R =", R)
print("Angle of relative wind =", angle_deg, "degrees")

    \end{lstlisting}
\end{frame}

\begin{frame}[fragile]
    \frametitle{Python Code}
    \begin{lstlisting}

# Create plot
plt.figure(figsize=(6,6))
plt.axhline(0, color='gray', linewidth=0.5)
plt.axvline(0, color='gray', linewidth=0.5)

    \end{lstlisting}
\end{frame}

\begin{frame}[fragile]
    \frametitle{Python Code}
    \begin{lstlisting}

# Plot vectors
plt.quiver(0,0, W[0], W[1], angles='xy', scale_units='xy', scale=1, color='blue', label="Wind (ground)")
plt.quiver(0,0, V[0], V[1], angles='xy', scale_units='xy', scale=1, color='green', label="Boat (ground)")
plt.quiver(0,0, R[0], R[1], angles='xy', scale_units='xy', scale=1, color='red', label="Relative Wind")

    \end{lstlisting}
\end{frame}
\begin{frame}[fragile]
    \frametitle{Python Code}
    \begin{lstlisting}

# Labels
plt.text(W[0], W[1], " W", fontsize=12)
plt.text(V[0], V[1], " V", fontsize=12)
plt.text(R[0], R[1], " R", fontsize=12)

plt.xlim(-10,80)
plt.ylim(-10,80)
plt.xlabel("East (+x)")
plt.ylabel("North (+y)")
plt.title("Relative Wind Seen from Boat")
plt.legend()
plt.grid(True)
plt.gca().set_aspect('equal', adjustable='box')
plt.show()

    \end{lstlisting}
\end{frame}


\begin{frame}{Plot}
    \centering
    \includegraphics[width=\columnwidth, height=0.8\textheight, keepaspectratio]{figs/python image.png}     
\end{frame}


\end{document}