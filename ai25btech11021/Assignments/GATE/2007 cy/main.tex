%iffalse
\let\negmedspace\undefined
\let\negthickspace\undefined
\documentclass[journal,12pt,onecolumn]{exam}
\usepackage[version=4]{mhchem}
\usepackage{chemformula} % for \ch if needed
\usepackage{chemfig}
\usepackage{chemmacros}
\chemsetup{modules = reactions} % Enables reaction arrows
\usepackage{graphicx}
\graphicspath{ {./images/} }

\usepackage{fancyhdr}
\usepackage{geometry}
\usepackage{lastpage}
\usepackage{cite}
\usepackage{amsmath,amssymb,amsfonts,amsthm}
\usepackage{enumitem,multicol}
\usepackage{algorithmic}
\usepackage{graphicx}
\usepackage{textcomp}
\usepackage{xcolor}
\usepackage{txfonts}
\usepackage{listings}
\usepackage{enumitem}
\usepackage{mathtools}
\usepackage{gensymb}
\usepackage{comment}
\usepackage[breaklinks=true]{hyperref}
\usepackage{tkz-euclide} 
\usepackage{listings}
\usepackage{gvv}                                        
%\def\inputGnumericTable{}                                 
\usepackage[latin1]{inputenc}                                
\usepackage{color}                                            
\usepackage{array}                                            
\usepackage{longtable}                                       
\usepackage{calc}                                             
\usepackage{multirow}                                         
\usepackage{hhline}                                           
\usepackage{ifthen}                                           
\usepackage{lscape}
\usepackage{tabularx}
\usepackage{array}
\usepackage{float}


\newtheorem{theorem}{Theorem}[section]
\newtheorem{problem}{Problem}
\newtheorem{proposition}{Proposition}[section]
\newtheorem{lemma}{Lemma}[section]
\newtheorem{corollary}[theorem]{Corollary}
\newtheorem{example}{Example}[section]
\newtheorem{definition}[problem]{Definition}
\newcommand{\BEQA}{\begin{eqnarray}}
\newcommand{\EEQA}{\end{eqnarray}}
\newcommand{\define}{\stackrel{\triangle}{=}}
\theoremstyle{remark}

\geometry{margin=1 in}

% Marks the beginning of the document
\begin{document}
\begin{center}
    \textbf{2007} \\[5pt]
    \textbf{CY: Chemistry} \\[10pt]
\end{center}
Duration: Three hours \hfill Maximum Marks :$150$
\vspace{0.5 cm}

\centering \textbf{Read the following instructions carefully.}



\begin{enumerate}
\item This question paper contains 85 objective type questions. Q.$1$ to Q.$20$ carry \textbf{one} mark
each and Q.21 to Q.85 carry \textbf{two} marks each.
\item Attempt all the questions.
\item  Questions must be answered on \textbf{O}bjective \textbf{R}esponse \textbf{S}heet (\textbf{ORS}) by darkening the
appropriate bubble (marked A, B, C, D) using HB pencil against the question number on
the left hand side of the \textbf{ ORS. Each question has only one correct answer.}  In case you
wish to change an answer, erase the old answer completely.
\item Wrong answers will carry NEGATIVE marks. In Q.$1$ to Q.$20$, \textbf{0.25} mark will be
deducted for each wrong answer. In Q.21 to Q.76, Q.78, Q.80, Q.82 and in Q.84, \textbf{0.5}
mark will be deducted for each wrong answer. However, there is no negative marking in
Q.77, Q.79, Q.81, Q.83 and in Q.85. More than one answer bubbled against a question
will be taken as an incorrect response. Unattempted questions will not carry any marks.
\item  Write your registration number, your name and name of the examination centre at the
specified locations on the right half of the \textbf{ORS}.
\item Using HB pencil, darken the appropriate bubble under each digit of your registration
number and the letters corresponding to your paper code.
\item Calculator is allowed in the examination hall.
\item Charts, graph sheets or tables are NOT allowed in the examination hall.
\item Rough work can be done on the question paper itself. Additionally blank pages are given
at the end of the question paper for rough work.
\item This question paper contains \textbf{24} printed pages including pages for rough work. Please
check all pages and report, if there is any discrepancy.
\end{enumerate}


\newpage

\centering
\textbf{Q. 1-Q. 20 carry one mark each.}

\vspace{0.5cm}

\noindent\textbf{Q.1} \quad The rate of sulphonation of benzene can be significantly enhanced by the use of
\hfill{\textbf{(gate ee 2025)}}
\begin{itemize}
    \item[(A)] a mixture of HNO$_3$ and H$_2$SO$_4$
    \item[(B)] conc. H$_2$SO$_4$
    \item[(C)] a solution of SO$_3$ in H$_2$SO$_4$
    \item[(D)] SO$_3$
\end{itemize}

\noindent\textbf{Q.2} \quad 
\schemestart
\chemfig{*6(=-=-=-)} 
\+ 2Na \+ 2\ce{C2H5OH}
\arrow{->[\text{Liq. NH3}]}
\chemfig{*6(--=--=)} 
\+ 2\ce{C2H5ONa}
\schemestop
\\is an example of \hfill{\textbf{(gate ee 2025)}}

\begin{itemize}
    \item[(A)] Birch reduction
    \item[(B)] Clemmenson reduction
    \item[(C)] Wolff-Kishner reduction
    \item[(D)] Hydride reduction
\end{itemize}

\noindent\textbf{Q.3} \quad 
The major product (X) of the monobromination reaction is
 \hfill{\textbf{(gate ee 2025)}}

\[
\chemfig{CH_3-[:0]*6(------)} \xrightarrow{\ce{NBS}} \text{(X)}
\]
\begin{multicols}{2}
    
\begin{itemize}
  \item[(A)] \chemfig{Br-CH_2-[:0]-*6(------)}
  \item[(B)] \chemfig{CH_3-[:0]*6(---(-Br)---)}
\item[(C)] \chemfig{*6(--((-Br))-((-CH_3))---)}
  \item[(D)] \chemfig{*6(-(-[:30](-Br)-[:-30](CH_3))-----)}
  
\end{itemize}

\end{multicols}

\noindent\textbf{Q.4} \quad  Benzene can not be iodinated with I2 directly. However, in presence of oxidants such as HNO3, iodination is possible. The electrophile formed in this case is
\hfill{\textbf{(gate ee 2025)}}

\begin{multicols}{4}
    
\begin{itemize}
  \item[(A)] \[
  \left[ \ce{I^+} \right]
  \]
  \item[(B)] \[
  \left[ \ce{I^-} \right]
  \]
\item[(C)] \[
  \left[ \ce{^{\delta+}I \cdots ^{\delta+}OH2} \right]^+
  \]

  \item[(D)] \[
  \left[ \ce{^{\delta+}I \cdots ^{\delta-}OH2} \right]^+
  \]
  
\end{itemize}

\end{multicols}

\noindent\textbf{Q.5} \quad Classification of species as Electrophiles (E) and Nucleophiles (N)\hfill{\textbf{(gate ee 2025)}}

Given species: \ce{SO3}, \ce{Cl^+}, \ce{CH3NH2}, \ce{H3O^+}, \ce{BH3}, \ce{CN^-} 
\vspace{0.5 cm}


\begin{tabular}{ll}
(A) & E = \ce{SO3}, \ce{Cl^+}, \ce{BH3} \quad ; \quad N = \ce{CH3NH2}, \ce{H3O^+}, \ce{CN^-} \\
(B) & E = \ce{Cl^+}, \ce{H3O^+} \quad ; \quad N = \ce{SO3}, \ce{CH3NH2}, \ce{BH3}, \ce{CN^-} \\
(C) & E = \ce{Cl^+}, \ce{H3O^+}, \ce{BH3} \quad ; \quad N = \ce{SO3}, \ce{CH3NH2}, \ce{H3O^+}, \ce{CN^-} \\
(D) & E = \ce{SO3}, \ce{Cl^+}, \ce{H3O^+}, \ce{BH3} \quad ; \quad N = \ce{CH3NH2}, \ce{CN^-}
\end{tabular}

\noindent\textbf{Q.6} \quad The major product obtained upon treatment of compound X with \ce{H2SO4} at 80\degree C is \hfill{\textbf{(gate ee 2025)}}

\vspace{0.5cm}

\textbf{(X)}

\[
\chemfig{CH_3-[:30]CH_2-[:30]C(-[:90]CH_3)(-[:330]OH)-[:30]CH_2-[:30]CH_3}
\]

\vspace{0.5 cm}

\begin{multicols}{2}
\noindent
\textbf{(A)} \\
\chemfig{CH_3-[:30]CH_2-[:30]CH=[:30]CH-[:30]CH_2-[:30]CH_3}

\vspace{0.5cm}

\noindent
\textbf{(B)} \\
\chemfig{CH_3-[:30]C(-[:90]CH_3)-[:330]*5(-----)}

\vspace{0.5cm}

\noindent
\textbf{(C)} \\
\chemfig{CH_2=[:30]C(-[:90]CH_3)-[:330]CH_2-[:30]CH_3}

\vspace{0.5cm}

\noindent
\textbf{(D)} \\
\chemfig{CH_3-[:30]CH=[:30]C(-[:90]CH_3)-[:330]CH_2-[:30]CH_3}

\end{multicols}

\noindent \textbf{Q.7} \ce{BaTi[Si3O9]} is a class of \hfill{\textbf{(gate ee 2025)}}
\begin{multicols}{4}
    

\begin{itemize}
    \item[(A)] ortho silicate
    \item[(B)] cyclic silicate
    \item[(C)] chain silicate
    \item[(D)] sheet silicate
\end{itemize}
\end{multicols}
\vspace{0.5 cm}

\noindent \textbf{Q.8} The ground state term for \ce{V^{3+}} ion is \hfill{\textbf{(gate ee 2025)}}
\begin{multicols}{4}
\begin{itemize}
    \item[(A)] $^3F$
    \item[(B)] $^2F$
    \item[(C)] $^3P$
    \item[(D)] $^2D$
\end{itemize}
\end{multicols}
\vspace{0.5 cm}

\noindent \textbf{Q.9} In photosynthesis, the predominant metal present in the reaction centre of photosystem II is \hfill{\textbf{(gate ee 2025)}}
\begin{multicols}{4}
    

\begin{itemize}
    \item[(A)] Zn
    \item[(B)] Cu
    \item[(C)] Mn
    \item[(D)] Fe
\end{itemize}
\end{multicols}
\vspace{0.5 cm}

\noindent \textbf{Q.10} The octahedral complex / complex ion which shows both facial and meridional isomers is \hfill{\textbf{(gate ee 2025)}}
\begin{multicols}{2}
    
\begin{itemize}
    \item[(A)] Triglycinatocobalt(III)
    \item[(B)] Tris(ethylenediamine)cobalt(III)
    \item[(C)] Dichlorodiglycinatocobalt(III)
    \item[(D)] Trioxalatocobaltate(III)
\end{itemize}
\end{multicols}
\vspace{0.5 cm}

\noindent \textbf{Q.11} Zn in carbonic anhydrase is coordinated by three histidine and one water molecule. The reaction of \ce{CO2} with this enzyme is an example of \hfill{\textbf{(gate ee 2025)}}
\begin{multicols}{4}
    

\begin{itemize}
    \item[(A)] electrophilic addition
    \item[(B)] electron transfer
    \item[(C)] nucleophilic addition
    \item[(D)] electrophilic substitution
\end{itemize}

\end{multicols}
\vspace{0.5 cm}

\noindent \textbf{Q.12} The difference in the measured and calculated magnetic moment (based on spin-orbit coupling) is observed for \hfill{\textbf{(GATE EE 2025)}}
\begin{multicols}{2}
\begin{itemize}[leftmargin=*,labelsep=1em]
    \item[(A)] Pm$^{3+}$
    \item[(B)] Eu$^{3+}$
    \item[(C)] Dy$^{3+}$
    \item[(D)] Lu$^{3+}$
\end{itemize}
\end{multicols}
\vspace{0.5cm}

\noindent \textbf{Q.13} For a redox reaction, \ce{Cd^{2+} + 2e^- -> Cd}, the \ce{E_{p,\ anodic}} observed in cyclic voltammetry at hanging mercury drop electrode is --650 mV vs. SCE. The expected value for \ce{(E_{p})_{cathodic}} is \hfill{\textbf{(GATE EE 2025)}}
\begin{multicols}{2}
\begin{itemize}[leftmargin=*,labelsep=1em]
    \item[(A)] --708 mV
    \item[(B)] --679 mV
    \item[(C)] --650 mV
    \item[(D)] --621 mV
\end{itemize}
\end{multicols}
\vspace{0.5cm}

\noindent \textbf{Q.14} The dimension of Planck constant is (M, L and T denote mass, length and time respectively) \hfill{\textbf{(GATE EE 2025)}}
\begin{multicols}{2}
\begin{itemize}[leftmargin=*,labelsep=1em]
    \item[(A)] ML$^2$T$^{-2}$
    \item[(B)] ML$^2$T$^{-1}$
    \item[(C)] M$^2$L$^{-1}$T$^{-1}$
    \item[(D)] M$^{-1}$L$^2$T$^{-2}$
\end{itemize}
\end{multicols}
\vspace{0.5cm}

\noindent \textbf{Q.15} For a homonuclear diatomic molecule, the bonding molecular orbital is \hfill{\textbf{(GATE EE 2025)}}
\begin{multicols}{2}
\begin{itemize}[leftmargin=*,labelsep=1em]
    \item[(A)] $\sigma_u$ of lowest energy
    \item[(B)] $\sigma_u$ of second lowest energy
    \item[(C)] $\pi_g$ of lowest energy
    \item[(D)] $\pi_u$ of lowest energy
\end{itemize}
\end{multicols}
\vspace{0.5cm}

\noindent \textbf{Q.16} The selection rules for the appearance of P branch in the rotational-vibrational absorption spectra of a diatomic molecule within rigid rotor-harmonic oscillator model are \hfill{\textbf{(GATE EE 2025)}}
\begin{multicols}{2}
\begin{itemize}[leftmargin=*,labelsep=1em]
    \item[(A)] $\Delta \nu = \pm 1$ and $\Delta J = \pm 1$
    \item[(B)] $\Delta \nu = +1$ and $\Delta J = +1$
    \item[(C)] $\Delta \nu = +1$ and $\Delta J = -1$
    \item[(D)] $\Delta \nu = -1$ and $\Delta J = -1$
\end{itemize}
\end{multicols}
\vspace{0.5 cm}

\noindent \textbf{Q.17} The S$_2$ operation on a molecule with the axis of rotation as the z axis, moves a nucleus at (x, y, z) to \hfill{\textbf{(GATE EE 2025)}}
\begin{multicols}{2}
\begin{itemize}[leftmargin=*,labelsep=1em]
    \item[(A)] (-x, -y, z)
    \item[(B)] (x, -y, -z)
    \item[(C)] (-x, y, -z)
    \item[(D)] (-x, -y, -z)
\end{itemize}
\end{multicols}
\vspace{0.5 cm}

\noindent \textbf{Q.18} The expression which represents the chemical potential of the $i^{\text{th}}$ species ($\mu_i$) in a mixture ($i \ne j$) is \hfill{\textbf{(GATE EE 2025)}}
\begin{multicols}{2}
\begin{itemize}[leftmargin=*,labelsep=1em]
    \item[(A)] $\left( \frac{\partial E}{\partial n_i} \right)_{S,V,n_j}$
    \item[(B)] $\left( \frac{\partial H}{\partial n_i} \right)_{S,P,n_j}$
    \item[(C)] $\left( \frac{\partial A}{\partial n_i} \right)_{T,V,n_j}$
    \item[(D)] $\left( \frac{\partial G}{\partial n_i} \right)_{T,P,n_j}$
\end{itemize}
\end{multicols}
\vspace{0.5 cm}

\noindent \textbf{Q.19} Which of the following statements is \textbf{NOT} correct for a catalyst? \hfill{\textbf{(GATE EE 2025)}}
\begin{multicols}{2}
\begin{itemize}[leftmargin=*,labelsep=1em]
    \item[(A)] It increases the rate of a reaction
    \item[(B)] It is not consumed in the course of a reaction
    \item[(C)] It provides an alternate pathway for the reaction
    \item[(D)] It increases the activation energy of the reaction
\end{itemize}
\end{multicols}
\vspace{0.5cm}

\noindent \textbf{Q.20} The value of the rate constant for the gas phase reaction \ch{2NO2 + F2 -> 2NO2F} is 38 dm$^{3}$ mol$^{-1}$ s$^{-1}$ at 300K. The order of the reaction is \hfill{\textbf{(GATE EE 2025)}}
\begin{multicols}{2}
\begin{itemize}[leftmargin=*,labelsep=1em]
    \item[(A)] 0
    \item[(B)] 1
    \item[(C)] 2
    \item[(D)] 3
\end{itemize}
\end{multicols}
\vspace{0.5cm}

\noindent \textbf {Q.21 to Q.75 carry two marks each.}

\vspace{0.5cm}

\noindent \textbf{Q.21} Boric acid in aqueous solution in presence of glycerol behaves as a strong acid due to the formation of \hfill{\textbf{(GATE EE 2025)}}
\begin{multicols}{2}
\begin{itemize}[leftmargin=*,labelsep=1em]
    \item[(A)] an anionic metal-chelate
    \item[(B)] borate anion
    \item[(C)] glycerate ion
    \item[(D)] a charge transfer complex
\end{itemize}
\end{multicols}
\vspace{0.5cm}

\noindent \textbf{Q.22} Match the compounds in \textbf{List I} with the corresponding structure / property given in \textbf{List II} \hfill{\textbf{(GATE EE 2025)}}

\begin{tabbing}
\hspace{3cm} \= \textbf{List I} \hspace{5cm} \= \textbf{List II} 

\\
(a) (Ph$_3$P)$_3$RhCl \> (i) Spinel \\
(b) LiCl$_6$ \> (ii) Intercalation \\
(c) PtF$_6$ \> (iii) Oxidising agent \\
(d) Ni$_3$S$_4$ \> (iv) Catalyst for alkane hydrogenation \\

\end{tabbing}


\begin{itemize}[leftmargin=*,labelsep=1em]
    \item[(A)] a - iii \quad b - i \quad c - ii \quad d - iv
    \item[(B)] a - iv \quad b - ii \quad c - iii \quad d - i
    \item[(C)] a - iii \quad b - ii \quad c - ii \quad d - iv
    \item[(D)] a - iv \quad b - iii \quad c - ii \quad d - i
\end{itemize}

\vspace{0.5cm}

\noindent \textbf{Q.23} \ch{W(CO)6} reacts with MeLi to give an intermediate which upon treatment with \ch{CH2N2} gives a compound \textbf{X}. \textbf{X} is represented as \hfill{\textbf{(GATE EE 2025)}}
\begin{multicols}{2}
\begin{itemize}[leftmargin=*,labelsep=1em]
    \item[(A)] WMe$_6$
    \item[(B)] (CO)$_5$W-Me
    \item[(C)] (CO)$_5$W=C(Me)OMe
    \item[(D)] (CO)$_5$W = CMe
\end{itemize}
\end{multicols}
\vspace{0.5cm}

\noindent \textbf{Q.24} Considering the quadrupolar nature of M-M bond in \ch{[Re2Cl8]^{2-}}, the M-M bond order in \ch{[Re2Cl4(PMe2Ph)4]^{+}} and \ch{[Re2Cl4(PMe2Ph)4]} respectively are \hfill{\textbf{(GATE EE 2025)}}
\begin{multicols}{2}
\begin{itemize}[leftmargin=*,labelsep=1em]
    \item[(A)] 3.0 and 3.0
    \item[(B)] 3.0 and 3.5
    \item[(C)] 3.5 and 3.5
    \item[(D)] 3.5 and 3.0
\end{itemize}
\end{multicols}
\vspace{0.5cm}

\noindent \textbf{Q.25} A student recorded a polarogram of 2.0 mM Cd$^{2+}$ solution and forgot to add KCl solution. What type of error do you expect in his results? \hfill{\textbf{(GATE EE 2025)}}
\begin{multicols}{2}
\begin{itemize}[leftmargin=*,labelsep=1em]
    \item[(A)] Only migration current will be observed
    \item[(B)] Only diffusion current will be observed
    \item[(C)] Both migration current as well as diffusion current will be observed
    \item[(D)] Both catalytic current as well as diffusion current will be observed
\end{itemize}
\end{multicols}
\vspace{0.5cm}

\noindent \textbf{Q.26} The separation of trivalent lanthanide ions, Lu$^{3+}$, Yb$^{3+}$, Dy$^{3+}$, Eu$^{3+}$ can be effectively done by a cation exchange resin using ammonium \textit{o}-hydroxy isobutyrate as the eluent. The order in which the ions will be separated is \hfill{\textbf{(GATE EE 2025)}}
\begin{multicols}{2}
\begin{itemize}[leftmargin=*,labelsep=1em]
    \item[(A)] Lu$^{3+}$, Yb$^{3+}$, Dy$^{3+}$, Eu$^{3+}$
    \item[(B)] Eu$^{3+}$, Dy$^{3+}$, Yb$^{3+}$, Lu$^{3+}$
    \item[(C)] Dy$^{3+}$, Yb$^{3+}$, Eu$^{3+}$, Lu$^{3+}$
    \item[(D)] Yb$^{3+}$, Dy$^{3+}$, Lu$^{3+}$, Eu$^{3+}$
\end{itemize}
\end{multicols}
\vspace{0.5cm}

\noindent \textbf{Q.27} Arrange the following metal complexes in order of their increasing hydration energy \hfill{\textbf{(GATE EE 2025)}}

\begin{center}
\begin{tabular}{cccc}
\ch{[Mn(H2O)6]^{2+}}$_P$ & \ch{[V(H2O)6]^{2+}}$_Q$ & \ch{[Ni(H2O)6]^{2+}}$_R$ & \ch{[Ti(H2O)6]^{2+}}$_S$
\end{tabular}
\end{center}

\begin{multicols}{2}
\begin{itemize}[leftmargin=*,labelsep=1em]
    \item[(A)] $P < S < Q < R$
    \item[(B)] $P < Q < R < S$
    \item[(C)] $Q < P < R < S$
    \item[(D)] $S < R < Q < P$
\end{itemize}
\end{multicols}
\vspace{0.5cm}

\noindent \textbf{Q.28} In the complex, [Ni$_2$(η$^5$-Cp)$_2$(CO)$_2$], the IR stretching frequency appears at 1857 cm$^{-1}$ (strong) and 1897 cm$^{-1}$ (weak). The valence electron count and the nature of the M-CO bond respectively are \hfill{\textbf{(GATE EE 2025)}}
\begin{multicols}{2}
\begin{itemize}[leftmargin=*,labelsep=1em]
    \item[(A)] 16 e$^-$, bridging
    \item[(B)] 17 e$^-$, bridging
    \item[(C)] 18 e$^-$, terminal
    \item[(D)] 18 e$^-$, bridging
\end{itemize}
\end{multicols}
\vspace{0.5cm}

\noindent \textbf{Q.29} The correct classification of [B$_5$H$_5$]$^{2-}$, B$_5$H$_9$ and B$_5$H$_{11}$ respectively is \hfill{\textbf{(GATE EE 2025)}}
\begin{multicols}{2}
\begin{itemize}[leftmargin=*,labelsep=1em]
    \item[(A)] closo, arachno, nido
    \item[(B)] arachno, closo, nido
    \item[(C)] closo, nido, arachno
    \item[(D)] nido, arachno, closo
\end{itemize}
\end{multicols}
\vspace{0.5cm}


\noindent \textbf{Q.30} The compounds \textbf{X} and \textbf{Y} in the following reaction are
\hfill \textbf{(GATE EE 2025)}

\[
\chemfig P_4S_{10} \xrightarrow{\ce{EtOH}} \text{(X)}
\xrightarrow{\ce{Cl_2}} \text{(Y)}
 \xrightarrow{\ce{p-O_2NC_6H_4ONa}} \text{Paration} 
\]


\begin{multicols}{2}
\begin{itemize}[leftmargin=*, labelsep=1em]
    \item[(A)] $X = (\mathrm{Et})_2\mathrm{P}(S)SH \quad ; \quad Y = (\mathrm{Et})_2\mathrm{P}(S)Cl$
    \item[(B)] $X = (\mathrm{EtO})_2\mathrm{P}(S)SH \quad ; \quad Y = (\mathrm{EtO})_2\mathrm{P}(S)Cl$
    \item[(C)] $X = (\mathrm{EtO})_2\mathrm{PSH} \quad ; \quad Y = (\mathrm{EtO})_2\mathrm{PCl}$
    \item[(D)] $X = (\mathrm{Et})_3\mathrm{PO} \quad ; \quad Y = (\mathrm{Et})_3\mathrm{PCl}$
\end{itemize}
\end{multicols}
\vspace{0.5 cm}

\noindent \textbf{Q.31} Consider the reactions \hfill{\textbf{(gate ee 2025)}}

\begin{enumerate}[label=\arabic*.]
\item \([Cr(H_2O)_6]^{2+} + [CoCl(NH_3)_5]^{2+} \rightarrow [Co(NH_3)_5(H_2O)]^{3+} + [CrCl(H_2O)_5]^{2+}\)
\item \([Fe(CN)_6]^{4-} + [Mo(CN)_8]^{3-} \rightarrow [Fe(CN)_6]^{3-} + [Mo(CN)_8]^{4-}\)
\end{enumerate}

Which one of the following is the correct statement?  

\begin{multicols}{2}
\begin{enumerate}[label=(\Alph*)]
\item[(A)] Both involve an inner sphere mechanism  
\item[(B)] Both involve an outer sphere mechanism  
\vspace{0.5 cm}
\item[(C)] Reaction 1 follows outer sphere and reaction 2 follows inner sphere mechanism  
\item[(D)] Reaction 1 follows inner sphere and reaction 2 follows outer sphere mechanism  
\end{enumerate}
\end{multicols}
\vspace{0.5cm}

\noindent \textbf{Q.32} The pair of compounds having the same hybridization for the central atom is  \hfill{\textbf{(gate ee 2025)}}

\begin{multicols}{2}
\begin{enumerate}[label=(\Alph*)]
\item[(A)] XeF$_4$ and [SiF$_6$]$^{2-}$  
\item[(B)] [NiCl$_4$]$^{2-}$ and [PtCl$_4$]$^{2-}$  
\item[(C)] Ni(CO)$_4$ and XeO$_2$F$_2$  
\item[(D)] [Co(NH$_3$)$_6$]$^{3+}$ and [Co(H$_2$O)$_6$]$^{3+}$  
\end{enumerate}
\end{multicols}
\vspace{0.5cm}

\noindent \textbf{Q.33} In the reaction shown below, X and Y respectively are  \hfill{\textbf{(gate ee 2025)}}

\[
\text{Mn}_2(\text{CO})_{10} \xrightarrow{\text{Na}} (X) \xrightarrow{\text{CH}_3\text{COCl}} (Y)
\]

\begin{multicols}{2}
\begin{enumerate}[label=(\Alph*)]
\item[(A)] [Mn(CO)$_4$]$^{2-}$ , [CH$_3$C(O)Mn(CO)$_5$]$^{-}$  
\item[(B)] [Mn(CO)$_5$]$^{-}$ , CH$_3$C(O)Mn(CO)$_5$  
\item[(C)] [Mn(CO)$_5$]$^{-}$ , C1Mn(CO)$_5$  
\item[(D)] [Mn(CO)$_4$]$^{2-}$ , C1Mn(CO)$_5$  
\end{enumerate}
\end{multicols}
\vspace{0.5cm}

\noindent \textbf{Q.34} The Lewis acid character of BF$_3$, BCl$_3$ and BBr$_3$ follows the order  \hfill{\textbf{(gate ee 2025)}}

\begin{multicols}{2}
\begin{enumerate}[label=(\Alph*)]
\item BF$_3$ < BBr$_3$ < BCl$_3$  
\item BCl$_3$ < BBr$_3$ < BF$_3$  
\item BF$_3$ < BCl$_3$ < BBr$_3$  
\item BBr$_3$ < BCl$_3$ < BF$_3$  
\end{enumerate}
\end{multicols}
\vspace{0.5cm}

\noindent \textbf{Q.35} The compound which shows L$\leftarrow$M charge transfer is  \hfill{\textbf{(gate ee 2025)}}

\begin{multicols}{2}
\begin{enumerate}[label=(\Alph*)]
\item Ni(CO)$_4$  
\item K$_2$Cr$_2$O$_7$  
\item HgO  
\item[(D)] [Ni(H$_2$O)$_6$]$^{2+}$  
\end{enumerate}
\end{multicols}
\vspace{0.5cm}

\noindent \textbf{Q.36} The reaction of [PtCl$_4$]$^{2-}$ with NH$_3$ gives rise to  \hfill{\textbf{(gate ee 2025)}}

\begin{multicols}{2}
\begin{enumerate}[label=(\Alph*)]
\item[(A)] [PtCl$_2$(NH$_3$)$_2$]$^{2-}$  
\item[(B)] trans-[PtCl$_2$(NH$_3$)$_2$]  
\item[(C)] [PtCl$_2$(NH$_3$)$_4$]  
\item[(D)] cis-[PtCl$_2$(NH$_3$)$_2$]  
\end{enumerate}
\end{multicols}
\vspace{0.5cm}

\noindent \textbf{Q.37} Zeise's salt is represented as  \hfill{\textbf{(gate ee 2025)}}

\begin{multicols}{2}
\begin{enumerate}[label=(\Alph*)]
\item[(A)] H$_2$PtCl$_6$  
\item[(b)] [PtCl$_4$]$^{2-}$  
\item[(c)] [ZnCl$_4$]$^{2-}$  
\item[(D)] [PtCl$_3$($\eta^2$-C$_2$H$_4$)]$^{-}$  
\end{enumerate}
\end{multicols}
\vspace{0.5cm}

\noindent \textbf{Q.38} The catalyst used in the conversion of ethylene to acetaldehyde using Wacker process is  \hfill{\textbf{(gate ee 2025)}}

\begin{multicols}{2}
\begin{enumerate}[label=(\Alph*)]
\item H$_2$Co(CO)$_4$  
\item PdCl$_4^{2-}$  
\item V$_2$O$_5$  
\item TiCl$_4$ in presence of Al(C$_2$H$_5$)$_3$  
\end{enumerate}
\end{multicols}
\vspace{0.5cm}

\noindent \textbf{Q.39} The temperature of 54 g of water is raised from 15°C to 75°C at constant pressure. The change in the enthalpy of the system (given that $C_{p,m}$ of water = 75 J K$^{-1}$ mol$^{-1}$) is \hfill{\textbf{(gate ee 2025)}}

\begin{multicols}{2}
\begin{itemize}[label={}, leftmargin=*, itemsep=0pt]
\item (A) 4.5 kJ
\item (B) 13.5 kJ
\item (C) 9.0 kJ
\item (D) 18.0 kJ
\end{itemize}
\end{multicols}
\vspace{0.5cm}

\noindent \textbf{Q.40} The specific volume of liquid water is 1.0001 mL g$^{-1}$ and that of ice is 1.0907 mL g$^{-1}$ at 0°C. If the heat of fusion of ice at this temperature is 333.88 J g$^{-1}$, the rate of change of melting point of ice with pressure in deg atm$^{-1}$ will be \hfill{\textbf{(gate ee 2025)}}

\begin{multicols}{2}
\begin{itemize}[label={}, leftmargin=*, itemsep=0pt]
\item (A) $-$0.0075
\item (B) 0.0075
\item (C) 0.075
\item (D) $-$0.075
\end{itemize}
\end{multicols}
\vspace{0.5cm}

\noindent \textbf{Q.41} Given that $E^\circ$(Fe$^{3+}$, Fe$^{2+}$) = $-$0.04 V and $E^\circ$(Fe$^{2+}$, Fe) = $-$0.44 V, the value of $E^\circ$(Fe$^{3+}$, Fe) is \hfill{\textbf{(gate ee 2025)}}

\begin{multicols}{2}
\begin{itemize}[label={}, leftmargin=*, itemsep=0pt]
\item (A) 0.76 V
\item (B) $-$0.40 V
\item (C) $-$0.76 V
\item (D) 0.40 V
\end{itemize}
\end{multicols}
\vspace{0.5cm}

\noindent \textbf{Q.42} For the reaction P + Q + R $\rightarrow$ S, experimental data for the measured initial rates is given below \hfill{\textbf{(gate ee 2025)}}

\begin{center}
\begin{tabular}{|c|c|c|c|c|}
\hline
Expt. & Initial conc. P (M) & Initial conc. Q (M) & Initial conc. R (M) & Initial rate (M s$^{-1}$) \\
\hline
1 & 0.2 & 0.5 & 0.4 & $8.0 \times 10^{-3}$ \\
2 & 0.4 & 0.5 & 0.4 & $3.2 \times 10^{-2}$ \\
3 & 0.4 & 0.25 & 0.4 & $1.28 \times 10^{-2}$ \\
4 & 0.1 & 0.25 & 1.6 & $4.0 \times 10^{-3}$ \\
\hline
\end{tabular}
\end{center}

The order of the reaction with respect to P, Q and R respectively is

\begin{multicols}{2}
\begin{itemize}[label={}, leftmargin=*, itemsep=0pt]
\item (A) 2, 2, 1
\item (B) 2, 1, 2
\item (C) 2, 1, 1
\item (D) 1, 1, 2
\end{itemize}
\end{multicols}
\vspace{0.5cm}

\noindent \textbf{Q.43} Sucrose is converted to a mixture of glucose and fructose in a pseudo first order process under alkaline conditions. The reaction has a half life of 28.4 min. The time required for the reduction of a 8.0 mM sample of sucrose to 1.0 mM is \hfill{\textbf{(gate ee 2025)}}

\begin{multicols}{2}
\begin{itemize}[label={}, leftmargin=*, itemsep=0pt]
\item (A) 56.8 min
\item (B) 170.4 min
\item (C) 85.2 min
\item (D) 227.2 min
\end{itemize}
\end{multicols}
\vspace{0.5cm}

\noindent\textbf{Q.44} The reaction \hfill{\textbf{(gate ee 2025)}}
\[
\text{2NO(g)} + \text{O}_2(g) \rightarrow \text{2NO}_2(g)
\]
proceeds via the following steps \hfill{   }
\[
\begin{aligned}
\text{NO + NO} &\xrightarrow{\ce{K_a}} \text{N}_2\text{O}_2 \\
\text{N}_2\text{O}_2 &\xrightarrow{\ce{K_{a'}}} \text{NO + NO} \\
\text{N}_2\text{O}_2 + \text{O}_2 &\xrightarrow{k_b} \text{NO}_2 + \text{NO}_2
\end{aligned}
\]
The rate of this reaction is equal to \hfill{   }
\begin{multicols}{2}
\begin{enumerate}[label=(\Alph*)]
    \item $2k_b[\text{NO}]^2[\text{O}_2]$
    \item $\frac{2k_a k_b[\text{NO}]^2[\text{O}_2]}{(k_{-f}+k_b[\text{O}_2])}$
    \item $2k_b[\text{NO}]^2[\text{O}_2]$
    \item $k_b[\text{NO}]^2[\text{O}_2]$
\end{enumerate}
\end{multicols}

\noindent\textbf{Q.45} 40 millimoles of NaOH are added to 100 mL of a 1.2 M HA and Y M NaA buffer resulting in a solution of pH 5.30. Assuming that the volume of the buffer remains unchanged, the pH of the buffer ($K_\text{HA} = 1.00 \times 10^{-5}$) is\hfill{\textbf{(gate ee 2025)}}
\begin{multicols}{2}
\begin{enumerate}[label=(\Alph*)]
    \item 5.30
    \item 5.00
    \item 0.30
    \item 10.30
\end{enumerate}
\end{multicols}

\noindent\textbf{Q.46} The entropy of mixing of 10 moles of helium and 10 moles of oxygen at constant temperature and pressure, assuming both to be ideal gases, is\hfill{\textbf{(gate ee 2025)}}
\begin{multicols}{2}
\begin{enumerate}[label=(\Alph*)]
    \item 115.3 J K$^{-1}$
    \item 5.8 J K$^{-1}$
    \item 382.9 J K$^{-1}$
    \item 230.6 J K$^{-1}$
\end{enumerate}
\end{multicols}

\noindent\textbf{Q.47} The ionisation potential of hydrogen atom is 13.6 eV. The first ionisation potential of a sodium atom, assuming that the energy of its outer electron can be represented by a H-atom like model with an effective nuclear charge of 1.84, is\hfill{\textbf{(gate ee 2025)}}
\begin{multicols}{2}
\begin{enumerate}[label=(\Alph*)]
    \item 46.0 eV
    \item 11.5 eV
    \item 5.1 eV
    \item 2.9 eV
\end{enumerate}
\end{multicols}

\noindent\textbf{Q.48} The quantum state of a particle moving in a circular path in a plane is given by
\[
\Psi_{m}(\phi) = (1/\sqrt{2\pi})e^{im\phi}, m=0,\pm1,\pm2,...
\]
When a perturbation $H_1 = P \cos \phi$ is applied ($P$ is a constant), what will be the first order correction to the energy of the $m^\text{th}$ state\hfill{\textbf{(gate ee 2025)}}
\begin{multicols}{2}
\begin{enumerate}[label=(\Alph*)]
    \item 0
    \item $P(2\pi)$
    \item $P(4\pi)$
    \item $Pm^2(4\pi^2)$
\end{enumerate}
\end{multicols}
 \vspace{0.5 cm}

 \noindent\textbf{Q.49} The correct statement(s) among the following is/are \\
(i) The vibrational energy levels of a real diatomic molecule are equally spaced. \\
(ii) At 500K, the reaction A → B is spontaneous when \DeltaH = 18.83 kJ mol$^{-1}$ and \DeltaS = 41.84 J K$^{-1}$ mol$^{-1}$. \\
(iii) The process of fluorescence involves transition from a singlet electronic state to another singlet electronic state by absorption of light. \\
(iv) When a constant P is added to each of the possible energies of a system, its entropy remains unchanged. \\
\hfill{\textbf{(gate ee 2025)}}
\begin{multicols}{2}
\begin{enumerate}[label=(\Alph*)]
    \item only i
    \item only ii
    \item both i and iii
    \item both ii and iv
\end{enumerate}
\end{multicols}

\noindent\textbf{Q.50} Assuming H$_2$ and HD molecules having equal bond lengths, the ratio of the rotational partition functions of these molecules, at temperatures above 100K is \hfill{\textbf{(gate ee 2025)}}
\begin{multicols}{2}
\begin{enumerate}[label=(\Alph*)]
    \item 3/8
    \item 3/4
    \item 1/2
    \item 2/3
\end{enumerate}
\end{multicols}

\noindent\textbf{Q.51} N noninteracting molecules are distributed among three nondegenerate energy levels $\varepsilon_0 = 0$, $\varepsilon_1 = 1.38 \times 10^{-21}$ J and $\varepsilon_2 = 2.76 \times 10^{-21}$ J at 100K. If the average total energy of the system at this temperature is $1.38 \times 10^{-18}$ J, the number of molecules in the system is \hfill{\textbf{(gate ee 2025)}}
\begin{multicols}{2}
\begin{enumerate}[label=(\Alph*)]
    \item 1000
    \item 1503
    \item 2354
    \item 2987
\end{enumerate}
\end{multicols} 
\vspace{1 cm}

\noindent\textbf{Q.52} The $J = 0 \rightarrow 1$ rotational transition for $^{1}H^{2}D^{+}$ occurs at 500.72 GHz. Assuming the molecule to be a rigid rotor, the $J = 3 \rightarrow 4$ transition occurs at \hfill{\textbf{(gate ee 2025)}}
\begin{multicols}{2}
\begin{enumerate}[label=(\Alph*)]
    \item 50.1 cm$^{-1}$
    \item 66.8 cm$^{-1}$
    \item 16.7 cm$^{-1}$
    \item 83.5 cm$^{-1}$
\end{enumerate}
\end{multicols}
\vspace{3 cm}

\noindent\textbf{Q.53} The rate constants of two reactions at temperature T are $k_1(T)$ and $k_2(T)$ and the corresponding activation energies are $E_1$ and $E_2$ with $E_2 > E_1$. When temperature is raised from $T_1$ to $T_2$, which one of the following relations is correct? \hfill{\textbf{(gate ee 2025)}}
\begin{multicols}{2}
\begin{enumerate}[label=(\Alph*)]
    \item $\dfrac{k_1(T_2)}{k_1(T_1)} > \dfrac{k_2(T_2)}{k_2(T_1)}$
    \item $\dfrac{k_2(T_2)}{k_2(T_1)} > \dfrac{k_1(T_2)}{k_1(T_1)}$
    \item $\dfrac{k_2(T_1)}{k_2(T_2)} > \dfrac{k_1(T_2)}{k_1(T_1)}$
    \item $\dfrac{k_1(T_1)}{k_1(T_2)} > \dfrac{k_2(T_1)}{k_2(T_2)}$
\end{enumerate}
\end{multicols}
\vspace{3 cm}
\noindent\textbf{Q.54} The number of degrees of freedom for a system consisting of NaCl(s), Na$^+$(aq) and Cl$^-$(aq) at equilibrium is \hfill{\textbf{(gate ee 2025)}}
\begin{multicols}{2}
\begin{enumerate}[label=(\Alph*)]
    \item 2
    \item 3
    \item 4
    \item 5
\end{enumerate}
\end{multicols} 
\vspace{2 cm}

\noindent\textbf{Q.55} Match the structures in \textbf{List I} with their correct names given in \textbf{List II} \hfill{\textbf{(gate ee 2025)}}

\vspace{0.5cm}

\begin{tabular}{m{0.5\textwidth} m{0.5\textwidth}}
\textbf{List I} & \textbf{List II} \\[10pt]

% (a) 2-methyl furan
(a) \quad
\chemfig{*5(-O-=(-CH_3)-=-)}\hspace{0.3cm}  
& 
(i) 2-methyl furan \\[15pt]

% (b) Imidazole
(b) \quad
\chemfig{*5(=N-=-N--)} 
&
(ii) Imidazole \\[15pt]

% (c) 5-hydroxybenzothiazole
(c) \quad
\chemfig{*6((-NH_2)=N-=-=-)} 
&
(iii) 5-hydroxybenzothiazole \\[15pt]

% (d) 2-amino piperidine
(d) \quad
\chemfig{*6((-NH_2)-=-=-=--)} 
&
(iv) 2-amino piperidine \\[15pt]

% (e) 2-amino morpholine
(e) \quad
\chemfig{*6(--(-NH_2)-NH---)} 
&
(v) 2-amino morpholine \\
\end{tabular}

\vspace{0.5cm}

\noindent Options:

\begin{multicols}{2}
\begin{enumerate}
    \item a-vii, b-ii, c-vi, d-iii, e-iv
    \item a-vii, b-ii, c-vi, d-viii, e-iv
    \item a-vii, b-ii, c-vi, d-iii, e-v
    \item a-i, b-ii, c-vi, d-iii, e-iv
\end{enumerate}
\end{multicols} 
\vspace{1 cm}

\noindent\textbf{Q.56} The result of the reduction of either (R) or (S) 2-methylcyclohexanone, in separate reactions, using LiAlH\textsubscript{4} is that the reduction of \hfill{\textbf{(gate ee 2025)}}

\vspace{0.5cm}

\begin{multicols}{2}
\begin{enumerate}
    \item the R enantiomer is stereoselective
    \item the R enantiomer is stereospecific
    \item the S enantiomer is stereospecific
    \item both the R and S enantiomers is stereoselective
\end{enumerate}
\end{multicols}

 \vspace{0.5 cm}


\noindent\textbf{Q.57} The increasing order of basicity among the following is \hfill{\textbf{(gate ee 2025)}}

\includegraphics[scale=0.75]{images/image1.png}

\begin{multicols}{2}
\begin{itemize}
    \item[(A)] Y < X < Z
    \item[(B)] Y < Z < X
    \item[(C)] X < Z < Y
    \item[(D)] X < Y < Z
\end{itemize}
\end{multicols}

\vspace{0.5cm}

\noindent\textbf{Q.58} In the reaction \hfill{\textbf{(gate ee 2025)}}

\includegraphics[scale=0.75]{images/image2.png}

If the concentration of both the reactants is doubled, then the rate of the reaction will

\begin{multicols}{2}
\begin{itemize}
    \item[(A)] remain unchanged
    \item[(B)] quadruple
    \item[(C)] reduce to one fourth
    \item[(D)] double
\end{itemize}
\end{multicols}

\vspace{0.5cm}

\noindent\textbf{Q.59} Match the structures in \textbf{List I} with the coupling constant [$^1$J] (Hz) given in \textbf{List II} \hfill{\textbf{(gate ee 2025)}}

\includegraphics[scale=0.75]{images/image3.png}

\begin{multicols}{2}
\begin{itemize}
    \item[(A)] a-i \quad b-ii \quad c-iii
    \item[(B)] a-ii \quad b-iii \quad c-i
    \item[(C)] a-iii \quad b-ii \quad c-i
    \item[(D)] a-iii \quad b-i \quad c-ii
\end{itemize}
\end{multicols}

\noindent\textbf{Q.60} Phenol on reaction with formaldehyde and dimethyl amine mainly gives \hfill{\textbf{(gate ee 2025)}}

\includegraphics[scale=1.5 ]{images/image4.png}

\vspace{0.5cm}

\noindent\textbf{Q.61} The mono protonation of adenine (X) in acidic solution \hfill{\textbf{(gate ee 2025)}}
\includegraphics[scale=1.5 ]{images/image5.png}


Mainly occurs at \hfill{  }

\begin{multicols}{2}
\begin{itemize}
    \item[(A)] position 1
    \item[(B)] position 2
    \item[(C)] position 3
    \item[(D)] either position 4 or 5
\end{itemize}
\end{multicols}

\vspace{0.5cm}

\noindent\textbf{Q.62} In the following reaction \hfill{\textbf{(gate ee 2025)}}

\includegraphics[scale=1.5 ]{images/image6.png}

X and Y respectively are

\begin{multicols}{2}
\begin{itemize}
    \item[(A)] \textsuperscript{1}:CH\textsubscript{2} and cis 1,2 dimethylcyclopropane
    \item[(B)] \textsuperscript{3}:CH\textsubscript{2} and cis 1,2 dimethylcyclopropane
    \item[(C)] \textsuperscript{1}:CH\textsubscript{2} and a mixture of cis/trans 1,2 dimethylcyclopropane
    \item[(D)] \textsuperscript{3}:CH\textsubscript{2} and a mixture of cis/trans 1,2 dimethylcyclopropane
\end{itemize}
\end{multicols}

\noindent\textbf{Q.63} The major products obtained upon treating a mixture of \hfill{\textbf{(gate ee 2025)}}

\includegraphics[scale=1]{images/image7.png} 
\vspace{1 cm}

\noindent\textbf{Q.64} Match the observed principal absorptions in the visible spectrum shown in \textbf{List I} with the bond that shows this absorption in \textbf{List II} \hfill{\textbf{(gate ee 2025)}}

\begin{center}
\begin{tabular}{c@{\hspace{3cm}}c}
\textbf{List I} & \textbf{List II} \\
(a) $\sigma \rightarrow \sigma^*$ & (i) C -- C \\
(b) $n \rightarrow \sigma^*$ & (ii) C -- O \\
(c) $n, \pi^*$ & (iii) C = O \\
(d) $\pi, \pi^*$ & (iv) C = C \\
\end{tabular}
\end{center}

\begin{multicols}{2}
\begin{itemize}
    \item[(A)] a-i \quad b-ii \quad c-iii \quad d-iv
    \item[(B)] a-i \quad b-iii \quad c-ii \quad d-iv
    \item[(C)] a-ii \quad b-i \quad c-iv \quad d-iii
    \item[(D)] a-iv \quad b-ii \quad c-iii \quad d-i
\end{itemize}
\end{multicols}
\vspace{1 cm}

\noindent\textbf{Q.66} The direction of rotation of the following thermal electrocyclic ring closures respectively is \hfill{\textbf{(gate ee 2025)}}

\includegraphics[scale=1.5]{images/image8.png}


\begin{multicols}{2}
\begin{itemize}
    \item[(A)] disrotatory, disrotatory, disrotatory
    \item[(B)] conrotatory, conrotatory, conrotatory
    \item[(C)] disrotatory, disrotatory, conrotatory
    \item[(D)] disrotatory, conrotatory, disrotatory
\end{itemize}
\end{multicols}

\vspace{0.5cm}


\noindent\textbf{Q.67} The molecule(s) that exist as \textit{meso} structure(s) is/are \hfill{\textbf{(gate ee 2025)}}

\includegraphics[scale=1]{images/image9.png}

\vspace{0.3cm}

\begin{multicols}{2}
\begin{itemize}
  \item[(A)] only M
  \item[(B)] both K and L
  \item[(C)] only L
  \item[(D)] only K
\end{itemize}
\end{multicols}

\vspace{1cm}

\noindent \textbf{Q.68} The stereochemical descriptors for the atoms labeled H\textsubscript{a} and H\textsubscript{b} in the structures X, Y and Z respectively are \hfill{\textbf{(gate ee 2025)}}
\includegraphics[scale=1]{images/image10.png}
\begin{multicols}{2}
\begin{enumerate}[label=(\Alph*)]
\item X-homotopic, Y-enantiotopic and Z-diastereotopic
\item X-enantiotopic, Y-homotopic and Z-diastereotopic
\item X-diastereotopic, Y-homotopic and Z-enantiotopic
\item X-homotopic, Y-diastereotopic and Z-enantiotopic
\end{enumerate}
\end{multicols}
\vspace{1cm}

\noindent \textbf{Q.69} Treatment of the pentapeptide Gly-Arg-Phe-Ala-Ala, in separate experiments, with the enzymes Trypsin, Chymotrypsin and Carboxypeptidase A respectively, gives \hfill{\textbf{(gate ee 2025)}}

\begin{multicols}{2}
\begin{enumerate}[label=(\Alph*)]
\item Gly-Arg + Phe-Ala-Ala ; Gly-Arg-Phe + Ala-Ala ; Gly-Arg-Phe-Ala + Ala
\item Gly-Arg-Phe + Ala-Ala ; Gly-Arg-Phe + Ala-Ala ; Gly-Arg-Phe-Ala + Ala
\item Gly-Arg + Phe-Ala-Ala ; Gly-Arg-Phe-Ala + Ala ; Gly-Arg-Phe-Ala-Ala
\item Gly-Arg + Phe-Ala-Ala ; Gly-Arg-Phe + Ala-Ala ; Gly + Arg-Phe-Ala + Ala
\end{enumerate}
\end{multicols}
\vspace{2cm}

\noindent \textbf{Q.70} Hordenine (X), an alkaloid, undergoes Hoffman degradation to give compound (Y). (Y) on treatment with alkaline permanganate gives (Z). Y and Z respectively are \hfill{\textbf{(gate ee 2025)}}

\includegraphics[scale=1]{images/image11.png}

\includegraphics[scale=1]{images/image12.png}

\noindent \textbf{Common Data for Questions 71, 72, 73:} \\
Trans 1,2 difluoroethylene molecule has a 2-fold rotational axis, a symmetry plane perpendicular to the rotational axis and an inversion centre.

\noindent \textbf{Q.71} The number of distinct symmetry operations that can be performed on the molecule is \hfill{\textbf{(gate ee 2025)}}

\begin{multicols}{2}
\begin{enumerate}[label=(\Alph*)]
\item 2
\item 4
\item 6
\item 8
\end{enumerate}
\end{multicols}
\vspace{0.5cm}

\noindent \textbf{Q.72} The number of irreducible representations of the point group of the molecule is \hfill{\textbf{(gate ee 2025)}}

\begin{multicols}{2}
\begin{enumerate}[label=(\Alph*)]
\item 1
\item 2
\item 3
\item 4
\end{enumerate}
\end{multicols}
\vspace{0.5cm}

\noindent \textbf{Q.73} When two H atoms of the above molecule are also replaced by F atoms, the point group of the resultant molecule will be \hfill{\textbf{(gate ee 2025)}}

\begin{multicols}{2}
\begin{enumerate}[label=(\Alph*)]
\item C$_i$
\item C$_{2h}$
\item C$_{2v}$
\item D$_{2h}$
\end{enumerate}
\end{multicols}
\vspace{0.5 cm}

\noindent \textbf{Common Data for Questions 74, 75:} \\
Reactivity of aryl amines towards electrophilic aromatic substitution is much higher than that of aliphatic amines. Hence differential reactivity of the amino group is desirable in many reactions.
\vspace{1cm}
\noindent \textbf{Q.74} The compound which on reacting with aniline will \textbf{NOT} form an acetanilide is \hfill{\textbf{(gate ee 2025)}}

\includegraphics[scale=1]{images/image13.png}
\vspace{1cm}

\noindent \textbf{Q.75} Aniline can be distinguished from methylamine by its reaction with \hfill{\textbf{(gate ee 2025)}}

\begin{multicols}{2}
\begin{enumerate}[label=(\Alph*)]
\item \textit{p}-toluenesulphonyl chloride / KOH
\item (i) NaNO$_2$ / HCl, 0-5$^\circ$C \quad (ii) alkaline $\beta$-naphthol
\item Sn / HCl
\item Acetyl chloride
\end{enumerate}
\end{multicols}

\noindent \textbf{Linked Answer Questions: Q.76 to Q.77 carry two marks each.}

\noindent \textbf{Q.76} In the reaction \hfill{\textbf{(gate ee 2025)}}

\includegraphics[scale=1.2]{images/image14.png}
\vspace{0.5cm}

\noindent \textbf{Q.77} Oxidation of \textbf{X} with chromic acid chiefly gives 
 \hfill{\textbf{(gate ee 2025)}}
 \includegraphics[width=15cm]{images/image15.png} 
 \vspace{1cm}

 \noindent \textbf{Q.78} In the reaction\\
\hspace*{1cm} AMP \hspace{0.5cm} $\xrightarrow[\text{175°C}]{\text{aq. NH}_3}$ \hspace{0.5cm} (X) + H$_3$PO$_4$\\
Compound X is \hfill{\textbf{(gate ee 2025)}}
\begin{multicols}{2}
\begin{enumerate}[label=(\Alph*)]
    \item Adenine
    \item Xanthine
    \item 2,6-diaminopurine
    \item Adenosine
\end{enumerate}
\end{multicols}
\vspace{0.5cm}

\noindent \textbf{Q.79} Compound X on treatment with conc. HCl gives \hfill{\textbf{(gate ee 2025)}}
\begin{multicols}{2}
\begin{enumerate}[label=(\Alph*)]
    \item Uric acid
    \item Adenine
    \item Hypoxanthine
    \item Guanine
\end{enumerate}
\end{multicols}
\vspace{0.5cm}

\noindent \textbf{Q.80} The reaction of ammonium chloride with BCl$_3$ at 140°C followed by treatment with NaBH$_4$ gives the product X. The formula of X is \hfill{\textbf{(gate ee 2025)}}
\begin{multicols}{2}
\begin{enumerate}[label=(\Alph*)]
    \item B$_3$N$_3$H$_3$
    \item B$_3$N$_3$H$_6$
    \item B$_3$N$_3$H$_{12}$
    \item[(D)] [BH-NH]$_n$
\end{enumerate}
\end{multicols}
\vspace{2cm}

\noindent \textbf{Q.81} Which of the following statement(s) is/are true for X?\hfill{\textbf{(gate ee 2025)}}\\
(i) X is not isoelectronic with benzene.\\
(ii) X undergoes addition reaction with HCl.\\
(iii) Electrophilic substitution reaction on X is much faster than that of benzene.\\
(iv) X undergoes polymerization at 90°C. \hfill{\textbf{(gate ee 2025)}}
\begin{multicols}{2}
\begin{enumerate}[label=(\Alph*)]
    \item i and ii
    \item only ii
    \item ii and iii
    \item i and iv
\end{enumerate}
\end{multicols}
\vspace{0.5cm}

\noindent \textbf{Q.82} Consider a particle of mass $m$ moving in a one-dimensional box under the potential $V=0$ for $0 \le x \le a$ and $V = \infty$ outside the box. When the particle is in its lowest energy state the average momentum $\langle p_x \rangle$ of the particle is \hfill{\textbf{(gate ee 2025)}}
\begin{multicols}{2}
\begin{enumerate}[label=(\Alph*)]
    \item $\langle p_x \rangle = 0$
    \item $\langle p_x \rangle = \dfrac{h}{a}$
    \item $\langle p_x \rangle = \dfrac{h}{2a}$
    \item $\langle p_x \rangle = \dfrac{h}{2\pi a}$
\end{enumerate}
\end{multicols}
\vspace{0.5cm}

\noindent \textbf{Q.83} The uncertainty in the momentum ($\Delta p_x$) of the particle in its lowest energy state is \hfill{\textbf{(gate ee 2025)}}
\begin{multicols}{2}
\begin{enumerate}[label=(\Alph*)]
    \item $\Delta p_x = 0$
    \item $\Delta p_x = \dfrac{h}{a}$
    \item $\Delta p_x = \dfrac{h}{2a}$
    \item $\Delta p_x = \dfrac{h}{2\pi a}$
\end{enumerate}
\end{multicols}
\vspace{0.5cm}


\noindent \textbf{Q.84} In the mixture obtained by mixing 25.0 mL 1.2 $\times$ 10$^{-3}$ M MnCl$_2$ and 35.0 mL of 6.0 $\times$ 10$^{-4}$ M KCl solution, the concentrations (M) of Mn$^{2+}$, K$^+$ and Cl$^-$ ions respectively are \hfill{\textbf{(gate ee 2025)}}
\begin{multicols}{2}
\begin{enumerate}[label=(\Alph*)]
    \item 6.0 $\times$ 10$^{-4}$, 3.0 $\times$ 10$^{-4}$, 1.5 $\times$ 10$^{-3}$
    \item 6.0 $\times$ 10$^{-4}$, 3.0 $\times$ 10$^{-4}$, 9.0 $\times$ 10$^{-4}$
    \item 5.0 $\times$ 10$^{-4}$, 3.5 $\times$ 10$^{-4}$, 1.35 $\times$ 10$^{-3}$
    \item 5.0 $\times$ 10$^{-4}$, 3.5 $\times$ 10$^{-4}$, 8.5 $\times$ 10$^{-4}$
\end{enumerate}
\end{multicols}
\vspace{0.5cm}
\noindent \textbf{Q.85} The activity (M) of Mn$^{2+}$ ions in the above solution is \hfill{\textbf{(gate ee 2025)}}
\begin{multicols}{2}
\begin{enumerate}[label=(\Alph*)]
    \item 1.0 $\times$ 10$^{-4}$
    \item 2.0 $\times$ 10$^{-4}$
    \item 3.0 $\times$ 10$^{-4}$
    \item 4.0 $\times$ 10$^{-4}$
\end{enumerate}
\end{multicols}
\vspace{0.5cm}
\begin{center}
    \textbf{END OF THE QUESTON PAPER}
\end{center}

\end{document}

