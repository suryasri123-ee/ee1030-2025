\documentclass[journal,12pt,onecolumn]{IEEEtran}
\usepackage{cite}
 \usepackage{caption}
\usepackage{graphicx}
\usepackage{amsmath,amssymb,amsfonts,amsthm}
\usepackage{algorithmic}
\usepackage{graphicx}
\usepackage{textcomp}
\usepackage{xcolor}
\usepackage{txfonts}
\usepackage{listings}
\usepackage{enumitem}
\usepackage{mathtools}
\usepackage{gensymb}
\usepackage{comment}
\usepackage[breaklinks=true]{hyperref}
\usepackage{tkz-euclide} 
\usepackage{listings}
\usepackage{gvv}
%\def\inputGnumericTable{}                                 
\usepackage[latin1]{inputenc} 
\usetikzlibrary{arrows.meta, positioning}
\usepackage{xparse}
\usepackage{color}                                            
\usepackage{array}                                            
\usepackage{longtable}                                       
\usepackage{calc}                                             
\usepackage{multirow}
\usepackage{multicol}
\usepackage{hhline}                                           
\usepackage{ifthen}                                           
\usepackage{lscape}
\usepackage{tabularx}
\usepackage{array}
\usepackage{float}

\usepackage{float}
%\newcommand{\define}{\stackrel{\triangle}{=}}
\theoremstyle{remark}
\usepackage{circuitikz}
\captionsetup{justification=centering}
\usepackage{tikz}

\title{Matrices in Geometry 1.5.23}
\author{EE25BTECH11035 - Kushal B N}
\begin{document}
\vspace{3cm}
\maketitle
{\let\newpage\relax\maketitle}
\textbf{Question: }
Show that the points $\vec{A}\brak{-2\hat{i}+3\hat{j}+5\hat{k}}$, $\vec{B}\brak{\hat{i}+2\hat{j}+3\hat{k}}$ and $\vec{C}\brak{7\hat{i}-\hat{k}}$ are collinear.\\

\textbf{Given: } 
$\vec{A}=\myvec{-2\\3\\5}$, $\vec{B}=\myvec{1\\2\\3}$ and $\vec{C}=\myvec{7\\0\\-1}$ are three points.\\

They are defined to be collinear if rank of the collinearity matrix is 1.\\
\begin{align*}
    \text{Collinearity matrix is }\myvec{\vec{A}- \vec{C} & \vec{B}-\vec{C}}^{\top}\\
\end{align*}
    \begin{equation}
    \vec{A}-\vec{C}=\myvec{-9\\3\\6}\\
    \end{equation}
    \begin{equation}
    \vec{B}-\vec{C}=\myvec{-6\\2\\4}\\
    \end{equation}
    \begin{equation}
    \implies \text{rank}\myvec{-9&3&6 \\ -6&2&4} = 1.
    \end{equation}
\begin{equation}
   \myvec{-9&3&6\\ -6&2&4}\overset{R_2 \rightarrow R_2 - \frac{2}{3}R_1}{\longrightarrow} \myvec{-9&3&6\\0&0&0}\\
\end{equation}
   
We know that for the rank of a matrix to be equal to 1, all the elements in the lower row of the matrix must be zero.\\
So it is proved that the given points are collinear.\\
\bigskip

\textbf{Conclusion: }
Hence, as the rank of the collinearity matrix is 1, it is proved that the given three points are collinear.


\begin{figure}[H]
    \centering
    \includegraphics[width=1\columnwidth]{figs/1.jpg}
    \caption{Plot for 1.5.23}
    \label{fig:placeholder}
\end{figure}
\end{document}