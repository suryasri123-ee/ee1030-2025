\documentclass{beamer}
\let\vec\mathbf
\mode<presentation>
\usepackage{amsmath}
\usepackage{amssymb}
%\usepackage{advdate}
\usepackage{adjustbox}
%\usepackage{subcaption}
\usepackage{enumitem}
\usepackage{multicol}
\usepackage{mathtools}
\usepackage{listings}
\usepackage{url}
\usetheme{Boadilla}
\usecolortheme{lily}
\setbeamertemplate{footline}
{
  \leavevmode%
  \hbox{%
  \begin{beamercolorbox}[wd=\paperwidth,ht=2.25ex,dp=1ex,right]{author in head/foot}%
    \insertframenumber{} / \inserttotalframenumber\hspace*{2ex} 
  \end{beamercolorbox}}%
  \vskip0pt%
}
\setbeamertemplate{navigation symbols}{}
\providecommand{\nCr}[2]{\,^{#1}C_{#2}} % nCr
\providecommand{\nPr}[2]{\,^{#1}P_{#2}} % nPr
\providecommand{\mbf}{\mathbf}
\providecommand{\pr}[1]{\ensuremath{\Pr\left(#1\right)}}
\providecommand{\qfunc}[1]{\ensuremath{Q\left(#1\right)}}
\providecommand{\sbrak}[1]{\ensuremath{{}\left[#1\right]}}
\providecommand{\lsbrak}[1]{\ensuremath{{}\left[#1\right.}}
\providecommand{\rsbrak}[1]{\ensuremath{{}\left.#1\right]}}
\providecommand{\brak}[1]{\ensuremath{\left(#1\right)}}
\providecommand{\lbrak}[1]{\ensuremath{\left(#1\right.}}
\providecommand{\rbrak}[1]{\ensuremath{\left.#1\right)}}
\providecommand{\cbrak}[1]{\ensuremath{\left\{#1\right\}}}
\providecommand{\lcbrak}[1]{\ensuremath{\left\{#1\right.}}
\providecommand{\rcbrak}[1]{\ensuremath{\left.#1\right\}}}
\theoremstyle{remark}
\newtheorem{rem}{Remark}
\newcommand{\sgn}{\mathop{\mathrm{sgn}}}

\providecommand{\res}[1]{\Res\displaylimits_{#1}} 
\providecommand{\norm}[1]{\lVert#1\rVert}
\providecommand{\mtx}[1]{\mathbf{#1}}

\providecommand{\fourier}{\overset{\mathcal{F}}{ \rightleftharpoons}}
%\providecommand{\hilbert}{\overset{\mathcal{H}}{ \rightleftharpoons}}
\providecommand{\system}{\overset{\mathcal{H}}{ \longleftrightarrow}}
	%\newcommand{\solution}[2]{\textbf{Solution:}{#1}}
%\newcommand{\solution}{\noindent \textbf{Solution: }}
\providecommand{\dec}[2]{\ensuremath{\overset{#1}{\underset{#2}{\gtrless}}}}
\newcommand{\myvec}[1]{\ensuremath{\begin{pmatrix}#1\end{pmatrix}}}

\title{Matrices in Geometry - 1.5.23}
\author{EE25BTECH11035  Kushal B N}
\date{Aug, 2025}

\begin{document}

\maketitle


\begin{frame}
\tableofcontents
\end{frame}
\section{Problem Statement}
\begin{frame}
\frametitle{Problem Statement}
Show that the points $\vec{A}\brak{-2\hat{i}+3\hat{j}+5\hat{k}}$, $\vec{B}\brak{\hat{i}+2\hat{j}+3\hat{k}}$ and $\vec{C}\brak{7\hat{i}-\hat{k}}$ are collinear.
\end{frame}

\section{Solution}
\begin{frame}{Solution}
   Given $\vec{A}=\myvec{-2\\3\\5}$, $\vec{B}=\myvec{1\\2\\3}$ and $\vec{C}=\myvec{7\\0\\-1}$ are three points.\\
   They are defined to be collinear if rank of the collinearity matrix is 1.
   \begin{align*}
    \text{Collinearity matrix is }\myvec{\vec{A}- \vec{C} & \vec{B}-\vec{C}}^{\top}\\
   \end{align*}
\begin{align}
\vec{A}-\vec{C}=\myvec{-9\\3\\6}
\end{align}
\begin{align}
\vec{B}-\vec{C}=\myvec{-6\\2\\4}
\end{align}
    
\end{frame}



\begin{frame}{Solution}
\begin{align}
\implies \text{rank}\myvec{-9&3&6 \\ -6&2&4} = 1
\end{align}
\begin{align}
 \myvec{-9&3&6\\ -6&2&4}\overset{R_2 \rightarrow R_2 - \frac{2}{3}R_1}{\longrightarrow} \myvec{-9&3&6\\0&0&0}
 \end{align}

We know that for the rank of a matrix to be equal to 1, all the elements in the lower row of the matrix must be zero.\\
So it is proved that the given points are collinear
\end{frame}

\section{Conclusion}
\begin{frame}{Conclusion}
Hence, as the rank of the collinearity matrix is 1, it is proved that the given three points are collinear.

\end{frame}
\end{document}