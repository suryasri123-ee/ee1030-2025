\let\negmedspace\undefined
\let\negthickspace\undefined
\documentclass[journal]{IEEEtran}
\usepackage[a5paper, margin=10mm, onecolumn]{geometry}
%\usepackage{lmodern} % Ensure lmodern is loaded for pdflatex
\usepackage{tfrupee} % Include tfrupee package

\setlength{\headheight}{1cm} % Set the height of the header box
\setlength{\headsep}{0mm}     % Set the distance between the header box and the top of the text

\usepackage{gvv-book}
\usepackage{gvv}
\usepackage{cite}
\usepackage{amsmath,amssymb,amsfonts,amsthm}
\usepackage{algorithmic}
\usepackage{graphicx}
\usepackage{textcomp}
\usepackage{xcolor}
\usepackage{txfonts}
\usepackage{listings}
\usepackage{enumitem}
\usepackage{mathtools}
\usepackage{gensymb}
\usepackage{comment}
\usepackage[breaklinks=true]{hyperref}
\usepackage{tkz-euclide} 
\usepackage{listings}
% \usepackage{gvv}                                        
\def\inputGnumericTable{}                                 
\usepackage[latin1]{inputenc}                                
\usepackage{color}                                            
\usepackage{array}                                            
\usepackage{longtable}                                       
\usepackage{calc}                                             
\usepackage{multirow}                                         
\usepackage{hhline}                                           
\usepackage{ifthen}                                           
\usepackage{lscape}
\begin{document}

\bibliographystyle{IEEEtran}
\vspace{3cm}

\title{1.5.6}
\author{EE25BTECH11018 - DARISY SREETEJ}
% \maketitle
% \newpage
% \bigskip
{\let\newpage\relax\maketitle}

\renewcommand{\thefigure}{\theenumi}
\renewcommand{\thetable}{\theenumi}
\setlength{\intextsep}{10pt} % Space between text and floats


\numberwithin{equation}{enumi}
\numberwithin{figure}{enumi}
\renewcommand{\thetable}{\theenumi}


\textbf{Question}:\\
\begin{enumerate}
\item The point which divides the line segment joining the points $\brak{7,-6}$ and $\brak{3, 4}$ in the ratio $1 : 2$ is \dots.
\end{enumerate}

\quad

\textbf{Solution:}
Let us consider the coordinates of $\vec{P}$  on $\vec{AB}$ such that $\vec{AP:PB}=1:2 ,$ where coordinates of A = $\myvec{7\\-6}$ and B are $\myvec{3\\4}$ are 
$P=\myvec{x\\y}$

\begin{table}[h!]    
  \centering
  \begin{center}
    \begin{tabular}{|c|c|} 
        \hline
            \textbf{Variable}  & \textbf{Formula} \\ 
        \hline
            $a$   & $a = \myvec{4 \\ -1 \\ 1}$ \\ 
        \hline
            $b$   &  $b = \myvec{2 \\ -2 \\ 1}$\\ 
        \hline
           \end{tabular}
\end{center}  

  \caption{Variables Used}
  \label{tab10.5.3.9.1}
\end{table}
\begin{align}
\vec{P}=\frac{k(\vec{B})+(\vec{A})}{k+1}=\myvec{x\\y}\\
\end{align}
Here according to problem value of k is 2\\
\begin{align}
\vec{P}=\frac{2(\vec{B})+(\vec{A})}{3}=\frac{2\myvec{7\\-6}+\myvec{3\\4}}{3}=\frac{\myvec{17\\-8}}{3}\\
\end{align}
\begin{align}
\vec{P}=\myvec{17/3\\-8/3}
\end{align}
Hence the coordinates of $\vec{P}$ are $\brak{17/3,-8/3}$
\begin{figure}
    \centering
    \includegraphics[width=0.7\linewidth]{figs/plot.png}
    \caption{Stem plot of y\brak{n}}
    \label{stemplot}
\end{figure}
\end{document}  
